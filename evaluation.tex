\subsection{Evaluation}
\label{sec:case-studies}

In this section, we present our experience in running \tool on two different applications: \emph{Courseware}: a
course registration system described by~\cite{gotsmanpopl16}, and
% 2. RUBiS: an eBay-like auction site~\cite{rubis} that allows users to
% bid for items on sale, and pay for items from a wallet modeled after a
% bank account, and 
the \emph{TPC-C} benchmark. 

\textbf{Courseware} The Courseware application allows new courses to be
added (via an \C{add\_course} transaction), and new students to be
registered (via a \C{register\_student} transaction) into a database. A registered
student can enroll (\C{enroll}) in an existing course,
provided that enrollment has not already exceeded the course
capacity (\C{c\_capacity}). A course with no enrollments can be
canceled (\C{cancel\_course}). Likewise, a student who is not enrolled
in any course can be deregistered (\C{deregister\_student}). Besides
\C{Student} and \C{Course} tables, there is also an \C{Enrollment}
table to track the many-to-many enrollment relationship between courses
and students. The application is required to preserve the following
invariants on the database:

\begin{enumerate}
\item  $I_1$: An enrollment record should always refer to an existing student and an existing course.
\item  $I_2$: The capacity (\C{c\_capacity}) of a course should always be a
  non-negative quantity.
\end{enumerate}
\noindent Both $I_1$ and $I_2$ can be violated under weak
isolation. $I_1$ can be violated, for example, when
\C{deregister\_student} runs concurrently with \C{enroll}, both at RC
isolation. While the former transaction removes the student record
after checking that no enrollments for that student exists, the latter
transaction concurrently adds an enrollment record after checking the
student exists.  Both can succeed concurrently, resulting in an
invalid state. Invariant $I_2$ can be violated by two \C{enroll}s,
both reading \C{c\_capacity}=1, and both (atomically) decrementing it,
resulting in \C{c\_capacity}=-1.  We ran \tool on the courseware
application written in our DSL, annotated with guarantees for all
transactions taken from ~\cite{gotsmanpopl16}, and asserted $I = I_1
\wedge I_2$ as the correctness condition. \tool was able to
automatically discover the anomalous executions described above, and
was subsequently able to infer that the anomalies can be preempted by
promoting the isolation level of \C{enroll} and
\C{deregister\_student} to SER (on both MySQL and PostgreSQL), leaving
the isolation levels of remaining transactions at RC.  The total time
for inference and verification took less than a minute running on a
conventional laptop.

\begin{table}[]
\centering
\begin{tabular}{l|c|c|c|c|c|}
\cline{2-6}
                                 & \multicolumn{1}{l|}{\C{new\_order}} & \multicolumn{1}{l|}{\C{delivery}} & \multicolumn{1}{l|}{\C{payment}} & \multicolumn{1}{l|}{\C{order\_status}} & \multicolumn{1}{l|}{\C{stock\_level}} \\ \hline
\multicolumn{1}{|l|}{MySQL}      & SER                                   & SER                                 & RC                                 & RC                                       & RC                                      \\ \hline
\multicolumn{1}{|l|}{PostgreSQL} & RR                                    & RR                                  & RC                                 & RC                                       & RC                                      \\ \hline
\end{tabular}
\caption{The discovered isolation levels for TPC-C transactions}
\label{tab:tpcc}
\end{table}

\textbf{TPC-C} The simplified schema of the TPC-C benchmark has been
described in Sec.~\ref{sec:motivation}. In addition to the tables
shown in Fig.~\ref{fig:schema}, the TPC-C schema also has
\C{Warehouse} and \C{New\_order} tables that are relevant for
verification.  To verify TPC-C, we examined four of the twelve
consistency conditions specified by the standard, which we name $I_1$
to $I_4$:

\begin{enumerate}
\item Consistency condition $I_1$  requires that the sales bottom line
of each warehouse equals the sum of the sales bottom lines of all
districts served by the warehouse.

\item Conditions $I_2$ and $I_3$ effectively enforce uniqueness of ids assigned
  to \C{Order} and \C{New\_order} records, respectively, under a district.


\item  Condition $I_4$ requires that the number of order lines under a district
  must match the sum of order line counts of all orders under the district.
% There are five transactions in TPC-C, out of which two are read-only.
% The three remaining transactions are \C{new\_order}, \C{delivery}, and
  % \C{payment}.
\end{enumerate}

Similar to the example discussed in Sec.~\ref{sec:motivation}, there
are a number of ways TPC-C's transactions violate the aforementioned
invariants under weak isolation. \tool was able to discover all such
violations when verifying the benchmark against $I =
\bigwedge_{i}I_i$, with guarantees of all three transactions
provided. The isolation levels were subsequently strengthened  as
shown in Table.~\ref{tab:tpcc}.  As before, inference and verification
took less than a minute.

%% \begin{table*}[t]\small
%% \centering
%% \begin{tabular}{|c|c|c|c|}
%%   \hline
%% \textbf{Transaction}   & \textbf{Invariant} 
%% & \textbf{MySQL-Isolation} & \textbf{PostgreSQL-Isolation} \\ 
%% \hline
%% \multirow{4}{*}{\C{New\_Order} }  & $I_1$ 
%% & RC &  RC\\ 
%% &  $I_2$ &SER & RR \\
%% &  $I_3$ & SER  &  RR  \\
%% & $I_4$ & RC & RC   \\
%% \hline
%% \multirow{4}{*}{\C{Payment}}  & $I_1 $ 
%% & RC &  RC\\ 
%% &  $I_2$  &RC & RC \\
%% &  $I_3 $ & RC  &  RC  \\
%% & $I_4$  & RC & RC   \\
%% \hline
%% \multirow{4}{*}{\C{Delivery}}  & $I_1$  
%% & RC &  RC \\ 
%% &  $I_2$ &SER & RR \\
%% &  $I_3$ & SER  &  RR \\
%% & $I_4$  & RC & RC   \\
%% \hline
%% \end{tabular}
%% \caption{Various invariant violations witnessed for the TPC-C
%%   benchmark on MySQL and PostgreSQL}
%% \label{tab:tpcc-eval}
%% \end{table*}

To sanity-check the results of \tool, we conducted experiments with a
high-contention OLTP workload on TPC-C aiming to explore the space of
correct isolation levels for different transactions. The workload
involves a mix of all five TPC-C transactions executing against a
TPC-C database with 10 warehouses. Each warehouse has 10 districts,
and each district serves 3000 customers. There are a total of 5
transactions in TPC-C, and given that MySQL and PostgreSQL support 3
isolation levels each, there are a total of $3^5 = 243$ different
configurations of isolation levels for TPC-C transactions on MySQL and
PostgreSQL. We executed the benchmark with all 243 configurations, and
found 171 of them violated at least one of the four invariants we
considered.  Table~\ref{fig:tpcc-eval} shows the summary of these
results. As expected, the isolation levels that \tool infers for the
TPC-C transactions do not result in invariant violations, either on
MySQL or on PostgreSQL, and were determined to be the weakest safe
assignments possible.

