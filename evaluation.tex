\subsection{Evaluation}

We evaluate \tool on two different applications: 1. Courseware: a
course registration system taken from~\cite{gotsmanpopl16}, and
% 2. RUBiS: an eBay-like auction site~\cite{rubis} that allows users to
% bid for items on sale, and pay for items from a wallet modeled after a
% bank account, and 
2. The TPC-C benchmark. 

\textbf{Courseware} Courseware application allows new courses to be
added (\C{add\_course} transaction), and new students to be
registered (\C{register\_student}) into the system. A registered
student can enroll (\C{enroll}) in one of the existing courses,
provided that the enrollments haven't already exceeded the course
capacity (\C{c\_capacity}). A course with no enrollments can be
canceled (\C{cancel\_course}). Likewise, a student who is not enrolled
in any course can be deregistered (\C{deregister\_student}). Besides
\C{Student} and \C{Course} tables, there is also an \C{Enrollment}
table tracks the many-to-many enrollment relationship between courses
and students. The application is required to preserve the following
invariants on the database: 1.  $I_1$: An enrollment record should
always refer to an existing student and an existing course, and 2.
$I_2$: the capacity (\C{c\_capacity}) of a course should always be a
non-negative quantity. Both $I_1$ and $I_2$ can be violated under weak
isolation. $I_1$ can be violated, for example, when
\C{deregister\_student} runs concurrently with \C{enroll}, both at RC
isolation. While the former transaction removes the student record
after checking that no enrollments for that student exist, the latter
transaction concurrently adds an enrollment record after checking the
student exist. Both can succeed concurrently, resulting in an invalid
state. Invariant $I_2$ can be violated by two \C{enroll}s, both
reading the \C{c\_capacity}=1, and both (atomically) decrementing it,
resulting in \C{c\_capacity}=-1.  We ran \tool on the courseware
application written in our DSL, annotated with guarantees for all
transactions, and provided the $I = I_1 \wedge I_2$ as the correctness
condition. \tool was able to automatically discover anomalous
executions described above, and was subsequently able to infer that
the anomalies can be preempted by promoting the isolation level of
\C{enroll} and \C{deregister\_student} to SER (on both MySQL and
PostgreSQL), leaving the isolation levels of remaining transactions at
RC.

\begin{table}[]
\centering
\begin{tabular}{l|c|c|c|c|c|}
\cline{2-6}
                                 & \multicolumn{1}{l|}{\C{new\_order}} & \multicolumn{1}{l|}{\C{delivery}} & \multicolumn{1}{l|}{\C{payment}} & \multicolumn{1}{l|}{\C{order\_status}} & \multicolumn{1}{l|}{\C{stock\_level}} \\ \hline
\multicolumn{1}{|l|}{MySQL}      & SER                                   & SER                                 & RC                                 & RC                                       & RC                                      \\ \hline
\multicolumn{1}{|l|}{PostgreSQL} & RR                                    & RR                                  & RC                                 & RC                                       & RC                                      \\ \hline
\end{tabular}
\caption{The discovered isolation levels for TPC-C transactions}
\label{tab:tpcc}
\end{table}

\textbf{TPC-C} The simplified schema of the TPC-C benchmark has been
described in Sec.~\ref{sec:motivation}. In addition to the tables
shown in Fig.~\ref{fig:schema}, the TPC-C schema also has
\C{Warehouse} and \C{New\_order} tables that are relevant for
verification. To verify TPC-C, we consider four of the twelve
consistency conditions specified by the standard, which we name $I_1$
to $I_4$. Consistency condition $I_1$  requires that sales bottom line
of each warehouse equals to the sum of the sales bottom lines of all
district served by the warehouse.  Conditions $I_2$ and $I_3$
effectively enforce uniqueness of ids assigned to \C{Order} and
\C{New\_order} records, respectively, under a district. Condition
$I_4$ requires that the number of order lines under a district must
match the sum of order line counts of all orders under the district.
% There are five transactions in TPC-C, out of which two are read-only.
% The three remaining transactions are \C{new\_order}, \C{delivery}, and
% \C{payment}.  
Similar to the example discussed in Sec.~\ref{sec:motivation}, there
are a number of ways TPC-C's transactions violate the aforementioned
invariants under weak isolation. \tool was able to discover all such
violations when verifying the benchmark against $I =
\bigwedge_{i}I_i$, with guarantees of all three transactions
provided. The isolation levels were subsequently strengthened  as
shown in Table.~\ref{tab:tpcc}.

\begin{table*}[t]\small
\centering
\begin{tabular}{|c|c|c|c|}
  \hline
\textbf{Transaction}   & \textbf{Invariant} 
& \textbf{MySQL-Isolation} & \textbf{PostgreSQL-Isolation} \\ 
\hline
\multirow{4}{*}{\C{New\_Order} }  & $I_1$ 
& RC &  RC\\ 
&  $I_2$ &SER & RR \\
&  $I_3$ & SER  &  RR  \\
& $I_4$ & RC & RC   \\
\hline
% \multirow{4}{*}{\C{Payment}}  & $I_1 $ 
% & RC &  RC\\ 
% &  $I_2$  &RC & RC \\
% &  $I_3 $ & RC  &  RC  \\
% & $I_4$  & RC & RC   \\
% \hline
\multirow{4}{*}{\C{Delivery}}  & $I_1$  
& RC &  RC \\ 
&  $I_2$ &SER & RR \\
&  $I_3$ & SER  &  RR \\
& $I_4$  & RC & RC   \\
\hline
\end{tabular}
\caption{Various invariant violations witnessed for the TPC-C
  benchmark on MySQL and PostgreSQL}
\label{tab:tpcc-eval}
\end{table*}

To sanity-check the results of \tool, we conducted experiments with
the high-contention OLTP workload  on TPC-C aiming to explore the
space of correct isolation levels for different transactions. The
workload involves  a mix of all five TPC-C transactions executing
against a TPC-C database with 10 warehouses. Each warehouse has 10
districts, and each district serves 3000 customers. There are a total
of 5 transactions in TPC-C, and given that MySQL and PostgreSQL
support 3 isolation levels each, there are a total of $3^5 = 243$
different configurations of isolation levels for TPC-C transactions on
MySQL and PostgreSQL. We executed the OLTP workload on the TPC-C
benchmark with all 243 configurations, and found 171 of them violating
at least one invariant.  Table~\ref{fig:tpcc-eval} shows the summary
of these results. As expected, the isolation levels that \tool infers
for the TPC-C transactions do not result in invariant violations,
either on MySQL or on PostgreSQL.

