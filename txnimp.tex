%\renewcommand{\ctxn}[3]{\C{TXN}_{#1}\langle #2 \rangle\{#3\}}
\begin{figure*}[!ht]
\raggedright
%
\textbf{Syntax}\\
%
\begin{smathpar}
\renewcommand{\arraystretch}{1.2}
\begin{array}{lclcl}
\multicolumn{5}{c} {
  {x,y} \in \mathtt{Variables}\qquad
  {f} \in \mathtt{Field\;Names} \qquad
% {\tau} \in \mathtt{Table\;Names}
  {i,j} \in \mathbb{N} \qquad
  {\odot} \in \{+,-,\le,\ge,=\}\qquad
  {k} \in \mathbb{Z}\cup\mathbb{B} \qquad
  {\rec} \in \{\bar{f}=\bar{k}\}\
}\\
{\stl,\stg,s} & \in & \mathtt{State} & \coloneqq &  \Pow{\{\bar{f}=\bar{k}\}} \\
{\I_e, \I_c }  & \in & \mathtt{Isolation Spec} & \coloneqq & (\stl,\stg,\stg') \rightarrow \Prop\\
v & \in & \mathtt{Values} & \coloneqq & k \ALT \rec \ALT s\\
e & \in & \mathtt{Expressions} & \coloneqq & k \ALT x \ALT x.f 
    \ALT \{\bar{f}=\bar{e}\} \ALT e_1 \odot e_2\\ 
c & \in & \mathtt{Commands} & \coloneqq & \cskip \ALT \lete{x}{e}{c}
    \ALT \ite{e}{c_1}{c_2}\ALT c_1;c_2 \ALT \inserte{x}  \\
&&&&\ALT \deletee{\lambda x.e}
    \ALT \lete{x}{\selecte{\lambda x.e}}{c}
    \ALT \updatee{\lambda x.e_1}{\lambda x.e_2}\\
&&&&\ALT \foreache{x}{\lambda y.\lambda z. c} 
    \ALT \foreachr{s_1}{s_2}{\lambda x.\lambda y. e}\\
&&&&\ALT \stxn{i}{\I_e,\I_c}{ c } \ALT \ctxn{i}{\I_e,\I_c,\stl,\stg}{c} \ALT c1 || c2\\
% t & \in & \mathtt{Terms} & \coloneqq & e \ALT c\\
\ectx & \in & \mathtt{Eval\;Ctx} & ::= & \bullet \ALT  
  \bullet || c_2 \ALT c_1 || \bullet \ALT \bullet;\,c_2 
  \ALT \ctxn{i}{\I_e,\I_c,\stl,\stg}{\bullet} \\
\end{array}
\end{smathpar}
%
\bigskip

\renewcommand{\arraystretch}{1.2}

%
\textbf{Local Reduction} \quad 
\fbox {\(\stg \vdash (c,\stl) \stepsto (c',\stl')\)}\\
%
\begin{minipage}{2.8in}
\rulelabel{E-Insert}
\begin{smathpar}
\begin{array}{c}
\RULE
{
  i \not\in \dom(\stl \cup \stg)\\
  r = \{\bar{f}=\bar{k};\,\idf=i;\,\delf=\C{false}\}
}
{
  \stg \vdash (\inserte{\{\bar{f}=\bar{k}\}},\stl) \stepsto
  (\cskip,\stl \cup \{r\})
}
\end{array}
\end{smathpar}
\end{minipage}
%
%
\begin{minipage}{2.8in}
\rulelabel{E-Delete}
\begin{smathpar}
\begin{array}{c}
\RULE
{
  s = \{r' \,|\, \exists(r\in\Delta).~ \eval([r/x]e)=\C{true} \\
        \hspace*{0.7in}\conj r'=\{\bar{f}=r.\bar{f}; \idf=r.\idf;
        \delf=\C{true}\}\}\\
% \dom(s) \cap \dom(\delta) = \emptyset
}
{
  \stg \vdash (\deletee{\lambda x.e},\stl) \stepsto (\cskip,\stl \cup s)
}
\end{array}
\end{smathpar}
\end{minipage}
%
\bigskip

%
\begin{minipage}{2.8in}
\rulelabel{E-Select}
\begin{smathpar}
\begin{array}{c}
\RULE
{
  s = \{r\in\Delta \,|\, \eval([r/x]e)=\C{true}\}\spc
  c' = [s/y]c
}
{
  \stg \vdash (\lete{y}{\selecte{\lambda x.e}}{c}, \stl) \stepsto 
              (c',\stl)
}
\end{array}
\end{smathpar}
\end{minipage}
%
%
\begin{minipage}{2.8in}
\rulelabel{E-Update}
\begin{smathpar}
\begin{array}{c}
\RULE
{
  s = \{r' \,|\, \exists(r\in\Delta).~ \eval([r/x]e_2)=\C{true} \conj r'=[r/x]e_1\}\\
}
{
  \stg \vdash (\updatee{\lambda x.e_1}{\lambda x.e_2},\stl) \stepsto 
              (\cskip,\stl \cup s)
}
\end{array}
\end{smathpar}
\end{minipage}
%

\begin{smathpar}
\begin{array}{ll}
  \rulelabel{E-Foreach1} & \stg \vdash (\foreache{s}{\lambda y.\lambda
  z.c},\stl) \stepsto (\foreachr{\emptyset}{s}{\lambda y.\lambda z. c})\\
  \rulelabel{E-Foreach2} & \stg \vdash (\foreachr{s_1}{\{r\} \uplus s_2}{\lambda y.\lambda
  z.c},\stl) \stepsto ([r/z][s_1/y]c;\,\foreachr{s_1 \cup \{r\}}{s_2}{\lambda y.\lambda z. c})\\
  \rulelabel{E-Foreach3} & \stg \vdash (\foreachr{s}{\emptyset}{\lambda y.\lambda
  z.c},\stl) \stepsto (\cskip,\stl)\\
\end{array}
\end{smathpar}
%
\bigskip

%
\textbf{Top-Level Reduction} \quad 
\fbox {\((c,\stg) \stepsto (c',\stg')\)}\\
%
\begin{minipage}{3in}
  \rulelabel{E-Txn-Start}
  \begin{smathpar}
  \begin{array}{c}
    \RULE{}
         {(\stxn{i}{\I_e,\I_c}{c},\stg) \stepsto (\ctxn{i}{\I_e,\I_c,\emptyset,\stg}{c},\stg)}
  \end{array}
  \end{smathpar}
\end{minipage}%
\hfill
\begin{minipage}{3in}
\rulelabel{E-Txn}
\begin{smathpar}
\begin{array}{c}
\RULE
{
  \I_e\,\,(\stl,\stg,\stg')\spc
  \stg \vdash (c,\stl) \stepsto (c',\stl')
}
{
  (\ctxn{i}{\I_e,\I_c,\stl,\stg}{c},\stg') \stepsto
  (\ctxn{i}{\I_e,\I_c,\stl',\stg'}{c'},\stg')
}
\end{array}
\end{smathpar}
\end{minipage}\\

\begin{center}
\begin{minipage}{3in}
\rulelabel{E-Commit}
\begin{smathpar}
\begin{array}{c}
\RULE
{
  \I_c\,\,(\stl,\stg,\stg')
}
{
  (\ctxn{i}{\I_e,\I_c,\stl,\stg}{\cskip},\stg') \stepsto (\cskip,\stl\gg\stg')
}
\end{array}
\end{smathpar}
\end{minipage}
\end{center}
\hfill
%
\caption{\small \txnimp: Syntax and Small-step semantics}
\label{fig:txnimp}
\end{figure*}

