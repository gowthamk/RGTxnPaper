\subsection{Isolation Specifications}
\label{sec:ansi-isolation}


% ISOLATION SPECS
% ---------------
\begin{figure*}[t!]
\begin{smathpar}
\begin{array}{lcl}
\underE{\C{RMWVis}(T_j)} & \defeq & \forall\eta_1,\eta_2.\,
       \underE{\{\eta_1,\eta_2\} \subseteq T_j} \conj
       \underE{\eta_1 \soar \eta_2} \Rightarrow \underE{\eta_1 \visar
       \eta_2}\\
\underE{\C{MonotonicVis}(T_j)} & \defeq & 
       \underE{\C{RMWVis}(T_j)} \conj 
       \forall\eta,\eta_1,\eta_2.\, \underE{\{\eta_1,\eta_2\} \in T_j} 
          \conj \\
  &   & \hspace*{1.2in}\underE{\eta \visar \eta_1} \conj
        \underE{\eta_1 \soar \eta_2} \Rightarrow \underE{\eta \visar
        \eta_2} \\
%  \C{CausalVis}(T_i) & \defeq & 
%         \C{MonotonicVis}(T_i) \conj \C{RMWVis}(T_i)\\
\underE{\C{AtomicVis}(T_j)} & \defeq & 
       \forall\eta_1,\eta_2.\, \neg(\underE{\eta_1 \in T_j}) \conj
       \underE{\eta_2 \in T_j} \conj
       \underE{\eta_1 \visar \eta_2} \Rightarrow \underE{\txn(\eta_1)
       \visar \eta_2}\\
\underE{\C{CommitVis}(T_j)} & \defeq & 
       \forall\eta_1,\eta_2.~ \neg(\underE{\eta_1 \in T_j}) \conj 
          \underE{\eta_2 \in T_j} \conj
       \underE{\eta_1 \visar \eta_2} \Rightarrow\\
  &   & \hspace*{1.2in}\exists\eta.\, \underE{\eta \in \txn(\eta_1)} 
        \conj \kind(\eta) = \C{COMMIT} 
        \conj \underE{\eta \visar \eta_2}\\
% \underE{\C{TransVis}(T_j)} & \defeq &  \forall
%        \eta_1,\eta_2,\eta_3.\, \underE{\eta_3 \in T_j} \conj
%        \underE{\eta_1 \visar \eta_2} \conj
%        \underE{\eta_2 \visar \eta_3} \Rightarrow \underE{\eta_1 \visar
%        \eta_3} \\
\underE{\C{RC}(T_j)} & \defeq & \underE{\C{AtomicVis}(T_j)} 
        \conj \underE{\C{CommitVis}(T_j)}\\
\underE{\C{MAV}(T_j)} & \defeq & \underE{\C{RC}(T_j)} \conj
      \underE{\C{MonotonicVis}(T_j)} \\
\underE{\C{SnapshotVis}(T_i,T_j)} & \defeq &  \underE{T_i
       \visar T_j} \disj \underE{T_i \invisar T_j}\\
\underE{\C{RR}(T_j)} & \defeq & \underE{\C{MAV}(T_j)}
       \conj \forall T_i.\,T_i \neq T_j \Rightarrow 
        \underE{\C{SnapshotVis}(T_i,T_j)} \\
% &   & \hspace*{2in}\conj  \C{SnapshotVis}(T_i,T_j)\\
\underE{\C{SnapshotSER}(T_i,T_j,X)} & \defeq &  \underE{T_i
       \visar T_j} \disj (\underE{T_i \invisar T_j} \conj \\
  &   &\hspace*{1.2in}\exists \eta.~\underE{\eta\in T_i} \conj \kind(\eta) =
       \C{WR(X)} \Rightarrow \underE{T_j \visar \eta})\\
\underE{\C{SI}(T_j)} & \defeq &  \underE{\C{RR}(T_j)}
       \conj \forall T_i.\,(T_i \neq T_j \conj \exists X.\, \underE{T_i \wrstoar X} \conj 
        \underE{T_j \wrstoar X})\\
%      (\exists X.\, T_i \wrstoar X \conj T_j \wrstoar X)
%      \Rightarrow  \C{TotalVis}(T_i,T_j)\\
  &   & \hspace*{2in} \Rightarrow \underE{\C{SnapshotSER}(T_i,T_j,X)}\\
\underE{\C{SER}(T_j)} & \defeq & \underE{\C{RR}(T_j)}
        \conj \forall T_i.\,(T_i \neq T_j \conj \exists X.\, 
        \underE{T_i \wrstoar X} \conj \underE{T_j \usesar X})\\
  &   & \hspace*{2in} \Rightarrow \underE{\C{SnapshotSER}(T_i,T_j,X)}\\
\end{array}
\end{smathpar}

\caption{Standard isolation guarantees expressed as trace
well-formedness constraints}
\label{fig:ansi-isolation}
\end{figure*}




We now describe specifications of standard isolation guarantees
expressed as constraints over trace well-formedness. For brevity and
convenience, we adopt few notations in this and subsequent sections.
An execution trace is destructed as ($\A$,$\visZ$) whenever individual
components of the pair are needed. Otherwise, it is written as $\E$.
Sometimes, the dot notation (\eg~$\E.\A$) is also used. Since $\A$ and
$\visZ$ are both sets, we lift the operations on sets to pairs of sets
when updating $\E$. For example, $\E' = \E \cup
(\{\eta_2\},\{(\eta_1,\eta_2)\})$ expands to $\E' = (\E.\A \cup
\{\eta_2\},\,\E.\visZ \cup \{(\eta_1,\eta_2)\})$.  When $\psi$ is a
formula, $\underE{\psi}$ denotes the interpretation of $\psi$ in the
context of the trace $\E$. Such interpretations are defined on a
case-by-case basis in Figs.~\ref{fig:rel-defs}
and~\ref{fig:ansi-isolation}. 

Fig.~\ref{fig:rel-defs} shows various relations defined over elements
in a trace. In the context of a trace $(\A,\visZ)$, an effect $\eta$
is said to belong to a transaction $T_i$ if $\eta$ belongs to the
effect set $A$ and its transaction identifier is $T_i$. The
containment relation is trivially lifted to the set of effects to
define $\underE{S \subseteq T_i}$.  Visibility and session order
relations are denoted by $\eta_1 \visar \eta_2$ and $\eta_1 \soar
\eta_2$, respectively. A transaction $T_i$ is said to be visible to an
effect $\eta$ if every effect $\eta_1$ of $T_i$ recorded by the trace
is visible to $\eta$.  $T_i$ may be visible to $\eta$ but may not be
visible to every other effect in the $\txn(\eta)$. For a transaction
$T_i$ to be considered to be visible to a transaction $T_j$ in the
context of a trace $\E$ (written $\underE{T_i \visar T_j}$), every
effect ($\eta_1$) of $T_i$ present in $\E$ must be visible to every
effect ($\eta_2$) of $T_j$ in $\E$.  Conversely, if none of the
effects of $T_i$ present in $\E$ are visible to any effect of $T_j$,
then $T_i$ is considered invisible to $T_j$ under $\E$ (written
$\underE{T_i \invisar T_j}$). Transaction $T_i$ is said to have
written to a variable $X$ under $\E$ (i.e., $\underE{T_i \wrstoar X}$)
if there exists a write-to-$X$ effect from $T_i$ in $\E$.
\emph{Reads-from} ($\underE{T_i \rdsfmar X}$) is defined similarly.
$T_i \usesar X$ if it reads or writes $X$.

Fig.~\ref{fig:ansi-isolation} shows various isolation guarantees
defined as propositions indexed by transaction identifiers.
Transaction $T_j$ is said to have experienced \emph{read-my-writes}
visibility under $\E$ if every effect ($\eta_1$) of $T_j$ is visible
to every subsequent effect ($\eta_2$) of the same transaction in $\E$.
This lets $T_j$ to never lose its own updates. Monotonic visibility
adds one more constraint to read-my-writes; besides requiring $\eta_1$
to be visible to $\eta_2$, it also requires every effect ($\eta$)
visible to $\eta_1$ in $\E$ to be visible to $\eta_2$ as well. Thus,
later operations in a transaction witness at least the same set of
effects witnessed by the earlier operations, if not more (hence,
``monotonic'' visibility). Atomic visibility allows an effect
$\eta_2$ of $T_j$ to witness an effect $\eta_1$ of $T_i$ only if all
effects of $T_i$ in $\E$ are also visible to $\eta_2$. Atomic
visibility thus prevents a transaction from being partially visible.
However, atomic visibility does not prevent an uncommitted transaction
from being visible. This is addressed by \C{CommitVis}, which requires
the commit effect of $T_i$ to be visible whenever any effect of $T_i$
is visible.

The ANSI SQL 92 standard requires \iso{Read Committed} isolation to
avoid the dirty reads phenomenon, which is achieved by enforcing
$\mathtt{AtomicVis}$ and $\mathtt{CommitVis}$ guarantees. The {\sc rc}
specification\footnote{Note the distinction in fonts: {\sc rc} is an
abbreviation of \iso{Read Committted}, whereas \C{RC} is its
specification.} (\C{RC}) is therefore a combination of these two
guarantees. The specification also agrees with the description and
implementation~\cite{bailishat,pldi15} of {\sc rc} for highly
available replicated stores. On relational databases, however, {\sc
rc} has also come to be associated with the $\mathtt{MonotonicVis}$
guarantee.  Nonetheless, $\mathtt{AtomicVis}$ and $\mathtt{CommitVis}$
are sufficient to reason about {\sc rc} isolation on relational stores
too. The combination of these guarantees with the {\sc sc} property of
relational stores (formalized in \S\ref{sec:store-consistency})
automatically leads to the monotonicity guarantee, which explains why
{\sc rc} comes with $\mathtt{MonotonicVis}$ on such stores regardless
of the implementation. On weakly consistent stores however,
$\mathtt{AtomicVis}$ and $\mathtt{CommitVis}$ do not imply
$\mathtt{MonotonicVis}$. A stronger isolation level called
\iso{Monotonic Atomic View} ({\sc mav} of
Fig.~\ref{fig:ansi-isolation})~\cite{bailishat} was proposed to
explicitly extend {\sc rc} with monotonicity on such stores. 

\begin{figure}
\centering
\subcaptionbox {
  {\sc rr}($T_1$), {\sc rr}($T_2$).
  \label{fig:ansi-iso-eg-rr}
} [
  0.33\columnwidth
] {
  \includegraphics[scale=0.36]{Figures/rr-eg}
}
\subcaptionbox {
  {\sc si}($T_1$), {\sc mav}($T_2$)
  \label{fig:ansi-iso-eg-si}
} [
  0.33\columnwidth
] {
  \includegraphics[scale=0.36]{Figures/si-eg-2}
}
\subcaptionbox {
  {\sc ser}($T_3$), {\sc mav}($T_4$)
  \label{fig:ansi-iso-eg-ser}
}{
  \includegraphics[scale=0.32]{Figures/ser-eg}
}

\caption{Sample executions of concurrent transactions at different
isolation levels. Solid (black) arrows indicate the timeline. Points
on the timeline mark the time when an operation is executed. Dotted
(blue) arrows denote $\visZ$. }
\label{fig:ansi-iso-eg}
\end{figure}

Snapshot visibility (\C{SnapshotVis}) captures the scenario where a
transaction executes against a static snapshot of the database.  A
transaction $T_i$ is said to be snapshot-visible to a transaction
$T_j$ if either it is visible to $T_j$ (i.e., $T_i$ is included in the
snapshot), or it is invisible (i.e., it is not included); it is
forbidden for only a suffix (more generally, a subset) of $T_j$ to
witness $T_i$. \iso{Repeatable Read} ({\sc rr}) isolation extends {\sc
mav} with the snapshot visibility guarantee.  Observe that snapshot
visibility permits $T_i$ and $T_j$ to execute and commit while being
oblivious of each other. This scenario is captured in
Fig.~\ref{fig:ansi-iso-eg-rr}, where transactions $T_1$ and $T_2$,
which perform conflicting writes, execute against a snapshot of the
database and commit concurrently. While the actual values read and
written by $T_1$ and $T_2$ are unimportant (hence, elided), it is
important to note the \emph{absence} of visiblity arrows from $T_2$ to
$T_1$, although $T_2$ commits before $T_1$'s read-from-$Y$.

\iso{Snapshot Isolation} ({\sc si}) proscribes this possibility. If
$T_i$ and $T_j$ both write to the same shared variable, then {\sc si}
insists that either $T_i$ be visible to $T_j$, or $T_j$ be visible to
the conflicting write of $T_i$ (this is captured by the auxiliary
definition \C{SnapshotSER}). Fig.~\ref{fig:ansi-iso-eg-si} shows an
execution where {\sc si} transaction $T_1$ witnesses a snapshot of the
database that doesn't include $T_2$. Due to \C{SI($T_1$)}, the
conflicting write to $X$ in $T_2$ is now required to witness $T_1$, as
captured by the direction of the $\visZ$ arrow (subsequent operations
also witness $T_1$ because $T_2$ is executing under {\sc mav}).
Lastly, \iso{Serializable} isolation extends $\mathtt{SnapshotSER}$ to
also cover non-conflicting transactions that write to variables read
in the current transaction.  Fig.~\ref{fig:ansi-iso-eg-ser} shows an
execution of an {\sc ser} transaction $T_3$ and a {\sc mav
transaction} $T_4$, which write to $X$ and $Y$, respectively. The
snapshot witnessed by the {\sc ser} transaction $T_3$ does not include
$T_4$, but the write to $Y$ in $T_4$, although non-conflicting,
witnesses $T_1$ because $Y$ is read by $T_1$. Visibility includes
other operations of $T_4$ because of {\sc mav}.

% Note that both $\mathtt{SnapshotVis}$ and $\mathtt{OneWaySER}$
% are asymmetric definitions that guarantee complete visiblity or
% invisibility of $T_i$ to $T_j$, but not the converse. While
% $\mathtt{SnapshotVis}$ provides no guarantees to $T_i$,
% $\mathtt{OneWaySER}$ guarantees that at least the commit effect of
% $T_i$ witnesses $T_j$. 

% \iso{Snapshot Isolation} spec ({\sc si}) extends {\sc rr} with a
% one-way serializability guarantee w.r.t. the transactions that perform
% conflicting writes (i.e., writes to the same shared variable).
% \footnote{Fig.~\ref{fig:ansi-isolation} presents slightly
% weaker versions of {\sc si} and {\sc ser} specs in the interest of
% clarity.}. Fig.~\ref{fig:ansi-iso-eg} shows sample executions of
% transactions $T_1$ and $T_2$. Both transactions read and write to
% shared variables $X$ and $Y$. In the first execution, $T_2$ commits
% while $T_1$ is still in progress, but {\sc rr} isolation prevents
% $T_1$ from witnessing the writes of $T_2$. In the second execution,

%% SJ: Not sure this paragraph is necessary here.
%% In contrast to recent proposals (\emph{e.g.},
%% ~\cite{gotsmanconcur15}), our specifications for {\sc si} and {\sc
%%   ser} do not necessarily impose a total order among transactions
%% (conflicting or otherwise). In reality, a total order under {\sc ser}
%% (resp. {\sc si}) is guaranteed only if the store executes all
%% transactions under {\sc ser} (resp. {\sc si}) isolation. Our
%% specifications admit this possibility, and derive a total order under
%% the assumption of homogenity. However, databases almost always allow
%% isolation levels to be configured on a per-transaction basis, allowing
%% transactions at various isolation levels to coexist.  Specifications
%% that assume homogenity are incorrect under this setting.

% Having specified isolation levels as trace well-formedness
% constraints, we can now construct trace invariants ($\I$) for
% \txnimp programs by composing isolation specifications for various
% transactions. For instance, the following trace invariant enforces
% {\sc si} for both transactions of the program in
% Fig.~\ref{fig:motiv-eg-1}, allowing it to satisfy its postcondition:
% \begin{smathpar}
% \I \;=\; \lambda\E.~ \underE{\C{SI(Wd1)}} \conj \underE{\C{SI(Wd2)}}
% \end{smathpar}
% In contrast, the following trace invariant enforces {\sc rc} for one and {\sc si}
% for another, leading to a possible violation of the postcondition:
% \begin{smathpar}
% \I \;=\; \lambda\E.~ \underE{\C{RC(Wd1)}} \conj \underE{\C{SI(Wd2)}}
% \end{smathpar}
