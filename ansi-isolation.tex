\subsection{Isolation Specifications}
\label{sec:ansi-isolation}

\begin{figure*}[t!]
\begin{smathpar}
\begin{array}{lcl}
(\A,\visZ) \Vdash \eta \in T_i & \Leftrightarrow & \eta \in \A \conj \txn(\eta) = T_i\\
(\A,\visZ) \Vdash \eta_1 \visar \eta_2 & \Leftrightarrow & \{\eta_1,\eta_2\}
        \subseteq \A \conj (\eta_1,\eta_2) \in \visZ\\
(\A,\visZ) \Vdash \eta_1 \soar \eta_2 & \Leftrightarrow & \{\eta_1,\eta_2\}
        \subseteq \A \conj \txn(\eta_1)=\txn(\eta_2) \conj \id(\eta_1)
        < \id(\eta_2)\\
\E \Vdash S \subseteq T_i & \Leftrightarrow & \forall \eta.~ \eta
        \in S \Rightarrow \underE{\eta \in T_i} \\
\E \Vdash T_i \visar \eta & \Leftrightarrow &\forall\eta_1
        .\,(\E \Vdash \eta_1 \in T_i) \Rightarrow \E \Vdash \eta_1 \visar \eta \\
\underE{T_i \visar T_j} & \Leftrightarrow &  \forall\eta_1,\eta_2.\,
    %\sameobj{\eta_1}{\eta_2}  \Rightarrow 
    \underE{\eta_1\in T_i} \conj \underE{\eta_2 \in T_j} \Rightarrow 
    \underE{\eta_1 \visar \eta_2} \\
\underE{T_i \invisar T_j} & \Leftrightarrow &  \forall\eta_1,\eta_2.\,
        %\sameobj{\eta_1}{\eta_2}\Rightarrow 
        \underE{\eta_1\in T_i} \conj \underE{\eta_2 \in T_j} \Rightarrow 
        \neg (\underE{\eta_1 \visar \eta_2})\\
\underE{T_i \wrstoar X} & \Leftrightarrow & \exists\eta.~
        \underE{\eta \in T_i} \conj \kind(\eta) = \C{WR}(X)\\
\underE{\C{RMWVis}(T_j)} & \Leftrightarrow & \forall\eta_1,\eta_2.\,
       \underE{\{\eta_1,\eta_2\} \subseteq T_j} \conj
       \underE{\eta_1 \soar \eta_2} \Rightarrow \underE{\eta_1 \visar
       \eta_2}\\
\underE{\C{MonotonicVis}(T_j)} & \Leftrightarrow & 
       \underE{\C{RMWVis}(T_j)} \conj 
       \forall\eta,\eta_1,\eta_2.\, \underE{\{\eta_1,\eta_2\} \in T_j} 
          \conj \\
  &   & \hspace*{1.2in}\underE{\eta \visar \eta_1} \conj
        \underE{\eta_1 \soar \eta_2} \Rightarrow \underE{\eta \visar
        \eta_2} \\
%  \C{CausalVis}(T_i) & \Leftrightarrow & 
%         \C{MonotonicVis}(T_i) \conj \C{RMWVis}(T_i)\\
\underE{\C{AtomicVis}(T_j)} & \Leftrightarrow & 
       \forall\eta_1,\eta_2.\, \neg(\underE{\eta_1 \in T_j}) \conj
       \underE{\eta_2 \in T_j} \conj
       \underE{\eta_1 \visar \eta_2} \Rightarrow \underE{\txn(\eta_1)
       \visar \eta_2}\\
\underE{\C{CommitVis}(T_j)} & \Leftrightarrow & 
       \forall\eta_1,\eta_2.~ \neg(\underE{\eta_1 \in T_j}) \conj 
          \underE{\eta_2 \in T_j} \conj
       \underE{\eta_1 \visar \eta_2} \Rightarrow\\
  &   & \hspace*{1.2in}\exists\eta.\, \underE{\eta \in \txn(\eta_1)} 
        \conj \kind(\eta) = \C{COMMIT} 
        \conj \underE{\eta \visar \eta_2}\\
% \underE{\C{TransVis}(T_j)} & \Leftrightarrow &  \forall
%        \eta_1,\eta_2,\eta_3.\, \underE{\eta_3 \in T_j} \conj
%        \underE{\eta_1 \visar \eta_2} \conj
%        \underE{\eta_2 \visar \eta_3} \Rightarrow \underE{\eta_1 \visar
%        \eta_3} \\
\underE{\C{SnapshotVis}(T_i,T_j)} & \Leftrightarrow &  \underE{T_i
       \visar T_j} \disj \underE{T_i \invisar T_j}\\
\underE{\C{OneWaySER}(T_i,T_j)} & \Leftrightarrow &  \underE{T_i
       \visar T_j} \disj (\underE{T_i \invisar T_j} \conj \\
  &   &\hspace*{1.2in}\exists \eta.~\underE{\eta\in T_i} \conj \kind(\eta) =
       \C{COMMIT} \Rightarrow \underE{T_j \visar \eta})\\
\underE{\C{RC}(T_j)} & \Leftrightarrow & \underE{\C{AtomicVis}(T_j)} 
        \conj \underE{\C{CommitVis}(T_j)}\\
\underE{\C{MAV}(T_j)} & \Leftrightarrow & \underE{\C{RC}(T_j)} \conj
      \underE{\C{MonotonicVis}(T_j)} \\
\underE{\C{RR}(T_j)} & \Leftrightarrow & \underE{\C{MAV}(T_j)}
       \conj \forall T_i.\,T_i \neq T_j \Rightarrow 
        \underE{\C{SnapshotVis}(T_i,T_j)} \\
% &   & \hspace*{2in}\conj  \C{SnapshotVis}(T_i,T_j)\\
\underE{\C{SI}(T_j)} & \Leftrightarrow &  \underE{\C{RR}(T_j)}
       \conj \forall T_i.\,(T_i \neq T_j \conj \exists X.\, \underE{T_i \wrstoar X} \conj 
        \underE{T_j \wrstoar X})\\
%      (\exists X.\, T_i \wrstoar X \conj T_j \wrstoar X)
%      \Rightarrow  \C{TotalVis}(T_i,T_j)\\
  &   & \hspace*{2in} \Rightarrow \underE{\C{OneWaySER}(T_i,T_j)}\\
\underE{\C{SER}(T_j)} & \Leftrightarrow & \underE{\C{MAV}(T_j)}
       \conj \forall T_i.\,T_i \neq T_j 
       \Rightarrow \underE{\C{OneWaySER}(T_i,T_j)}\\
% &   & \hspace*{2in}\conj  \C{TotalVis}(T_i,T_j)\\
\end{array}
\end{smathpar}

\caption{Standard isolation guarantees expressed as trace
well-formedness constraints}
\label{fig:ansi-isolation}
\end{figure*}

Fig.~\ref{fig:ansi-isolation} shows the specification of standard
isolation guarantees expressed as constraints over trace
well-formedness. For brevity and convenience, we adopt few notations
in this and subsequent sections.  An execution trace is destructed
into $\A$ and $\visZ$ whenever individual components of the pair are
needed. Otherwise, it is written as $\E$. Sometimes, the dot notation
(\eg~$\E.\A$) is also used. Since $\A$ and $\visZ$ are both sets, we
lift the operations on sets to pairs of sets when updating $\E$. For
example, $\E' = \E \cup (\{\eta_2\},\{(\eta_1,\eta_2)\})$ expands to
$\E' = (\E.\A \cup \{\eta_2\},\,\E.\visZ \cup \{(\eta_1,\eta_2)\})$.
When $\psi$ is a formula, $\underE{\psi}$ denotes the
interpretation of $\psi$ in the context of the trace $\E$. Such
interpretations are defined on a case-by-case basis in
Fig.~\ref{fig:ansi-isolation}. In the following, we give informal
explanations for each definition.

In the context of a trace $(\A,\visZ)$, an effect $\eta$ is said to
belong to a transaction $T_i$ if $\eta$ belongs to the effect set $A$
and its transaction identifier is $T_i$. The containment relation is
trivially lifted to the set of effects to define $\underE{S \subseteq
  T_i}$.  The interpretation of $\eta_1 \visar \eta_2$ and $\eta_1
\soar \eta_2$ are as explained earlier.  A transaction $T_i$ is said
to be visible to an effect $\eta$ if every effect $\eta_1$ of $T_i$
recorded by the trace is visible to $\eta$.  $T_i$ may be visible to
$\eta$ but may not be visible to every other effect in the
transaction. For a transaction $T_i$ to be considered to be visible to
a transaction $T_j$ in the context of a trace $\E$ (written
$\underE{T_i \visar T_j}$), every effect ($\eta_1$) of $T_i$ present
in $\E$ must be visible to every effect ($\eta_2$) of $T_j$ in
$\E$. Conversely, if none of the effects of $T_i$ present in $\E$ are
visible to any effect of $T_j$, then $T_i$ is considered invisible to
$T_j$ under $\E$ (written $\underE{T_i \invisar T_j}$). Transaction
$T_i$ is said to have written to a variable $X$ under $\E$ (i.e.,
$\underE{T_i \wrstoar X}$) if there exists a write-to-$X$ effect from
$T_i$ in $\E$.

Various isolation guarantees are defined as propositions indexed by
a transaction identifier $T_j$. Transaction $T_j$ is said to have
experienced \emph{read-my-writes} visibility under $\E$ if every
effect ($\eta_1$) of $T_j$ is visible to every subsequent effect
($\eta_2$) of the same transaction in $\E$. This lets $T_j$ to never
lose its own updates. Monotonic visibility adds one more constraint to
read-my-writes; besides requiring $\eta_1$ to be visible to $\eta_2$,
it also requires every effect ($\eta$) visible to $\eta_1$ in $\E$ to
be visible to $\eta_2$ as well. Thus $T_j$ witnesses monotonically
increasing state as time progresses. Atomic visibility allows an
effect $\eta_2$ of $T_j$ to witness an effect $\eta_1$ of $T_i$ only
if all effects of $T_i$ in $\E$ are also visible to $\eta_2$. Atomic
visibility thus prevents a transaction from being partially visible.
However, atomic visiblity does not prevent an uncommitted transaction
from being visible. To confine visibility to only committed
transactions, we need an additional constraint that requires an effect
($\eta_2$) of $T_j$ to also witness the commit effect of $T_i$
whenever it witnesses some effect ($\eta_1$) of $T_i$. This additional
constraint is captured by the $\mathtt{CommitVis}$ specification. A
transaction $T_i$ is said to be snapshot-visible to a transaction
$T_j$ if either it is visible to $T_j$, or it is invisible; it is
forbidden for only a suffix (more generally, a subset) of $T_j$ to
witness $T_i$. Observe that snapshot visibility permits $T_i$ and $T_j$ to
execute and commit while being  oblivious of each other. One-way
serializability proscribes this possiblity. $T_i$
is said to be one-way-serializable w.r.t $T_j$ if either it is visible
to $T_j$ or $T_j$ is visible to the commit effect of $T_i$.  Note that
both $\mathtt{SnapshotVis}$ and $\mathtt{OneWaySER}$ are asymmetric
definitions that guarantee complete visiblity or invisibility of $T_i$
to $T_j$, but not the converse. While $\mathtt{SnapshotVis}$ provides
no guarantees to $T_i$, $\mathtt{OneWaySER}$ guarantees that at least
the commit effect of $T_i$ witnesses $T_j$.

The ANSI SQL 92 standard requires \iso{Read Committted} isolation to
avoid the dirty reads phenomenon, which is achieved by enforcing
$\mathtt{AtomicVis}$ and $\mathtt{CommitVis}$ guarantees. The {\sc rc}
specification is therefore a combination of these two guarantees. The
specification also agrees with the description and
implementation~\cite{bailishat,pldi15} of {\sc rc} for highly
available replicated stores. On relational databases, however, {\sc
rc} has also come to be associated with the $\mathtt{MonotonicVis}$
guarantee.  Nonetheless, $\mathtt{AtomicVis}$ and $\mathtt{CommitVis}$
are sufficient to reason about {\sc rc} isolation on relational stores
too. The combination of these guarantees with the {\sc sc} property of
relational stores (formalized in \S\ref{sec:store-consistency})
automatically leads to the monotonicity guarantee, which 
explains why {\sc rc} comes with $\mathtt{MonotonicVis}$ on such
stores. On weakly consistent stores however, $\mathtt{AtomicVis}$ and
$\mathtt{CommitVis}$ do not imply $\mathtt{MonotonicVis}$. A stronger
isolation level called \iso{Monotonic Atomic View} ({\sc mav} of
Fig.~\ref{fig:ansi-isolation})~\cite{bailishat,pldi15} has been proposed to
explicitly extend {\sc rc} with monotonicity on such stores. Sample
{\sc rc} executions shown in Fig.~\ref{fig:rc-executions} are also
{\sc mav}. \iso{Repeatable Read} ({\sc rr}) isolation  extends {\sc mav}
with the snapshot visibility guarantee as described before.
\iso{Snapshot Isolation} spec ({\sc si}) extends {\sc rr} with a
one-way serializability guarantee w.r.t the transactions that perform
conflicting writes (i.e., writes to the same shared variable).
Finally, \iso{Serializable} isolation extends one-way
$\mathtt{SnapshotVis}$ to all transactions, including those that do
not conflict\footnote{Fig.~\ref{fig:ansi-isolation} presents slightly
weaker versions of {\sc si} and {\sc ser} specs in the interest of
clarity.}. \GK{ToDo: Figures. Need to figure out a convention to
uniformly represent various isolation levels. Current idea: concurrent
timelines with arrows in both directions.}

%% SJ: Not sure this paragraph is necessary here.
%% In contrast to recent proposals (\emph{e.g.},
%% ~\cite{gotsmanconcur15}), our specifications for {\sc si} and {\sc
%%   ser} do not necessarily impose a total order among transactions
%% (conflicting or otherwise). In reality, a total order under {\sc ser}
%% (resp. {\sc si}) is guaranteed only if the store executes all
%% transactions under {\sc ser} (resp. {\sc si}) isolation. Our
%% specifications admit this possibility, and derive a total order under
%% the assumption of homogenity. However, databases almost always allow
%% isolation levels to be configured on a per-transaction basis, allowing
%% transactions at various isolation levels to coexist.  Specifications
%% that assume homogenity are incorrect under this setting.

% Having specified isolation levels as trace well-formedness
% constraints, we can now construct trace invariants ($\I$) for
% \txnimp programs by composing isolation specifications for various
% transactions. For instance, the following trace invariant enforces
% {\sc si} for both transactions of the program in
% Fig.~\ref{fig:motiv-eg-1}, allowing it to satisfy its postcondition:
% \begin{smathpar}
% \I \;=\; \lambda\E.~ \underE{\C{SI(Wd1)}} \conj \underE{\C{SI(Wd2)}}
% \end{smathpar}
% In contrast, the following trace invariant enforces {\sc rc} for one and {\sc si}
% for another, leading to a possible violation of the postcondition:
% \begin{smathpar}
% \I \;=\; \lambda\E.~ \underE{\C{RC(Wd1)}} \conj \underE{\C{SI(Wd2)}}
% \end{smathpar}
