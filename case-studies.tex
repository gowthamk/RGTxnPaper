\section{Case Studies}
\label{sec:case-studies}

In this section, we present two case studies that demonstrate the
applicability of our reasoning approach in varied settings, and its
consequent practical utility. Both our case studies target
Ruby-on-Rails applications. To analyze these, and other similar
open-source applications, we developed a symbolic execution engine
called {\sc MagLev} that compiles a Rails program to an abstract
program in an extended version of \txnimp\footnote{The details of {\sc
MagLev} can be found in the supplementary. Its source code, along with
that of our applications and experiments, is available at the
following anonymized url: \url{...}}. The focus of our formal
verification efforts are such \txnimp programs.

\paragraph{A video view counter.} Our first application is a Rails
implementation of a counter to count the number of views a video has
garnered on a video-sharing website like YouTube. The application
stores the video count in a MySQL database, and supports a read
operation (a database read transaction), and an increment operation (a
read-increment-write sequence wrapped in a transaction).  The
application is expected to preserve the \emph{monotonicity invariant},
namely that the view count should never appear to be decreasing.
However, the default \iso{Repeatable Read} isolation level of MySQL
leads to anomalous executions - with 32 concurrent writers performing
single increments, and one reader issuing a constant stream of reads,
we observed 12 violations of monotonicity in 10 rounds. We
subsequently instantiated our reasoning framework for an {\sc sc}
store with ANSI SQL isolation levels, and spent 5 man hours
determining appropriate isolation levels needed to enforce the
invariant. In particular, we were able to formally prove that to
preserve the invariant, writers need to execute using
\iso{Serializable} isolation, while readers can execute using
\iso{Read Uncommitted}. Repeating the experiments with this
configuration led to no violations, as expected. While executing all
transactions at {\sc ser} also prevents violations, it led to an
additional 113\% increase in the latency of writes averaged over 5
rounds with 32 concurrent writers.  We repeated the experiments with
the same configuration on Postgres, and found no violations. However,
Postgres' isolation levels~\cite{postgresiso} are stronger than ANSI
SQL's in non-trivial ways. We therefore formalized Postgres'
guarantees and repeated relevant parts of the proof to discover that
\iso{Repeatable Read} is in fact sufficient for writers - an
observation we confirmed through experiments. As usual, using {\sc
ser} for all transactions also suffices, but on Postgres results in an
additional 47\% increase in write latency.

% To understand whether a straightforward implementation of the view
% counter preserves the invariant, we considered a \txnimp program
% that is representative of the counter application. In particular, we
% considered four concurrent transactions that operate on a shared
% counter: a reader transaction that reads the counter twice obtaining
% the values $k_1$ and $k_2$, respectively, and three writer
% transactions, each incrementing the counter once. If the reader
% witnesses a monotonically increasing view count, then $k_2\ge k_1$.
% We then instantiated our reasoning framework for an {\sc sc} store
% with {\sc rc} as the default isolation level (Postgres is one such
% store), with the aim of verifying the program under this
% instantiation.  Our attempts were however unsuccessful owing to our
% inability to prove that a writer transaction always commits a value
% no smaller than the current value of the counter. This points to the
% existence of anamolous executions where  To understand the
% prevalence of such anomalies, we implemented a view counter
% application in Ruby-on-Rails using standard templates and
% techniques.   
To understand the effect of changing store consistency levels, we
re-implemented the counter as an op-based \emph{replicated data type}
(RDT)~\cite{crdt,burckhardt14} on top of Cassandra
\emph{bolted-on}~\cite{bailisbolton} with {\sc cc} and {\sc sc}
consistency levels (both on-demand; {\sc ec} is default), and ANSI SQL
isolation levels. Keeping the consistency level at {\sc ec}, we then
repeated our experiments with committed reads and serializable
increments - a combination which was proved to preserve monotonicity
on an {\sc sc} store, but nonetheless observed monotonicity
violations. The reason for such violations became evident when
we instantiated our proof framework for an {\sc ec} store.
Under this instantiation, it is impossible to derive the
\C{MonotonicVis} (\S\ref{sec:ansi-isolation}) property needed to show
that successive reads in a session witness monotonically increasing
state. However, assuming {\sc cc} store consistency for reads allowed
us to complete the proof, and recover the monotonicity invariant. In
fact, we were able to complete the proof even after relaxing the
serializability requirement for increments, just by changing the
definition of the $\interp{\E}(X)$ (\S\ref{sec:opsem}) to return the
value of the largest write-to-$X$ instead of the latest write-to-$X$.
In other words, if the reads chose the largest valued write effect,
instead of the latest write effect they witness, then counter
increments need not be serializable for reads to witness a
monotonically increasing count.\footnote{The trade-off is that the view
count is now no longer accurate, although it grows monotonically.}
We implemented this version of the counter as an op-based RDT, which
uses {\sc cc} reads and {\sc ec} increments, both un-isolated (i.e.,
\emph{Read Uncommitted}), and obtained a 52\% improvement in overall
latency compared to the version that uses serializable increments.
\vspace*{-3pt}

\paragraph{Microblog.} Our second set of experiments focused on a
Twitter-like microblogging application taken from~\cite{railsbook}.
An important invariant that must be preserved is the referential
integrity between microposts and authors. The application enforces
this invariant entirely in user space by relying on database-backed
transactions (this behavior is typical of Rails
applications~\cite{bailisferal}).  For instance, deletion of a user is
carried out in a transaction that also includes the deletion of
microposts authored by the user.  However, it is possible for
concurrent observers to nonetheless witness invariant violations. To
measure the likelihood of such anomalies, we first populated a
Postgres database with 1000 user accounts, each user with 50
microposts and 20 followers, chosen uniformly at random. We then
constructed experiments with 64 concurrent clients performing
deletions and timeline reads of randomly chosen users in a 1:7 ratio,
and witnessed 250 violations of referential integrity (the isolation
level was left at the default {\sc rc} level for experiments). To
determine appropriate isolation levels, we then instantiated our
framework for an {\sc sc} store with various ANSI SQL isolation levels
for user deletions and timeline reads, and obtained a proof that {\sc
rc} is sufficient for deletions, whereas reads require {\sc rr}. The
proof is immediately applicable for MySQL, which implements ANSI SQL
isolation levels. Repeating our experiments for MySQL with this
configuration resulted in no violations (as expected), while yielding
a 38\% reduction in latency than an {\sc ser} configuration.  Since
Postgres's isolation levels are stronger than the corresponding ANSI
levels, \rulelabel{RG-Conseq} rule (Fig.~\ref{fig:rg-rules}) lets us
immediately deduce that the same choice of isolation levels also work
on Postgres.

Clearly, there exists a significant diversity among the weak
consistency and isolation variants implemented on various real-world
data stores. More such variants have been developed in recent years to
meet the need for high availability (\emph{e.g.}, \emph{Parallel
  Snapshot Isolation}~\cite{psi} (2011) or \emph{Non-Monotonic
  Snapshot Isolation}~\cite{nmsi} (2013)).  As more commercial
weakly-consistent stores adopt various kinds of transactional
semantics, like Cassandra's recently introduced \emph{lightweight
  transactions}, it is reasonable to expect new isolation
definitions to continue to be proposed.  While it may be possible to
carefully engineer a reasoning framework for each combination of
consistency and isolation, such a strategy would lead to multiple
semantic definitions and proof systems with no obvious way to compare
and relate them. As illustrated here, having a parameterized semantics
and a proof framework built on top of it allows us to support these
variants as distinct, yet comparable, instantiations.

