\section{Case Studies}
\label{sec:case-studies}

In this section, we present two case studies that demonstrate the
applicability of our reasoning approach in varied settings, and the
consequent practical utility.

\paragraph{A video view counter} The view count of a video on a
YouTube-like video-sharing website must never appear to be decreasing,
although it may not necessarily reflect the actual number of views.
We call this the monotonicity invariant. To understand whether a
straightforward implementation of the view counter preserves the
invariant, we considered a \txnimp program that is representative of
the counter application. In particular, we considered four concurrent
transactions that operate on a shared counter: a reader transaction
that reads the counter twice obtaining the values $k_1$ and $k_2$,
respectively, and three writer transactions, each incrementing the
counter once. If the reader witnesses a monotonically increasing view
count, then $k_2\ge k_1$. We then instantiated our reasoning framework
for an {\sc sc} store with {\sc rc} as the default isolation level
(Postgres is one such store), with the aim of verifying the program
under this instantiation.  Our attempts were however unsuccessful
owing to our inability to prove that a writer transaction always
commits a value no smaller than the current value of the counter. This
points to the existence of anamolous executions where writers may
commit a value smaller than the current value of the counter. To
understand the prevalence of such anomalies, we implemented a view
counter application in Ruby-on-Rails using standard templates and
techniques\footnote{Source code of the applications, experiments, and
the tools we developed to analyze Ruby can be found at the following
anonymized url: \url{...}}. The application stores the video count in
a Postgres database, and supports an increment operation and a read
operation on the count. In our experiments with 16 (resp. 32)
concurrent writers, readers observed 7 (17) violations of
monotonicity, in 10 rounds of operation, thus confirming that
anomalies are indeed possible. 

To determine the weakest ANSI SQL isolation levels needed to restore
the monotonicity invariant, we instantiated our reasoning framework
for an {\sc sc} store with different isolation levels for readers and
writers.  With half-a-day of effort, we were able to prove that
\iso{Read Uncommmitted}, the weakest isolation level, is sufficient
for readers, whereas \iso{Serializable} isolation is needed for
writers. Our experiments with uncommitted reads and serializable
writes resulted in no observed violations of monotonicity, confirming
our judgment. While this exercise let us determine the appropriate
isolation levels for an {\sc sc} store with ANSI SQL isolation levels,
we were however surprised to learn from Postgres's
documentation~\cite{postgres} that its isolation levels are stronger
than ANSI SQL's in non-trivial ways. We subsequently formalized the
guarantees and repeated the proof effort to discover that Postgres's
\iso{Repeatable Read} is in fact sufficient for writers - an
observation we confirmed through experiments. However, the same choice
of isolation levels led to monotonicity violations on MySQL, because
MySQL's {\sc rr} is weaker than Postgres's.

To understand the effect of changing the store consistency, we
reimplemented the counter application for Cassandra, an {\sc ec} store
(We used the \emph{shim layer} of~\cite{pldi15} to equip Cassandra
with {\sc cc} and {\sc sc} consistency levels, and {\sc rc}, {\sc
mav}, and {\sc rr} isolation levels). We then repeated our experiments
with uncommitted reads and serializable writes - a combination which
was proved to preserve monotonicity on an {\sc sc} store, but
nonetheless observed monotonicity violations. The reason for such
violations became evident as we considered an instantiation our proof
framework for an {\sc ec} store. Under this instantiation, it is
impossible to derive the \C{MonotonicVis} (\S\ref{sec:ansi-isolation})
property needed to show that successive reads in a session witness
monotonically increasing state. We were however able to derive the
property assuming either causal consistency or {\sc mav} isolation
(both supported by the shim layer). We experimented with both these
options and were able to recover the invariant.


\paragraph{Microblog}

Next, we focus our attention on a Twitter-like microblogging site that
is one the sample applications shipped with a popular textbook on
Ruby-on-Rails~\cite{railsbook}. The application is built to
demonstrate various capabilities of Rails' ORM framework, and, as
such, is a good representative of Ruby-on-Rails applications.  For
instance, instead of relying on database capabilities, it uses Rails's
\emph{has\_many} association to describe the relationship between a
user entity and a micropost entity. Further, it qualifies the
association with \emph{dependent: destroy}, thus directing Rails to
destroy all microposts if the corresponding author is deleted. Thus,
the application entrusts Rails with the responsibility to enforce
referential integrity between microposts and users. Rails in turn
relies on database-backed transactions to enforce such constraints,
thus making application semantics susceptible to the changes in
isolation level.  For instance, deletion of a user is done within a
transaction ($T_1$) that also deletes the microposts authored by
(hence, \emph{dependent} on) the user. While this is effective if all
transactions are {\sc ser}, under the default {\sc rc} isolation, it
lets concurrent transactions witness referential integrity violations.
For instance, consider a transaction ($T_2$) that populates a timeline
by first reading microposts, and then reading the author information
associated with each micropost. It is possible for an author to be
deleted (due to the interference from $T_1$) between the two reads,
leading $T_2$ to witness a referential integrity violation, although
none really exists in the database.  Indeed, we have verified this
hypothesis by instantiating our reasoning framework for {\sc sc}
machine and {\sc rc} isolation, and extracting the aforementioned
anomalous execution from our failure to prove referential
integrity\footnote{We modeled referential integrity using two
variables, $X$ and $Y$: $Y$ is 1 only if $X$ is 1.  Otherwise, both
are 0.}. To measure the likelihood of such anomalies in practice, we
first populated a Postgres database with 1000 user accounts, each user
with 50 microposts and 20 followers, choosen uniformly at random. We
then constructed experiments with 64 concurrent clients performing
deletions and timeline reads of randomly choosen users in 1:7 ratio,
and witnessed a maximum of 250 violations of referential integrity.

Clearly, {\sc rc} is insufficient. In order to determine the weakest
isolation levels needed to enforce referential integrity, we
instantiated our framework for an {\sc sc} store with various Postgres
isolation levels for user deletions and timeline reads, and obtained a
proof that {\sc rc} is sufficient for deletions, whereas reads require
{\sc rr}. Repeating the experiment with these isolation settings
resulted in no violations, as expected.
