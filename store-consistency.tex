\subsection{Store Consistency}
\label{sec:store-consistency}

Under a trivial trace invariant (i.e., $\I(\E) = true$), the abstract
machine of Fig.~\ref{fig:txnimp} assumes the semantics of an EC store;
it allows operations of a transaction witness arbitrary subsets of the
global state. A non-trivial $\I$ composed of isolation specifications
from Fig.~\ref{fig:ansi-isolation} induces the machine to provide
non-trivial isolation guarantees for transactions. However, weak
isolation levels often only constrain the visibility sets of
operations by dictating what \emph{not} to see; not what to see.  For
instance, \iso{Repeatable Read} isolation prohibits operations of a
transaction from witnessing different states. It however does not
prohibit all operations of a transaction from witnessing an aribitrary
subset of the global state. Consequently, the machine can remain an EC
store even while providing non-trivial isolation. How then to model
the semantics of an SC store, such as a relational database, with
variable isolation?

The answer lies in enforcing store-specific consistency constraints,
along with transaction-specific isolation constraints, via the trace
invariant $\I$. In particular, we split $\I$ into two components: (1).
$\I_s$, the store-specific invariant, and (2). $\I_c$, the
program-specific (or, client-specific) invariant, to capture
consistency and isolation constraints, respectively. $\I$ is now a
conjunction:
\begin{smathpar}
  \I \;=\; \lambda\E.~\I_s(\E) \wedge \I_c(\E)
\end{smathpar}
The trace invariant we wrote in \S\ref{sec:ansi-isolation} for the
withdraw program of Fig.~\ref{fig:motiv-eg-1} now becomes its $\I_c$
and remains an invariant regardless of the store. $\I_s$ however
changes from store to store. We now consider various stores and
describe their corresponding $\I_s$.

\paragraph{An EC store} An EC store provides no additional consistency
guarantees besides those provided by the machine. Hence, its
$\I_s(\E)$ is $true$ for any $\E$.

\paragraph{An SC store} An SC store guarantees total order of all
operations w.r.t $\visZ$ that is consistent with their chronological
order. A straightforward $\I_s$ for this store formalizes this
guarantee:
\begin{smathpar}
\begin{array}{l}
  \I_s \;=\; \lambda\E.~\forall\eta_1,\eta_2.~\{\eta_1,\eta_2\}
  \subseteq \E.\A \conj \id(\eta_1) < \id(\eta_2) \\
  \hspace*{2in}\Rightarrow \underE{\eta_1 \visar \eta_2}
\end{array}
\end{smathpar}
Unfortunately, the above $\I_s$ is inconsistent with all isolation
specifications of Fig.~\ref{fig:ansi-isolation}. For instance, let
$\eta_1$ be an effect of an uncommitted transaction and let $\eta_2$
be an effect of a \iso{Read Committed} transaction such that
$\id(\eta_1) < \id(\eta_2)$. While \C{RC} prohibits $\eta_1$ from
being visible $\eta_2$, $\I_s$ requires $\eta_1$ to be visible to
$\eta_2$. The machine, which cannot satisfy both constraints
simultaneously, gets stuck.

The way out of this impasse is to recognize that SC property of the
store needs to be weakened to admit weak isolation.

