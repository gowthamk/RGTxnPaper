\section{Inference}
\label{sec:inference}

The rely-guarantee framework presented in the previous section
facilitates modular proofs for weakly-isolated transactions, but
imposes a non-trivial annotation burden.  In particular, it requires
each statement ($c$) of the transaction to be annotated with a stable
pre- ($P$) and post-condition ($Q$), and loops to be annotated with
stable inductive invariants ($\psi$). While weakest pre-condition
style predicate transformers can help in inferring intermediate
assertions for regular statements, loop invariant inference remains
challenging, even for the simple form of loops considered here.  As an
alternative, we present an inference algorithm based on state
transformers that alleviates this burden.  The idea is to infer the
logical effect that each statement has on the transaction-local
database state $\stl$ (i.e., how it transforms $\stl$), and compose
multiple such effects together to describe the effect of the
transaction as a whole.  Importantly, this approach generalizes to
loops, where the effect of a loop can be inferred as a well-defined
function of the effect of its body, thanks to certain pleasant
properties enjoyed by the database programs modeled by $\txnimp$.
Interpreting database semantics as functional transformations on sets
(described in terms of their logical effects) enables an inference
mechanism that can leverage off-the-shelf SMT solvers for automated
verification.

\begin{figure}[h]
\begin{smathpar}
\renewcommand{\arraystretch}{1.2}
\begin{array}{lcl}
\multicolumn{3}{c}{
  x,y,\stl,\stg \in \texttt{Variables}\spc
  \varphi \in \Prop^{0}\spc
  \phi \in \Prop^{1}
}\\
s & \coloneqq & x \ALT \stl \ALT \stg \ALT \{x \,|\, \varphi\} 
  \ALT \existsl(\stg,\phi,s) 
  \ALT s_1 \bind \lambda x. s_2 \ALT \itel{\varphi}{s_1}{s_2} 
  \ALT s_1 \cup s_2 \\
\end{array}
\end{smathpar}
%
\caption{Syntax of the set language $\SL$}
\label{fig:logic-syntax}
\end{figure}



At the core of our approach is a simple language ($\SL$) to express
set transformations (see Fig.~\ref{fig:logic-syntax}). The language
admits only set expressions that include variables ($x$), literals of
the form $\{x \,|\, \varphi\}$ where $\varphi$ is a propositional
(quantifier-free) formula on $x$, a restricted form of existential
quantification that binds a set $x$ satisfying proposition $\phi$ in a
set expression $s$, a monadic composition of two set expressions
($s_1$ and $s_2$) composed using a bind ($\bind$) operation, a
conditional set expression where the condition is a propositional
formula, and a union of two set expressions. Symbols $\stl$ and $\stg$
are also variables in $\SL$, but are used to denote local and database
states, respectively.  The language is carefully chosen to be
expressive enough to capture the semantics of $\txnimp$ statements (as
well as SQL operations more generally), yet simple enough to have a
semantics-preserving translation to a decidable fragment of
first-order logic.
\begin{figure}
\raggedright
%
\fbox {\(c \elabsto \F \)}\\
%

% \renewcommand{\arraystretch}{1.2}
% \begin{smathpar}
% \begin{array}{lcl}
% % \inserte{x} & \elabsto & \stabilize{\lambda(\stl,\stg).~
% %   \{ r \,|\, r = \{x \with \idf = \uid(x);\,r.\delf=\C{false};\,
% %                 \txnf = i\} \conj \uid(x) \notin \dom(\stl\cup\stg) \}}\\ 
% %
% %
% \updatee{\lambda x.e_1}{\lambda x.e_2} & \elabsto & \stabilize{
%   \lambda(\stl,\stg).~ \stg \bind (\lambda r.~ 
%   \{r' \,|\, [r/x]e_2 \conj r' = \{[r/x]e_1 \with 
%       \idf=r.\idf;\,\delf = r.\delf;\,\txnf=i \}) \} }\\ 
% %
% \deletee{\lambda x.e} & \elabsto & \stabilize{
%   \lambda(\stl,\stg).~ \stg \bind (\lambda r.~ 
%   \{r' \,|\, [r/x]e \conj r'=\{r \with \delf=\C{true}\})\} }\\ 
% %
% \end{array}
% \end{smathpar}

\begin{center} 
%
%\begin{minipage}{3.5in}
\begin{smathpar}
\begin{array}{c}
\RULE
{
}
{
  \inserte{x} ~\elabsto~ \stabilize{\lambda(\stl,\stg).~
    \{ y \,|\, y = \{\langle x \with r.\delf=\mathtt{false};\, \txnf = i\rangle \}}
}
\end{array}
\end{smathpar}
%\end{minipage}
%\begin{minipage}{3.5in}
\begin{smathpar}
\begin{array}{c}
\RULE
{
  G = \lambda y.~ \itel{[y/x]e_2}
      {\{\langle[y/x]e_1 \with \idf=r.\idf;\,\delf = y.\delf;\,\txnf=i\rangle \}}
      {\emptyset}
}
{
  \updatee{\lambda x.e_1}{\lambda x.e_2}  ~\elabsto~
  \stabilize{\lambda(\stl,\stg).~ \stg \bind G } 
}
\end{array}
\end{smathpar}
%\end{minipage}
%\begin{minipage}{3in}
\begin{smathpar}
\begin{array}{c}
\RULE
{
  G = \lambda y.~ \itel{[y/x]e}
      {\{\langle y \with \delf = \mathtt{true};\,\txnf=i\rangle \}}
      {\emptyset}
}
{
  \deletee{\lambda x.e}  ~\elabsto~
  \stabilize{
  \lambda(\stl,\stg).~ \stg \bind G \} }
}
\end{array}
\end{smathpar}
%\end{minipage}

%\begin{minipage}{2.7in}
\begin{smathpar}
\begin{array}{c}
\RULE
{
  c \elabsto \F \spc
}
{
  \lete{x}{e}{c} ~\elabsto~  
    \stabilize{\lambda(\stl,\stg).~ [e/x]\,\F(\stl,\stg)}\\
}
\end{array}
\end{smathpar}
%\end{minipage}

%\begin{minipage}{3in}
\begin{smathpar}
\begin{array}{c}
\RULE
{
  c \elabsto \F \spc
  G = \lambda r.~ \itel{[r/x]e}{\{x\ |\ x = r \}}{\emptyset}
}
{
  \lete{y}{\selecte{\lambda x.e}}{c} ~\elabsto~  
    \stabilize{\lambda (\stl,\stg).~ [(\stg \bind G)/y]\,\F(\stl,\stg)}\\
}
\end{array}
\end{smathpar}
%\end{minipage}

%\begin{minipage}{3.2in}
\begin{smathpar}
\begin{array}{c}
\RULE
{
  c_1 \elabsto \F_1 \spc
  c_2 \elabsto \F_2 
}
{
  \ite{e}{c_1}{c_2} ~\elabsto~
    \lambda(\stl,\stg).~\itel{e}{\F_1(\stl,\stg)}{\F_2(\stl,\stg)}\\
}
\end{array}
\end{smathpar}
%\end{minipage}

%\begin{minipage}{3in}
\begin{smathpar}
\begin{array}{c}
\RULE
{
  c_1 \elabsto \F_1 \spc
  c_2 \elabsto \F_2 
}
{
  c_1;c_2 ~\elabsto~  \lambda(\stl,\stg).~\F_1(\stl,\stg) \cup \F_2(\stl \cup \F_1(\stl,\stg),\stg)
}
\end{array}
\end{smathpar}
%\end{minipage}
%

%
%\begin{minipage}{3in}
\begin{smathpar}
\begin{array}{c}
\RULE
{
  c \elabsto \F \spc
}
{
  \foreache{x}{\lambda y.\lambda z.~c} ~\elabsto~
  \lambda(\stl,\stg).~ x\bind(\lambda z.~\F(\stl,\stg))
}
\end{array}
\end{smathpar}
%\end{minipage}

\end{center}
%

\caption{\txnimp: State transformer semantics. }
\label{fig:inference-rules}
\end{figure}

Fig.~\ref{fig:inference-rules} shows the syntax-directed state
transformer inference rules for $\txnimp$ commands inside a
transaction $\C{TXN}_i$.  Inference depends on the $\I$-constrained
rely relation $\R$, and the high-level invariant $I$ for the reasons
that will become clear shortly.  Inference rules compute, for each
command $c$, a (meta) function $\F$ that returns a set as an
expression in $\SL$, given a pair of sets ($\stl$ and $\stg$) that
describe local and a global database states, respectively. The
expression returned by $\F(\stl,\stg)$ (abstractly) describes the set
of records that get added to $\stl$ as a result of executing $c$ under
$\stl$ and $\stg$.  Thus, $\F$ captures the \emph{effect} part of the
state transformer of $c$, which is the function
$\lambda(\stl,\stg).~\stl \cup \F(\stl,\stg)$\footnote{Recall that the
  operational semantics treats deletion of records as the addition of
  the deleted record with its \C{del} field set to true in the local
  store.}. For $\F$ to be useful in RG verification, it needs to be
stable w.r.t the rely relation $\R$. The stability condition on
effects can be defined thus:
\begin{smathpar}
\begin{array}{lcl}
  \stable(\R,\F) & \Leftrightarrow & \forall \stl,\stg,\stg',\bar{v}.~
  \R(\stl,\stg,\stg') \Rightarrow \F(\stl,\stg) \equiv \F(\stl,\stg')
\end{array}
\end{smathpar}
where $\bar{v}$ are the variables that occur free in $\F$; this is
possible because of how the inference rules are structured.
Intuitively, the stability condition requires an effect to describe
the same set of records before and after the interference. The
equivalence in $\SL$ translates to equality in first-order logic,
as we describe below. In the inference rules, stability is
enforced constructively by a meta function $\stabilize{\cdot}$, which
accepts an effect and returns a new effect that is guaranteed to be
stable under $\R$.  $\stabilize{\cdot}$ achieves the stability
guarantee by abstracting away the bound $\stg$ in an unstable $\F$ to
an existentially bound $\stg'$ as described below:
\begin{smathpar}
\begin{array}{lcll}
  \stabilize{\F} & = & \F & \texttt{if } \stable(\R,\F).\\
  & = & \lambda (\stl,\stg).~\existsl(\stg',I(\stg'),\F(\stl,\stg')) 
      & \texttt{otherwise. }\stg'\texttt{ is a fresh name.}\\
\end{array}
\end{smathpar}
Observe that when $\F$ is not stable, then $\stabilize{\F}$ returns a
transformer $\F'$ that simply ignores its $\stg$ argument in favor of a generic
$\stg'$, making $\F'$ trivially stable. It is safe to assume
$I(\stg')$ because all verified transactions preserve the invariant,
and hence only valid database states will ever be witnessed. From the
perspective of RG reasoning, $\stabilize{\cdot}$ effectively weakens
the post-condition of a statement, as done by the
\rulelabel{RG-Conseq} rule for transaction-bound commands.  The weakening semantics chosen by
$\stabilize{\cdot}$, while being simple, is nonetheless useful because
of the $I(\stg')$ assumption on the existentially bound $\stg'$. The
example in Fig.~\ref{fig:weakening-example} demonstrates. 
\begin{figure}[h]
\begin{ocaml}
let add_interest acc_id pc = atomically @@ do
  let a = SQL.select1 BankAccount (fun acc -> acc.id = acc_id);
  let y = a.bal + pc*a.bal;
  SQL.update BankAccount (fun acc -> {acc with bal = acc.bal + y})
                         (fun acc -> acc.id = acc_id)
\end{ocaml}
\caption{A transaction that deposits an interest to a bank account.}
\label{fig:weakening-example}
\end{figure}
Here, an \C{add\_interest} transaction adds a positive interest
(\C{pc}) to the balance of a bank account, which is required to be
non-negatives ($I(\stg) \Leftrightarrow \forall(r\in\stg).~r.\C{bal}\ge
0$). The transaction starts by issuing a \C{select1} query, whose
effect implicitly asserts that there exists a record in the bank
account database that is equal to $a$
($\exists(r\in\stg).~r=a$). However, this assertion is unstable
because a concurrent \C{withdraw} or \C{deposit} transactions might
update the account balance, so that such a record no longer exists in
the database.  Fortunately, the weakening operator
($\stabilize{\cdot}$) allows us to weaken the assertion to $\exists
\stg'.I(\stg') \conj \exists(r\in\stg').~a=r$ (Fig.~\ref{fig:logic}
formalizes the encoding of {\sf exists} to logic), which is enough to
prove that $\C{a.bal + pc*a.bal}\ge 0$, and verify that
\C{add\_interest} preserves the positive balance invariant.

The state transformer rules, like the earlier RG rules, closely follow
the corresponding reduction rules in Fig.~\ref{fig:txnimp}, except
that their language of expression is $\SL$. For instance, while the
reduction rule for \C{UPDATE} declaratively specifies the set of
updated records, the state transformer rule uses $\SL$'s bind
operation to \emph{compute} the set. Other SQL rules do likewise. The
rules for \C{LET} binders, conditionals, and sequences compose the
effects inferred for their subcommands. Thus, the effect of a sequence
of commands $c_1;c_2$ is the union of effects $F_1$ and $F_2$ of $c_1$
and $c_2$, respectively, except that $F_2$ is applied to a state
($\stl$) to which $F_1$ has already been applied, reflecting their
order of reduction. The inference rule for \C{FOREACH} takes advantage
of the $\SL$'s bind operator to lift the effect inferred for the loop
body to the level of the loop. Since records added to $\stl$ in each
iteration of \C{FOREACH} are independent of the previous iteration,
sequential composition of the effects of different iterations is same
as their parallel composition. Since the loop body is executed once
per each $z\in x$, the effect of the the loop is the union of effects
($\F$) for all $z\in x$, all applied to the same state ($\stl$ and
$\stg$). That is, $\F_{loop}(\stl,\stg) = \bigcup_{z\in
  x}\F_{body}(\stl,\stg)$. From the definition of the set monad's bind
operator, $\F_{loop}(\stl,\stg) = x \bind (\lambda
z.~F_{body}(\stl,\stg))$, which mirrors the definition of the rule.
This development mirrors ideas found in type checkers for certain
dependent type systems~\cite{KJ14}.


% \begin{figure}[h]
% \begin{smathpar}
% \begin{array}{l}
%   \foreache{xs}{(\lambda y.\lambda x.~ 
%         \updatee{(\lambda z.~ \{\idf=z.\idf;\,\C{name}=z.\C{name};\,
%                                \C{s\_id}=x.\idf\})}
%                 {(\lambda z.~ z.\idf = x.\C{s\_id})};\\
%         \hspace*{1.27in} \inserte{x})}
% \end{array}
% \end{smathpar}
% \caption{A $\txnimp$ program that populates a database
%   of couples.}
% \label{fig:people_code}
% \end{figure}

%% \begin{example}
%% Consider a database of couples that tracks the spouse
%% relationship. Each record has an $\idf$ field, a \C{name} field, and a
%% spouse id (\C{s\_id}) field. A $\txnimp$ transaction bearing id $i$
%% that populates the database with a given set ($xs$) of couples will
%% include the following code snippet:
%% \begin{smathpar}
%% \begin{array}{l}
%%   \foreache{xs}{(\lambda y.\lambda x.~ 
%%         \updatee{(\lambda z.~ \{\idf=z.\idf;\,\C{name}=z.\C{name};\,
%%                                \C{s\_id}=x.\idf\})}
%%                 {(\lambda z.~ z.\idf = x.\C{s\_id})};\\
%%         \hspace*{1.27in} \inserte{x})}
%% \end{array}
%% \end{smathpar}
%% The effect captured by a state transformer inferred for the program, assuming no interference would be:
%% \begin{smathpar}
%% \begin{array}{l}
%%   \lambda(\stl,\stg).~xs \bind 
%%     (\lambda x. \stg \bind 
%%       (\lambda z. \itel{z.\idf = x.\C{s\_id}}
%%                        {\{\{\idf=z.\idf;\,\C{name}=z.\C{name};\,
%%                                \C{s\_id}=x.\idf\}\}}
%%                        {\emptyset}) \\
%%       \hspace*{1cm} ~\cup~ \{y \,|\, y = \{x \with \delf=\mathit{false}; \}\})
%% \end{array}
%% \end{smathpar}
%% Under the possibility of an interference affecting the stability, the
%% following stable effect is inferred ($I$ is the database invariant):
%% \begin{smathpar}
%% \begin{array}{l}
%%   \lambda(\stl,\stg).~xs \bind 
%%     (\lambda x.~\existsl(\stg',I,\stg' \bind 
%%       (\lambda z. \itel{z.\idf = x.\C{s\_id}}
%%                        {\{\{\idf=z.\idf;\,\C{name}=z.\C{name};\,
%%                                \C{s\_id}=x.\idf\}\}}
%%                        {\emptyset})) \\
%%       \hspace*{1cm} ~\cup~ \{y \,|\, y = \{x \with \delf=false; \}\})
%% \end{array}
%% \end{smathpar}
%% \end{example}

\begin{figure}
\raggedright
%
\begin{smathpar}
\begin{array}{lclcr}
  \mssemof{x \ALT \stl \ALT \stg \ALT \ldots} & = & \lambda
  r.\,r \in x \ALT \lambda r.\,r\in\stl \ALT 
  \lambda r.\,r\in\stg \ALT \ldots & \texttt{  }
  & \\
%\fresh(g_x) \spc \fresh(g_\stl) \spc \fresh(g_\stg)\\
%
  \mssemof{\{x \,|\, \varphi\}} & = & \lambda r.\,[r/x]\varphi
  & \texttt{  } & \\
%
\mssemof{\existsl(a,\phi,s)} & = &  (\lambda r.\, [a'/a]\phi \conj
  \mssemof{s}(r)
% \conj \forall b.~ g(b) \Leftrightarrow g_1(b)
  & \texttt{  } & \fresh(a') 
% \spc \fresh(g)\\
  \\
%
\mssemof{s_1 \bind \lambda x. s_2} & = & \lambda r.~
  (\forall a.\forall b.~ \mssemof{s_1}(a) \conj \mssemof{[a/x]s_2}(b) 
   \Leftrightarrow g(b)) \conj g(r) & \texttt{  } &  \fresh(g)\\
%
\mssemof{\itel{\varphi}{s_1}{s_2}} & = & \lambda r.\,
  (\varphi \Rightarrow \mssemof{s_1}(r)) \conj 
  (\neg\varphi \Rightarrow \mssemof{s_2}(r)) & \texttt{  } & \\
%
  \mssemof{s_1 \cup s_2} & = & \lambda r.\, \mssemof{s_1}(r) \disj
    \mssemof{s_2}(r) & \texttt{ } & \\
\end{array}
\end{smathpar}

\caption{Encoding $\SL$ in first-order logic}
\label{fig:logic}
\end{figure}


\subsection{Soundness of Inference}

The correspondence between the inference rules and the RG judgment of
\S~\ref{sec:reasoning} is stated thus\footnote{The proof can be found in
the supplementary material.}:
\begin{theorem}
\label{thm:inference-sound}
  \emph{Forall} i,$R$,$I$,$c$,$\F$, if $\stable(\R,I)$ and $c \elabsto \F$,
  then:\\\vspace*{-0.2cm}
  \begin{smathpar}
  \begin{array}{c}
  \R \vdash \hoare{\lambda(\stl,\stg).~\stl=\emptyset \conj
  I(\stg)}{c}{\lambda(\stl,\stg).\stl = \F(\emptyset,\stg)}
  \end{array}
  \end{smathpar}
\end{theorem}
\noindent where the set expression $\F(\emptyset,\stg)$ has a natural
interpretation as discussed below.
% Where, ${\sf true} = \lambda(\stl,\stg).~true$ and $\stable(\R,I)
% \Leftrightarrow \forall\stl,\stg,\stg'. I(\stg) \conj
% \R(\stl,\stg,\stg') \Rightarrow \I(\stg')$.

\subsection{From $\SL$ to the first-order logic}

Theorem~\ref{thm:inference-sound} lets us replace the local judgment
of the \rulelabel{RG-Txn} rule (Fig.~\ref{fig:rg-rules}) by a state
transformer inference judgment. The soundness of a transaction's guarantee can now
be established w.r.t the effect $\F$ of the body. The
\rulelabel{RG-Txn} rule so updated is shown
below:
\begin{smathpar}
\begin{array}{c}
\RULE
{
  \stable(R,\I)\spc
  \stable(R,I)\spc
   \R_e(\stl,\stg,\stg') \Leftrightarrow \exists \stg_1.  R(\stg, \stg') \wedge \I_e(\stl, \stg_1, \stg') \spc  c \elabsto \F\\
  \R_c(\stl,\stg,\stg') \Leftrightarrow \exists \stg_1.  R(\stg, \stg') \wedge \I_c(\stl, \stg_1, \stg') \spc \stable(\R_c,\F)\\
  \forall \stl,\stg.~ \stl \in F(\emptyset,\stg) \Rightarrow 
    G(\stg, \stl\gg\stg)\spc
  \forall \stg,\stg'.~I(\stg) \wedge G(\stg,\stg') \Rightarrow I(\stg')\\
}
{
  \rg{I,R}{\ctxn{i}{\I}{c}}{G,I}
}
\end{array}
\end{smathpar}
Automating the application of the \rulelabel{RG-Txn} rule for a
transaction requires automating the multiple implication checks in
the premise. While $R$, $\R$, $\R_c$, $\I$ and $I$ are formulas in
first-order logic (FOL) with a relatively simple structure, $\F$
is an expression in the set language $\SL$
(Fig.~\ref{fig:logic-syntax}) with a possibly complex structure.
Fortunately, however, there exists a semantics-preserving translation
from $\SL$ to a restricted subset of first-order logic (FOL) that
lends itself to automatic reasoning. 

The algorithm ($\mssemof{\cdot}$) shown in Fig.~\ref{fig:logic}
encodes an $\SL$ expression in a decidable fragment of FOL. Given a
set expression $s$ of $\SL$, $\mssemof{s}$ is a pair $(\phi,f)$, where
$\phi$ encodes $s$ in FOL, and $f$ is a Boolean function that names
the set defined by $s$.  Naming provides an easy handle to $s$,
simplifying $\mssemof{\cdot}$.  Set variables ($x$, $\stl$, $\stg$,
$\dots$) are encoded as unconstrained Boolean functions ($g_x$,
$g_{\stl}$, $g_{\stg}$, $\ldots$) that are named after the variable.
Encoding of a set literal $\{a\,|\,\varphi\}$ introduces a new Boolean
function $g$ that is \emph{true} for only those $a$ that satisfy $\varphi$.
Thus $g$ faithfully encodes the literal. The encoding of the
$\existsl{}$ expression first eliminates the second-order
quantification over sets.  Existentially bound set variable ($a$) is
skolemized, i.e., instantiated with a new unbound variable $a'$,
before encoding the bound expression $s$ and the first-order
constraint $\phi$. The algorithm for encoding $\phi$, which is already
a first-order formula, is straightforward. It basically involves
replacing set variables with Boolean functions named after the
variables. The first-order encoding for a bind expression describes
the semantics of set monad's bind operator in FOL. Let $s_1$ be a set, and
let $f$ be a function that maps each variable in $s_1$ to a new
set. Then, $s_2 = s_1 \bind f$ if and only if for all $y\in s_2$,
there exists an $x \in s_1$ such that $y = f(x)$, and forall $x\in
s_1$, $f(x)\in s_2$. The encoding adds new constraints to this
effect. Conditional set expressions and set union are encoded
straightforwardly.

Observe that the encoding shown in Fig.~\ref{fig:logic} maps to
a subset of logic that satisfies the following syntactic properties:
\begin{itemize}
  \item All quantification is first-order; second-order objects, such
    as sets and functions, are never quantified.
  \item Quantifiers appear only at the prenex position, i.e., at the
    beginning of a quantified formula.  
  \item All function symbols, modulo those that might appear in
    $\varphi\in\Prop^0$, are uninterpreted and Boolean.
\end{itemize}
This fragment of FOL known as EPR (Effectively Propositional logic)
is known to be decidable~\cite{z3epr}. The language of encoding, however, is a
combination of {\sf EPR} with (a). $\Prop^0$, the theory from which
quantifier-free propositions ($\varphi$) that encode object language
expressions are drawn, and (b).  $\Prop^1$, the theory from which
invariants ($I$) are drawn. We write $\SL[\Prop^0,\Prop^1]$ to
highlight the parameterization of $\SL$ on $\Prop^0$ and $\Prop^1$,
and state the following theorem:
\begin{theorem}
  $\SL[\Prop^0,\Prop^1]$ is decidable if ${\sf EPR}+\Prop^0+\Prop^1$
  is decidable.
\end{theorem}
A useful instantiation of $\SL$ is $\SL[{\sf SLA},{\sf EPR}+{\sf
SLA}]$, where ${\sf SLA}$ is the theory of simple linear arithmetic.
Since {\sf EPR}+{\sf SLA} is known to be decidable~\cite{eprsla}:
\begin{theorem}
  $\SL[{\sf SLA},{\sf EPR}+{\sf SLA}]$ is decidable.
\end{theorem}
The $\SL[{\sf SLA},{\sf EPR}+{\sf SLA}]$ instantiation requires $I$ to
be drawn from {\sf EPR}+{\sf SLA}, which is expressive enough to
describe common database integrity constraints, such as referential
integrity, non-negativeness of all integer values in a column etc.
The isolation specifications presented in \S\ref{sec:isolation} are
already simple first-order formulas that can be encoded in {\sf EPR}.
Furthermore, it is also reasonable to expect the guarantee ($G$) of a
transaction to be expressible in the same logic as its inferred $\F$,
since $\F$ (without the stability check) is essentially a complete
characterization of the transaction, while $G$ is only an abstraction.
Thus, with $\SL[{\sf SLA},{\sf EPR}+{\sf SLA}]$ as the language of
inference, the verification problem for weakly isolated transactions
is decidable. Moreover, off-the-shelf SMT solvers (e.g., Z3) are
equipped with efficient decision procedures for ${\sf EPR}+{\sf SLA}$,
making automated verification a practical exercise.

