\section{The Reasoning Framework}
\label{sec:reasoning}

We now describe a proof system that lets us demonstrate that a \txnimp
program satisfies its high-level invariants in all small-step
executions that satisfy a chosen trace invariant ($\I$).\footnote{It is
important to keep in mind the distinction between high-level invariants
(\eg~\C{X$\ge$0}) and trace invariants (\eg~\C{RC}). The former
constitute \emph{proof} \emph{obligations} for the programmer, whereas
the latter captures \emph{assumptions} about the operational characteristics
of transactions and the underlying data store. } 
Since the trace invariant captures transaction-isolation constraints,
the demonstration is a proof that the given selection of isolation
levels for transactions is sufficient to enforce a program's invariants.

As mentioned earlier, our proof system is a rely-guarantee framework
that admits compositional reasoning by abstracting away environmental
interference as a rely relation. A conspicuous difference between a
standard development of rely-guarantee and ours is that, while the
former reasons in terms of program states (variable to value
bindings), we reason in terms of executions as captured by their
traces ($\E$). In particular, our rely ($R$) and guarantee ($G$)
relations relate executions (i.e., $R,\,Q\,\subseteq\,\E\times\E$),
and our pre- ($P$) and post- ($Q$) conditions are assertions over
executions (i.e., $P,\,Q\, : \E \rightarrow \Prop$). Our development
also facilitates state-based reasoning via the interpretation function
($\interp{\cdot}$) introduced in \S\ref{sec:syntax}. For example, if a
\emph{bi-state} rely relation relates every pair of states $\sigma$
and $\sigma'$ such that $\sigma'(X) \ge \sigma(X) \ge 0$, the
corresponding \emph{bi-execution} rely relation relates every pair of
executions $\E$ and $\E'$ such that $\interp{\E}(X) \ge
\interp{\E'}(X) \ge 0$.  Assertions on states are also written
similarly. For instance, consider the post-condition
$\C{B\,=\,k-a1-a2}$ of the program in Fig.~\ref{fig:motiv-eg-1}. The
corresponding assertion on the post-state
($\lambda\sigma.~\sigma(\C{B}) \,=\, \C{k-a1-a2}$) asserts that in all
states resulting from executing the program, the value of \C{B} is
\C{k-a1-a2}. The equivalent execution-based assertion
($\lambda\E.~\interp{\E}(\C{B})\,=\,\C{k-a1-a2}$) asserts that in all
executions of the program, the value of \C{B} is \C{k-a1-a2}.
However, having access to an execution facilitates assertions that go
beyond the state. An example is an invariant of \C{Wd1} described
abstractly in \S\ref{sec:motivation}, and reified as an
execution-based assertion below:

\begin{smathpar}
\begin{array}{l}
  \lambda\E.~ \neg(\underE{\committed(\C{Wd1}}) \conj 
      \underE{\C{Wd1} \wrstoar \C{B}}\\
  \hspace*{0.2in}\conj(\neg(\underE{\committed(\C{Wd2})}) \Rightarrow 
      \C{\interp{\E}(B) = k-a1}) \\
   \hspace*{0.2in}\conj (\underE{\committed(\C{Wd2})} \Rightarrow 
      \C{\interp{\E}(B) = k-a1-a2})
\end{array}
\end{smathpar}
% Where, $\committed$ is defined as:
% \begin{smathpar}
% \begin{array}{lcl}
% \underE{\committed(T_i)} & \Leftrightarrow &
%   \exists\eta.~\underE{\eta\in T_i} \conj \kind(\eta) = \C{COMMIT}\\
% \end{array}
% \end{smathpar}

\noindent One of the conjuncts asserts that \C{Wd1} has written to \C{B} - a
fact which cannot be deduced solely from the value of \C{B} (esp. in the
presence of interference), but can be expressed as a proposition over
the execution trace.

\subsection{The Rely-Guarantee Judgment}
\label{sec:rely-guarantee}

\begin{figure}[t]
\raggedright
%
\fbox {\( \R \vdash \hoare{P}{c}{Q} \)} 
\quad \fbox {\( \rg{I,R}{c}{G,I} \)}\\[4pt]
%\fbox{\( \rg{\mathbb{I},P,R}{\txnbox{c}_i}{G,Q}\)} \quad
%\fbox{\( \rg{\mathbb{I},P,R}{c}{G,Q}\)} \quad\\
%
\begin{minipage}{3.1in}
\rulelabel{RG-Update}
\begin{smathpar}
\begin{array}{c}
\RULE
{
  \stable(\R,Q)\\
  \forall\stl,\stl',\stg.~P(\stl,\stg) \conj \\
  \stl' = \stl \cup \{r' \,|\, \exists(r\in\Delta).[r/x]e_2 \conj\\
  \hspace*{.7in} r'=[r/x]e_1\} \Rightarrow   Q(\stl',\stg)
}
{
  \R \vdash \hoare{P}{\updatee{\lambda x.e_1}{\lambda x.e_2}}{Q}
}
\end{array}
\end{smathpar}
\end{minipage}
%
%
\begin{minipage}{3in}
\rulelabel{RG-Select}
\begin{smathpar}
\begin{array}{c}
\RULE
{
  \\
  \R \vdash \hoare{P'}{c}{Q}\spc
  \stable(\R,P')\\
  P'(\stl,\stg) \Leftrightarrow P(\stl,\stg) \wedge
  x = \{r' \,|\, \exists(r\in\Delta).~ [r/y]e_2\} \\
}
{
  \R \vdash \hoare{P}{\lete{x}{\selecte{\lambda y.e}}{c}}{Q}
}
\end{array}
\end{smathpar}
\end{minipage}
%
\bigskip

%
\begin{minipage}{3.2in}
\rulelabel{RG-Delete}
\begin{smathpar}
\begin{array}{c}
\RULE
{
  \stable(\R,Q)\\
  \forall\stl,\stl',\stg.~P(\stl,\stg) \conj 
  \stl' = \stl \cup \{r' \,|\, \exists(r\in\Delta).~ [r/x]e
        \conj r'=\{\bar{f}=r.\bar{f}; \idf=r.\idf;
        \delf=\C{true}\}\}
  \Rightarrow 
  Q(\stl',\stg)
}
{
  \R \vdash \hoare{P}{\deletee{\lambda x.e}}{Q}
}
\end{array}
\end{smathpar}
\end{minipage}
%
\bigskip

%
\begin{minipage}{3.2in}
\rulelabel{RG-Insert}
\begin{smathpar}
\begin{array}{c}
\RULE
{
  \stable(\R,Q)\\
  \hspace*{-1.1in}\forall\stl,\stl',\stg,i.~P(\stl,\stg) \conj i \not\in
  \dom(\stl\cup\stg) \\
  \conj \stl'=\stl \cup 
  \{\{\bar{f}=x.\bar{f};\,\idf=i;\,\delf=\C{false}\} \Rightarrow 
  Q(\stl',\stg)
}
{
  \R \vdash \hoare{P}{\inserte{x}}{Q}
}
\end{array}
\end{smathpar}
\end{minipage}
%
%
\begin{minipage}{3in}
\rulelabel{RG-Foreach}
\begin{smathpar}
\begin{array}{c}
\RULE
{
  \stable(\R,Q)\spc
  \stable(\R,\psi)\\
  P \wedge y=\emptyset \Rightarrow \psi\spc
  \R \vdash \hoare{\psi \wedge z\in x}{c}{Q_c}\\
  Q_c \wedge z\in y \Rightarrow \psi\spc
  \psi \wedge y=x \Rightarrow Q
}
{
  \R \vdash \hoare{P}{\foreache{x}{\lambda y.\lambda z.c}}{Q}
}
\end{array}
\end{smathpar}
\end{minipage}
%
\bigskip

%
\begin{minipage}{3.9in}
\rulelabel{RG-Txn}
\begin{smathpar}
\begin{array}{c}
\RULE
{
  \stable(R,\I)\spc
  P(\stl,\stg) \Leftrightarrow \stl=\emptyset \wedge I(\stg)\\
  \R_l(\stl,\stg,\stg') \Leftrightarrow \exists \stg_1. \I_e(\stl, \stg_1, \stg) \wedge R(\stg, \stg') \wedge \I_e(\stl, \stg_1, \stg')\\
  \R_c(\stl,\stg,\stg') \Leftrightarrow \exists \stg_1. \I_c(\stl, \stg_1, \stg) \wedge R(\stg, \stg') \wedge \I_c(\stl, \stg_1, \stg')\\
  \R_l \vdash \rg{P}{c}{Q} \spc \stable(\R_c,Q)\\
  \forall \stl,\stg.~ Q(\stl,\stg) \Rightarrow 
    G(\stg, \stl \gg \stg)\spc
  \forall \stg,\stg'.~I(\stg) \wedge G(\stg,\stg') \Rightarrow I(\stg')\\
}
{
  \rg{I,R}{\ctxn{i}{\I}{c}}{G,I}
}
\end{array}
\end{smathpar}
\end{minipage}
%
%
\begin{minipage}{2in}
\rulelabel{RG-Conseq}
\begin{smathpar}
\begin{array}{c}
\RULE
{
  \rg{I,R}{\ctxn{i}{\I}{c}}{G,I}\\
  \I' \Rightarrow \I \spc 
  R' \subseteq R \\
  \stable(R',\I')\spc
  G \subseteq G' \\
  \forall \stg,\stg'.~I(\stg) \wedge G'(\stg,\stg') \Rightarrow I(\stg')\\
}
{
  \rg{I,R'}{\ctxn{i}{\I'}{c}}{G',I}
}
\end{array}
\end{smathpar}
\end{minipage}
%

\caption{\small \txnimp: Rely-Guarantee Rules}
\label{fig:rg-rules}
\vspace*{-12pt}
\end{figure}


The standard rely-guarantee judgment is a quintuple
$\rg{P,R}{c}{G,Q}$, which informally asserts that if a command $c$ is
executed in a state that satisfies its pre-condition $P$, provided
that every interference step during the execution is contained inside
the rely relation $R$, then the effect that each step of executing $c$
has on the state is captured by $G$, and the final state of execution
satisfies the post-condition $Q$. The terms ``execution'' and ``step''
are defined with respect to an operational semantics, which, in our
case, is parameterized on the trace invariant $\I$. Consequently, our
rely-guarantee judgment for commands is the sextuple
$\rg{\I,P,R}{c}{G,Q}$, whose semantics differs from the quintuple in
that it requires every step of the execution to additionally preserve
the trace invariant $\I$ for the post-condition to be valid.  As
usual, if $c$ is inside a transaction $T_i$, we write $\txnbox{c}_i$.
\txnimp expressions are side-effecting (they generate \C{RD} effects),
and admit interference during their evaluation.  Therefore, like
commands, expressions need to be reasoned about explicitly. But,
unlike commands, expressions evaluate to values, and the reasoning
framework should also admit assertions on such values.  We therefore
define a separate judgment for expressions - the septuple
$\rg{\I,P,R}{e}{G,C,Q}$. The new entrant $C : \mathbb{N} \rightarrow
\Prop$ is an assertion on the return value of $e$.

The rules that define rely-guarantee judgment are shown in
Fig.~\ref{fig:rg-rules}. Like standard rely-guarantee definitions,
these definitions also require a \emph{stability} condition, which
requires pre- and post- conditions to hold despite any interference
from concurrent threads (captured by $R$). Stability can be predicated
on the assumption that interference preserves the trace invariant
$\I$. Formally:\vspace*{-10pt}

\begin{smathpar}
\begin{array}{lcl}
  \underI{\stable(R,P)} & \defeq & \forall \E, \E'.\, 
  \I(\E) \conj P(\E) \conj R(\E,\E') \\
  &   & \hspace*{1in}\conj \I(\E') \Rightarrow P(\E')\\
\end{array}
\end{smathpar}

\noindent However, the assumption that interference preserves $\I$ (or,
dually, $\I$ withstands interference) needs to be justified
separately. We call this the stability requirement on $\I$:\vspace*{-10pt}

\begin{smathpar}
\begin{array}{lcl}
\stable(R,\I)& \defeq & \forall \E, \E'.\, 
  \I(\E) \conj R(\E,\E')\Rightarrow \I(\E')\\
\end{array}
\end{smathpar}

\noindent The rule \rulelabel{RG-Var} defines the rely-guarantee judgment for
shared variable reads inside a transaction $T_i$. It requires $\I$ to
be stable, and pre- and post- conditions to be stable relative to
$\I$.  The quantified premise effectively requires a proof that if the
abstract machine of Fig.~\ref{fig:txnimp} takes a step starting from
an execution $\E$ that satisfies the pre-condition $P$, then the
resultant execution $\E'$ satisfies the post-condition $Q$, and that
the guarantee $G$ faithfully captures the transition from $\E$ to
$\E'$. It is informative to compare this premise with the premise of
the \rulelabel{E-Aux} reduction rule of Fig.~\ref{fig:txnimp}. Similar
premises also appear in the \rulelabel{RG-Asgn} and \rulelabel{RG-Txn}
rules, which define rely-guarantee judgments for assignments and
transactions, respectively. \rulelabel{RG-Var} however also requires
the return value ($\interp{S}(X)$) of the read to satisfy the
assertion $C$ meant for the value. \rulelabel{RG-Arith} defines the RG
judgment for an arithmetic expression $e_1\pm e_2$ in terms of the
corresponding judgments for the constituent expressions $e_1$ and
$e_2$. The quantified premise requires any value resulting from
evaluating $e_1 \pm e_2$ to satisfy the assertion $C$, provided that
$e_1$ and $e_2$ always evaluate to values that satisfy $C_1$ and
$C_2$, respectively. The rules for sequential and parallel composition
of commands are essentially the same as their counterparts in a
standard rely-guarantee formulation and hence elided.

% RG judgment for the assignment $X:=e$ makes use of the corresponding
% judgment for the RHS expression $e$. The quantified premise asserts
% that evaluating the assigment after evaluating $e$ to a value $v$ and
% an execution $\E$, should result in an execution $\E'$ that satisfies
% the post-condition $Q$, while being related to $\E$ via the guarantee
% $G$. RG judgment for the \C{txn} lexical block is similar.  It uses
% the judgment for the transaction-bound command $c$ (i.e.,
% $\txnbox{c}_i$) to obtain an invariant $Q'$ for the execution $\E$
% before the commit, and verifies that committing the transaction under
% $\E$ results in an execution $\E'$ that satisfies transaction's
% post-condition $Q$. As usual, $\E$ and $\E'$ need to be related by $G$.
% The rules for sequential composition of transaction-bound comands
% (\rulelabel{RG-Seq}) and parallel composition of top-level commands
% (\rulelabel{RG-Par}) are straightforward, and more-or-less same as the
% corresponding rules in classical rely-guarantee.

The \rulelabel{RG-Conseq} rule define ways to strengthen or
weaken relations and assertions associated with the RG judgment of
transaction-bound expressions. Similar rules exist for
transaction-bound and top-level commands, but are not
shown.\footnote{The supplementary materials provide the complete set
  of rules.} As is the case with a standard rely-guarantee
formulation,the rules allow the pre-condition $P$ and the rely relation
$R$ to be strengthened, and the post-condition $Q$ (also, $C$ in the
case of expressions) and the guarantee relation $G$ to be weakened.
The most notable aspect of the \rulelabel{RG-Conseq} rules is that
they allow the trace invariant $\I$ to be strengthened. Considering
that $\I$ captures isolation properties, this means that a program
proven correct under weaker isolation levels is also correct under
stronger ones.  Parametricity over the trace invariant $\I$, combined
with the ability to strengthen $\I$ as needed, allows our proof system
to support a highly flexible proof strategy to prove programs correct
over various isolation variants. For example, programmers can
\emph{define} isolation guarantees of their choice (by defining $\I$
appropriately) and then prove programs correct assuming the guarantees
hold.  The soundness of strengthening $\I$ ensures that a program can
be safely executed on any system that offers isolation guarantees
at least as strong as those assumed.

\subsection{Semantics and Soundness}

\begin{definition}[\bfseries Step-indexed reflexive transitive closure]
For all $A:\text{Type}$, $R: A \rightarrow A \rightarrow \mathbb{P}$, and $n :
\mathbb{N}$, the step-indexed reflexive transitive closure $R^n$ of $R$ is
the smallest relation satisfying the following
properties:
\begin{itemize}
\item $\forall (x:A).\, R^0 (x,x)$
\item $\forall (n:\mathbb{N})(x,y,z : A).\, R(x,y) \conj R^n(y,z) \Rightarrow
R^{n+1}(x,z)$
\end{itemize}
\end{definition}

\begin{definition}[\bfseries Interleaved step and multi-step relations]
An interleaved step relation interleaves thread-local reductions with
interference from concurrent threads captured as the rely relation
($R$).  It is defined thus:\vspace*{-10pt}

\begin{smathpar}
\begin{array}{lcl}
\I \vdash (c,\E) \rstepsto (c',\E') & \defeq & \I \vdash 
  (c,\E) \stepsto (c',\E') \\
  &   & \disj (c' = c \conj R(\E, \E') \conj \I(\E'))\\
\end{array}
\end{smathpar}

\noindent The interleaved step relation for transaction bound expressions
($\txnbox{e}_i$) and commands ($\txnbox{c}_i$) is defined similarly.
An interleaved multi-step relation ($\stepssto{n}$) is the step-indexed
reflexive transitive closure of the interleaved step relation.
\end{definition}

\begin{definition}[\bfseries Semantics of the RG judgment]
\label{def:rg-semantics}
The semantics of the RG sextuple $\rg{\mathbb{I},P,R}{c}{G,Q}$ is defined
in terms of the interleaved step relation thus:\vspace*{-10pt}

\begin{smathpar}
\begin{array}{l}
\hspace*{-0.3in}
\rg{\mathbb{I},P,R}{c}{G,Q} \;\defeq\; \forall \E.\, P(\E)
  \wedge \mathbb{I}(\E) \\
\hspace*{0.4in}\Rightarrow (\forall n,\E'.\; \I \vdash (c,\E) 
    \rstepssto{n} (\cskip,\E') \Rightarrow Q(\E')) \\
\hspace*{0.5in}\conj \texttt{step-guaranteed}(\I,R,G,c,\E)\\
\end{array}
\end{smathpar}

\noindent The first conjunct in the consequent is called the \emph{Hoare
consequent} since it ascribes Hoare triple semantics to an RG sextuple.
The second conjunct, called the \emph{guarantee consequent}, uses the
$\texttt{step-guaranteed}$ predicate defined below:\vspace*{-10pt}

\begin{smathpar}
\begin{array}{l}
\texttt{step-guaranteed}(\I,R,G,c,\E) \;\defeq\; \forall n,\E',c'',\E''.\\
\hspace*{0.2in}\I \vdash (c,\E) \rstepssto{n} (c',\E') \conj \I \vdash (c',\E') \stepsto
  (c'',\E'') \Rightarrow G(\E',\E'')\\
\end{array}
\end{smathpar}

\noindent The guarantee consequent requires $G$ to capture the trace effect of
every small-step of $c$, where the reduction can be interleaved by the
interference ($R$) from concurrent threads. The semantics of the RG
sextuple for transaction-bound commands ($\txnbox{c}_i$) is defined
similarly. Expressions, unlike commands, evaluate to a value $v$, and
the semantics of their RG septuple ($\rg{\I,P,R}{\txnbox{e}_i}{G,C,Q}$) differs slightly in that its
Hoare consequent requires the value $v$ to satisfy the assertion $C$. 
\end{definition}

Note that the semantics of all RG judgments, including the judgments
for transaction-bound terms, make similar demands of the guarantee
relation. Given that transactions are atomic (though not isolated), it
is not immediately apparent why a transaction's guarantee is required
to make explicit every step of its reduction. This requirement is
justified however because, in reality, a transaction's atomicity is
predicated on the isolation settings of the observer. A \iso{Read
  Uncommitted} transaction, for example, is permitted to observe the
internal state of a transaction $T$ even if $T$ is claimed to execute
atomically.  In the interest of modular verification, the transaction
must therefore make its internal state available via its guarantee
relation.

\begin{theorem}[\bfseries Soundness] 
The rely-guarantee judgments defined by the rules in
Fig.~\ref{fig:rg-rules} are sound with respect to
the semantics of Definition~\ref{def:rg-semantics}.\footnote{Full proofs are provided in the supplementary material.}
\end{theorem}

\noindent In particular, if $\rg{\I,P,R}{c}{G,Q}$ can be derived using
the rules of Fig.~\ref{fig:rg-rules}, then (a) every interleaved
multi-step reduction of $c$ starting from a trace that satisfies $P$
and $\I$, results in a trace that satisfies $Q$, and (b) the effect
that every small-step of $c$ has on the trace is contained in $G$.
Soundness of the RG judgment for transaction-bound commands
($\txnbox{c}_i$) is stated similarly.  For expressions, soundness of
the judgment $\rg{\I,P,R}{\txnbox{e}_i}{G,C,Q}$ also proves that $e$
is always evaluated to a value that satisfies $C$.

