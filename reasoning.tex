\section{The Reasoning Framework}
\label{sec:reasoning}

% \begin{figure}
% \centering
% \begin{tabular}{l|l}
% \begin{txnimpcode}
% $\begin{decoration}
%   \{X = k\}
% \end{decoration}$
%   atomic {
%     v1 := 0; v2 := 1;
%     while(v1<X) {
%       v2 := v2 * (X-v1);
%       v1 := v1+1;
%     }
%     X := v2; Y := v2;
%   }
% $\begin{decoration}
%   \{X = k! \conj Y = X\}
% \end{decoration}$
% \end{txnimpcode}
% &
% \begin{txnimpcode}
% $\begin{decoration}
%   \{X = k\}
% \end{decoration}$
%   transaction {
%     v1 := 0; v2 := 1;
%     while(v1<X) {
%       v2 := v2 * (X-v1);
%       v2 := v2 * v1;
%       v1 := v1+1;
%     }
%     X := v2; Y := v2;
%   }
% $\begin{decoration}
%   \{Y = X\}
% \end{decoration}$
% \end{txnimpcode}
% \end{tabular}
%   \caption{Factorial program written as an atomic block and a weakly
%   isolated transaction (shared variables have uppercase names).
%   Effect of multiple iterations of the \C{atomic} loop can be
%   summarized as $k!$, while same cannot be done with the
%   \C{transaction} loop, where weak isolation allows reads in different
%   iterations to return different values for $X$.  Nonetheless,
%   post-state satisfies $X=Y$ in both cases due to atomicity of
%   writes.}
% \label{fig:atomic-vs-transaction}
% \end{figure}

We now describe a proof system that lets us prove the correctness of a
\txnimp program $c$ w.r.t its high-level consistency conditions $I$,
on an implementation that satisfies the isolation specifications
($\I$) of its transactions\footnote{Note the difference between $I$
  and $\I$. The former constitute \emph{proof} \emph{obligations} for
  the programmer, whereas the latter describes a transaction's
  \emph{assumptions} about the operational characteristics of the
  underlying system.}.  Our proof system is essentially an adaptation
of a rely-guarantee reasoning framework~\cite{rgjones} to the setting
of weakly isolated database transactions.  The primary challenge in
the formulation deals with how we relate a transaction's isolation
specification ($\I$) to its rely relation ($R$) that describes the
transaction's environment, so that interference is considered only
insofar as allowed by the isolation level.  Another characteristic of
the transaction setting that affects the structure of the proof system
is atomicity; we do not permit a transaction's writes to be visible
until it commits.  In the context of rely-guarantee, this means that
the transaction's guarantee ($G$) should capture the aggregate effect
of a transaction, and not its individual writes.  While shared memory
\C{atomic} blocks also have the same characteristic, the fact that
transactions are weakly-isolated introduces non-trivial complexity.
Unlike an \C{atomic} block, the effect of a transaction is \emph{not}
a sequential composition of the effects of its statements because each
statement can witness a potentially different version of the state.

\subsection{The Rely-Guarantee Judgment}
\label{sec:rely-guarantee}

\begin{figure}[t]
\raggedright
%
\fbox {\( \R \vdash \hoare{P}{c}{Q} \)} 
\quad \fbox {\( \rg{I,R}{c}{G,I} \)}\\[4pt]
%\fbox{\( \rg{\mathbb{I},P,R}{\txnbox{c}_i}{G,Q}\)} \quad
%\fbox{\( \rg{\mathbb{I},P,R}{c}{G,Q}\)} \quad\\
%
\begin{minipage}{3.1in}
\rulelabel{RG-Update}
\begin{smathpar}
\begin{array}{c}
\RULE
{
  \stable(\R,Q)\\
  \forall\stl,\stl',\stg.~P(\stl,\stg) \conj \\
  \stl' = \stl \cup \{r' \,|\, \exists(r\in\Delta).[r/x]e_2 \conj\\
  \hspace*{.7in} r'=[r/x]e_1\} \Rightarrow   Q(\stl',\stg)
}
{
  \R \vdash \hoare{P}{\updatee{\lambda x.e_1}{\lambda x.e_2}}{Q}
}
\end{array}
\end{smathpar}
\end{minipage}
%
%
\begin{minipage}{3in}
\rulelabel{RG-Select}
\begin{smathpar}
\begin{array}{c}
\RULE
{
  \\
  \R \vdash \hoare{P'}{c}{Q}\spc
  \stable(\R,P')\\
  P'(\stl,\stg) \Leftrightarrow P(\stl,\stg) \wedge
  x = \{r' \,|\, \exists(r\in\Delta).~ [r/y]e_2\} \\
}
{
  \R \vdash \hoare{P}{\lete{x}{\selecte{\lambda y.e}}{c}}{Q}
}
\end{array}
\end{smathpar}
\end{minipage}
%
\bigskip

%
\begin{minipage}{3.2in}
\rulelabel{RG-Delete}
\begin{smathpar}
\begin{array}{c}
\RULE
{
  \stable(\R,Q)\\
  \forall\stl,\stl',\stg.~P(\stl,\stg) \conj 
  \stl' = \stl \cup \{r' \,|\, \exists(r\in\Delta).~ [r/x]e
        \conj r'=\{\bar{f}=r.\bar{f}; \idf=r.\idf;
        \delf=\C{true}\}\}
  \Rightarrow 
  Q(\stl',\stg)
}
{
  \R \vdash \hoare{P}{\deletee{\lambda x.e}}{Q}
}
\end{array}
\end{smathpar}
\end{minipage}
%
\bigskip

%
\begin{minipage}{3.2in}
\rulelabel{RG-Insert}
\begin{smathpar}
\begin{array}{c}
\RULE
{
  \stable(\R,Q)\\
  \hspace*{-1.1in}\forall\stl,\stl',\stg,i.~P(\stl,\stg) \conj i \not\in
  \dom(\stl\cup\stg) \\
  \conj \stl'=\stl \cup 
  \{\{\bar{f}=x.\bar{f};\,\idf=i;\,\delf=\C{false}\} \Rightarrow 
  Q(\stl',\stg)
}
{
  \R \vdash \hoare{P}{\inserte{x}}{Q}
}
\end{array}
\end{smathpar}
\end{minipage}
%
%
\begin{minipage}{3in}
\rulelabel{RG-Foreach}
\begin{smathpar}
\begin{array}{c}
\RULE
{
  \stable(\R,Q)\spc
  \stable(\R,\psi)\\
  P \wedge y=\emptyset \Rightarrow \psi\spc
  \R \vdash \hoare{\psi \wedge z\in x}{c}{Q_c}\\
  Q_c \wedge z\in y \Rightarrow \psi\spc
  \psi \wedge y=x \Rightarrow Q
}
{
  \R \vdash \hoare{P}{\foreache{x}{\lambda y.\lambda z.c}}{Q}
}
\end{array}
\end{smathpar}
\end{minipage}
%
\bigskip

%
\begin{minipage}{3.9in}
\rulelabel{RG-Txn}
\begin{smathpar}
\begin{array}{c}
\RULE
{
  \stable(R,\I)\spc
  P(\stl,\stg) \Leftrightarrow \stl=\emptyset \wedge I(\stg)\\
  \R_l(\stl,\stg,\stg') \Leftrightarrow \exists \stg_1. \I_e(\stl, \stg_1, \stg) \wedge R(\stg, \stg') \wedge \I_e(\stl, \stg_1, \stg')\\
  \R_c(\stl,\stg,\stg') \Leftrightarrow \exists \stg_1. \I_c(\stl, \stg_1, \stg) \wedge R(\stg, \stg') \wedge \I_c(\stl, \stg_1, \stg')\\
  \R_l \vdash \rg{P}{c}{Q} \spc \stable(\R_c,Q)\\
  \forall \stl,\stg.~ Q(\stl,\stg) \Rightarrow 
    G(\stg, \stl \gg \stg)\spc
  \forall \stg,\stg'.~I(\stg) \wedge G(\stg,\stg') \Rightarrow I(\stg')\\
}
{
  \rg{I,R}{\ctxn{i}{\I}{c}}{G,I}
}
\end{array}
\end{smathpar}
\end{minipage}
%
%
\begin{minipage}{2in}
\rulelabel{RG-Conseq}
\begin{smathpar}
\begin{array}{c}
\RULE
{
  \rg{I,R}{\ctxn{i}{\I}{c}}{G,I}\\
  \I' \Rightarrow \I \spc 
  R' \subseteq R \\
  \stable(R',\I')\spc
  G \subseteq G' \\
  \forall \stg,\stg'.~I(\stg) \wedge G'(\stg,\stg') \Rightarrow I(\stg')\\
}
{
  \rg{I,R'}{\ctxn{i}{\I'}{c}}{G',I}
}
\end{array}
\end{smathpar}
\end{minipage}
%

\caption{\small \txnimp: Rely-Guarantee Rules}
\label{fig:rg-rules}
\vspace*{-12pt}
\end{figure}


Fig.~\ref{fig:rg-rules} shows an illustrative subset of the
rely-guarantee (RG) reasoning rules for $\txnimp$. We define two RG
judgments: top-level ($\rg{I,R}c{G,I}$), and transaction-local ($\R
\vdash \hoare{P}c{Q}$).  Recall that the standard RG judgment is the
quintuple $\rg{P,R}{c}{G,Q}$. Instead of separate $P$ and $Q$
assertions, our top-level judgment uses $I$ as both a pre- and
post-condition, because our focus is on verifying that a
\txnimp\ program \emph{preserves} a databases' consistency
conditions\footnote{The terms \emph{consistency condition},
  \emph{high-level invariant}, and \emph{integrity constraint} are
  used interchangeably throughout the paper.}.  A transaction-local RG
judgment does not include a guarantee relation because
transaction-local effects are not visible outside a transaction. Also,
the rely relation ($\R$) of the transaction-local judgment is not the
same as the top-level rely relation ($R$) because it must take into
account the transaction's isolation specification ($\I$). Intuitively,
$\R$ is $R$ modulo $\I$.  Recall that a transaction writes to its
local database ($\stl$), which is then flushed when the transaction
commits. Thus, the guarantee of a transaction depends on the state of
its local database at the commit point. The pre- and post-condition
assertions ($P$ and $Q$) in the local judgment facilitate tracking the
changes to the transaction-local state, which eventually helps us
prove the validity of the transaction's guarantee.  Both $P$ and $Q$
are bi-state assertions; they relate transaction-local database state
($\stl$) to the global database state ($\stg$). Thus, the
transaction-local judgment effectively tracks how transaction-local
and global states change in relation to each other.

\subsubsection{Stability}

A central feature of a rely-guarantee judgment is a stability
condition that requires the validity of an assertion $\phi$ to be
unaffected by interference from other concurrently executing
transactions, i.e., the rely relation $R$. In conventional RG,
stability is defined as follows, where $\sigma$ and $\sigma'$ denote
states:
\begin{smathpar}
\begin{array}{lcl}
\stable(R,\phi) & \Leftrightarrow & \forall \sigma,\sigma'.~
\phi(\sigma) \conj R(\sigma,\sigma') \Rightarrow \phi(\sigma')\\
\end{array}
\end{smathpar}
Due to the presence of local and global database states, and the
availability of an isolation specification, we use multiple
definitions of stability in Fig.~\ref{fig:rg-rules}, but they all
convey the same intuition as above. In our setting, we only need to
prove the stability of an assertion ($\phi$) against those environment
steps which lead to a global database state on which the transaction
itself can take its next step according to its isolation specification
($\I$). 
\begin{smathpar}
\begin{array}{lcl}
\stable(R, \phi) & \Leftrightarrow & \forall \stl, \stg, \stg'. \phi(\stl, \stg) 
  \wedge R^{*}(\stg, \stg') \wedge \I(\stl, \stg, \stg') \Rightarrow \phi(\stl, \stg')
\end{array}
\end{smathpar}
\noindent A characteristic of RG reasoning is that stability of an
assertion is always proven w.r.t to $R$, and not $R^{*}$, although
interference may include multiple environment steps, and $R$ only
captures a single step. This is nonetheless sound due to inductive
reasoning: if $\phi$ is preserved by every step of $R$, then $\phi$ is
preserved by $R^{*}$, and vice-versa.  However, such reasoning does
not extend naturally to isolation-constrained interference because
$R^{*}$ modulo $\I$ is not same as $\R^{*}$; the former is a
transitive relation constrained by $\I$, whereas the latter is the
transitive closure of a relation constrained by $\I$. This means,
unfortunately, that we cannot directly replace $R^{*}$ by $R$ in the
above condition.

To obtain an equivalent form in our setting, we require an additional
condition on the isolation specification, which we call the
\emph{stability condition on $\I$}.  The condition requires $\I$ to
admit the interference of multiple $R$ steps (i.e.,
$R^{*}(\stg,\stg'')$, for two database states $\stg$ and $\stg''$),
only if it also admits interference of each $R$ step along the way.
Formally:
\begin{smathpar}
\begin{array}{lcl}
  \stable(R,\I) & \Leftrightarrow & \forall \stl,\stg,\stg',\stg''.~
  \I(\stl,\stg,\stg'') \conj R(\stg',\stg'') \Rightarrow
  \I(\stl,\stg,\stg') \conj \I(\stl,\stg',\stg'')
\end{array}
\end{smathpar}
It can be easily verified that the above stability condition is
satisfied by the isolation axioms from Sec.~\ref{sec:isolation}. For
instance, $\I_{ss}$, the snapshot axiom, is stable because if a the
state is unmodified between $\stg$ and $\stg''$, then it is clearly
unmodified between $\stg$ and $\stg'$, and also between $\stg'$ and
$\stg''$, where $\stg'$ is an intermediary state.  Modifying and
restoring the state $\stg$ is not possible because each new commit
bears a new transaction id different from the transaction ids (\C{txn}
fields) present in $\stg$. 

The stability condition on $\I$ guarantees that an interference from
$R^*$ is admissible only if the interference due to each individual
$R$ step is admissible. In other words, it makes isolation-constrained
$R^*$ relation equal to the transitive closure of the
isolation-constrained $R$ relation. We call $R$ constrained by
isolation $\I$ as $R$ modulo $\I$ ($R\backslash \I$; written equivalently
as $\R$), which is the following ternary relation:
\begin{smathpar}
  \begin{array}{lcl}
    (R \backslash \I)(\stl, \stg, \stg') & \Leftrightarrow & R(\stg,
      \stg') \wedge \I(\stl, \stg, \stg')\\
  \end{array}
\end{smathpar}
It is now enough to prove the stability of an RG assertion $\phi$
w.r.t $R\backslash \I$:
\begin{smathpar}
  \begin{array}{lcl}
    \stable((R \backslash \I), \phi) & \Leftrightarrow & \forall
      \stl,\stg,\stg'.~ \phi(\stl,\stg) \wedge (R \backslash \I)(\stl,\stg,\stg') \Rightarrow \phi(\stl,\stg')\\
  \end{array}
\end{smathpar}
This condition often significantly simplifies the form of $R
\backslash \I$ irrespective of $R$. For example, when a transaction is
executed against a snapshot of the database (i.e. $\mathbb{I}_{ss}$),
$R \backslash \I_{ss}$ will be the identity function, since any
non-trivial interference will violate the $\stg' = \stg$ condition
imposed by $\I_{ss}$.

\subsubsection{Rules}

\rulelabel{RG-Txn} is the top-level rule that lets us prove a
transaction preserves the high-level invariant $I$ when executed under
the required isolation as specified by $\I$. It depends on a
transaction-local judgment to verify the body ($c$) of a transaction
with id $i$. The precondition $P$ of $c$ must follow from the fact
that the transaction-local database ($\stl$) is initially empty, and
the global database satisfies the high-level invariant $I$. The rely
relation ($\R_e$) is obtained from the global rely relation $R$ and
the isolation specification $\I_e$ as explained above. Recall that
$\I_e$ constrains the global effects visible to the transaction while
it is executing but has not yet committed, and $P$ and $Q$ of the
transaction-local RG judgment are binary assertions; they relate local
and global database states. The local judgment rules require one or
both of them to be stable with respect to the constrained rely
relation $\R_e$.


%for the local judgment is a ternary relation computed as $R$
%modulo $\I_e$:
%\begin{smathpar}
%\begin{array}{lcl}
%\R_e(\stl,\stg,\stg') & \Leftrightarrow & R(\stg,\stg') \conj
%\I_e(\stl,\stg,\stg')
%  \R_e(\stl,\stg,\stg') & \Leftrightarrow & \exists \stg_1.  R(\stg, \stg') \wedge \I_e(\stl, \stg_1, \stg) \wedge \I_e(\stl, \stg_1, \stg')
%\end{array}
%\end{smathpar}
%\noindent While $R$ describes changes in the global database state, $\I_e$
%constrains the global effects visible to the transaction.
%Recall however that $\I_e$ uses the global state in effect at the
%point the transaction last took a step ($\stg_1$) to determine whether
%the transaction can proceed from the current global state ($\stg$).
%Moreover, the global rely relation $R$ defines the permissible
%interferences from $\Delta$ to a future global state ($\Delta'$).  The
%local rely relation $\R_e$ determines allowed interferences as
%specified by $\I_e$ assuming $\stg_1$.  In effect, the existential on
%$\stg_1$ defines this rely relation to be an over-approximation on
%the interferences in the presence of which the transaction can
%proceed, simplifying verification burden.


%In words, if there exists $\stg_1$ which was the global database state during the last step by a transaction and if the transaction can take its next step at the current global database state $\stg$, then even after the interference, the transaction should still be able to take its next step. 


%\begin{smathpar}
%\exists \stg_1. \I_e(\stl, \stg_1, \stg) \wedge R(\stg, \stg') \wedge \I_e(\stl, \stg_1, \stg') \Rightarrow \forall \stg_2 
%\end{smathpar}
%Thus, $\R_l$ allows an interference only if it does not violate the
%execution-time isolation specification ($\I_e$) of the transaction. If a
%certain interference violates the isolation spec, i.e., $R(\stg,\stg')
%\Rightarrow \neg(\I_e(\stl,\stg,\stg'))$, then $\R_l(\stl,\stg,\stg')
%\Leftrightarrow false$, and any assertion is trivially stable w.r.t
%that interference. This is sensible considering such interference is
%prevented in the operational semantics. 


%The
%stability of a binary assertion $Q$ w.r.t a ternary rely relation $\R$
%is defined as:
%\begin{smathpar}
%\begin{array}{c}
%\forall \stl,\stg,\stg'.~ Q(\stl,\stg) \conj \R(\stl,\stg,\stg')
%\Rightarrow Q(\stl,\stg')
%\end{array}
%\end{smathpar}
%That is, if $Q$ relates $\stl$ to $\stg$, and an interference allowed
%by the isolation specification (which implicitly considers the local
%state $\stl$) takes $\stg$ to $\stg'$, then $Q$ must also relate $\stl$
%to $\stg'$.
\raggedbottom
For the guarantee $G$ of a transaction to be valid, it must follow
from the post-condition $Q$ of the body, provided that $Q$ is stable
w.r.t the commit-time interference captured by $\R_c$. $\R_c$, like
$\R_e$, is computed as a rely relation modulo isolation, except that
commit-time isolation ($\I_c$) is considered. The validity of
$G$ is captured by the following implication:
\begin{smathpar}
\begin{array}{c}
  \forall \stl,\stg.~ Q(\stl,\stg) \Rightarrow G(\stg, \stl \rhd \stg)\spc
\end{array}
\end{smathpar}
In other words, if $Q$ relates the transaction-local database state
($\stl$) to the state of the global database ($\stg$) before a transaction
commits, then $G$ must relate the states of the global database before
and after the commit. The act of commit is captured by the flush
action ($\stl\rhd\stg$). Once we establish the validity of $G$ as a
faithful representative of the transaction, we can verify that the
transaction preserves the high-level invariant $I$ by checking the
stability of $I$ w.r.t $G$, i.e., $\forall \stg,\stg'.~I(\stg) \wedge
G(\stg,\stg') \Rightarrow I(\stg')$.

%A characteristic of RG reasoning is that stability of an assertion is
%always proven w.r.t to $R$, and not $R^{*}$, although interference may
%include multiple environment steps, and $R$ only captures a single
%step. This is nonetheless sound due to inductive reasoning: if
%$Q$ is preserved by every step of $R$, then $Q$ is preserved by
%$R^{*}$, and vice-versa.  However, such reasoning does not extend
%naturally to isolation-constrained interference because $R^{*}$ modulo
%$\I$ is not same as $\R^{*}$; the former is a transitive relation
%constrained by $\I$, whereas the latter is the transitive closure of an
%$\I$-constrained relation. We therefore introduce a side-condition on
%$\I$ that restores the equality. The condition requires $\I$ to allow
%an interference $R^{*}(\stg,\stg'')$, for two database states $\stg$
%and $\stg''$, only if it also allows interference for every prefix of
%$R^{*}(\stg,\stg'')$. In other words, if $\I$ disallows interference
%from $\stg$ to $\stg'$, then an $R$-step from $\stg'$ to $\stg''$
%should not make the interference from $\stg$ to $\stg''$ valid.  This
%stability condition on $\I$ is defined formally thus:
%\begin{smathpar}
%\begin{array}{lcl}
%  \stable(R,\I) & \Leftrightarrow & \forall \stl,\stg,\stg',\stg''.~
%  \neg\I(\stl,\stg,\stg') \conj R(\stg',\stg'') \Rightarrow
%  \neg\I(\stl,\stg,\stg'')
%\end{array}
%\end{smathpar}
%It can be easily verified that the above stability condition is
%satisfied by the isolation axioms from Sec.~\ref{sec:isolation}. For
%instance, $\I_{ss}$, the snapshot axiom, is stable because if
%$\I_{ss}$ is invalid ($\neg\I_{ss}$), then an interference has already
%modified a record, and no further interference will restore the
%original record, because the original record bears the id of a
%transaction that has already committed. 

The \rulelabel{RG-Conseq} rule lets us safely weaken the guarantee
$G$, and strengthen the rely $R$ of a transaction. Importantly, it
also allows its isolation specification $\I$ to be strengthened (both
$\I_e$ and $\I_c$). This means that a transaction proven correct under
a weaker isolation level is also correct under a stronger level.
Parametricity over the isolation specification $\I$, combined with the
ability to strengthen $\I$ as needed, admits a flexible proof strategy
to prove database programs correct. For example, programmers can
declare isolation requirements of their choice through $\I$, and then
prove programs correct assuming the guarantees hold. The soundness of
strengthening $\I$ ensures that a program can be safely executed on
any system that offers isolation guarantees at least as strong as
those assumed.

Salient rules of transaction-local RG judgments are shown in
Fig.~\ref{fig:rg-rules}. These rules (\rulelabel{RG-Update},
\rulelabel{RG-Select}, \rulelabel{RG-Delete}, and
\rulelabel{RG-Insert}) reflect the structure of the corresponding
reduction rule from Fig.~\ref{fig:txnimp}.  The rule
\rulelabel{RG-Foreach} defines the RG judgment for a \C{FOREACH} loop.
As is characteristic of loops, the reasoning is pivoted on a loop
invariant $\psi$ that needs to be stable w.r.t $\R$. $\psi$ must be
implied by $P$, the pre-condition of \C{FOREACH}, when no elements
have been iterated, i.e, when $y=\emptyset$. The body of the loop can
assume the loop invariant, and the fact that $z$ is an element from
the set $x$ being iterated, to prove its post-condition $Q_c$. The
operational semantics ensures that $z$ is added to $y$ at the end of
the iteration, hence $Q_c$ must imply $[y\cup\{z\}/y]\psi$. When the
loop has finished execution, $y$, the set of iterated items, is the
entire set $x$. Thus $[x/y]\psi$ is true at the end of the loop, from
which the post-condition $Q$ must follow. As with the other rules, $Q$
needs to be stable. The rules for conditionals, sequencing etc., are
standard, and hence elided.

\subsection{Semantics and Soundness}

We now formalize the semantics of the RG judgments defined in
Fig.~\ref{fig:rg-rules}, and state their soundness guarantees.

\begin{definition}[\bfseries Interleaved step and multi-step relations]
Interleaved step relations interleave global and transaction-local
reductions with interference as captured by the corresponding rely
relations. They are defined thus:
\begin{smathpar}
\begin{array}{lclr}
(c,\stg) \rstepsto (c',\stg') & \defeq &  
  (c,\stg) \stepsto (c',\stg') \disj (c' = c \conj R(\stg, \stg'))&
  \texttt{global}\\
(\tbox{c}_i,\stl,\stg) \rstepsto (\tbox{c'}_i,\stl',\stg') & \defeq & \stg \vdash 
  (\tbox{c}_i,\stl) \stepsto (\tbox{c'}_i,\stl') \conj \stg'=\stg& \texttt{transaction-local}\\
  &   & \disj (c' = c \conj \stl'=\stl \conj \R(\stl, \stg, \stg'))
\end{array}
\end{smathpar}

\noindent An interleaved multi-step relation ($\stepssto{n}$) is the
reflexive transitive closure of the interleaved step relation.  
\end{definition}

\begin{definition}[\bfseries Semantics of RG judgments]
\label{def:rg-semantics}
The semantics of the global and transaction-local RG judgments are
defined thus:
\begin{smathpar}
\begin{array}{lclr}
\R \vdash \hoare{P}{c}{Q} & \defeq & \forall
  \stl,\stl',\stg,\stg'.~ P(\stl,\stg) \conj (\tbox{c}_i,\stl,\stg) \rstepssto{n}
  (\tbox{\cskip}_i, \stl',\stg')
  \Rightarrow Q(\stl',\stg') &\\
\rg{I,R}{c}{G,I} & \defeq &  \forall \stg.\, I(\stg)
  \Rightarrow (\forall \stg'.\; (c,\stg) 
    \rstepssto{n} (\cskip,\stg') \Rightarrow I(\stg')) \\
&&\hspace*{0.6in}\conj \texttt{txn-guaranteed}(R,G,c,\stg)\\
\end{array}
\end{smathpar}

\noindent The
$\texttt{txn-guaranteed}$ predicate used in the semantics of the
global RG judgment is defined below:\vspace*{-10pt}

\begin{smathpar}
\begin{array}{lcl}
\texttt{txn-guaranteed}(R,G,c,\stg) &\defeq& \forall c',c''\stg',\stg''.
(c,\stg) \rstepssto{n} (c',\stg') \conj (c',\stg') \stepsto
  (c'',\stg'') \Rightarrow G(\stg',\stg'')\\
\end{array}
\end{smathpar}
\end{definition}


\noindent Thus, if $\rg{I,R}{c}{G,I}$ is a valid RG judgment, then (a)
every interleaved multi-step reduction of $c$ preserves the database
integrity constraint (consistency condition) $I$, and (b) the effect
that every transaction in $c$ has on the database state is captured by
$G$.
\noindent We can now assert the soundness of the RG judgments in
Fig.~\ref{fig:rg-rules} as follows\footnote{Full proofs for the major
  theorems and lemmas defined in this paper are available from
  ~\cite{KN+18_arxiv}.}:

\begin{theorem}[\bfseries Soundness] 
The rely-guarantee judgments defined by the rules in
Fig.~\ref{fig:rg-rules} are sound with respect to the semantics of
Definition~\ref{def:rg-semantics}.
\end{theorem}


\paragraph{{\sc Proof Sketch.}}  The most important rule is the top-level
rule \rulelabel{RG-Txn}, which proves that a transaction $c$ which
begins its execution in global database state satisfying $I$ and
encountering interference $R$ while executing under isolation
specification $\I$ finishes its execution in a database state also
satisfying $I$, and also guarantees that its commit step satisfies
$G$. The rule uses the transaction-local RG judgment $\R_e \vdash
\rg{P}{c}{Q}$. By \rulelabel{E-Txn-Start}, the local and global
database states at the start of a transaction satisfy $P$, and the
only challenge is that environment steps in an execution covered by
$\R_e \vdash \rg{P}{c}{Q}$ are in $\R_e$, while the top-level
judgment requires environment steps in $R$. We show that it is enough
to consider only those environment steps in $\R_e$. First, we use an
inductive argument and stability of $\I_e$ ($\stable(R, \I_e)$) to
show that any execution in which the transaction completes all its
steps must always preserve the isolation specification $\I_e$ after
every environment step. Intuitively, this is because once $\I_e$ gets
broken after some environment step, it will continue to remain broken
and the transaction would not be able to proceed (according to
\rulelabel{E-Txn}). Since $\R_e$ contains exactly those environment
steps which preserve $\I_e$, the local-level RG judgment can be
soundly used, which guarantees that after the transaction finishes its
execution, its local state $\stl$ and global state $\stg$ will satisfy
the assertion $Q$. Environment steps between the last step of the
transaction and its commit step can modify the global state, and hence
we also require $Q$ to be stable against $R$. Again, we use an
inductive argument, the stability of $\I_c$, and the fact that the
transaction must execute its commit step to show that all environment
steps must preserve $\I_c$, and hence it is enough to require
$\stable(\R_c, Q)$. $Q$ guarantees that the commit step is in $G$, and
$G$ in turn guarantees that after execution, the global database state
will obey the invariant $I$.

%% The local RG-judgement rules are mostly straightforward and directly
%% follow from the operational semantics. At this point, the transaction
%% can only change its local database state by adding new records to it,
%% and hence all the rules have the form : $P(\stl, \stg)$ and $\stl'$ is
%% the new state should satisfy $Q(\stl', \stg)$. Note that note is that
%% the {\sc RG-Update}, {\sf RG-Select}, {\sc RG-Delete} and {\sc
%%   RG-Insert} rules only require the pre-condition $P$ to be stable
%% against interference. This is because the these rules only make
%% assertions about the database state just after the transaction takes
%% its last step. For the aforementioned commands, the last step would be
%% the only step taken by the transaction, and hence $Q$ would not be
%% affected by any environment steps, but instead directly reflect the
%% changes made by the command (the environment steps between the last
%% execution step of the transaction and the commit step would be
%% controlled by $\I_c$ and would be taken care of by $\stable(\R_c, Q)$
%% at the top level).

%% The rule $\rulelabel{RG-Foreach}$ makes use of an inductive loop
%% invariant $\psi$ which assumes that the variable $y$ is bound to
%% records which have been processed in previous iterations and $z$ is
%% bound to the record processed in the current iteration. Since the
%% operational semantics ensure that every iteration processes a unique
%% record, the proof is straightforward.


