\section{\txnimp}
\label{sec:opsem}

\renewcommand{\ctxn}[3]{\C{TXN}_{#1}\langle #2 \rangle\{#3\}}
\begin{figure*}[!ht]
\raggedright
%
\textbf{Syntax}\\
%
\begin{smathpar}
\renewcommand{\arraystretch}{1.2}
\begin{array}{lclcl}
\multicolumn{5}{c} {
  {x,y} \in \mathtt{Variables}\qquad
  {f} \in \mathtt{Field\;Names} \qquad
% {\tau} \in \mathtt{Table\;Names}
  {i,j} \in \mathbb{N} \qquad
  {\odot} \in \{+,-,\le,\ge,=\}\qquad
}\\
\multicolumn{5}{c}{
  {k} \in \mathbb{Z}\cup\mathbb{B} \qquad
  {\rec} \in \{\bar{f}=\bar{k}\}\qquad
  {\stl,\stg,s} \in \Pow{\rec} \qquad
  \I \in \mathbb{B}\times\Pow{\rec}\times\Pow{\rec}\times\Pow{\rec}
  \rightarrow \Prop
}\\
v & \in & \mathtt{Values} & \coloneqq & k \ALT \rec \ALT s\\
e & \in & \mathtt{Expressions} & \coloneqq & c \ALT x \ALT x.f 
    \ALT \{\bar{f}=\bar{e}\} \ALT e_1 \odot e_2\\ 
c & \in & \mathtt{Commands} & \coloneqq & \cskip \ALT \lete{x}{e}{c}
    \ALT \ite{e}{c_1}{c_2}\ALT c_1;c_2 \ALT \inserte{e}  \\
&&&&\ALT \deletee{\lambda x.e}
    \ALT \lete{x}{\selecte{\lambda x.e}}{c}
    \ALT \updatee{\lambda x.e_1}{\lambda x.e_2}\\
&&&&\ALT \foreache{e_1}{\lambda x.\lambda y. e_2} 
    \ALT \foreachr{s_1}{s_2}{\lambda x.\lambda y. e}\\
&&&&\ALT \ctxn{i}{\I}{ c } \ALT \ctxn{i}{\I,\stl,\stg}{c} \ALT c1 || c2\\
% t & \in & \mathtt{Terms} & \coloneqq & e \ALT c\\
\ectx & \in & \mathtt{Eval\;Ctx} & ::= & \bullet \ALT  
  \bullet || c_2 \ALT c_1 || \bullet \ALT \bullet;\,c_2 
  \ALT \ctxn{i}{\I,\stl,\stg}{\bullet} \\
\end{array}
\end{smathpar}
%
\bigskip

\renewcommand{\arraystretch}{1.2}

%
\textbf{Local Reduction} \quad 
\fbox {\(\stg \vdash (c,\stl) \stepsto (c',\stl')\)}\\
%
\begin{minipage}{2.8in}
\rulelabel{E-Insert}
\begin{smathpar}
\begin{array}{c}
\RULE
{
  i \not\in \dom(\stl \cup \stg)\\
  r = \{\bar{f}=\bar{k};\,\idf=i;\,\delf=\C{false}\}
}
{
  \stg \vdash (\inserte{\{\bar{f}=\bar{k}\}},\stl) \stepsto
  (\cskip,\stl \cup \{r\})
}
\end{array}
\end{smathpar}
\end{minipage}
%
%
\begin{minipage}{2.8in}
\rulelabel{E-Delete}
\begin{smathpar}
\begin{array}{c}
\RULE
{
  s = \{r' \,|\, \exists(r\in\Delta).~ \eval([r/x]e)=\C{true} \\
        \hspace*{0.7in}\conj r'=\{\bar{f}=r.\bar{f}; \idf=r.\idf;
        \delf=\C{true}\}\}\\
% \dom(s) \cap \dom(\delta) = \emptyset
}
{
  \stg \vdash (\deletee{\lambda x.e},\stl) \stepsto (\cskip,\stl \cup s)
}
\end{array}
\end{smathpar}
\end{minipage}
%
\bigskip

%
\begin{minipage}{2.8in}
\rulelabel{E-Select}
\begin{smathpar}
\begin{array}{c}
\RULE
{
  s = \{r\in\Delta \,|\, \eval([r/x]e)=\C{true}\}\spc
  c' = [s/x]c
}
{
  \stg \vdash (\lete{x}{\selecte{\lambda x.e}}{c}, \stl) \stepsto 
              (c',\stl)
}
\end{array}
\end{smathpar}
\end{minipage}
%
%
\begin{minipage}{2.8in}
\rulelabel{E-Update}
\begin{smathpar}
\begin{array}{c}
\RULE
{
  s = \{r' \,|\, \exists(r\in\Delta).~ \eval([r/x]e_2)=\C{true} \conj r'=[r/x]e_1\}\\
}
{
  \stg \vdash (\updatee{\lambda x.e_1}{\lambda x.e_2},\stl) \stepsto 
              (\cskip,\stl \cup s)
}
\end{array}
\end{smathpar}
\end{minipage}
%

\begin{smathpar}
\begin{array}{ll}
  \rulelabel{E-Foreach1} & \stg \vdash (\foreache{e}{\lambda x.\lambda
  y.c},\stl) \stepsto (\foreachr{\emptyset}{\eval(e)}{\lambda x.\lambda y. c})\\
  \rulelabel{E-Foreach2} & \stg \vdash (\foreachr{s_1}{\{r\} \cup s_2}{\lambda x.\lambda
  y.c},\stl) \stepsto ([r/y][s_1/x]c;\,\foreachr{s_1 \cup \{r\}}{s_2}{\lambda x.\lambda y. c})\\
  \rulelabel{E-Foreach3} & \stg \vdash (\foreachr{s}{\emptyset}{\lambda x.\lambda
  y.c},\stl) \stepsto (\cskip,\stl)\\
\end{array}
\end{smathpar}
%
\bigskip

%
\textbf{Top-Level Reduction} \quad 
\fbox {\((c,\stg) \stepsto (c',\stg')\)}\\
%
\begin{minipage}{3in}
\rulelabel{E-Txn}
\begin{smathpar}
\begin{array}{c}
\RULE
{
  \I(\C{true},\stl,\stg,\stg')\spc
  \stg \vdash (c,\stl) \stepsto (c',\stl')
}
{
  (\ctxn{i}{\I,\stl,\stg}{c},\stg') \stepsto
  (\ctxn{i}{\I,\stl',\stg'}{c'},\stg')
}
\end{array}
\end{smathpar}
\end{minipage}
%
%
\begin{minipage}{2.8in}
\rulelabel{E-Commit}
\begin{smathpar}
\begin{array}{c}
\RULE
{
  \I(\C{false},\stl,\stg,\stg')
}
{
  (\ctxn{i}{\I,\stl,\stg}{\cskip},\stg') \stepsto (\cskip,\stl\gg\stg')
}
\end{array}
\end{smathpar}
\end{minipage}
%


\caption{\small \txnimp: Syntax and Small-step semantics}
\label{fig:txnimp}
\end{figure*}



\subsection{Syntax and Semantics}
\label{sec:syntax}

Fig.~\ref{fig:txnimp} shows the syntax and small-step semantics of
\txnimp. Natural numbers, (shared) variables and arithmetic
expressions constitute the syntactic class of expressions ($e$).
Commands ($c$) include $\cskip$, assignment statements, transaction
(\C{txn}) lexical blocks, and their sequential and parallel
composition. We let $T_i$ for $i \in \mathbb{N}$ range over
transaction identifiers. When it is evident we are referring to a
transaction, we use the number $i$ instead of $T_i$ for identification
(\eg, in $\C{txn}\langle i \rangle$). For notational convenience we let
$t$ range over both expressions and commands.

Small-step semantics of \txnimp are defined in terms of an abstract
machine that generates an execution trace, also called an execution
($\E$). First artifact of the trace is a set ($\A$) of \emph{effects},
where an effect ($\eta$) documents a read (\C{RD(X)}), or a write
(\C{WR(X)}), or a transaction commit (\C{COMMIT}) operation performed
during the execution. Any value associated with the operation (\eg,
value read or value written) is also documented, and can be accessed
via $\rval$. Every effect has a unique identifier accessible via
$\id$, and a transction idenfier accessible via $\txn$.  The latter
identifies the transaction that generated the effect. In every step of
the evaluation, abstract machine reduces a \txnimp term by executing a
read, write or commit operation, generating an effect, and extending
the trace. Since effects include transaction identifiers, semantics
distinguishes between terms ($t$) of different transactions. For
example, $\txnbox{t}_i$ denotes a term $t$ inside a transaction $T_i$.
Evaluation contexts are also appropriately marked. For example,
$\ectx_i$ denotes evaluation context for a term inside $T_i$. The
other component of the execution trace is the visibility relation
($\visZ$) between the effects. The intent and mechanics of $\visZ$
become clear as we describe the evaluation rules in the following.

Fundamental to our operational semantics is the notation of a trace
invariant ($\I$), which is a function from traces ($\E$) to
propositions ($\texttt{Prop}$) that declare well-formedness
constraints over traces. Abstract machine takes a step only if the
resulting trace satisfies the well-formedness constraints imposed by
$\I$. This behaviour is captured by the auxiliary reduction rule
\rulelabel{E-Aux} that factors out the trace extension aspect of the
evaluation by abstracting away operation-specific behaviour as a
function ($\mathcal{F}$) that generates appropriate effect.
\rulelabel{E-Aux} uses $\mathcal{F}$ to generate a new effect and
extend the trace ($\E = (\A,\visZ)$) \emph{only if} the
well-formedness constraints imposed by $\I$ on $\E$ (i.e., $\I(\E')$)
are satisfied. Otherwise, it gets stuck. In an execution that runs to
completion, every small-step preserves well-formedness of trace, thus
ensuring the invariance of $\I$. Note that the operational semantics
makes no assumptions about $\I$ other than its type. As such it is
parametric over $\I$, and can be instantiated with any
trace-parametric proposition that expresses constraints over the given
trace. While the machine takes a step only if the constraints are
satisfied, it neither defines nor explicitly assumes an oracle to
check the satisfaction; there is no motivation for doing so in the
semantics.

As explained in \S\ref{sec:motivation}, semantics of various isolation
levels can be captured as constraints over the happens-before ($\hbZ$)
relation. $\hbZ$ is however a derived relation in our model, composed
of more fundamental \emph{session order} ($\soZ$) and
\emph{visibility} ($\visZ$) relations. Session order captures the
sequential order of operations within a transaction. In particular, it
relates two effects, $\eta_1$ and $\eta_2$, such that $\txn(\eta_1) =
\txn(\eta_2)$ and $\id(\eta_1) < \id(\eta_2)$. Small-step semantics
assigns monotonically increasing identifiers to effects, as captured
by the $\id(\eta) > \maxId(\A)$ condition of \rulelabel{E-Aux}.
Evaluation contexts ($\ectx_i$) for transaction-bound terms are
defined so as to enforce a deterministic sequential order of execution
within a transaction, leading to a deterministic total order among
effect ids - the session order. Visibility ($\visZ$) on the other
hand relates effects across concurrent transactions, thus reflects the
non-determinism inherent in concurrency. Intuitively, $\visZ$ relates
$\eta_1$ to $\eta_2$ if and only if $\eta_1$ was \emph{visible} to the
operation that generated $\eta_2$ during its execution, thus 
effecting its return value ($\rval(\eta_2)$). For example, a read
operation over \C{X} may pick the value ($\rval$) of the write effect
with highest id among the visible effects (this is made possible by
appropriately defining $\interp{\cdot}$ in \rulelabel{E-Var},
as we show later). Thus, the value of a read depends on what write
effects it can witness. An operation can only witness the effects of
already concluded operations, which varies between executions due to
the non-deterministic order of evaluating the parallel composition
(cf. evaluation contexts ($\ectx$) for parallel composition). A more
notable source of non-determinism, however, is the \rulelabel{E-Aux}
rule, which allows the machine to expose an arbitrary subset ($S$) of
existing effects ($\A$) to the incoming operation. In other words, the
machine is not obligated to reveal the effects of all previous
operations to an incoming operation. This relaxation allows the
abstract machine to model the semantics of weakly consistent data
stores. For instance, operations issued to an eventually consistent
(EC) replicated store could be dispatched to different replicas whose
states may not be in any well-defined relationship. By allowing
operations to witness arbitrary subsets of the global state, the
abstract machine effectively models an EC store.

However, a machine that lets operations witness arbitrary subset of
the global state offers no isolation whatsoever. For example, it lets
a read operation witness writes of an uncommitted transaction,
violating RC isolation. Fortunately, our ability to express isolation
semantics as constraints over happens-before order through $\visZ$ and
$\soZ$ relations, and the property of the abstract machine to be
parametric over the trace invariant ($\I$) lets us solve this problem.
In particular, we continue defining isolation semantics as constraints
over $\visZ$ and $\soZ$, but confine their domain of interpretation to
the given trace so that they now become trace well-formedness
constraints. Well-formedness constraints can be combined into a trace
invariant ($\I$). Since the abstract machine is guaranteed to preserve
$\I$ at every step of the execution, operations are guaranteed to
experience the level of isolation specified by $\I$.  Thus, in
executions that run to completion abstract machine models a store that
provides the required levels of isolation. Notably, the machine
achieves this without confining to any particular implementation of
the isolation, by relying on a declarative specification of isolation
as trace well-formedness constraints.  \S\ref{sec:ansi-isolation}
shows specifications of various ANSI SQL isolation levels stated as
trace well-formedness constraints.

As described previously, \rulelabel{E-Aux} abstracts away the
operation-specific behaviour of a machine step as a function ($\F$)
that accepts a set ($S$) of effects chose by the machine to make
visible to the operation, interprets the operation w.r.t $S$, and
returns an appropriate effect that encodes its return value. Rules
\rulelabel{E-Var}, \rulelabel{E-Asgn} and \rulelabel{E-Commit} define
such functions for read, write and commit operations, respectively.
The effect returned by the function in each case includes its
transaction id ($T_i$) along with an arbitrarily chosen effect id
($j$) that is later verified to be unique in \rulelabel{E-Aux}. The
$\rval$ for a write is the value being written, and for commit it is
$\bot$. In case of a read, the value read depends on the how the read
operation chooses to interpret the given set ($S$) of visible effects.
The interpretation may depend on the application semantics. For
example, a monotonically increasing counter application may choose to
let a write with largest value determine the value of a
read. A more straightforward interpretation has already been described
before. To accommodate multiple interpretations, operational semantics
is made parametric over an interpretation function ($\interp{\cdot}$)
that accepts a set of effects and a variable name, and returns the
value associated with the variable. A straightforward interpretation
function that chooses the last write (i.e., write with largest id) is
shown below:
\begin{smathpar}
\begin{array}{lcl}
  \isMax(S,\eta) & \Leftrightarrow &  \forall (\eta'\in S).  
  \kind(\eta') = \kind(\eta) \\
  & & \hspace*{0.4in}\Rightarrow \eta' = \eta \disj \id(\eta') < \id(\eta)\\

\interp{S}(X) & = & \C{if}\;(\exists (\eta \in S). \kind(\eta) = \C{WR}(X) 
  \wedge \isMax(S,\eta)) \\
  & & \C{then}\;\rval(\eta)\;\C{else}\;0\\
\end{array}
\end{smathpar}
Rules \rulelabel{E-Top-Ctx} and \rulelabel{E-Txn-Ctx} rules define
congruence rules for top-level terms and transaction-bound terms,
respectively. The rules and evaluation contexts ($\ectx$ and
$\ectx_i$) are defined such that only a certain kinds of terms are
allowed at the top-level and inside a transaction. In particular, a
\txnimp program at the top-level can either be a transaction, or a
parallel composition of transactions. A command inside a \C{txn} block
can either be an assignment, or a sequential composition of
assignments. Nested transactions and transactions that fork multiple
threads are forbidden to focus on the problems relevant to this paper.

\subsection{Isolation Specifications}
\label{sec:ansi-isolation}

\subsection{Store Consistency}
\label{sec:store-consistency}
