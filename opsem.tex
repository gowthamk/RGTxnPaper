\section{\txnimp: Syntax and Semantics}
\label{sec:opsem}

\label{sec:syntax}

%\renewcommand{\ctxn}[3]{\C{TXN}_{#1}\langle #2 \rangle\{#3\}}
\begin{figure*}[!ht]
\raggedright
%
\textbf{Syntax}\\
%
\begin{smathpar}
\renewcommand{\arraystretch}{1.2}
\begin{array}{lclcl}
\multicolumn{5}{c} {
  {x,y} \in \mathtt{Variables}\qquad
  {f} \in \mathtt{Field\;Names} \qquad
% {\tau} \in \mathtt{Table\;Names}
  {i,j} \in \mathbb{N} \qquad
  {\odot} \in \{+,-,\le,\ge,=\}\qquad
  {k} \in \mathbb{Z}\cup\mathbb{B} \qquad
  {\rec} \in \{\bar{f}=\bar{k}\}\
}\\
{\stl,\stg,s} & \in & \mathtt{State} & \coloneqq &  \Pow{\{\bar{f}=\bar{k}\}} \\
{\I_e, \I_c }  & \in & \mathtt{Isolation Spec} & \coloneqq & (\stl,\stg,\stg') \rightarrow \Prop\\
v & \in & \mathtt{Values} & \coloneqq & k \ALT \rec \ALT s\\
e & \in & \mathtt{Expressions} & \coloneqq & k \ALT x \ALT x.f 
    \ALT \{\bar{f}=\bar{e}\} \ALT e_1 \odot e_2\\ 
c & \in & \mathtt{Commands} & \coloneqq & \cskip \ALT \lete{x}{e}{c}
    \ALT \ite{e}{c_1}{c_2}\ALT c_1;c_2 \ALT \inserte{x}  \\
&&&&\ALT \deletee{\lambda x.e}
    \ALT \lete{y}{\selecte{\lambda x.e}}{c}
    \ALT \updatee{\lambda x.e_1}{\lambda x.e_2}\\
&&&&\ALT \foreache{x}{\lambda y.\lambda z. c} 
    \ALT \foreachr{s_1}{s_2}{\lambda x.\lambda y. e}\\
&&&&\ALT \ctxn{i}{\I}{ c } \ALT \ctxnr{i}{\I,\stl,\stg}{c} \ALT c1 || c2\\
% t & \in & \mathtt{Terms} & \coloneqq & e \ALT c\\
\ectx & \in & \mathtt{Eval\;Ctx} & ::= & \bullet \ALT  
  \bullet || c_2 \ALT c_1 || \bullet \ALT \bullet;\,c_2 
  \ALT \ctxnr{i}{\I,\stl,\stg}{\bullet} \\
\end{array}
\end{smathpar}
%
\bigskip

\renewcommand{\arraystretch}{1.2}

%
\textbf{Local Reduction} \quad 
\fbox {\(\stg \vdash (\tbox{c}_i,\stl) \stepsto (\tbox{c'}_i,\stl')\)}\\
%
\begin{minipage}{2.8in}
\rulelabel{E-Insert}
\begin{smathpar}
\begin{array}{c}
\RULE
{
  r.\idf \not\in \dom(\stl \cup \stg)\\
  r' = \{r \;\C{with}\; \txnf=i;\,\delf=\C{false}\}
}
{
  \stg \vdash (\tbox{\inserte{r}}_i,\stl) \stepsto
  (\tbox{\cskip}_i,\stl \cup \{r'\})
}
\end{array}
\end{smathpar}
\end{minipage}
%
%
\begin{minipage}{2.8in}
\rulelabel{E-Delete}
\begin{smathpar}
\begin{array}{c}
\RULE
{
  \hspace*{-1in}
  s = \{r' \,|\, \exists(r\in\Delta).~ \eval([r/x]e)=\C{true} \\
      \hspace*{0.1in}\conj r'=\{r \with \delf=\C{true};\, \txnf=i\}\}\spc
  \dom(\stl)\cap\dom(s) = \emptyset \spc
% \dom(s) \cap \dom(\delta) = \emptyset
}
{
  \stg \vdash (\tbox{\deletee{\lambda x.e}}_i,\stl) \stepsto 
  (\tbox{\cskip}_i,\stl \cup s)
}
\end{array}
\end{smathpar}
\end{minipage}
%
\bigskip

%
\begin{minipage}{2.8in}
\rulelabel{E-Select}
\begin{smathpar}
\begin{array}{c}
\RULE
{
  \\
  s = \{r\in\Delta \,|\, \eval([r/x]e)=\C{true}\}\spc
  c' = [s/y]c
}
{
  \stg \vdash (\tbox{\lete{y}{\selecte{\lambda x.e}}{c}}_i, \stl) \stepsto 
              (\tbox{c'}_i,\stl)
}
\end{array}
\end{smathpar}
\end{minipage}
%
%
\begin{minipage}{2.8in}
\rulelabel{E-Update}
\begin{smathpar}
\begin{array}{c}
\RULE
{
  \hspace*{-0.2in}s = \{r' \,|\, \exists(r\in\Delta).~ 
    \eval([r/x]e_2)=\C{true} \conj r'= \{ [r/x]e_1 \;\C{with}\\
    \idf=r.\idf;\,\txnf=i;\,\delf = r.\delf \}\} \spc
  \dom(\stl) \cap \dom(s) = \emptyset
}
{
  \stg \vdash (\tbox{\updatee{\lambda x.e_1}{\lambda x.e_2}}_i,\stl) 
      \stepsto (\tbox{\cskip}_i,\stl \cup s)
}
\end{array}
\end{smathpar}
\end{minipage}
%

\begin{smathpar}
\begin{array}{ll}
  \rulelabel{E-Foreach1} & \stg \vdash (\tbox{\foreache{s}{\lambda y.\lambda
    z.c}}_i,\stl) \stepsto (\tbox{\foreachr{\emptyset}{s}{\lambda
    y.\lambda z. c}}_i,\stl)\\
  \rulelabel{E-Foreach2} & \stg \vdash (\tbox{\foreachr{s_1}{\{r\} \uplus s_2}
    {\lambda y.\lambda z.c}}_i,\stl) \stepsto (\tbox{[r/z][s_1/y]c;\,
    \foreachr{s_1 \cup \{r\}}{s_2}{\lambda y.\lambda z. c}}_i,\stl)\\
  \rulelabel{E-Foreach3} & \stg \vdash (\tbox{\foreachr{s}{\emptyset}
    {\lambda y.\lambda z.c}}_i,\stl) \stepsto (\tbox{\cskip}_i,\stl)\\
\end{array}
\end{smathpar}
%
\bigskip

%
\textbf{Top-Level Reduction} \quad 
\fbox {\((c,\stg) \stepsto (c',\stg')\)}\\
%
\begin{minipage}{3in}
  \rulelabel{E-Txn-Start}
  \begin{smathpar}
  \begin{array}{c}
    \RULE{}
         {(\ctxn{i}{\I}{c},\stg) \stepsto (\ctxnr{i}{\I,\emptyset,\stg}{c},\stg)}
  \end{array}
  \end{smathpar}
\end{minipage}%
\hfill
\begin{minipage}{3in}
\rulelabel{E-Txn}
\begin{smathpar}
\begin{array}{c}
\RULE
{
  \I_e\,\,(\stl,\stg,\stg')\spc
  \stg \vdash (\tbox{c}_i,\stl) \stepsto (\tbox{c'}_i,\stl')
}
{
  (\ctxnr{i}{\I,\stl,\stg}{c},\stg') \stepsto
  (\ctxnr{i}{\I,\stl',\stg'}{c'},\stg')
}
\end{array}
\end{smathpar}
\end{minipage}\\

\begin{center}
\begin{minipage}{3in}
\rulelabel{E-Commit}
\begin{smathpar}
\begin{array}{c}
\RULE
{
  \I_c\,\,(\stl,\stg,\stg')
}
{
  (\ctxnr{i}{\I,\stl,\stg}{\cskip},\stg') \stepsto (\cskip,\stl\gg\stg')
}
\end{array}
\end{smathpar}
\end{minipage}
\end{center}
\hfill
%
\caption{\small \txnimp: Syntax and Small-step semantics}
\label{fig:txnimp}
\end{figure*}



Fig.~\ref{fig:txnimp} shows the syntax and small-step semantics of
\txnimp, a core language that we use to formalize our intuitions about
reasoning under weak isolation. Variables ($x$), Integer and boolean
constants ($k$), records ($r$) of named constants, sets ($s$) of such
records, arithmetic and boolean expressions ($e_1 \odot e_2$), and
record expressions ($\{\bar{f}=\bar{e}\}$) constitute the syntactic
class of expressions ($e$). Commands ($c$) include $\cskip$,
conditional statements, \C{LET} constructs to bind names, \C{FOREACH}
loops, SQL statements, their sequential composition ($c_1;c_2$), and
transactions ($\ctxn{i}{\I}{c}$) and their parallel composition
($c_1\,||\,c_2$). Certain terms that only appear at run-time are also
present in $c$. These include a \C{TXN} block tagged with sets ($\stl$
and $\stg$) of records that assume special meaning in operational
semantics, and a \C{FOREACH} loop that keeps track of the set ($s_1$)
of items already iterated, and the set ($s_2$) of items yet to be
iterated. Note that even the surface-level \C{FOREACH} shown here is a
little different from the one used in previous sections; its
higher-order argument has two arguments, $y$ and $z$, which are
invoked (during the reduction) with the set of already-iterated items,
and the current item, respectively. This form of \C{FOREACH} lends
itself to inductive reasoning, facilitating inductive proofs
(Sec.~\ref{sec:reasoning}).

% also We let $T_i$ for $i \in \mathbb{N}$
% range over transaction identifiers. When it is evident we are
% referring to a transaction, we use the number $i$ instead of $T_i$ for
% identification (\eg in $\C{txn}\langle i \rangle$). Like variables,
% transaction identifiers are globally accessible. For notational
% convenience, we let $t$ range over both expressions and commands.

We define a small-step operational semantics for this language in
terms of an abstract machine that executes a command, and updates
either a transaction-local database ($\stl$), or the global database
($\stg$). Database ($\stg$) is a modeled as a set of records of a
pre-defined type, i.e., they all belong to a single table.
Generalization to multiple tables is straightforward. Records in
$\stg$ are uniquely identifiable through their $\idf$ field, which is
auto-generated and does not belong to the surface language, i.e.,
$\C{id}\notin f$. For a set $S$ of records, we define $\dom(S)$ as the
set of unique ids of all the records in $S$. Thus $|\dom(\stg)| =
|\stg|$. During its execution, a transaction may write to multiple
records in $\stg$. Atomicity dictates that such writes shouldn't be
available in $\stg$ until the transaction commits. We therefore
associate each transaction with a local database ($\stl$) that stores
\emph{only} the uncommitted records\footnote{While SQL's \C{UPDATE}
admits writes at the granularity of record fields, databases, in
reality, enforce record-level locking, allowing us to think of
``uncommitted writes'' as ``uncommitted records''. }. Uncommitted
records include deleted records, which are identified by a hidden
$\delf$ field set to \C{true}. When the transaction commits, its local
database is atomically \emph{flushed} to the global database,
committing the uncommitted records. The flush operation ($\gg$) is defined as
following:
\begin{smathpar}
\begin{array}{c}
\forall r.~ r \in (\stl\gg\stg) ~\Leftrightarrow~ 
  (r.\idf \notin \dom(\stl) \conj r \in \stg)
\disj (r \in \stl \conj \neg r.\delf) 
\end{array}
\end{smathpar}
Let $\stg' = \stl\gg\stg$. A record $r$ belongs to $\stg'$ iff it
belongs to $\stg$ and not been updated in $\stl$, i.e., $r.\idf \notin
\dom(\stl)$, or it belongs to $\stl$, i.e., it is either a new record,
or an updated version of an old record, but the updatation is not a
deletion ($\neg r.\delf$). Thus, flush defines the result of
atomically applying transaction's local writes to the global database.
Besides the commit, flush also helps a transaction read its own
writes. Intuitively, the result of a read operation inside a
transaction must be computed on the database resulting from flushing
the current local state ($\stl$) to the global state ($\stg$). The
abstract machine of Fig.~\ref{fig:txnimp}, however, doesn't let a
transaction read its own writes. This reduces verbosity and simplifies
semantics, without losing any generality; substituting $\stl\gg\stg$
for $\stg$ at select places in reduction rules recovers the real

Small-step semantics is stratified into a transaction-local reduction
relation, and a top-level reduction relation. Transaction-local
relation ($\stg \vdash (c,\stl) \stepsto (c',\stl')$) defines a
small-step reduction for a command inside a transaction, when the
database state is $\stg$; the command $c$ reduces to $c'$, while
updating the transaction-local database $\stl$ to $\stl'$. The
definition assumes a meta function $\eval$ that evaluates expressions
with no free variables to values. The reduction relation for SQL
statements is defined straightforwardly.  \C{INSERT} adds a new record
to $\stl$ after adding a unique identifier (more discussion on
uniqueness later). \C{DELETE} finds the records in $\stg$ that match
the search criteria defined by its boolean function argument, and adds
the records to $\stl$ after marking them for deletion. \C{SELECT}
bounds the name introduced by \C{LET} to the set of records from
$\stg$ that match the search criteria, and then executes the bound
command $c$. \C{UPDATE} uses its first function argument to compute
the updated version of the records that match the search criteria
defined by its second function argument. Updated records are added to
$\stl$. 

The reduction of \C{FOREACH} starts by first converting it to its
run-time form that also keeps track of the iterated items ($s_1$),
besides the yet-to-be-iterated items ($s_2$). Initially, $s_1$ is
empty. As the elements are iterated, they are removed from $s_2$ and
added to $s_1$ . Iteration involves invoking the higher-order function
with $s_1$ and the current element $x$ (note: $\uplus$ in $\{x\}
\uplus s_2$ denotes a disjoint union). The reduction ends when the
$s_2$ becomes empty. The reduction rules for conditionals, \C{LET}
binders, and sequences are straightforward, hence ommitted.

The top-level reduction relation


% Fundamental to our development is the notion of a trace invariant
% ($\I$). $\I$ is a function from traces ($\E$) to first-order logical
% formulas ($\Prop$) that define well-formedness
% constraints over traces.  The machine takes a step only if the
% resulting trace satisfies the constraints imposed by $\I$. This
% behaviour is captured by the auxiliary reduction rule
% \rulelabel{E-Aux} that factors out the trace extension aspect of the
% evaluation by abstracting away the operation-specific behaviour as a
% function that generates an appropriate effect. We let $\mathcal{F}$
% denote this function.  \rulelabel{E-Aux} uses $\mathcal{F}$ to
% generate a new effect and extend the trace ($\E = (\A,\visZ)$)
% \emph{only if} the well-formedness constraints imposed by $\I$ on $\E$
% (i.e., $\I(\E')$) are satisfied. Otherwise, it gets stuck. In an
% execution that runs to completion, every small-step preserves the
% well-formedness of a trace, thus ensuring the invariance of $\I$.
% Note that the semantics makes no assumptions about $\I$ other than its
% type. As such, it can be instantiated with any trace-parametric
% proposition that expresses constraints over the given trace. For
% instance, consider the $\psi_{RC}$ specification from
% \S\ref{sec:motivation}, but with bounded $T_1$ and $T_2$ instantiated
% with \C{Wd1} and \C{Wd2}, respectively. The instantiated specification
% is the following term:\vspace*{-8pt}

% \begin{smathpar}
% \begin{array}{l}
%   \forall \eta_1,\eta_2.\; \txn(\eta_1) = \C{Wd1}
%   \conj \txn(\eta_2) = \C{Wd2} \\
%   \hspace*{0.6in}\conj \C{Wd1} \neq \C{Wd2} \conj \eta_1 \hboar
%   \eta_2 \Rightarrow \C{Wd1} \hboar \eta_2 \\
% \end{array}
% \end{smathpar}

% \noindent It is easy to interpret the above specification in the
% context of a trace $\E$ that captures an execution of the program in
% Fig.~\ref{fig:motiv-eg-1}. Such a trace-parametric formula can be
% used to instantiate the trace invariant $\I$ in Fig.~\ref{fig:txnimp}.
% The resultant operational semantics describes an abstract machine that
% gets stuck if an operation of \C{Wd2} is executed in a state that
% incorporates some, but not all the effects (including the \C{COMMIT})
% of \C{Wd1}.
% % While the machine takes a step only if the constraints are
% % satisfied, it neither defines nor explicitly assumes an oracle to
% % check satisfaction.

% As described in \S\ref{sec:motivation}, the semantics of various
% isolation levels can be captured as constraints over the
% happens-before ($\hbZ$) relation. $\hbZ$ is however a derived relation
% in our model, composed of more fundamental \emph{session order}
% ($\soZ$) and \emph{visibility} ($\visZ$) relations. In particular,
% $\hbZ \,=\, (\soZ \cup \visZ)^{+}$. Unlike $\hbZ$, $\visZ$ (defined
% below) is not a transitive relation, and hence lets us capture
% finer-grained isolation properties than $\hbZ$, which we leverage in
% our development.  The session order relation captures the sequential
% order of operations within a transaction. In particular, it relates
% two effects, $\eta_1$ and $\eta_2$, such that $\txn(\eta_1) =
% \txn(\eta_2)$ and $\id(\eta_1) < \id(\eta_2)$.  The semantics assigns
% monotonically increasing identifiers to effects, as defined by the
% $\id(\eta) > \maxId(\A)$ condition of \rulelabel{E-Aux} ($\maxId(\A)$
% returns the maximum number identifying an effect in $\A$). Evaluation
% contexts ($\ectx_i$) for transaction-bound terms are defined so as to
% enforce a deterministic sequential order of execution within a
% transaction, leading to a deterministic total order among effect ids,
% which defines the session order relation. Visibility ($\visZ$) on the
% other hand relates effects across concurrent transactions.
% Intuitively, $\visZ$ relates $\eta_1$ to $\eta_2$ if and only if
% $\eta_1$ was \emph{visible} to the operation that generated $\eta_2$
% during its execution, thus effecting its return value
% ($\rval(\eta_2)$). For example, a read operation over \C{X} may pick
% the value ($\rval$) of the write effect with the highest id among its
% visible effects (this is made possible by appropriately defining
% $\interp{\cdot}$ in \rulelabel{E-Var}, as we show later). Thus, the
% value of a read depends on what write effects it can witness. An
% operation can only witness the effects of already concluded
% operations, which varies between executions due to the
% non-deterministic order of evaluating the parallel composition of
% transactions.

% A more notable source of non-determinism, however, is the
% \rulelabel{E-Aux} rule, which allows the machine to expose an
% arbitrary subset ($S$) of existing effects ($\A$) to the incoming
% operation. In other words, the machine is not obligated to reveal the
% effects of all previous operations to an incoming operation. This
% relaxation allows the abstract machine to model the semantics of
% weakly-consistent data stores. For instance, operations issued to an
% eventually consistent ({\sc ec}) replicated store could be dispatched
% to different replicas whose states may not be in any well-defined
% relationship. By allowing operations to witness arbitrary subsets of
% the global state, the semantics models the weak visibility properties
% of such stores; we elaborate on the implication of this style of
% definition in \S\ref{sec:store-consistency}.  Stronger visibility
% properties can be expressed by imposing well-formedness constraints
% over $\visZ$ via the trace invariant ($\I$). Since the abstract
% machine is obligated to satisfy $\I$ at every step of the execution,
% operations are guaranteed to experience the level of isolation
% specified by $\I$.  Thus, in executions that run to completion, the
% abstract machine models a store that provides the required levels of
% isolation.  Notably, the machine achieves this without defining an
% operational semantics for isolation levels, instead solely relying on
% their declarative characterization as trace well-formedness
% constraints to enforce isolation guarantees.
% \S\ref{sec:ansi-isolation} specifies various ANSI SQL isolation levels
% stated as trace well-formedness constraints.

% % TRACE RELATIONS
% % ---------------
% \begin{figure*}[t!]
\begin{tabular}{l|l}
\begin{minipage}{\columnwidth}
\begin{smathpar}
\begin{array}{lcl}
(\A,\visZ) \Vdash \eta \in T_i & \defeq & \eta \in \A \conj \txn(\eta) = T_i\\
\E \Vdash S \subseteq T_i & \defeq & \forall \eta.~ \eta
        \in S \Rightarrow \underE{\eta \in T_i} \\
(\A,\visZ) \Vdash \eta_1 \visar \eta_2 & \defeq & \{\eta_1,\eta_2\}
        \subseteq \A \conj (\eta_1,\eta_2) \in \visZ\\
(\A,\visZ) \Vdash \eta_1 \soar \eta_2 & \defeq & \{\eta_1,\eta_2\}
        \subseteq \A \conj \txn(\eta_1)=\txn(\eta_2) \\
        &   & \hspace*{0.3in} \conj \id(\eta_1) < \id(\eta_2)\\
(\A,\visZ) \Vdash \eta_1 \hboar \eta_2 & \defeq & \{\eta_1,\eta_2\}
        \subseteq \A \conj (\eta_1,\eta_2) \in (\visZ \cup \soZ)^{+}\\
\E \Vdash T_i \visar \eta & \defeq &\forall\eta_1
        .\,(\E \Vdash \eta_1 \in T_i) \Rightarrow \E \Vdash \eta_1 \visar \eta \\
\end{array}
\end{smathpar}
\end{minipage}
&
\begin{minipage}{\columnwidth}
\begin{smathpar}
\begin{array}{lcl}
\underE{T_i \visar T_j} & \defeq &  \forall\eta_1,\eta_2.\,
    %\sameobj{\eta_1}{\eta_2}  \Rightarrow 
    \underE{\eta_1\in T_i} \conj \underE{\eta_2 \in T_j} \\
    &   & \hspace*{1in}\Rightarrow \underE{\eta_1 \visar \eta_2} \\
\underE{T_i \invisar T_j} & \defeq &  \forall\eta_1,\eta_2.\,
        %\sameobj{\eta_1}{\eta_2}\Rightarrow 
        \underE{\eta_1\in T_i} \conj \underE{\eta_2 \in T_j}\\
    &   & \hspace*{1in} \Rightarrow \neg (\underE{\eta_1 \visar \eta_2})\\
\underE{T_i \wrstoar X} & \defeq & \exists\eta.~
        \underE{\eta \in T_i} \conj \kind(\eta) = \C{WR}(X)\\
\underE{T_i \rdsfmar X} & \defeq & \exists\eta.~
        \underE{\eta \in T_i} \conj \kind(\eta) = \C{RD}(X)\\
\underE{T_i \usesar X} & \defeq & \underE{T_i \wrstoar X} \disj
      \underE{T_i \rdsfmar X}\\
\end{array}
\end{smathpar}
\end{minipage}
\\
\end{tabular}

\caption{Relations defined over a trace}
\label{fig:rel-defs}
\end{figure*}




% % However, a machine that lets operations witness arbitrary subset of
% % the global state offers no isolation whatsoever. For example, it may
% % allow a read operation to witness writes of an uncommitted
% % transaction, violating RC isolation. Fortunately, our ability to
% % express an isolation semantics as constraints over happens-before
% % order through $\visZ$ and $\soZ$ relations, and the property of the
% % abstract machine to be parametric over the trace invariant ($\I$),
% % lets us solve this problem.
% % In particular, we continue to define isolation semantics as
% % constraints over $\visZ$ and $\soZ$, but confine their domain of
% % interpretation to the given trace so that they now become trace
% % well-formedness constraints. Well-formedness constraints can be
% % combined into a trace invariant ($\I$). 


% As described previously, \rulelabel{E-Aux} abstracts away the
% operation-specific behaviour of a machine step as a function ($\F$)
% that accepts a set ($S$) of effects chosen by the machine to make
% visible to the operation, interprets the operation w.r.t. $S$, and
% returns an appropriate effect that encodes its return value. Rules
% \rulelabel{E-Var}, \rulelabel{E-Asgn} and \rulelabel{E-Commit} define
% such functions for read, write and commit operations, respectively.
% The effect returned by the function in each case includes its
% transaction id ($T_i$) along with an arbitrarily chosen effect id
% ($j$) that is later verified to be unique in \rulelabel{E-Aux}. The
% $\rval$ for a write is the value being written, and for commit it is
% $\bot$. In case of a read, the value read depends on how the read
% operation chooses to interpret the given set ($S$) of visible effects.
% The interpretation may depend on application semantics. For example, a
% monotonically increasing counter application may choose to let a write
% with the largest value determine the value of a read. To accommodate
% multiple interpretations, the semantics is made parametric over an
% interpretation function ($\interp{\cdot}$) that accepts a set of
% effects and a variable name, and returns the value associated with the
% variable. A straightforward interpretation function that chooses the
% last write (i.e., write with largest id) is shown below:

% \begin{smathpar}
% \begin{array}{lcl}
%   \isMax(S,\eta) & \Leftrightarrow &  \forall (\eta'\in S).  
%   \kind(\eta') = \kind(\eta) \\
%   & & \hspace*{0.4in}\Rightarrow \eta' = \eta \disj \id(\eta') < \id(\eta)\\

% \interp{S}(X) & = & \C{if}\;(\exists (\eta \in S). \kind(\eta) = \C{WR}(X) 
%   \wedge \isMax(S,\eta)) \\
%   & & \C{then}\;\rval(\eta)\;\C{else}\;0\\
% \end{array}
% \end{smathpar}

% \noindent Rules \rulelabel{E-Top-Ctx} and \rulelabel{E-Txn-Ctx} define
% congruence properties for top-level terms and transaction-bound terms,
% respectively. The rules and evaluation contexts ($\ectx$ and
% $\ectx_i$) are defined such that only certain kinds of terms are
% allowed at the top-level and inside a transaction. In particular, a
% \txnimp program at the top-level can either be a transaction, or a
% parallel composition of transactions. A command inside a \C{txn} block
% can either be an assignment, or a sequential composition of
% assignments. 

% \subsection{Isolation Specifications}
\label{sec:ansi-isolation}

\begin{figure*}[t]
\begin{smathpar}
\begin{array}{lcl}
(\A,\visZ) \Vdash \eta \in T_i & \Leftrightarrow & \eta \in \A \conj \txn(\eta) = T_i\\
(\A,\visZ) \Vdash \eta_1 \visar \eta_2 & \Leftrightarrow & \{\eta_1,\eta_2\}
        \subseteq \A \conj (\eta_1,\eta_2) \in \visZ\\
(\A,\visZ) \Vdash \eta_1 \soar \eta_2 & \Leftrightarrow & \{\eta_1,\eta_2\}
        \subseteq \A \conj \txn(\eta_1)=\txn(\eta_2) \conj \id(\eta_1)
        < \id(\eta_2)\\
\E \Vdash S \subseteq T_i & \Leftrightarrow & \forall \eta.~ \eta
        \in S \Rightarrow \underE{\eta \in T_i} \\
\E \Vdash T_i \visar \eta & \Leftrightarrow &\forall\eta_1
        .\,(\E \Vdash \eta_1 \in T_i) \Rightarrow \E \Vdash \eta_1 \visar \eta \\
\underE{T_i \visar T_j} & \Leftrightarrow &  \forall\eta_1,\eta_2.\,
    %\sameobj{\eta_1}{\eta_2}  \Rightarrow 
    \underE{\eta_1\in T_i} \conj \underE{\eta_2 \in T_j} \Rightarrow 
    \underE{\eta_1 \visar \eta_2} \\
\underE{T_i \invisar T_j} & \Leftrightarrow &  \forall\eta_1,\eta_2.\,
        %\sameobj{\eta_1}{\eta_2}\Rightarrow 
        \underE{\eta_1\in T_i} \conj \underE{\eta_2 \in T_j} \Rightarrow 
        \neg (\underE{\eta_1 \visar \eta_2})\\
\underE{T_i \wrstoar X} & \Leftrightarrow & \exists\eta.~
        \underE{\eta \in T_i} \conj \kind(\eta) = \C{WR}(X)\\
\underE{\C{RMWVis}(T_j)} & \Leftrightarrow & \forall\eta_1,\eta_2.\,
       \underE{\{\eta_1,\eta_2\} \subseteq T_j} \conj
       \underE{\eta_1 \soar \eta_2} \Rightarrow \underE{\eta_1 \visar
       \eta_2}\\
\underE{\C{MonotonicVis}(T_j)} & \Leftrightarrow & 
       \underE{\C{RMWVis}(T_j)} \conj 
       \forall\eta,\eta_1,\eta_2.\, \underE{\{\eta_1,\eta_2\} \in T_j} 
          \conj \\
  &   & \hspace*{1.2in}\underE{\eta \visar \eta_1} \conj
        \underE{\eta_1 \soar \eta_2} \Rightarrow \underE{\eta \visar
        \eta_2} \\
%  \C{CausalVis}(T_i) & \Leftrightarrow & 
%         \C{MonotonicVis}(T_i) \conj \C{RMWVis}(T_i)\\
\underE{\C{AtomicVis}(T_j)} & \Leftrightarrow & 
       \forall\eta_1,\eta_2.\, \neg(\underE{\eta_1 \in T_j}) \conj
       \underE{\eta_2 \in T_j} \conj
       \underE{\eta_1 \visar \eta_2} \Rightarrow \underE{\txn(\eta_1)
       \visar \eta_2}\\
\underE{\C{CommitVis}(T_j)} & \Leftrightarrow & 
       \forall\eta_1,\eta_2.~ \neg(\underE{\eta_1 \in T_j}) \conj 
          \underE{\eta_2 \in T_j} \conj
       \underE{\eta_1 \visar \eta_2} \Rightarrow\\
  &   & \hspace*{1.2in}\exists\eta.\, \underE{\eta \in \txn(\eta_1)} 
        \conj \kind(\eta) = \C{COMMIT} 
        \conj \underE{\eta \visar \eta_2}\\
% \underE{\C{TransVis}(T_j)} & \Leftrightarrow &  \forall
%        \eta_1,\eta_2,\eta_3.\, \underE{\eta_3 \in T_j} \conj
%        \underE{\eta_1 \visar \eta_2} \conj
%        \underE{\eta_2 \visar \eta_3} \Rightarrow \underE{\eta_1 \visar
%        \eta_3} \\
\underE{\C{SnapshotVis}(T_i,T_j)} & \Leftrightarrow &  \underE{T_i
       \visar T_j} \disj \underE{T_i \invisar T_j}\\
\underE{\C{OneWaySER}(T_i,T_j)} & \Leftrightarrow &  \underE{T_i
       \visar T_j} \disj (\underE{T_i \invisar T_j} \conj \\
  &   &\hspace*{1.2in}\exists \eta.~\underE{\eta\in T_i} \conj \kind(\eta) =
       \C{COMMIT} \Rightarrow \underE{T_j \visar \eta})\\
\underE{\C{RC}(T_j)} & \Leftrightarrow & \underE{\C{AtomicVis}(T_j)} 
        \conj \underE{\C{CommitVis}(T_j)}\\
\underE{\C{MAV}(T_j)} & \Leftrightarrow & \underE{\C{RC}(T_j)} \conj
      \underE{\C{MonotonicVis}(T_j)} \\
\underE{\C{RR}(T_j)} & \Leftrightarrow & \underE{\C{MAV}(T_j)}
       \conj \forall T_i.\,T_i \neq T_j \Rightarrow 
        \underE{\C{SnapshotVis}(T_i,T_j)} \\
% &   & \hspace*{2in}\conj  \C{SnapshotVis}(T_i,T_j)\\
\underE{\C{SI}(T_j)} & \Leftrightarrow &  \underE{\C{RR}(T_j)}
       \conj \forall T_i.\,(T_i \neq T_j \conj \exists X.\, \underE{T_i \wrstoar X} \conj 
        \underE{T_j \wrstoar X})\\
%      (\exists X.\, T_i \wrstoar X \conj T_j \wrstoar X)
%      \Rightarrow  \C{TotalVis}(T_i,T_j)\\
  &   & \hspace*{2in} \Rightarrow \underE{\C{OneWaySER}(T_i,T_j)}\\
\underE{\C{SER}(T_j)} & \Leftrightarrow & \underE{\C{MAV}(T_j)}
       \conj \forall T_i.\,T_i \neq T_j 
       \Rightarrow \underE{\C{OneWaySER}(T_i,T_j)}\\
% &   & \hspace*{2in}\conj  \C{TotalVis}(T_i,T_j)\\
\end{array}
\end{smathpar}

\caption{Standard isolation guarantees expressed as trace
well-formedness constraints}
\label{fig:ansi-isolation}
\end{figure*}

Fig.~\ref{fig:ansi-isolation} shows the specification of standard
isolation guarantees expressed as constraints over trace
well-formedness. For brevity and convenience, we adopt some notation.
An execution trace is destructed into $\A$ and $\visZ$ whenever
individual components of the pair are needed.  Otherwise, it is written
as $\E$. When $\psi$ is a proposition, we write $\underE{\psi}$ to denote
an interpretation of $\psi$ in the context of the trace $\E$. Such
interpretations are defined on a case-by-case basis in
Fig.~\ref{fig:ansi-isolation}. In the following, we give informal
explanations for each definition.

In the context of a trace $(\A,\visZ)$, an effect $\eta$ is said to
belong to a transaction $T_i$ if $\eta$ belongs to the effect set $A$
and its transaction identifier is $T_i$. The containment relation is
trivially lifted to the set of effects to define $\underE{S \subseteq
T_i}$.  The interpretation of $\eta_1 \visar \eta_2$ and $\eta_1
\soar \eta_2$ are straightforward and explained in \S\ref{sec:syntax}.
A transaction $T_i$ is said to be visible to an effect $\eta$ if every
effect $\eta_1$ of $T_i$ recorded by the trace is visible to $\eta$.
$T_i$ may be visible to $\eta$ but may not be visible to every other
effect in the transaction. For a transaction $T_i$ to be considered to
be visible to a transaction $T_j$ in the context of a trace $\E$
(written $\underE{T_i \visar T_j}$), every effect ($\eta_1$) of $T_i$
present in $\E$ must be visible to every effect ($\eta_2$) of $T_j$ in
$\E$. Conversely, if none of the effects of $T_i$ present in $\E$ are
visible to any effect of $T_j$, then $T_i$ is considered invisible to
$T_j$ under $\E$ (written $\underE{T_i \invisar T_j}$). Transaction
$T_i$ is said to have written to a variable $X$ under $\E$ (i.e.,
$\underE{T_i \wrstoar X}$) if there exists a write-to-$X$ effect from
$T_i$ in $\E$.

Various isolation guarantees are defined as propositions indexed by
a transaction identifier $T_j$. Transaction $T_j$ is said to have
experienced \emph{read-my-writes} visibility under $\E$ if every
effect ($\eta_1$) of $T_j$ is visible to every subsequent effect
($\eta_2$) of the same transaction in $\E$. This lets $T_j$ to never
lose its own updates. Monotonic visibility adds one more constraint to
read-my-writes; besides requiring $\eta_1$ to be visible to $\eta_2$,
it also requires every effect ($\eta$) visible to $\eta_1$ in $\E$ to
be visible to $\eta_2$ as well. Thus $T_j$ witnesses monotonically
increasing state as time progresses. Atomic visibility allows an
effect $\eta_2$ of $T_j$ to witness an effect $\eta_1$ of $T_i$ only
if all effects of $T_i$ in $\E$ are also visible to $\eta_2$. Atomic
visibility thus prevents a transaction from being partially visible.
However, atomic visiblity does not prevent an uncommitted transaction
from being visible. To confine visibility to only committed
transactions, we need an additional constraint that requires an effect
($\eta_2$) of $T_j$ to also witness the commit effect of $T_i$
whenever it witnesses some effect ($\eta_1$) of $T_i$. This additional
constraint is captured by the $\mathtt{CommitVis}$ specification. A
transaction $T_i$ is said to be snapshot-visible to a transaction
$T_j$ if either it is visible to $T_j$, or it is invisible; it is
forbidden for only a suffix (more generally, a subset) of $T_j$ to
witness $T_i$. Observe that snapshot visibility permits $T_i$ and $T_j$ to
execute and commit while being  oblivious of each other. One-way
serializability proscribes this possiblity. $T_i$
is said to be one-way-serializable w.r.t $T_j$ if either it is visible
to $T_j$ or $T_j$ is visible to the commit effect of $T_i$.  Note that
both $\mathtt{SnapshotVis}$ and $\mathtt{OneWaySER}$ are asymmetric
definitions that guarantee complete visiblity or invisibility of $T_i$
to $T_j$, but not the converse. While $\mathtt{SnapshotVis}$ provides
no guarantees to $T_i$, $\mathtt{OneWaySER}$ guarantees that at least
the commit effect of $T_i$ witnesses $T_j$.

The ANSI SQL 92 standard requires \iso{Read Committted} isolation to avoid
the dirty reads phenomenon, which is achieved by enforcing
$\mathtt{AtomicVis}$ and $\mathtt{CommitVis}$ guarantees. The {\sc rc}
specification is therefore a combination of these two guarantees. The
specification also agrees with the description and
implementation~\cite{bailishat,pldi15} of {\sc rc} for highly available
replicated stores. On relational databases, however, {\sc rc} has also come
to be associated with the $\mathtt{MonotonicVis}$ guarantee.
Nonetheless, $\mathtt{AtomicVis}$ and $\mathtt{CommitVis}$ are
sufficient to reason about {\sc rc} isolation on relational stores too. As
we demonstrate in \S\ref{sec:store-consistency}, the combination of
these guarantees with the {\sc sc} property of relational stores
automatically leads to the monotonicity guarantee, which probably
explains why {\sc rc} comes with $\mathtt{MonotonicVis}$ on such stores. On
weakly consistent stores however, $\mathtt{AtomicVis}$ and
$\mathtt{CommitVis}$ do not imply $\mathtt{MonotonicVis}$. A stronger
isolation level called \iso{Monotonic Atomic View} ({\sc mav} of
Fig.~\ref{fig:ansi-isolation})~\cite{bailishat,pldi15} was proposed to
explicitly extend {\sc rc} with monotonicity on such stores. \iso{Repeatable
Read} spec ({\sc rr}) extends {\sc mav} with the snapshot visibility guarantee
as described before. \iso{Snapshot Isolation} spec ({\sc si}) extends
{\sc rr} with a one-way serializability guarantee w.r.t the transactions
that perform conflicting writes (i.e., writes to the same shared
variable). Finally, \iso{Serializable} isolation extends one-way
$\mathtt{SnapshotVis}$ to all transactions, including those that do
not conflict\footnote{Fig.~\ref{fig:ansi-isolation} presents slightly
weaker versions of {\sc si} and {\sc ser} specs in the interest of
clarity.}. 

In contrast to recent proposals (\emph{e.g.},
~\cite{gotsmanconcur15}), our specifications for {\sc si} and {\sc
  ser} do not necessarily impose a total order among transactions
(conflicting or otherwise). In reality, a total order under {\sc ser}
(resp. {\sc si}) is guaranteed only if the store executes all
transactions under {\sc ser} (resp. {\sc SI}) isolation. Our
specifications admit this possibility, and derive a total order under
the assumption of homogenity. However, databases almost always allow
isolation levels to be configured on a per-transaction basis, allowing
transactions at various isolation levels to coexist.  Specifications
that assume homogenity are incorrect under this setting.

% Having specified isolation levels as trace well-formedness
% constraints, we can now construct trace invariants ($\I$) for
% \txnimp programs by composing isolation specifications for various
% transactions. For instance, the following trace invariant enforces
% {\sc si} for both transactions of the program in
% Fig.~\ref{fig:motiv-eg-1}, allowing it to satisfy its postcondition:
% \begin{smathpar}
% \I \;=\; \lambda\E.~ \underE{\C{SI(Wd1)}} \conj \underE{\C{SI(Wd2)}}
% \end{smathpar}
% In contrast, the following trace invariant enforces {\sc rc} for one and {\sc si}
% for another, leading to a possible violation of the postcondition:
% \begin{smathpar}
% \I \;=\; \lambda\E.~ \underE{\C{RC(Wd1)}} \conj \underE{\C{SI(Wd2)}}
% \end{smathpar}


