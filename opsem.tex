\section{\txnimp}
\label{sec:opsem}

%\renewcommand{\ctxn}[3]{\C{TXN}_{#1}\langle #2 \rangle\{#3\}}
\begin{figure*}[!ht]
\raggedright
%
\textbf{Syntax}\\
%
\begin{smathpar}
\renewcommand{\arraystretch}{1.2}
\begin{array}{lclcl}
\multicolumn{5}{c} {
  {x,y} \in \mathtt{Variables}\qquad
  {f} \in \mathtt{Field\;Names} \qquad
% {\tau} \in \mathtt{Table\;Names}
  {i,j} \in \mathbb{N} \qquad
  {\odot} \in \{+,-,\le,\ge,=\}\qquad
  {k} \in \mathbb{Z}\cup\mathbb{B} \qquad
  {\rec} \in \{\bar{f}=\bar{k}\}\
}\\
{\stl,\stg,s} & \in & \mathtt{State} & \coloneqq &  \Pow{\{\bar{f}=\bar{k}\}} \\
{\I_e, \I_c }  & \in & \mathtt{Isolation Spec} & \coloneqq & (\stl,\stg,\stg') \rightarrow \Prop\\
v & \in & \mathtt{Values} & \coloneqq & k \ALT \rec \ALT s\\
e & \in & \mathtt{Expressions} & \coloneqq & k \ALT x \ALT x.f 
    \ALT \{\bar{f}=\bar{e}\} \ALT e_1 \odot e_2\\ 
c & \in & \mathtt{Commands} & \coloneqq & \cskip \ALT \lete{x}{e}{c}
    \ALT \ite{e}{c_1}{c_2}\ALT c_1;c_2 \ALT \inserte{x}  \\
&&&&\ALT \deletee{\lambda x.e}
    \ALT \lete{y}{\selecte{\lambda x.e}}{c}
    \ALT \updatee{\lambda x.e_1}{\lambda x.e_2}\\
&&&&\ALT \foreache{x}{\lambda y.\lambda z. c} 
    \ALT \foreachr{s_1}{s_2}{\lambda x.\lambda y. e}\\
&&&&\ALT \ctxn{i}{\I}{ c } \ALT \ctxnr{i}{\I,\stl,\stg}{c} \ALT c1 || c2\\
% t & \in & \mathtt{Terms} & \coloneqq & e \ALT c\\
\ectx & \in & \mathtt{Eval\;Ctx} & ::= & \bullet \ALT  
  \bullet || c_2 \ALT c_1 || \bullet \ALT \bullet;\,c_2 
  \ALT \ctxnr{i}{\I,\stl,\stg}{\bullet} \\
\end{array}
\end{smathpar}
%
\bigskip

\renewcommand{\arraystretch}{1.2}

%
\textbf{Local Reduction} \quad 
\fbox {\(\stg \vdash (\tbox{c}_i,\stl) \stepsto (\tbox{c'}_i,\stl')\)}\\
%
\begin{minipage}{2.8in}
\rulelabel{E-Insert}
\begin{smathpar}
\begin{array}{c}
\RULE
{
  r.\idf \not\in \dom(\stl \cup \stg)\\
  r' = \{r \;\C{with}\; \txnf=i;\,\delf=\C{false}\}
}
{
  \stg \vdash (\tbox{\inserte{r}}_i,\stl) \stepsto
  (\tbox{\cskip}_i,\stl \cup \{r'\})
}
\end{array}
\end{smathpar}
\end{minipage}
%
%
\begin{minipage}{2.8in}
\rulelabel{E-Delete}
\begin{smathpar}
\begin{array}{c}
\RULE
{
  \hspace*{-1in}
  s = \{r' \,|\, \exists(r\in\Delta).~ \eval([r/x]e)=\C{true} \\
      \hspace*{0.1in}\conj r'=\{r \with \delf=\C{true};\, \txnf=i\}\}\spc
  \dom(\stl)\cap\dom(s) = \emptyset \spc
% \dom(s) \cap \dom(\delta) = \emptyset
}
{
  \stg \vdash (\tbox{\deletee{\lambda x.e}}_i,\stl) \stepsto 
  (\tbox{\cskip}_i,\stl \cup s)
}
\end{array}
\end{smathpar}
\end{minipage}
%
\bigskip

%
\begin{minipage}{2.8in}
\rulelabel{E-Select}
\begin{smathpar}
\begin{array}{c}
\RULE
{
  \\
  s = \{r\in\Delta \,|\, \eval([r/x]e)=\C{true}\}\spc
  c' = [s/y]c
}
{
  \stg \vdash (\tbox{\lete{y}{\selecte{\lambda x.e}}{c}}_i, \stl) \stepsto 
              (\tbox{c'}_i,\stl)
}
\end{array}
\end{smathpar}
\end{minipage}
%
%
\begin{minipage}{2.8in}
\rulelabel{E-Update}
\begin{smathpar}
\begin{array}{c}
\RULE
{
  \hspace*{-0.2in}s = \{r' \,|\, \exists(r\in\Delta).~ 
    \eval([r/x]e_2)=\C{true} \conj r'= \{ [r/x]e_1 \;\C{with}\\
    \idf=r.\idf;\,\txnf=i;\,\delf = r.\delf \}\} \spc
  \dom(\stl) \cap \dom(s) = \emptyset
}
{
  \stg \vdash (\tbox{\updatee{\lambda x.e_1}{\lambda x.e_2}}_i,\stl) 
      \stepsto (\tbox{\cskip}_i,\stl \cup s)
}
\end{array}
\end{smathpar}
\end{minipage}
%

\begin{smathpar}
\begin{array}{ll}
  \rulelabel{E-Foreach1} & \stg \vdash (\tbox{\foreache{s}{\lambda y.\lambda
    z.c}}_i,\stl) \stepsto (\tbox{\foreachr{\emptyset}{s}{\lambda
    y.\lambda z. c}}_i,\stl)\\
  \rulelabel{E-Foreach2} & \stg \vdash (\tbox{\foreachr{s_1}{\{r\} \uplus s_2}
    {\lambda y.\lambda z.c}}_i,\stl) \stepsto (\tbox{[r/z][s_1/y]c;\,
    \foreachr{s_1 \cup \{r\}}{s_2}{\lambda y.\lambda z. c}}_i,\stl)\\
  \rulelabel{E-Foreach3} & \stg \vdash (\tbox{\foreachr{s}{\emptyset}
    {\lambda y.\lambda z.c}}_i,\stl) \stepsto (\tbox{\cskip}_i,\stl)\\
\end{array}
\end{smathpar}
%
\bigskip

%
\textbf{Top-Level Reduction} \quad 
\fbox {\((c,\stg) \stepsto (c',\stg')\)}\\
%
\begin{minipage}{3in}
  \rulelabel{E-Txn-Start}
  \begin{smathpar}
  \begin{array}{c}
    \RULE{}
         {(\ctxn{i}{\I}{c},\stg) \stepsto (\ctxnr{i}{\I,\emptyset,\stg}{c},\stg)}
  \end{array}
  \end{smathpar}
\end{minipage}%
\hfill
\begin{minipage}{3in}
\rulelabel{E-Txn}
\begin{smathpar}
\begin{array}{c}
\RULE
{
  \I_e\,\,(\stl,\stg,\stg')\spc
  \stg \vdash (\tbox{c}_i,\stl) \stepsto (\tbox{c'}_i,\stl')
}
{
  (\ctxnr{i}{\I,\stl,\stg}{c},\stg') \stepsto
  (\ctxnr{i}{\I,\stl',\stg'}{c'},\stg')
}
\end{array}
\end{smathpar}
\end{minipage}\\

\begin{center}
\begin{minipage}{3in}
\rulelabel{E-Commit}
\begin{smathpar}
\begin{array}{c}
\RULE
{
  \I_c\,\,(\stl,\stg,\stg')
}
{
  (\ctxnr{i}{\I,\stl,\stg}{\cskip},\stg') \stepsto (\cskip,\stl\gg\stg')
}
\end{array}
\end{smathpar}
\end{minipage}
\end{center}
\hfill
%
\caption{\small \txnimp: Syntax and Small-step semantics}
\label{fig:txnimp}
\end{figure*}



\subsection{Syntax and Semantics}
\label{sec:syntax}

Fig.~\ref{fig:txnimp} shows the syntax and small-step semantics of
\txnimp. Natural numbers, (shared) variables and arithmetic
expressions constitute the syntactic class of expressions ($e$).
Commands ($c$) include $\cskip$, assignment statements, transaction
(\C{txn}) lexical blocks, and their sequential and parallel
composition. We let $T_i$ for $i \in \mathbb{N}$ range over
transaction identifiers. When it is evident we are referring to a
transaction, we use the number $i$ instead of $T_i$ for identification
(\eg, in $\C{txn}\langle i \rangle$). For notational convenience we let
$t$ range over both expressions and commands.

Small-step semantics of \txnimp are defined in terms of an abstract
machine that generates an execution trace ($\E$). First artifact of
the trace is a set ($\A$) of \emph{effects}, where an effect ($\eta$)
documents a read (\C{RD(X)}), or a write (\C{WR(X)}), or a transaction
commit (\C{COMMIT}) operation performed during the execution. Any
value associated with the operation (\eg, value read or value written)
is also documented, and can be accessed via $\rval$. Every effect has
a unique identifier accessible via $\id$, and a transction identifier
accessible via $\txn$.  The latter identifies the transaction that
generated the effect. In every step of the evaluation, abstract
machine reduces a \txnimp term by executing a read, write or commit
operation, generating an effect, and extending the trace. Since
effects include transaction identifiers, semantics distinguishes
between terms ($t$) of different transactions. For example,
$\txnbox{t}_i$ denotes a term $t$ inside a transaction $T_i$.
Evaluation contexts are also appropriately marked. For example,
$\ectx_i$ denotes evaluation context for a term inside $T_i$. The
other component of the execution trace is the visibility relation
($\visZ$) between the effects. The intent and mechanics of $\visZ$
become clear as we describe the evaluation rules in the following.

Fundamental to our operational semantics is the notion of a trace
invariant ($\I$), which is a function from traces ($\E$) to
propositions ($\texttt{Prop}$) that declare well-formedness
constraints over traces. Abstract machine takes a step only if the
resulting trace satisfies the well-formedness constraints imposed by
$\I$. This behaviour is captured by the auxiliary reduction rule
\rulelabel{E-Aux} that factors out the trace extension aspect of the
evaluation by abstracting away operation-specific behaviour as a
function ($\mathcal{F}$) that generates appropriate effect.
\rulelabel{E-Aux} uses $\mathcal{F}$ to generate a new effect and
extend the trace ($\E = (\A,\visZ)$) \emph{only if} the
well-formedness constraints imposed by $\I$ on $\E$ (i.e., $\I(\E')$)
are satisfied. Otherwise, it gets stuck. In an execution that runs to
completion, every small-step preserves well-formedness of trace, thus
ensuring the invariance of $\I$. Note that the operational semantics
makes no assumptions about $\I$ other than its type. As such it is
parametric over $\I$, and can be instantiated with any
trace-parametric proposition that expresses constraints over the given
trace. While the machine takes a step only if the constraints are
satisfied, it neither defines nor explicitly assumes an oracle to
check the satisfaction; there is no need for doing so in the
semantics.

As explained in \S\ref{sec:motivation}, semantics of various isolation
levels can be captured as constraints over the happens-before ($\hbZ$)
relation. $\hbZ$ is however a derived relation in our model, composed
of more fundamental \emph{session order} ($\soZ$) and
\emph{visibility} ($\visZ$) relations. Session order captures the
sequential order of operations within a transaction. In particular, it
relates two effects, $\eta_1$ and $\eta_2$, such that $\txn(\eta_1) =
\txn(\eta_2)$ and $\id(\eta_1) < \id(\eta_2)$. Small-step semantics
assigns monotonically increasing identifiers to effects, as captured
by the $\id(\eta) > \maxId(\A)$ condition of \rulelabel{E-Aux}.
Evaluation contexts ($\ectx_i$) for transaction-bound terms are
defined so as to enforce a deterministic sequential order of execution
within a transaction, leading to a deterministic total order among
effect ids - the session order. Visibility ($\visZ$) on the other
hand relates effects across concurrent transactions, thus reflects the
non-determinism inherent in concurrency. Intuitively, $\visZ$ relates
$\eta_1$ to $\eta_2$ if and only if $\eta_1$ was \emph{visible} to the
operation that generated $\eta_2$ during its execution, thus 
effecting its return value ($\rval(\eta_2)$). For example, a read
operation over \C{X} may pick the value ($\rval$) of the write effect
with highest id among the visible effects (this is made possible by
appropriately defining $\interp{\cdot}$ in \rulelabel{E-Var},
as we show later). Thus, the value of a read depends on what write
effects it can witness. An operation can only witness the effects of
already concluded operations, which varies between executions due to
the non-deterministic order of evaluating the parallel composition
(cf. evaluation contexts ($\ectx$) for parallel composition). A more
notable source of non-determinism, however, is the \rulelabel{E-Aux}
rule, which allows the machine to expose an arbitrary subset ($S$) of
existing effects ($\A$) to the incoming operation. In other words, the
machine is not obligated to reveal the effects of all previous
operations to an incoming operation. This relaxation allows the
abstract machine to model the semantics of weakly consistent data
stores. For instance, operations issued to an eventually consistent
(EC) replicated store could be dispatched to different replicas whose
states may not be in any well-defined relationship. By allowing
operations to witness arbitrary subsets of the global state, the
abstract machine effectively models an EC store.

However, a machine that lets operations witness arbitrary subset of
the global state offers no isolation whatsoever. For example, it lets
a read operation witness writes of an uncommitted transaction,
violating RC isolation. Fortunately, our ability to express isolation
semantics as constraints over happens-before order through $\visZ$ and
$\soZ$ relations, and the property of the abstract machine to be
parametric over the trace invariant ($\I$) lets us solve this problem.
In particular, we continue defining isolation semantics as constraints
over $\visZ$ and $\soZ$, but confine their domain of interpretation to
the given trace so that they now become trace well-formedness
constraints. Well-formedness constraints can be combined into a trace
invariant ($\I$). Since the abstract machine is guaranteed to preserve
$\I$ at every step of the execution, operations are guaranteed to
experience the level of isolation specified by $\I$.  Thus, in
executions that run to completion abstract machine models a store that
provides the required levels of isolation. Notably, the machine
achieves this without confining to any particular implementation of
the isolation, by relying on a declarative specification of isolation
as trace well-formedness constraints.  \S\ref{sec:ansi-isolation}
shows specifications of various ANSI SQL isolation levels stated as
trace well-formedness constraints.

As described previously, \rulelabel{E-Aux} abstracts away the
operation-specific behaviour of a machine step as a function ($\F$)
that accepts a set ($S$) of effects chose by the machine to make
visible to the operation, interprets the operation w.r.t $S$, and
returns an appropriate effect that encodes its return value. Rules
\rulelabel{E-Var}, \rulelabel{E-Asgn} and \rulelabel{E-Commit} define
such functions for read, write and commit operations, respectively.
The effect returned by the function in each case includes its
transaction id ($T_i$) along with an arbitrarily chosen effect id
($j$) that is later verified to be unique in \rulelabel{E-Aux}. The
$\rval$ for a write is the value being written, and for commit it is
$\bot$. In case of a read, the value read depends on the how the read
operation chooses to interpret the given set ($S$) of visible effects.
The interpretation may depend on the application semantics. For
example, a monotonically increasing counter application may choose to
let a write with largest value determine the value of a
read. A more straightforward interpretation has already been described
before. To accommodate multiple interpretations, operational semantics
is made parametric over an interpretation function ($\interp{\cdot}$)
that accepts a set of effects and a variable name, and returns the
value associated with the variable. A straightforward interpretation
function that chooses the last write (i.e., write with largest id) is
shown below:
\begin{smathpar}
\begin{array}{lcl}
  \isMax(S,\eta) & \Leftrightarrow &  \forall (\eta'\in S).  
  \kind(\eta') = \kind(\eta) \\
  & & \hspace*{0.4in}\Rightarrow \eta' = \eta \disj \id(\eta') < \id(\eta)\\

\interp{S}(X) & = & \C{if}\;(\exists (\eta \in S). \kind(\eta) = \C{WR}(X) 
  \wedge \isMax(S,\eta)) \\
  & & \C{then}\;\rval(\eta)\;\C{else}\;0\\
\end{array}
\end{smathpar}
Rules \rulelabel{E-Top-Ctx} and \rulelabel{E-Txn-Ctx} rules define
congruence rules for top-level terms and transaction-bound terms,
respectively. The rules and evaluation contexts ($\ectx$ and
$\ectx_i$) are defined such that only a certain kinds of terms are
allowed at the top-level and inside a transaction. In particular, a
\txnimp program at the top-level can either be a transaction, or a
parallel composition of transactions. A command inside a \C{txn} block
can either be an assignment, or a sequential composition of
assignments. Nested transactions and transactions that fork multiple
threads are forbidden to focus on the problems relevant to this paper.

\subsection{Isolation Specifications}
\label{sec:ansi-isolation}

\begin{figure*}[t]
\begin{smathpar}
\begin{array}{lcl}
(\A,\visZ) \Vdash \eta \in T_i & \Leftrightarrow & \eta \in \A \conj \txn(\eta) = T_i\\
(\A,\visZ) \Vdash \eta_1 \visar \eta_2 & \Leftrightarrow & \{\eta_1,\eta_2\}
        \subseteq \A \conj (\eta_1,\eta_2) \in \visZ\\
(\A,\visZ) \Vdash \eta_1 \soar \eta_2 & \Leftrightarrow & \{\eta_1,\eta_2\}
        \subseteq \A \conj \txn(\eta_1)=\txn(\eta_2) \conj \id(\eta_1)
        < \id(\eta_2)\\
\E \Vdash S \subseteq T_i & \Leftrightarrow & \forall \eta.~ \eta
        \in S \Rightarrow \underE{\eta \in T_i} \\
\E \Vdash T_i \visar \eta & \Leftrightarrow &\forall\eta_1
        .\,(\E \Vdash \eta_1 \in T_i) \Rightarrow \E \Vdash \eta_1 \visar \eta \\
\underE{T_i \visar T_j} & \Leftrightarrow &  \forall\eta_1,\eta_2.\,
    %\sameobj{\eta_1}{\eta_2}  \Rightarrow 
    \underE{\eta_1\in T_i} \conj \underE{\eta_2 \in T_j} \Rightarrow 
    \underE{\eta_1 \visar \eta_2} \\
\underE{T_i \invisar T_j} & \Leftrightarrow &  \forall\eta_1,\eta_2.\,
        %\sameobj{\eta_1}{\eta_2}\Rightarrow 
        \underE{\eta_1\in T_i} \conj \underE{\eta_2 \in T_j} \Rightarrow 
        \neg (\underE{\eta_1 \visar \eta_2})\\
\underE{T_i \wrstoar X} & \Leftrightarrow & \exists\eta.~
        \underE{\eta \in T_i} \conj \kind(\eta) = \C{WR}(X)\\
\underE{\C{RMWVis}(T_j)} & \Leftrightarrow & \forall\eta_1,\eta_2.\,
       \underE{\{\eta_1,\eta_2\} \subseteq T_j} \conj
       \underE{\eta_1 \soar \eta_2} \Rightarrow \underE{\eta_1 \visar
       \eta_2}\\
\underE{\C{MonotonicVis}(T_j)} & \Leftrightarrow & 
       \underE{\C{RMWVis}(T_j)} \conj 
       \forall\eta,\eta_1,\eta_2.\, \underE{\{\eta_1,\eta_2\} \in T_j} 
          \conj \\
  &   & \hspace*{1.2in}\underE{\eta \visar \eta_1} \conj
        \underE{\eta_1 \soar \eta_2} \Rightarrow \underE{\eta \visar
        \eta_2} \\
%  \C{CausalVis}(T_i) & \Leftrightarrow & 
%         \C{MonotonicVis}(T_i) \conj \C{RMWVis}(T_i)\\
\underE{\C{AtomicVis}(T_j)} & \Leftrightarrow & 
       \forall\eta_1,\eta_2.\, \neg(\underE{\eta_1 \in T_j}) \conj
       \underE{\eta_2 \in T_j} \conj
       \underE{\eta_1 \visar \eta_2} \Rightarrow \underE{\txn(\eta_1)
       \visar \eta_2}\\
\underE{\C{CommitVis}(T_j)} & \Leftrightarrow & 
       \forall\eta_1,\eta_2.~ \neg(\underE{\eta_1 \in T_j}) \conj 
          \underE{\eta_2 \in T_j} \conj
       \underE{\eta_1 \visar \eta_2} \Rightarrow\\
  &   & \hspace*{1.2in}\exists\eta.\, \underE{\eta \in \txn(\eta_1)} 
        \conj \kind(\eta) = \C{COMMIT} 
        \conj \underE{\eta \visar \eta_2}\\
% \underE{\C{TransVis}(T_j)} & \Leftrightarrow &  \forall
%        \eta_1,\eta_2,\eta_3.\, \underE{\eta_3 \in T_j} \conj
%        \underE{\eta_1 \visar \eta_2} \conj
%        \underE{\eta_2 \visar \eta_3} \Rightarrow \underE{\eta_1 \visar
%        \eta_3} \\
\underE{\C{SnapshotVis}(T_i,T_j)} & \Leftrightarrow &  \underE{T_i
       \visar T_j} \disj \underE{T_i \invisar T_j}\\
\underE{\C{OneWaySER}(T_i,T_j)} & \Leftrightarrow &  \underE{T_i
       \visar T_j} \disj (\underE{T_i \invisar T_j} \conj \\
  &   &\hspace*{1.2in}\exists \eta.~\underE{\eta\in T_i} \conj \kind(\eta) =
       \C{COMMIT} \Rightarrow \underE{T_j \visar \eta})\\
\underE{\C{RC}(T_j)} & \Leftrightarrow & \underE{\C{AtomicVis}(T_j)} 
        \conj \underE{\C{CommitVis}(T_j)}\\
\underE{\C{MAV}(T_j)} & \Leftrightarrow & \underE{\C{RC}(T_j)} \conj
      \underE{\C{MonotonicVis}(T_j)} \\
\underE{\C{RR}(T_j)} & \Leftrightarrow & \underE{\C{MAV}(T_j)}
       \conj \forall T_i.\,T_i \neq T_j \Rightarrow 
        \underE{\C{SnapshotVis}(T_i,T_j)} \\
% &   & \hspace*{2in}\conj  \C{SnapshotVis}(T_i,T_j)\\
\underE{\C{SI}(T_j)} & \Leftrightarrow &  \underE{\C{RR}(T_j)}
       \conj \forall T_i.\,(T_i \neq T_j \conj \exists X.\, \underE{T_i \wrstoar X} \conj 
        \underE{T_j \wrstoar X})\\
%      (\exists X.\, T_i \wrstoar X \conj T_j \wrstoar X)
%      \Rightarrow  \C{TotalVis}(T_i,T_j)\\
  &   & \hspace*{2in} \Rightarrow \underE{\C{OneWaySER}(T_i,T_j)}\\
\underE{\C{SER}(T_j)} & \Leftrightarrow & \underE{\C{MAV}(T_j)}
       \conj \forall T_i.\,T_i \neq T_j 
       \Rightarrow \underE{\C{OneWaySER}(T_i,T_j)}\\
% &   & \hspace*{2in}\conj  \C{TotalVis}(T_i,T_j)\\
\end{array}
\end{smathpar}

\caption{Standard isolation guarantees expressed as trace
well-formedness constraints}
\label{fig:ansi-isolation}
\end{figure*}

Fig.~\ref{fig:ansi-isolation} shows the specification of standard
isolation guarantees expressed as constraints over trace
well-formedness. For brevity and convenience, we adopt some notation.
An execution trace is destructed into $\A$ and $\visZ$ whenever
individual components of the pair are needed.  Otherwise, it is written
as $\E$. When $\psi$ is a proposition, we write $\underE{\psi}$ to denote
an interpretation of $\psi$ in the context of the trace $\E$. Such
interpretations are defined on a case-by-case basis in
Fig.~\ref{fig:ansi-isolation}. In the following, we give informal
explanations for each definition.

In the context of a trace $(\A,\visZ)$, an effect $\eta$ is said to
belong to a transaction $T_i$ if $\eta$ belongs to the effect set $A$
and its transaction identifier is $T_i$. The containment relation is
trivially lifted to the set of effects to define $\underE{S \subseteq
T_i}$.  The interpretation of $\eta_1 \visar \eta_2$ and $\eta_1
\soar \eta_2$ are straightforward and explained in \S\ref{sec:syntax}.
A transaction $T_i$ is said to be visible to an effect $\eta$ if every
effect $\eta_1$ of $T_i$ recorded by the trace is visible to $\eta$.
$T_i$ may be visible to $\eta$ but may not be visible to every other
effect in the transaction. For a transaction $T_i$ to be considered to
be visible to a transaction $T_j$ in the context of a trace $\E$
(written $\underE{T_i \visar T_j}$), every effect ($\eta_1$) of $T_i$
present in $\E$ must be visible to every effect ($\eta_2$) of $T_j$ in
$\E$. Conversely, if none of the effects of $T_i$ present in $\E$ are
visible to any effect of $T_j$, then $T_i$ is considered invisible to
$T_j$ under $\E$ (written $\underE{T_i \invisar T_j}$). Transaction
$T_i$ is said to have written to a variable $X$ under $\E$ (i.e.,
$\underE{T_i \wrstoar X}$) if there exists a write-to-$X$ effect from
$T_i$ in $\E$.

Various isolation guarantees are defined as propositions indexed by
a transaction identifier $T_j$. Transaction $T_j$ is said to have
experienced \emph{read-my-writes} visibility under $\E$ if every
effect ($\eta_1$) of $T_j$ is visible to every subsequent effect
($\eta_2$) of the same transaction in $\E$. This lets $T_j$ to never
lose its own updates. Monotonic visibility adds one more constraint to
read-my-writes; besides requiring $\eta_1$ to be visible to $\eta_2$,
it also requires every effect ($\eta$) visible to $\eta_1$ in $\E$ to
be visible to $\eta_2$ as well. Thus $T_j$ witnesses monotonically
increasing state as time progresses. Atomic visibility allows an
effect $\eta_2$ of $T_j$ to witness an effect $\eta_1$ of $T_i$ only
if all effects of $T_i$ in $\E$ are also visible to $\eta_2$. Atomic
visibility thus prevents a transaction from being partially visible.
However, atomic visiblity does not prevent an uncommitted transaction
from being visible. To confine visibility to only committed
transactions, we need an additional constraint that requires an effect
($\eta_2$) of $T_j$ to also witness the commit effect of $T_i$
whenever it witnesses some effect ($\eta_1$) of $T_i$. This additional
constraint is captured by the $\mathtt{CommitVis}$ specification. A
transaction $T_i$ is said to be snapshot-visible to a transaction
$T_j$ if either it is visible to $T_j$, or it is invisible; it is
forbidden for only a suffix (more generally, a subset) of $T_j$ to
witness $T_i$. Observe that snapshot visibility permits $T_i$ and $T_j$ to
execute and commit while being  oblivious of each other. One-way
serializability proscribes this possiblity. $T_i$
is said to be one-way-serializable w.r.t $T_j$ if either it is visible
to $T_j$ or $T_j$ is visible to the commit effect of $T_i$.  Note that
both $\mathtt{SnapshotVis}$ and $\mathtt{OneWaySER}$ are asymmetric
definitions that guarantee complete visiblity or invisibility of $T_i$
to $T_j$, but not the converse. While $\mathtt{SnapshotVis}$ provides
no guarantees to $T_i$, $\mathtt{OneWaySER}$ guarantees that at least
the commit effect of $T_i$ witnesses $T_j$.

The ANSI SQL 92 standard requires \iso{Read Committted} isolation to avoid
the dirty reads phenomenon, which is achieved by enforcing
$\mathtt{AtomicVis}$ and $\mathtt{CommitVis}$ guarantees. The {\sc rc}
specification is therefore a combination of these two guarantees. The
specification also agrees with the description and
implementation~\cite{bailishat,pldi15} of {\sc rc} for highly available
replicated stores. On relational databases, however, {\sc rc} has also come
to be associated with the $\mathtt{MonotonicVis}$ guarantee.
Nonetheless, $\mathtt{AtomicVis}$ and $\mathtt{CommitVis}$ are
sufficient to reason about {\sc rc} isolation on relational stores too. As
we demonstrate in \S\ref{sec:store-consistency}, the combination of
these guarantees with the {\sc sc} property of relational stores
automatically leads to the monotonicity guarantee, which probably
explains why {\sc rc} comes with $\mathtt{MonotonicVis}$ on such stores. On
weakly consistent stores however, $\mathtt{AtomicVis}$ and
$\mathtt{CommitVis}$ do not imply $\mathtt{MonotonicVis}$. A stronger
isolation level called \iso{Monotonic Atomic View} ({\sc mav} of
Fig.~\ref{fig:ansi-isolation})~\cite{bailishat,pldi15} was proposed to
explicitly extend {\sc rc} with monotonicity on such stores. \iso{Repeatable
Read} spec ({\sc rr}) extends {\sc mav} with the snapshot visibility guarantee
as described before. \iso{Snapshot Isolation} spec ({\sc si}) extends
{\sc rr} with a one-way serializability guarantee w.r.t the transactions
that perform conflicting writes (i.e., writes to the same shared
variable). Finally, \iso{Serializable} isolation extends one-way
$\mathtt{SnapshotVis}$ to all transactions, including those that do
not conflict\footnote{Fig.~\ref{fig:ansi-isolation} presents slightly
weaker versions of {\sc si} and {\sc ser} specs in the interest of
clarity.}. 

In contrast to recent proposals (\emph{e.g.},
~\cite{gotsmanconcur15}), our specifications for {\sc si} and {\sc
  ser} do not necessarily impose a total order among transactions
(conflicting or otherwise). In reality, a total order under {\sc ser}
(resp. {\sc si}) is guaranteed only if the store executes all
transactions under {\sc ser} (resp. {\sc SI}) isolation. Our
specifications admit this possibility, and derive a total order under
the assumption of homogenity. However, databases almost always allow
isolation levels to be configured on a per-transaction basis, allowing
transactions at various isolation levels to coexist.  Specifications
that assume homogenity are incorrect under this setting.

% Having specified isolation levels as trace well-formedness
% constraints, we can now construct trace invariants ($\I$) for
% \txnimp programs by composing isolation specifications for various
% transactions. For instance, the following trace invariant enforces
% {\sc si} for both transactions of the program in
% Fig.~\ref{fig:motiv-eg-1}, allowing it to satisfy its postcondition:
% \begin{smathpar}
% \I \;=\; \lambda\E.~ \underE{\C{SI(Wd1)}} \conj \underE{\C{SI(Wd2)}}
% \end{smathpar}
% In contrast, the following trace invariant enforces {\sc rc} for one and {\sc si}
% for another, leading to a possible violation of the postcondition:
% \begin{smathpar}
% \I \;=\; \lambda\E.~ \underE{\C{RC(Wd1)}} \conj \underE{\C{SI(Wd2)}}
% \end{smathpar}


\section{Store Consistency}
\label{sec:store-consistency}

Under a trivial trace invariant (i.e., $\I(\E) = true$), the abstract
machine of Fig.~\ref{fig:txnimp} assumes the semantics of an {\sc ec} store;
it allows operations of a transaction to witness arbitrary subsets of the
global state. A non-trivial $\I$ composed of isolation specifications
from Fig.~\ref{fig:ansi-isolation} induces the machine to provide
non-trivial isolation guarantees for transactions. However, weak
isolation levels often only constrain the visibility sets of
operations by dictating what \emph{not} to see; not what to see.  For
instance, \iso{Repeatable Read} isolation prohibits operations of a
transaction from witnessing different states. It, however, does not
prohibit all operations of a transaction from witnessing an aribitrary
subset of the global state. Consequently, the machine can remain an {\sc ec}
store even while providing non-trivial isolation. How then to model
the semantics of an {\sc sc} store, such as a relational database, with
variable (weak) isolation?

The answer lies in enforcing store-specific consistency constraints,
along with transaction-specific isolation constraints, via the trace
invariant $\I$. In particular, we split $\I$ into two components: (1).
$\I_s$, the store-specific invariant, and (2). $\I_c$, the
program-specific (or, client-specific) invariant, to capture
consistency and isolation constraints, respectively. $\I$ is now a
conjunction:
\begin{smathpar}
  \I \;=\; \lambda\E.~\I_s(\E) \wedge \I_c(\E)
\end{smathpar}
The trace invariant defined in \S\ref{sec:ansi-isolation} for the
withdraw program of Fig.~\ref{fig:motiv-eg-1} now becomes its $\I_c$
and remains an invariant regardless of the store. $\I_s$ however
changes from store to store. We now consider various stores and
describe their corresponding store invariants.

\paragraph{An EC store} An eventually-consistent {\sc ec} store provides no
additional consistency guarantees besides those provided by the operational
semantics.  Hence, its trace invariant, $\I_s(\E)$, for any execution
$\E$ is always \emph{true}.

\paragraph{An SC store} A strongly-consistent {\sc sc} store guarantees a total order of
all operations w.r.t $\visZ$ consistent with their chronological
order. A straightforward definition of $\I_s$ for this store is:
\begin{smathpar}
\begin{array}{l}
  \C{SC}(\E) \;=\; \forall\eta_1,\eta_2.~\{\eta_1,\eta_2\}
  \subseteq \E.\A \conj \id(\eta_1) < \id(\eta_2) \\
  \hspace*{2in}\Rightarrow \underE{\eta_1 \visar \eta_2}
\end{array}
\end{smathpar}
Unfortunately, $\I_s=\C{SC}$ conflicts with all isolation
specifications of Fig.~\ref{fig:ansi-isolation}. For instance,
consider a case where $\I_c(\E) \;=\; \forall
T_i.~\underE{\C{RC}(T_i)}$, a constraint that dictates all transactions execute under
\iso{Read Committed} isolation. Imagine a sample execution where
$\eta_1$'s transaction is not yet committed when $\eta_2$ is
generated. Letting $\eta_1$ be visible to $\eta_2$ violates $\I_c$,
whereas not letting it be visible violates $\I_{s}$. The only way to
satisfy both the invariants is to rule out all the executions that
interleave the operations of one transaction with the other, thereby
enforcing serializability and the ACID model of
transactions\footnote{Thus, serializability is a natural
generalization of {\sc sc} to transactions.}. In general, when $\I_s$ is in
conflict (but not inconsistent) with $\I_c$, the only way to enforce
both invariant sets is to restrict concurrency. Clearly, this is unacceptable since
it defeats the very purpose of supporting weak isolation. 
% How then do we enforce weak isolation on a strongly consistent
% machine?

Relational databases demonstrate a way of out this impasse.
Implementations of weak isolation levels on relational databases
implicitly relax the {\sc sc} requirement so as to maximize
concurrency and improve performance. Consequently, execution traces do
not necessarily satisfy {\sc sc}. This approach hints at a general
solution reconciling $\I_s$ with $\I_c$ that entails weakening the
former sufficiently to satisfy the latter. However, there are many
ways $\I_s$ can be weakened; for instance, by setting it to \emph{true}
which makes the store EC. In reality though, weak isolation on
databases does not cause so drastic a reduction in consistency. It is
therefore possible that there exists a principled approach to weaken
$\I_s$ that adequately captures the underlying properties provided by the store. Such
a principle can be gleaned by observing that data stores often make
\emph{recency} commitments~\cite{bailishat}, by which they aim to make
the most recent data available to clients; {\sc sc} is, in fact, an
extreme form of a recency commitment, which guarantees that reads
\emph{always} witness the most recent writes to a data item. In case this conflicts
with weak isolation guarantees, such as {\sc rr} which only requires that
reads in a transaction witness the same state, the natural way to
weaken consistency while prioritizing recency is to let reads witness
the most recent state that \emph{does not} violate isolation
constraints. In the context of the operational model of
Fig.~\ref{fig:txnimp}, this translates to making the largest subset
($S$) of the global state ($\A$) that does not lead to the violation
of {\sc rr} isolation constraints visible to an operation.

Generalizing this intuition to any $\I_s$ and $\I_c$ yields a
\emph{maximum visiblity principle}, which requires the weakened
consistency guarantee ($\I_s'$) of a store to enforce all visibility
relationships imposed by the actual consistency guarantee ($\I_s$) on
a given trace, unless enforcing such a relationship violates $\I_c$.
Formally, the maximum visibility weakening of $\I_s$ is a new store
consistency relation $\I_s'$, such that:
\begin{itemize}
  \item $\I_s'$ is weaker than $\I_s$: 
      $\forall\E.~ \I_s(\E) \Rightarrow \I_s'(\E)$, and
  \item In every trace $\E$ that satisfies $\I_c$, and for every pair
  of effects $\eta_1$ and $\eta_2$ in $\E$, if $\I_s(\E)$ requires
  $\eta_1$ to be visible to $\eta_2$, then so does $\I_s'(\E)$ unless
  extending $\E$ with $\visZ(\eta_1,\eta_2)$ violates
  $\I_c$\footnote{\GK{ToDo: consider other well-formedness conditions
  on trace, such as acyclicity of $\visZ$ and $\soZ$. Are they needed
  (considering that the machine never violates them)? Encode the
  specifications in Z3 and make sure they are consistent.}}:
  \begin{smathpar}
  \begin{array}{l}
  \forall\E,\eta_1,\eta_2.~ \I_c(\E) \Rightarrow (\I_s(\E)
    \Rightarrow \underE{\eta_1 \visar \eta_2}) \Rightarrow \\
    \hspace*{0.5in}(\I_s'(\E) \Rightarrow \underE{\eta_1 \visar
    \eta_2} \disj \neg\I_c(\E\,\cup\,(\emptyset,\{(\eta_1,\eta_2)\})))
  \end{array}
  \end{smathpar}
\end{itemize}

Applying this principle, we can weaken {\sc sc} to obtain the
following store trace invariant ($\I_s$) for an {\sc sc} store whose
isolation constraints are captured by $\I_c$:

\begin{smathpar}
\begin{array}{lcl}
\I_s(\E) & = & \forall \eta_1,\eta_2.\, \{\eta_1,\eta_2\},
    \subseteq \E.\A \conj \id(\eta_1) <
    \id(\eta_2) \\
    & & \hspace*{0.5in} \Rightarrow 
      \underE{\eta_1 \visar \eta_2} \disj \neg\I_c(\E
    \cup (\emptyset,\{(\eta_1,\eta_2)\}))\\
\end{array}
\end{smathpar}

\noindent As usual $\I = \I_s \wedge \I_c$. The resultant operational model
hides an existing effect ($\eta\in\A$) from being visible to the
current operation if and only if showing it results in the violation
of $\I_c$.

It is worthwhile to note that although the conflict-free definition of
$\I_s$ shown above captures the behaviour of an {\sc sc} database in
the presence of isolation constraints, it does not necessarily describe
how the database achieves this behaviour (i.e., its
implementation). It is unusual for databases to check for satisfaction
of isolation constraints every time an operation is executed. Efficient
implementation strategies (\eg multi-versioning) are employed to
achieve the twin goals of isolation and consistency/recency. \footnote{\SJ{This
  paragraph reads poorly - not quite sure what the reader should conclude
  from this.  A negative interpretation is that we can formalize the
  relationship between $I_s$ and $I_c$ but the formalization may be
  far removed from reality and practicality.}}

\paragraph{A CC store} A causally consistent data store~\cite{gotsmanpopl16,LBC16}
allows operations to only witness a causally consistent snapshot of the
global state. A straightforward trace invariant ($\I_s$) for this
store is the causally consistent guarantee that requires an effect to witness all
causally preceding effects regardless of the isolation requirements of
its transaction:
\begin{smathpar}
\begin{array}{lccl}
\C{CC}(\E) & \;=\; &  & \forall \eta_1,\eta_2.\, 
      \E \Vdash \eta_1 \soar \eta_2 \Rightarrow  \underE{\eta_1 \visar
      \eta_2}\\
    &   & \wedge & \forall\eta_1,\eta_2,\eta_3.\,\underE{\eta_1 \visar
      \eta_2} \conj \underE{\eta_2 \visoar \eta_3} \\
    &   & &\hspace*{0.5in} \Rightarrow \underE{\eta_1 \visar \eta_3}\\
\end{array}
\end{smathpar}
In order to admit weak isolation behavior without restricting concurrency, we weaken $\I_s$ while
being guided by the maximum visibility principle. The weakened $\I_s$
is shown below:

\begin{smathpar}
\begin{array}{lccl}
\C{CC}(\E) & \;=\; &  & \forall \eta_1,\eta_2.\, 
      \E \Vdash \eta_1 \soar \eta_2 \Rightarrow  \underE{\eta_1 \visar
      \eta_2} \\
    & & & \hspace*{0.6in}\disj \neg\I_c(\E \cup 
                (\emptyset,\{(\eta_1,\eta_2)\}))\\
    &   & \wedge & \forall\eta_1,\eta_2,\eta_3.\,\underE{\eta_1 \visar
      \eta_2} \conj \underE{\eta_2 \visoar \eta_3} \\
    &   & &\hspace*{0.3in} \Rightarrow \underE{\eta_1 \visar \eta_3}
      \disj \neg\I_c(\E \cup (\emptyset,\{(\eta_1,\eta_2)\}))\\
\end{array}
\end{smathpar}

\noindent Instantiating the parameter $\I$  with $\I_s \wedge \I_c$ in
Fig.~\ref{fig:txnimp} results in an operational semantics that admits
violation of causal consistency if and only if the violation is
inevitable to enforce $\I_c$.

\subsection{Eventually Consistent Replication}

\begin{figure}
\centering
\subcaptionbox {
  Replica model
  \label{fig:ec-theirs}
} [
  0.4\columnwidth
] {
  \includegraphics[scale=0.7]{Figures/ec-theirs}
 
}
\hspace*{0.1in}
\subcaptionbox {
  Subset model
  \label{fig:ec-ours}
}{
  \includegraphics[scale=0.45]{Figures/ec-ours}
} \caption{In the replica model, operation $\op$ generates effect
$\eta$ at replica $R_1$, which is then merged to $R_2$. If the
\emph{store is {\sc cc}}, then $R_2$'s state at merge event is same or
larger than $R_1$'s state at generation event (the difference is
highlighted). In our subset model, $\op$ witnesses $S_1 \subseteq
\E.\A$ and generates $\eta$, which is immediately added to $\E.\A$. A
later operation may witness $S_2 \subseteq \E.\A$, and if the
\emph{operation is} {\sc cc} and $\eta \in S_2$, then it also
witnesses $S_1$ (i.e., $S_1 \subseteq S_2$). } 
% Moreover, Like $R_2 - R_1$, if all effects in $S_2 - S_1$ are
% concurrent with $\eta$, i.e., $\not\exists\eta'.~\eta' \in S_2 - S_1
% \conj % visZ(\eta,\eta')$, then any precondition $P$ that is valid
% when $\op$ executed is also valid when $\eta$ is witnessed because
% of the stability condition.
\label{fig:ec-theirs-vs-ours}
\end{figure}

In this section, we present an intuitive explanation of the soundness
result in context of eventually consistent replication. 
% In other words, we explain why our approach is a sound way of
% reasoning about transactions under eventually consistent replication
% despite our system model not including the standard artifacts
% associated with a replicated store. 


Existing approaches to reason about eventually consistent replication
necessarily involve reasoning in terms of \emph{replicas} of data. The
primary challenge in this setting is to ensure that the assumptions
made and guarantees enforced by an operation at one replica carry over
to other replicas that merge its effects, thus preserving the overall
integrity of the system. Reasoning frameworks, such
as~\cite{gotsmanpopl16}, address this challenge by imposing
restrictions on how various replica states differ, i.e., by
strengthening the consistency of the store. Our view of eventually
consistent replication however does not explicitly involve replicas.
Fig.~\ref{fig:ec-theirs-vs-ours} contrasts our model of EC replication
with the conventional replica-based model.  Under our model, the
notion of a replica is subsumed by the concept of visibility; a
replica is defined by the subset ($S$) of global state ($\E.\A$) that
an operation witnesses.  Constraints over replica states therefore
manifest as constraints over the visibility relation. For example,
instead of requiring the store to be causally consistent, an operation
can request to witness a causally consistent subset of the state; such
demands can be made via the trace invariant $\I$. For a precondition
($P$) of the operation to be useful, it has to be an assertion over
every causally consistent subset of the global state. Since any
replica that eventually executes the operation has to expose one such
subset ($S$), the precondition is guaranteed to hold regardless of the
replica. There is however one problem with this explanation; by
considering subsets of just one global state, it ignores the fact that
the global state (hence, the replica states) change during the
execution of the operation. Existing approaches account for this
change by distinguishing between \emph{effect generation} event at one
replica and \emph{effect merge} event at another replica, and
requiring that certain invariants be preserved between these two
events at different replicas. Our framework folds all of this into a
stability condition (\S\ref{sec:rely-guarantee}). Since any change to
the global state during the execution of the operation is an
interference, and $P$ is required to be stable with respect to any
such interference, it follows that $P$ is valid on every replica, thus
ensuring that assumptions made at the generation event are also valid
at the merge event.

\subsection{Example}

\GK{Here, I plan to include the full example that we dropped from
\S\ref{sec:motivation}.}

