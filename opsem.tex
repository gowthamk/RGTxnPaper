\section{\txnimp: Syntax and Semantics}
\label{sec:opsem}

\label{sec:syntax}

\renewcommand{\ctxn}[3]{\C{TXN}_{#1}\langle #2 \rangle\{#3\}}
\begin{figure*}[!ht]
\raggedright
%
\textbf{Syntax}\\
%
\begin{smathpar}
\renewcommand{\arraystretch}{1.2}
\begin{array}{lclcl}
\multicolumn{5}{c} {
  {x,y} \in \mathtt{Variables}\qquad
  {f} \in \mathtt{Field\;Names} \qquad
% {\tau} \in \mathtt{Table\;Names}
  {i,j} \in \mathbb{N} \qquad
  {\odot} \in \{+,-,\le,\ge,=\}\qquad
}\\
\multicolumn{5}{c}{
  {k} \in \mathbb{Z}\cup\mathbb{B} \qquad
  {\rec} \in \{\bar{f}=\bar{k}\}\qquad
  {\stl,\stg,s} \in \Pow{\rec} \qquad
  \I \in \mathbb{B}\times\Pow{\rec}\times\Pow{\rec}\times\Pow{\rec}
  \rightarrow \Prop
}\\
v & \in & \mathtt{Values} & \coloneqq & k \ALT \rec \ALT s\\
e & \in & \mathtt{Expressions} & \coloneqq & c \ALT x \ALT x.f 
    \ALT \{\bar{f}=\bar{e}\} \ALT e_1 \odot e_2\\ 
c & \in & \mathtt{Commands} & \coloneqq & \cskip \ALT \lete{x}{e}{c}
    \ALT \ite{e}{c_1}{c_2}\ALT c_1;c_2 \ALT \inserte{e}  \\
&&&&\ALT \deletee{\lambda x.e}
    \ALT \lete{x}{\selecte{\lambda x.e}}{c}
    \ALT \updatee{\lambda x.e_1}{\lambda x.e_2}\\
&&&&\ALT \foreache{e_1}{\lambda x.\lambda y. e_2} 
    \ALT \foreachr{s_1}{s_2}{\lambda x.\lambda y. e}\\
&&&&\ALT \ctxn{i}{\I}{ c } \ALT \ctxn{i}{\I,\stl,\stg}{c} \ALT c1 || c2\\
% t & \in & \mathtt{Terms} & \coloneqq & e \ALT c\\
\ectx & \in & \mathtt{Eval\;Ctx} & ::= & \bullet \ALT  
  \bullet || c_2 \ALT c_1 || \bullet \ALT \bullet;\,c_2 
  \ALT \ctxn{i}{\I,\stl,\stg}{\bullet} \\
\end{array}
\end{smathpar}
%
\bigskip

\renewcommand{\arraystretch}{1.2}

%
\textbf{Local Reduction} \quad 
\fbox {\(\stg \vdash (c,\stl) \stepsto (c',\stl')\)}\\
%
\begin{minipage}{2.8in}
\rulelabel{E-Insert}
\begin{smathpar}
\begin{array}{c}
\RULE
{
  i \not\in \dom(\stl \cup \stg)\\
  r = \{\bar{f}=\bar{k};\,\idf=i;\,\delf=\C{false}\}
}
{
  \stg \vdash (\inserte{\{\bar{f}=\bar{k}\}},\stl) \stepsto
  (\cskip,\stl \cup \{r\})
}
\end{array}
\end{smathpar}
\end{minipage}
%
%
\begin{minipage}{2.8in}
\rulelabel{E-Delete}
\begin{smathpar}
\begin{array}{c}
\RULE
{
  s = \{r' \,|\, \exists(r\in\Delta).~ \eval([r/x]e)=\C{true} \\
        \hspace*{0.7in}\conj r'=\{\bar{f}=r.\bar{f}; \idf=r.\idf;
        \delf=\C{true}\}\}\\
% \dom(s) \cap \dom(\delta) = \emptyset
}
{
  \stg \vdash (\deletee{\lambda x.e},\stl) \stepsto (\cskip,\stl \cup s)
}
\end{array}
\end{smathpar}
\end{minipage}
%
\bigskip

%
\begin{minipage}{2.8in}
\rulelabel{E-Select}
\begin{smathpar}
\begin{array}{c}
\RULE
{
  s = \{r\in\Delta \,|\, \eval([r/x]e)=\C{true}\}\spc
  c' = [s/x]c
}
{
  \stg \vdash (\lete{x}{\selecte{\lambda x.e}}{c}, \stl) \stepsto 
              (c',\stl)
}
\end{array}
\end{smathpar}
\end{minipage}
%
%
\begin{minipage}{2.8in}
\rulelabel{E-Update}
\begin{smathpar}
\begin{array}{c}
\RULE
{
  s = \{r' \,|\, \exists(r\in\Delta).~ \eval([r/x]e_2)=\C{true} \conj r'=[r/x]e_1\}\\
}
{
  \stg \vdash (\updatee{\lambda x.e_1}{\lambda x.e_2},\stl) \stepsto 
              (\cskip,\stl \cup s)
}
\end{array}
\end{smathpar}
\end{minipage}
%

\begin{smathpar}
\begin{array}{ll}
  \rulelabel{E-Foreach1} & \stg \vdash (\foreache{e}{\lambda x.\lambda
  y.c},\stl) \stepsto (\foreachr{\emptyset}{\eval(e)}{\lambda x.\lambda y. c})\\
  \rulelabel{E-Foreach2} & \stg \vdash (\foreachr{s_1}{\{r\} \cup s_2}{\lambda x.\lambda
  y.c},\stl) \stepsto ([r/y][s_1/x]c;\,\foreachr{s_1 \cup \{r\}}{s_2}{\lambda x.\lambda y. c})\\
  \rulelabel{E-Foreach3} & \stg \vdash (\foreachr{s}{\emptyset}{\lambda x.\lambda
  y.c},\stl) \stepsto (\cskip,\stl)\\
\end{array}
\end{smathpar}
%
\bigskip

%
\textbf{Top-Level Reduction} \quad 
\fbox {\((c,\stg) \stepsto (c',\stg')\)}\\
%
\begin{minipage}{3in}
\rulelabel{E-Txn}
\begin{smathpar}
\begin{array}{c}
\RULE
{
  \I(\C{true},\stl,\stg,\stg')\spc
  \stg \vdash (c,\stl) \stepsto (c',\stl')
}
{
  (\ctxn{i}{\I,\stl,\stg}{c},\stg') \stepsto
  (\ctxn{i}{\I,\stl',\stg'}{c'},\stg')
}
\end{array}
\end{smathpar}
\end{minipage}
%
%
\begin{minipage}{2.8in}
\rulelabel{E-Commit}
\begin{smathpar}
\begin{array}{c}
\RULE
{
  \I(\C{false},\stl,\stg,\stg')
}
{
  (\ctxn{i}{\I,\stl,\stg}{\cskip},\stg') \stepsto (\cskip,\stl\gg\stg')
}
\end{array}
\end{smathpar}
\end{minipage}
%


\caption{\small \txnimp: Syntax and Small-step semantics}
\label{fig:txnimp}
\end{figure*}



Fig.~\ref{fig:txnimp} shows the syntax and small-step semantics of
\txnimp, a core language that we use to formalize our intuitions about
reasoning under weak isolation. Variables ($x$), integer and Boolean
constants ($k$), records ($r$) of named constants, sets ($s$) of such
records, arithmetic and boolean expressions ($e_1 \odot e_2$), and
record expressions ($\langle \bar{f}=\bar{e} \rangle$) constitute the
syntactic class of expressions ($e$). Commands ($c$) include $\cskip$,
conditional statements, \C{LET} constructs to bind names, \C{FOREACH}
loops, SQL statements, their sequential composition ($c_1;c_2$),
transactions ($\ctxn{i}{\I}{c}$) and their parallel composition
($c_1\,||\,c_2$). Each transaction is assumed to have a unique
identifier $i$, and executes at the top-level; our semantics does not
support nested transactions. The $\I$ in the \C{TXN} block syntax is
the transaction's isolation specification, whose purpose is explained
below.  Certain terms that only appear at run-time are also present in
$c$.  These include a {\sf txn} block tagged with sets ($\stl$ and
$\stg$) of records representing local and global database state, and a
runtime {\sf foreach} expression that keeps track of the set ($s_1$)
of items already iterated, and the set ($s_2$) of items yet to be
iterated. Note that the surface-level syntax of the \C{FOREACH}
command shown here is slightly different from the one used in previous
sections; its higher-order function has two arguments, $y$ and $z$,
which are invoked (during the reduction) with the set of
already-iterated items, and the current item, respectively. This form
of \C{FOREACH} lends itself to inductive reasoning that will be useful
for verification (Sec.~\ref{sec:reasoning}). Our language ensures
that all effectful actions are encapsulated within database commands,
and that all shared state among processes are only manipulated via
transactions and its supported operations.  In particular, we do not
consider programs in which objects resident on e.g., the OCaml heap
are concurrently manipulated by OCaml expressions as well as database
actions.


% also We let $T_i$ for $i \in \matshbb{N}$
% range over transaction identifiers. When it is evident we are
% referring to a transaction, we use the number $i$ instead of $T_i$ for
% identification (\eg in $\C{txn}\langle i \rangle$). Like variables,
% transaction identifiers are globally accessible. For notational
% convenience, we let $t$ range over both expressions and commands.

We define a small-step operational semantics for this language in
terms of an abstract machine that executes a command, and updates
either a transaction-local ($\stl$), or global ($\stg$) database, both
of which are modeled as a set of records of a pre-defined type, i.e.,
they all belong to a single table.  The generalization to multiple
tables is straightforward, e.g., by having the machine manipulate a
set of sets, one for each table.  The semantics assumes that records
in $\stg$ can be uniquely identified via their $\idf$ field, and
enforces this property wherever necessary. Certain hidden fields are
treated specially by the operational semantics, and are hidden
from the surface language. These include a $\txnf$ field that tracks
the identifier of the transaction that last updated the record, and a
$\delf$ field that flags deleted records in $\stl$.  For a set $S$ of
records, we define $\dom(S)$ as the set of unique ids of all records
in $S$. Thus $|\dom(\stg)| = |\stg|$. During its execution, a
transaction may write to multiple records in $\stg$. Atomicity
dictates that such writes should not be visible in $\stg$ until the
transaction commits. We therefore associate each transaction with a
local database ($\stl$) that stores such uncommitted
records\footnote{While SQL's \C{UPDATE} admits writes at the
  granularity of record fields, most popular databases enforce
  record-level locking, allowing us to think of ``uncommitted writes''
  as ``uncommitted records''. }. Uncommitted records include deleted
records, whose $\delf$ field is set to \C{true}. When the transaction
commits, its local database is atomically \emph{flushed} to the global
database, committing these heretofore uncommitted records. The flush
operation ($\rhd$) is defined as follows:
\begin{smathpar}
\begin{array}{c}
\forall r.~ r \in (\stl\rhd\stg) ~\Leftrightarrow~ 
  (r.\idf \notin \dom(\stl) \conj r \in \stg)
\disj (r \in \stl \conj \neg r.\delf) 
\end{array}
\end{smathpar}
Let $\stg' = \stl\rhd\stg$. A record $r$ belongs to $\stg'$ iff it
belongs to $\stg$ and has not been updated in $\stl$, i.e., $r.\idf
\notin \dom(\stl)$, or it belongs to $\stl$, i.e., it is either a new
record, or an updated version of an old record, provided the update is
not a deletion ($\neg r.\delf$).  Besides the commit, flush also helps
a transaction read its own writes. Intuitively, the result of a read
operation inside a transaction must be computed on the database
resulting from flushing the current local state ($\stl$) to the global
state ($\stg$). The abstract machine of Fig.~\ref{fig:txnimp},
however, does not let a transaction read its own writes. This
simplifies the semantics, without losing any generality, since
substituting $\stl\rhd\stg$ for $\stg$ at select places in the
reduction rules effectively allows reads of uncommitted transaction
writes to be realized, if so desired.

The small-step semantics is stratified into a transaction-local
reduction relation, and a top-level reduction relation. The
transaction-local relation ($\stg \vdash (c,\stl) \stepsto
(c',\stl')$) defines a small-step reduction for a command inside a
transaction, when the database state is $\stg$; the command $c$
reduces to $c'$, while updating the transaction-local database $\stl$
to $\stl'$. The definition assumes a meta-function $\eval$ that
evaluates closed terms to values. The reduction relation for SQL
statements is defined straightforwardly. \C{INSERT} adds a new record
to $\stl$ after checking the uniqueness of its id. \C{DELETE} finds
the records in $\stg$ that match the search criteria defined by its
Boolean function argument, and adds the records to $\stl$ after
marking them for deletion. \C{SELECT} bounds the name introduced by
\C{LET} to the set of records from $\stg$ that match the search
criteria, and then executes the bound command $c$. \C{UPDATE} uses its
first function argument to compute the updated version of the records
that match the search criteria defined by its second function
argument. Updated records are added to $\stl$.

The reduction of \C{FOREACH} starts by first converting it to its
run-time form to keep track of iterated items ($s_1$), as well as
yet-to-be-iterated items ($s_2$).  Iteration involves invoking its
function argument with $s_1$ and the current element $x$ (note:
$\uplus$ in $\{x\} \uplus s_2$ denotes a disjoint union). The
reduction ends when $s_2$ becomes empty. The reduction rules for
conditionals, \C{LET} binders, and sequences are standard, and
omitted for brevity.

The top-level reduction relation defines the small-step semantics of
transactions, and their parallel composition. A transaction comes
tagged with an \emph{isolation specification} $\I$, which has two components
$\I_e$ and $\I_c$, that dictate the timing and nature of interferences
that the transaction can witness, during its execution ($\I_e$), and
when it is about to commit ($\I_c$).  Formally, $\I_e$ and $\I_c$ are
predicates over the (current) transaction-local database state
($\stl$), the state ($\stg$) of the global database when the
transaction last took a step, and the current state ($\stg'$) of the
global database.  Intuitively, $\stg'\neq\stg$ indicates an
interference from another concurrent transaction, and the predicates
$\I_e$ and $\I_c$ decide if this interference is allowed or not,
taking into account the local database state ($\stl$). For instance,
as described in \S\ref{sec:motivation}, an SI transaction on
PostgreSQL defines $\I$ as follows:
\begin{smathpar}
\begin{array}{lcl}
\I_e\,\,(\stl,\stg,\stg') & = & \stg' = \stg\\
\I_c\,\,(\stl,\stg,\stg') & = & \forall(r\in\stl)(r'\in\stg).~ r'.\idf = r.\idf \Rightarrow r'\in\stg'.
\end{array}
\end{smathpar}
This definition dictates that no change to the global database state
can be visible to an SI transaction while it executes ($\I_e$), and
there should be no concurrent updates to records written by the
transaction by other concurrently executing ones ($\I_c$).
To simplify the presentation, we use $\I$ instead of $\I_e$ and $\I_c$
when its destructed form is not required.

The reduction of a $\ctxn{i}{\I}{c}$ begins by first converting it to
its run-time form $\ctxnr{i}{\I,\stl,\stg}{c}$, where $\stl =
\emptyset$, and $\stg$ is the current (global) database.  Rule
\rulelabel{E-Txn} reduces $\ctxnr{i}{\I,\stl,\stg}{c}$ under a
database state ($\stg'$), only if the transaction-body isolation
specification ($\I_e$) allows the interference between $\stg$ and
$\stg'$.   Rule \rulelabel{E-Commit} commits the
transaction $\ctxnr{i}{\I,\stl,\stg}{c}$ by flushing its uncommitted
records to the database. This is done only if the interference between
$\stg$ and $\stg'$ is allowed at the commit point by the isolation
specification ($\I_c$).  The distinction between $\I_e$ and $\I_c$
allows us to model the snapshot semantics of realistic isolation
levels that isolate a transaction from interference during its
execution, but expose interferences at the commit point.

\textbf{Local Context Independence} As mentioned previously, our
operational semantics does not let a transaction read its own writes.
It also does not let a transaction overwrite its own writes, due to
the premise $\dom(\stl)\cap\dom(s) = \emptyset$ on the
\rulelabel{E-Delete} and \rulelabel{E-Update} rules. We refer to this
restriction as \emph{local context independence}.  This restriction is
easy to relax in the operational semantics and the reasoning framework
presented in the next section; our inference procedure described in
\S\ref{sec:inference}, however, has a non-trivial dependence on this
assumption.  Nonetheless, we have encountered few instances in
practice where enforcing local context independence turns out to be a
severe restriction. Indeed, all of the transactions we have considered
in our benchmarks (e.g., TPC-C) satisfy this assumption.

\subsection{Isolation Specifications}
\label{sec:isolation}

A distinctive characteristic of our development is that it is
parameterized on a weak isolation specification $\I$ that can be
instantiated with the declarative characterization of an isolation
guarantee or a concurrency control mechanism, regardless of the actual
implementation used to realize it. This allows us to model a range of
isolation properties that are relevant to the theory and practice of
transaction processing systems without appealing to specific
implementation artifacts like locks, versions, logs, speculation, etc.
A few well-known properties are discussed below:

% Databases implement isolation levels through a combination of various
% concurrency control mechanisms, which are absent from
% Fig.~\ref{fig:txnimp}. In this section, we first axiomatize the
% semantics of various concurrency control constructs, and then define
% the precise semantics of the isolation levels as they are implemented
% on two real-world databases -  PostgreSQL and MySQL.
% Any additional concurrency control provided by the database
% implementation, beyond that which is already incuded in the
% isolation specification, also needs to be axiomatized in $\I$. We
% first describe such axiomatizations, and then present the
% specifications of the standard isolation levels as implemented by
% popular off-the-shelf databases.

\textbf{Unique Ids}. As the \C{new\_order} example
(\S\ref{sec:motivation}) demonstrates, enforcing global uniqueness
of ordered identifiers requires stronger isolation levels than the
ones that are default on most databases (e.g., Read
Committed). Alternatively, globally unique sequence numbers,
regardless of the isolation level, can be requested from a relational
database via SQL's \C{UNIQUE} and \C{AUTO\_INCREMENT} keywords. Our
development crucially relies on the uniqueness of record
identifiers\footnote{The importance of unique ids is recognized in
  real-world implementations.  For example, MySQL's InnoDB engine
  automatically adds a 6-byte unique identifier if none exists for a
  record.}, which are checked locally for uniqueness by the
\rulelabel{E-Insert} rule.  The global uniqueness of locally unique
identifiers can be captured as an isolation property thus:
\begin{smathpar}
\begin{array}{lcl}
  \I_{id}(\stl,\stg,\stg') & = & \forall(r\in\stl).~
      r.\idf\notin \dom(\stg) \Rightarrow r.\idf\notin \dom(\stg').
\end{array}
\end{smathpar}
$\I_{id}$ ensures that if the id of a record is globally unique when
it is added to a transaction's $\stl$, it remains globally unique
until the transaction commits. This would be achieved within our
semantic framework by prohibiting the interference from a concurrent
transaction that adds the same id. The axiom thus simulates a global
counter protected by an exclusive lock without explicitly appealing to
an implementation artifact.

\textbf{Write-Write Conflicts}. Databases often employ a combination
of concurrency control methods, both optimistic (e.g., speculation and
rollback) and pessimistic (e.g., various degrees of locking), to
eliminate write-write (\emph{ww}) conflicts among concurrent
transactions. We can specify the absence of such conflicts using our
tri-state formulation thus:
\begin{smathpar}
\begin{array}{lcl}
  \I_{ww}(\stl,\stg,\stg') & = & \forall(r'\in\stl)(r \in \stg).~
      r.\idf = r'.\idf  \Rightarrow r\in\stg'.
\end{array}
\end{smathpar}
That is, given a record $r'\in\stl$, if there exists an $r\in\stg$
with the same id (i.e., $r'$ is an updated version of $r$), then $r$
must be present unmodified in $\stg'$. This prevents a concurrent
transaction from changing $r$, thus simulating the behavior of an
exclusive lock or a speculative execution that succeeded (Note: a
transaction writing to $r$ always changes $r$ because its $\txnf$
field is updated). 
% There is however a caveat: if we assume extensional
% equality over records, a write by a concurrent transaction that
% doesn't change $r$'s contents is still allowed. While this in itself
% is not a problem, the concurrent transaction may modify other records,
% which then become visible in the current transaction. A write lock
% prevents this behavior, whereas our axiomatization ($\I_{ww}$) allows
% it. An easy fix for this is to add version timestamps to records that
% effectively intensionalizes equality. Nonetheless, imprecise
% axiomatization of write locks hasn't been a problem in practice.

\textbf{Snapshots} Almost all major relational databases implement
isolation levels that execute transactions against a static snapshot
of the database that can be axiomatized thus:
\begin{smathpar}
\begin{array}{lcl}
  \I_{ss}(\stl,\stg,\stg') & = & \stg' = \stg.
\end{array}
\end{smathpar}

\textbf{Read-Only Transactions}. Certain databases implement special
privileges for read-only transactions. Read-only behavior can be
enforced on a transaction by including the following proposition as
part of its isolation invariant:
\begin{smathpar}
\begin{array}{lcl}
  \I_{ro}(\stl,\stg,\stg') & = & \stl = \emptyset\\
\end{array}
\end{smathpar}

In addition to these properties, various specific isolation levels
proposed in the database or distributed systems literature, or
implemented by commercial vendors can also be specified within this
framework:

\textbf{Read Committed (RC) and Monotonic Atomic View (MAV).} RC
isolation allows a transaction to witness writes of committed
transactions at any point during the transaction's execution.
Although it offers only weak isolation guarantees, it nonetheless
prevents witnessing \emph{dirty writes} (i.e., writes performed by
uncommitted transactions).  Monotonic Atomic View
(MAV)~\cite{bailishat} is an extension to RC that guarantees the
continuous visibility of a committed transaction's writes once they
become visible in the current transaction. That is, a MAV transaction
does not witness \emph{disappearing writes}, which can happen on a
weakly consistent machine. Due to the SC nature of our abstract
machine (there is always a single global database state $\stg$; not a
vector of states indexed by vector clocks), and our choice to never
violate atomicity of a transaction's writes, both RC and MAV are
already guaranteed by our semantics.  Thus, defining $\I_e$ and $\I_c$
to \emph{true} ensures RC and MAV behavior under our semantics.

\textbf{Repeatable Read (RR)} By definition, multiple reads to a
transactional variable in a Repeatable Read transaction are required
to return the same value.  RR is often implemented (for e.g., in
~\cite{mysqliso,bailishat}) by executing the transaction against a
(conceptual) snapshot of the database, but committing its writes to
the actual database. This implementation of RR can be axiomatized as
$\I_e = \I_{ss}$ and $\I_{c}=true$. However, this specification of RR
is stronger than the ANSI SQL specification, which requires no more
than the invariance of already read records. In particular, ANSI SQL
RR allows \emph{phantom reads}, a phenomenon in which a repeated
\C{SELECT} query might return newly inserted records that were not
previously returned. This specification is implemented, for e.g., in
Microsoft's SQL server, using record-level exclusive read locks, that
prevent a record from being modified while it is read by an
uncommitted transaction, but which does not prohibit insertion of new
records. The ANSI SQL RR specification can be axiomatized in our
framework, but it requires a minor extension to our operational
semantics to track a transaction's reads. In particular, the records
returned by \C{SELECT} should be added to the local database $\stl$,
but without changing their transaction identifiers ($\txnf$ fields),
and flush ($\rhd$) should only flush the records that bear the current
transaction's identifier. With this extension, ANSI SQL RR can be
axiomatized thus:
\begin{smathpar}
\begin{array}{lcl}
  \I_e(\stl,\stg,\stg') & \Leftrightarrow & \forall(r\in\stl).
      r \in \Delta \Rightarrow r \in \Delta'\\
  \I_c(\stl,\stg,\stg') & \Leftrightarrow & true\\
\end{array}
\end{smathpar}
If a record $r$ belongs to both $\stl$ and $\stg$, then it must be a
record written by a different transaction and read by the current
transaction (since the current transaction's records are not yet
present in $\stg$). By requiring $r\in\stg'$, $\I_e$ guarantees the
invariance of $r$, thus the repeatability of the read. 

\textbf{Snapshot Isolation (SI)} The concept of executing a
transaction against a consistent snapshot of the database was first
proposed as Snapshot Isolation in~\cite{berenson}. SI doesn't admit
write-write conflicts, and the original proposal, which is implemented
in Microsoft SQL Server, required the database to roll-back an SI
transaction if conflicts are detected during the commit. This behavior
can be axiomatized as $\I_e = \I_{ss}$ (execution against a snapshot),
and $\I_c = \I_{ww}$ (avoiding write-write conflicts during the
commit).
% \begin{smathpar}
% \begin{array}{lcl}
% \I_e\,\,(\stl,\stg,\stg') & = & \stg' = \stg\\
% \I_c\,\,(\stl,\stg,\stg') & = & \forall(r\in\stl)(r'\in\stg).~ r'.\idf = r.\idf \Rightarrow r'\in\stg'.
% \end{array}
% \end{smathpar}
Note that the same axiomatization applies to PostgreSQL's RR,
although its implementation (described in Sec.~\ref{sec:motivation})
differs considerably from the original proposal. Thus, reasoning done
for an SI transaction on MS SQL server carries over to PostgreSQL's RR
and vice-versa, demonstrating the benefits of reasoning axiomatically
about isolation properties.

\textbf{Serializability (SER)} The specification of serializability is
straightforward:
\begin{smathpar}
\begin{array}{lcl}
\I_e\,\,(\stl,\stg,\stg') & = & \stg' = \stg\\
\I_c\,\,(\stl,\stg,\stg') & = & \stg' = \stg.
\end{array}
\end{smathpar}



%% %\usepackage[table,xcdraw]{xcolor}
%% \begin{table}[]
%% \centering
%% \begin{tabular}{l|l|l|l|l|l|l|l|l|l|l|l|l|}
%% \cline{2-13}
%%                                 & \multicolumn{6}{c|}{{\color[HTML]{333333} PostgreSQL}}                         & \multicolumn{6}{c|}{MySql}                                                   \\ \cline{2-13} 
%%                                 & \multicolumn{2}{c|}{RC} & \multicolumn{2}{c|}{RR} & \multicolumn{2}{c|}{SER} & \multicolumn{2}{c|}{RC} & \multicolumn{2}{c|}{RR} & \multicolumn{2}{c|}{SER} \\ \cline{2-13} 
%%                                 & T          & F          & T          & F          & T           & F          & T          & F          & T          & F          & T           & F          \\ \hline
%% \multicolumn{1}{|l|}{$\I_{ss}$} &            &            & \checkmark &            & \checkmark  & \checkmark &            &            & \checkmark &            & \checkmark  & \checkmark \\ \hline
%% \multicolumn{1}{|l|}{$\I_{rb}$} &            &            &            & \checkmark &             &            &            &            &            &            &             &            \\ \hline
%% \multicolumn{1}{|l|}{$\I_{ro}$} &            &            &            &            &             &            &            &            & \checkmark & \checkmark &             &            \\ \hline
%% \multicolumn{1}{|l|}{$\I_{id}$} & \checkmark & \checkmark & \checkmark & \checkmark & \checkmark  & \checkmark & \checkmark & \checkmark & \checkmark & \checkmark & \checkmark  & \checkmark \\ \hline
%% \multicolumn{1}{|l|}{$\I_{ww}$} & \checkmark & \checkmark & \checkmark & \checkmark & \checkmark  & \checkmark & \checkmark & \checkmark & \checkmark & \checkmark & \checkmark  & \checkmark \\ \hline
%% \end{tabular}

%% \caption{The semantics of isolation levels on PostgreSQL and MySQL}
%% \label{fig:iso-table}
%% \end{table}

%% With help of the above axiomatizations, we can now define the precise
%% semantics of various isolation levels on PostgreSQL and MySQL. The table
%% in Table~\ref{fig:iso-table} captures the summary.


