\section{\txnimp: Syntax and Semantics}
\label{sec:opsem}

\label{sec:syntax}

%\renewcommand{\ctxn}[3]{\C{TXN}_{#1}\langle #2 \rangle\{#3\}}
\begin{figure*}[!ht]
\raggedright
%
\textbf{Syntax}\\
%
\begin{smathpar}
\renewcommand{\arraystretch}{1.2}
\begin{array}{lclcl}
\multicolumn{5}{c} {
  {x,y} \in \mathtt{Variables}\qquad
  {f} \in \mathtt{Field\;Names} \qquad
% {\tau} \in \mathtt{Table\;Names}
  {i,j} \in \mathbb{N} \qquad
  {\odot} \in \{+,-,\le,\ge,=\}\qquad
  {k} \in \mathbb{Z}\cup\mathbb{B} \qquad
  {\rec} \in \{\bar{f}=\bar{k}\}\
}\\
{\stl,\stg,s} & \in & \mathtt{State} & \coloneqq &  \Pow{\{\bar{f}=\bar{k}\}} \\
{\I_e, \I_c }  & \in & \mathtt{Isolation Spec} & \coloneqq & (\stl,\stg,\stg') \rightarrow \Prop\\
v & \in & \mathtt{Values} & \coloneqq & k \ALT \rec \ALT s\\
e & \in & \mathtt{Expressions} & \coloneqq & k \ALT x \ALT x.f 
    \ALT \{\bar{f}=\bar{e}\} \ALT e_1 \odot e_2\\ 
c & \in & \mathtt{Commands} & \coloneqq & \cskip \ALT \lete{x}{e}{c}
    \ALT \ite{e}{c_1}{c_2}\ALT c_1;c_2 \ALT \inserte{x}  \\
&&&&\ALT \deletee{\lambda x.e}
    \ALT \lete{y}{\selecte{\lambda x.e}}{c}
    \ALT \updatee{\lambda x.e_1}{\lambda x.e_2}\\
&&&&\ALT \foreache{x}{\lambda y.\lambda z. c} 
    \ALT \foreachr{s_1}{s_2}{\lambda x.\lambda y. e}\\
&&&&\ALT \ctxn{i}{\I}{ c } \ALT \ctxnr{i}{\I,\stl,\stg}{c} \ALT c1 || c2\\
% t & \in & \mathtt{Terms} & \coloneqq & e \ALT c\\
\ectx & \in & \mathtt{Eval\;Ctx} & ::= & \bullet \ALT  
  \bullet || c_2 \ALT c_1 || \bullet \ALT \bullet;\,c_2 
  \ALT \ctxnr{i}{\I,\stl,\stg}{\bullet} \\
\end{array}
\end{smathpar}
%
\bigskip

\renewcommand{\arraystretch}{1.2}

%
\textbf{Local Reduction} \quad 
\fbox {\(\stg \vdash (\tbox{c}_i,\stl) \stepsto (\tbox{c'}_i,\stl')\)}\\
%
\begin{minipage}{2.8in}
\rulelabel{E-Insert}
\begin{smathpar}
\begin{array}{c}
\RULE
{
  r.\idf \not\in \dom(\stl \cup \stg)\\
  r' = \{r \;\C{with}\; \txnf=i;\,\delf=\C{false}\}
}
{
  \stg \vdash (\tbox{\inserte{r}}_i,\stl) \stepsto
  (\tbox{\cskip}_i,\stl \cup \{r'\})
}
\end{array}
\end{smathpar}
\end{minipage}
%
%
\begin{minipage}{2.8in}
\rulelabel{E-Delete}
\begin{smathpar}
\begin{array}{c}
\RULE
{
  \hspace*{-1in}
  s = \{r' \,|\, \exists(r\in\Delta).~ \eval([r/x]e)=\C{true} \\
      \hspace*{0.1in}\conj r'=\{r \with \delf=\C{true};\, \txnf=i\}\}\spc
  \dom(\stl)\cap\dom(s) = \emptyset \spc
% \dom(s) \cap \dom(\delta) = \emptyset
}
{
  \stg \vdash (\tbox{\deletee{\lambda x.e}}_i,\stl) \stepsto 
  (\tbox{\cskip}_i,\stl \cup s)
}
\end{array}
\end{smathpar}
\end{minipage}
%
\bigskip

%
\begin{minipage}{2.8in}
\rulelabel{E-Select}
\begin{smathpar}
\begin{array}{c}
\RULE
{
  \\
  s = \{r\in\Delta \,|\, \eval([r/x]e)=\C{true}\}\spc
  c' = [s/y]c
}
{
  \stg \vdash (\tbox{\lete{y}{\selecte{\lambda x.e}}{c}}_i, \stl) \stepsto 
              (\tbox{c'}_i,\stl)
}
\end{array}
\end{smathpar}
\end{minipage}
%
%
\begin{minipage}{2.8in}
\rulelabel{E-Update}
\begin{smathpar}
\begin{array}{c}
\RULE
{
  \hspace*{-0.2in}s = \{r' \,|\, \exists(r\in\Delta).~ 
    \eval([r/x]e_2)=\C{true} \conj r'= \{ [r/x]e_1 \;\C{with}\\
    \idf=r.\idf;\,\txnf=i;\,\delf = r.\delf \}\} \spc
  \dom(\stl) \cap \dom(s) = \emptyset
}
{
  \stg \vdash (\tbox{\updatee{\lambda x.e_1}{\lambda x.e_2}}_i,\stl) 
      \stepsto (\tbox{\cskip}_i,\stl \cup s)
}
\end{array}
\end{smathpar}
\end{minipage}
%

\begin{smathpar}
\begin{array}{ll}
  \rulelabel{E-Foreach1} & \stg \vdash (\tbox{\foreache{s}{\lambda y.\lambda
    z.c}}_i,\stl) \stepsto (\tbox{\foreachr{\emptyset}{s}{\lambda
    y.\lambda z. c}}_i,\stl)\\
  \rulelabel{E-Foreach2} & \stg \vdash (\tbox{\foreachr{s_1}{\{r\} \uplus s_2}
    {\lambda y.\lambda z.c}}_i,\stl) \stepsto (\tbox{[r/z][s_1/y]c;\,
    \foreachr{s_1 \cup \{r\}}{s_2}{\lambda y.\lambda z. c}}_i,\stl)\\
  \rulelabel{E-Foreach3} & \stg \vdash (\tbox{\foreachr{s}{\emptyset}
    {\lambda y.\lambda z.c}}_i,\stl) \stepsto (\tbox{\cskip}_i,\stl)\\
\end{array}
\end{smathpar}
%
\bigskip

%
\textbf{Top-Level Reduction} \quad 
\fbox {\((c,\stg) \stepsto (c',\stg')\)}\\
%
\begin{minipage}{3in}
  \rulelabel{E-Txn-Start}
  \begin{smathpar}
  \begin{array}{c}
    \RULE{}
         {(\ctxn{i}{\I}{c},\stg) \stepsto (\ctxnr{i}{\I,\emptyset,\stg}{c},\stg)}
  \end{array}
  \end{smathpar}
\end{minipage}%
\hfill
\begin{minipage}{3in}
\rulelabel{E-Txn}
\begin{smathpar}
\begin{array}{c}
\RULE
{
  \I_e\,\,(\stl,\stg,\stg')\spc
  \stg \vdash (\tbox{c}_i,\stl) \stepsto (\tbox{c'}_i,\stl')
}
{
  (\ctxnr{i}{\I,\stl,\stg}{c},\stg') \stepsto
  (\ctxnr{i}{\I,\stl',\stg'}{c'},\stg')
}
\end{array}
\end{smathpar}
\end{minipage}\\

\begin{center}
\begin{minipage}{3in}
\rulelabel{E-Commit}
\begin{smathpar}
\begin{array}{c}
\RULE
{
  \I_c\,\,(\stl,\stg,\stg')
}
{
  (\ctxnr{i}{\I,\stl,\stg}{\cskip},\stg') \stepsto (\cskip,\stl\gg\stg')
}
\end{array}
\end{smathpar}
\end{minipage}
\end{center}
\hfill
%
\caption{\small \txnimp: Syntax and Small-step semantics}
\label{fig:txnimp}
\end{figure*}



Fig.~\ref{fig:txnimp} shows the syntax and small-step semantics of
\txnimp, a core language that we use to formalize our intuitions about
reasoning under weak isolation. Variables ($x$), Integer and boolean
constants ($k$), records ($r$) of named constants, sets ($s$) of such
records, arithmetic and boolean expressions ($e_1 \odot e_2$), and
record expressions ($\{\bar{f}=\bar{e}\}$) constitute the syntactic
class of expressions ($e$). Commands ($c$) include $\cskip$,
conditional statements, \C{LET} constructs to bind names, \C{FOREACH}
loops, SQL statements, their sequential composition ($c_1;c_2$), and
transactions ($\ctxn{i}{\I}{c}$) and their parallel composition
($c_1\,||\,c_2$). The $\I$ in the \C{TXN} block syntax is the
transaction's isolation specification, which will be explained later.
Certain terms that only appear at run-time are also present in $c$.
These include a \R{TXN} block tagged with sets ($\stl$ and $\stg$) of
records whose meaning is explained below, and a runtime \R{FOREACH}
expression that keeps track of the set ($s_1$) of items already
iterated, and the set ($s_2$) of items yet to be iterated. Note that
even the surface-level syntax of the \C{FOREACH} command shown here is
a little different from the one used in previous sections; its
higher-order function has two arguments, $y$ and $z$, which are
invoked (during the reduction) with the set of already-iterated items,
and the current item, respectively. This form of \C{FOREACH} lends
itself to inductive reasoning that will be useful for verification
(Sec.~\ref{sec:reasoning}).

% also We let $T_i$ for $i \in \matshbb{N}$
% range over transaction identifiers. When it is evident we are
% referring to a transaction, we use the number $i$ instead of $T_i$ for
% identification (\eg in $\C{txn}\langle i \rangle$). Like variables,
% transaction identifiers are globally accessible. For notational
% convenience, we let $t$ range over both expressions and commands.

We define a small-step operational semantics for this language in
terms of an abstract machine that executes a command, and updates
either a transaction-local ($\stl$), or global ($\stg$) database, both
of which are modeled as a set of records of a pre-defined type, i.e.,
they all belong to a single table.  The generalization to multiple
tables is straightforward.  Records in $\stg$ are uniquely identifiable
through their $\idf$ field, which is auto-generated and does not
belong to the surface language, i.e., $\C{id}\notin f$. For a set $S$
of records, we define $\dom(S)$ as the set of unique ids of all 
records in $S$. Thus $|\dom(\stg)| = |\stg|$. During its execution, a
transaction may write to multiple records in $\stg$. Atomicity
dictates that such writes should not be visible in $\stg$ until the
transaction commits. We therefore associate each transaction with a
local database ($\stl$) that stores such uncommitted
records\footnote{While SQL's \C{UPDATE} admits writes at the
  granularity of record fields, databases, in reality, enforce
  record-level locking, allowing us to think of ``uncommitted writes''
  as ``uncommitted records''. }. Uncommitted records include deleted
records, which are identified by a hidden $\delf$ field set to
\C{true}. When the transaction commits, its local database is
atomically \emph{flushed} to the global database, committing these
heretofore uncommitted records. The flush operation ($\gg$) is defined as
follows:
\begin{smathpar}
\begin{array}{c}
\forall r.~ r \in (\stl\gg\stg) ~\Leftrightarrow~ 
  (r.\idf \notin \dom(\stl) \conj r \in \stg)
\disj (r \in \stl \conj \neg r.\delf) 
\end{array}
\end{smathpar}
Let $\stg' = \stl\gg\stg$. A record $r$ belongs to $\stg'$ iff it
belongs to $\stg$ and has not been updated in $\stl$, i.e., $r.\idf
\notin \dom(\stl)$, or it belongs to $\stl$, i.e., it is either a new
record, or an updated version of an old record, but the update is not
a deletion ($\neg r.\delf$). Thus, flush defines the result of
atomically applying a transaction's local writes to the global
database.  Besides the commit, flush also helps a transaction read its
own writes. Intuitively, the result of a read operation inside a
transaction must be computed on the database resulting from flushing
the current local state ($\stl$) to the global state ($\stg$). The
abstract machine of Fig.~\ref{fig:txnimp}, however, does not let a
transaction read its own writes. This simplifies the semantics,
without losing any generality, since substituting $\stl\gg\stg$ for
$\stg$ at select places in the reduction rules effectively allows
reads of uncommitted transaction writes to be realized.

The small-step semantics is stratified into a transaction-local reduction
relation, and a top-level reduction relation. Transaction-local
relation ($\stg \vdash (c,\stl) \stepsto (c',\stl')$) defines a
small-step reduction for a command inside a transaction, when the
database state is $\stg$; the command $c$ reduces to $c'$, while
updating the transaction-local database $\stl$ to $\stl'$. The
definition assumes a meta function $\eval$ that evaluates expressions
with no free variables to values. The reduction relation for SQL
statements is defined straightforwardly.  \C{INSERT} adds a new record
to $\stl$ after adding a unique identifier. \C{DELETE} finds the records in $\stg$ that match
the search criteria defined by its boolean function argument, and adds
the records to $\stl$ after marking them for deletion. \C{SELECT}
bounds the name introduced by \C{LET} to the set of records from
$\stg$ that match the search criteria, and then executes the bound
command $c$. \C{UPDATE} uses its first function argument to compute
the updated version of the records that match the search criteria
defined by its second function argument. Updated records are added to
$\stl$. 

The reduction of \C{FOREACH} starts by first converting it to its
run-time form to keep track of iterated items ($s_1$),
as well as  yet-to-be-iterated items ($s_2$). Initially, $s_1$ is
empty. As the elements are iterated, they are removed from $s_2$ and
added to $s_1$ . Iteration involves invoking its function argument
with $s_1$ and the current element $x$ (note: $\uplus$ in $\{x\}
\uplus s_2$ denotes a disjoint union). The reduction ends when 
$s_2$ becomes empty. The reduction rules for conditionals, \C{LET}
binders, and sequences are straightforward, and ommitted for brevity.

The top-level reduction relation defines the small-step semantics of
transactions, and their parallel composition. A transaction always
comes tagged with an isolation specification $\I_e$ or $\I_c$ that
dictates the timing and nature of interferences that a transaction can
witness, both during its execution ($\I_e$), and when it is about to
commit ($\I_c$).  Formally, $\I_e$ and $\I_c$ are functions that map
the (current) transaction-local database state ($\stl$), the state
($\stg$) of the global database when the transaction last took a step,
and the current state ($\stg'$) of the global database.  Intuitively,
$\stg'\neq\stg$ indicates an interference from another concurrent
transaction, and the predicates $\I_e$ and $\I_c$ decide if this
interference is allowed or not, taking into account the local database
state ($\stl$). For instance, as described in
(Sec.~\ref{sec:motivation}), an RR transaction on PostgreSQL defines
$\I$ as following:
\begin{smathpar}
\begin{array}{lcl}
\I_e\,\,(\stl,\stg,\stg') & = & \stg' = \stg\\
\I_c\,\,(\stl,\stg,\stg') & = & \forall(r\in\stl)(r'\in\stg).~ r'.\idf = r.\idf \Rightarrow r'\in\stg'.
\end{array}
\end{smathpar}

The reduction of a $\ctxn{i}{\I_e}{c}$ begins by first converting it
to its run-time form $\ctxn{i}{\I_e,\stl,\stg}{c}$, where $\stl =
\emptyset$, and $\stg$ the current (global) database.  Rule
\rulelabel{E-Txn} reduces $\ctxn{i}{\I_e,\stl,\stg}{c}$ under a
database state ($\stg'$), only if the isolation specification ($\I$)
allows the interference between $\stg$ and $\stg'$ during that
step. Rule \rulelabel{E-Commit} commits the transaction
$\ctxn{i}{\I_c,\stl,\stg}{c}$ by flushing its uncommitted records to
the database. This is done only if the interference between $\stg$ and
$\stg'$ is allowed at the commit point by the isolation specification.
The distinction between $\I_e$ and $\I_c$ allows us to model realistic
database features such as snapshot isolation that isolates the effects
a transaction can witness during its execution but exposes
interferences from concurrent transactions at the commit point.

\subsection{Isolation Specifications}
\label{sec:isolation}

Operational semantics of Fig.~\ref{fig:txnimp} does not employ
concurrency control constructs in any form. Any additional concurrency
control provided by the database implementation, beyond that which is
already incuded in the isolation specification, also needs to be
axiomatized in $\I$. We first describe such axiomatizations, and then
present the specifications of the standard isolation levels as
implemented by popular off-the-shelf databases.

\textbf{Unique Ids}. Most relational databases associate every record
in a table with a unique identifier. Often, auto-generated unique
identifiers are part of the schema itself (via SQL's \C{UNIQUE} and
\C{AUTO\_INCREMENT} keywords), but if not, database adds one (e.g.,
MySQL's InnoDB engine automatically adds a 6-byte identifier if none
exists). Enforcing global uniqueness of identifiers requires exclusive
locks, which are absent from Fig.~\ref{fig:txnimp}. The
\rulelabel{E-Insert} rule assigns an id to the newly added record
after checking that it is not already present in $\stg$ and $\stl$.
But the new record is only added to $\stl$, which is invisible outside
the transaction. Thus, the uniqueness check for the same id also
passes in a concurrent transaction, resulting in duplicate ids
after both commits. To prevent this, we need the following proposition
in $\I$ of every transaction, regardless of whether it is executing
or committing:
\begin{smathpar}
\begin{array}{lcl}
  \I_{id}(\stl,\stg,\stg') & = & \forall(r\in\stl).~
      r.\idf\notin \dom(\stg) \Rightarrow r.\idf\notin \dom(\stg').
\end{array}
\end{smathpar}
$\I_{id}$ makes sure that a globally unique id generated during a
transaction's execution remains globally unique until it commits. An
interference from a concurrent transaction that adds the same id to
$\stg$ is prohibited.

\textbf{Write Locks}. Transactions in MySQL and Postgres (effectively)
obtain exclusive locks on records they update, and do not release them
until they commit, thus preventing write-write (ww) conflicts. To
obtain this behavior on the abstract machine of Fig.~\ref{fig:txnimp},
the following proposition needs to be in $\I$ for every transaction:
\begin{smathpar}
\begin{array}{lcl}
  \I_{ww}(\stl,\stg,\stg') & = & \forall(r'\in\stl)(r \in \stg).~
      r.\idf = r'.\idf  \Rightarrow r\in\stg'.
\end{array}
\end{smathpar}
That is, given a record $r'\in\stl$, if there exists an $r\in\stg$
with the same id (i.e., $r'$ is an updated version of $r$), then $r$
must be present unmodified in $\stg'$. This prevents a concurrent
transaction from changing $r$, thus simulating the behavior of an
exclusive lock. There is however a caveat: if we assume extensional
equality over records, a write by a concurrent transaction that
doesn't change $r$'s contents is still allowed. While this in itself
is not a problem, the concurrent transaction may modify other records,
which then become visible in the current transaction. A write lock
prevents this behavior, whereas our axiomatization ($\I_{ww}$) allows
it. An easy fix for this is to add version timestamps to records that
effectively intensionalizes equality. Nonetheless, imprecise
axiomatization of write locks hasn't been a problem in practice.

\textbf{Read-Only Transactions}. Certain databases implement special
previleges for read-only transactions. Read-only behavior can be
enforced on a transaction by including the following proposition in
its $\I$:
\begin{smathpar}
\begin{array}{lcl}
  \I_{ro}(\stl,\stg,\stg') & = & \stl = \emptyset\\
\end{array}
\end{smathpar}
If a transaction declared as read-only performs a write, then its
$\stl\neq \emptyset$, and the transaction never commits.




