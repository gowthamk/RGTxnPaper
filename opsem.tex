\section{\txnimp}

%\renewcommand{\ctxn}[3]{\C{TXN}_{#1}\langle #2 \rangle\{#3\}}
\begin{figure*}[!ht]
\raggedright
%
\textbf{Syntax}\\
%
\begin{smathpar}
\renewcommand{\arraystretch}{1.2}
\begin{array}{lclcl}
\multicolumn{5}{c} {
  {x,y} \in \mathtt{Variables}\qquad
  {f} \in \mathtt{Field\;Names} \qquad
% {\tau} \in \mathtt{Table\;Names}
  {i,j} \in \mathbb{N} \qquad
  {\odot} \in \{+,-,\le,\ge,=\}\qquad
  {k} \in \mathbb{Z}\cup\mathbb{B} \qquad
  {\rec} \in \{\bar{f}=\bar{k}\}\
}\\
{\stl,\stg,s} & \in & \mathtt{State} & \coloneqq &  \Pow{\{\bar{f}=\bar{k}\}} \\
{\I_e, \I_c }  & \in & \mathtt{Isolation Spec} & \coloneqq & (\stl,\stg,\stg') \rightarrow \Prop\\
v & \in & \mathtt{Values} & \coloneqq & k \ALT \rec \ALT s\\
e & \in & \mathtt{Expressions} & \coloneqq & k \ALT x \ALT x.f 
    \ALT \{\bar{f}=\bar{e}\} \ALT e_1 \odot e_2\\ 
c & \in & \mathtt{Commands} & \coloneqq & \cskip \ALT \lete{x}{e}{c}
    \ALT \ite{e}{c_1}{c_2}\ALT c_1;c_2 \ALT \inserte{x}  \\
&&&&\ALT \deletee{\lambda x.e}
    \ALT \lete{y}{\selecte{\lambda x.e}}{c}
    \ALT \updatee{\lambda x.e_1}{\lambda x.e_2}\\
&&&&\ALT \foreache{x}{\lambda y.\lambda z. c} 
    \ALT \foreachr{s_1}{s_2}{\lambda x.\lambda y. e}\\
&&&&\ALT \ctxn{i}{\I}{ c } \ALT \ctxnr{i}{\I,\stl,\stg}{c} \ALT c1 || c2\\
% t & \in & \mathtt{Terms} & \coloneqq & e \ALT c\\
\ectx & \in & \mathtt{Eval\;Ctx} & ::= & \bullet \ALT  
  \bullet || c_2 \ALT c_1 || \bullet \ALT \bullet;\,c_2 
  \ALT \ctxnr{i}{\I,\stl,\stg}{\bullet} \\
\end{array}
\end{smathpar}
%
\bigskip

\renewcommand{\arraystretch}{1.2}

%
\textbf{Local Reduction} \quad 
\fbox {\(\stg \vdash (\tbox{c}_i,\stl) \stepsto (\tbox{c'}_i,\stl')\)}\\
%
\begin{minipage}{2.8in}
\rulelabel{E-Insert}
\begin{smathpar}
\begin{array}{c}
\RULE
{
  r.\idf \not\in \dom(\stl \cup \stg)\\
  r' = \{r \;\C{with}\; \txnf=i;\,\delf=\C{false}\}
}
{
  \stg \vdash (\tbox{\inserte{r}}_i,\stl) \stepsto
  (\tbox{\cskip}_i,\stl \cup \{r'\})
}
\end{array}
\end{smathpar}
\end{minipage}
%
%
\begin{minipage}{2.8in}
\rulelabel{E-Delete}
\begin{smathpar}
\begin{array}{c}
\RULE
{
  \hspace*{-1in}
  s = \{r' \,|\, \exists(r\in\Delta).~ \eval([r/x]e)=\C{true} \\
      \hspace*{0.1in}\conj r'=\{r \with \delf=\C{true};\, \txnf=i\}\}\spc
  \dom(\stl)\cap\dom(s) = \emptyset \spc
% \dom(s) \cap \dom(\delta) = \emptyset
}
{
  \stg \vdash (\tbox{\deletee{\lambda x.e}}_i,\stl) \stepsto 
  (\tbox{\cskip}_i,\stl \cup s)
}
\end{array}
\end{smathpar}
\end{minipage}
%
\bigskip

%
\begin{minipage}{2.8in}
\rulelabel{E-Select}
\begin{smathpar}
\begin{array}{c}
\RULE
{
  \\
  s = \{r\in\Delta \,|\, \eval([r/x]e)=\C{true}\}\spc
  c' = [s/y]c
}
{
  \stg \vdash (\tbox{\lete{y}{\selecte{\lambda x.e}}{c}}_i, \stl) \stepsto 
              (\tbox{c'}_i,\stl)
}
\end{array}
\end{smathpar}
\end{minipage}
%
%
\begin{minipage}{2.8in}
\rulelabel{E-Update}
\begin{smathpar}
\begin{array}{c}
\RULE
{
  \hspace*{-0.2in}s = \{r' \,|\, \exists(r\in\Delta).~ 
    \eval([r/x]e_2)=\C{true} \conj r'= \{ [r/x]e_1 \;\C{with}\\
    \idf=r.\idf;\,\txnf=i;\,\delf = r.\delf \}\} \spc
  \dom(\stl) \cap \dom(s) = \emptyset
}
{
  \stg \vdash (\tbox{\updatee{\lambda x.e_1}{\lambda x.e_2}}_i,\stl) 
      \stepsto (\tbox{\cskip}_i,\stl \cup s)
}
\end{array}
\end{smathpar}
\end{minipage}
%

\begin{smathpar}
\begin{array}{ll}
  \rulelabel{E-Foreach1} & \stg \vdash (\tbox{\foreache{s}{\lambda y.\lambda
    z.c}}_i,\stl) \stepsto (\tbox{\foreachr{\emptyset}{s}{\lambda
    y.\lambda z. c}}_i,\stl)\\
  \rulelabel{E-Foreach2} & \stg \vdash (\tbox{\foreachr{s_1}{\{r\} \uplus s_2}
    {\lambda y.\lambda z.c}}_i,\stl) \stepsto (\tbox{[r/z][s_1/y]c;\,
    \foreachr{s_1 \cup \{r\}}{s_2}{\lambda y.\lambda z. c}}_i,\stl)\\
  \rulelabel{E-Foreach3} & \stg \vdash (\tbox{\foreachr{s}{\emptyset}
    {\lambda y.\lambda z.c}}_i,\stl) \stepsto (\tbox{\cskip}_i,\stl)\\
\end{array}
\end{smathpar}
%
\bigskip

%
\textbf{Top-Level Reduction} \quad 
\fbox {\((c,\stg) \stepsto (c',\stg')\)}\\
%
\begin{minipage}{3in}
  \rulelabel{E-Txn-Start}
  \begin{smathpar}
  \begin{array}{c}
    \RULE{}
         {(\ctxn{i}{\I}{c},\stg) \stepsto (\ctxnr{i}{\I,\emptyset,\stg}{c},\stg)}
  \end{array}
  \end{smathpar}
\end{minipage}%
\hfill
\begin{minipage}{3in}
\rulelabel{E-Txn}
\begin{smathpar}
\begin{array}{c}
\RULE
{
  \I_e\,\,(\stl,\stg,\stg')\spc
  \stg \vdash (\tbox{c}_i,\stl) \stepsto (\tbox{c'}_i,\stl')
}
{
  (\ctxnr{i}{\I,\stl,\stg}{c},\stg') \stepsto
  (\ctxnr{i}{\I,\stl',\stg'}{c'},\stg')
}
\end{array}
\end{smathpar}
\end{minipage}\\

\begin{center}
\begin{minipage}{3in}
\rulelabel{E-Commit}
\begin{smathpar}
\begin{array}{c}
\RULE
{
  \I_c\,\,(\stl,\stg,\stg')
}
{
  (\ctxnr{i}{\I,\stl,\stg}{\cskip},\stg') \stepsto (\cskip,\stl\gg\stg')
}
\end{array}
\end{smathpar}
\end{minipage}
\end{center}
\hfill
%
\caption{\small \txnimp: Syntax and Small-step semantics}
\label{fig:txnimp}
\end{figure*}



As evident from the above example, a proof system for weak isolation
has to be flexible enough to admit different kinds of reasoning for
different isolation and consistency levels. One way this can be
achieved is by having a separate proof rule for each combination of
transaction isolation level and store consistency level, such that the
rule reflects the operational semantics of the isolation level on the
store. Given the large and ever-increasing diversity among isolation
and consistency semantics, this approach is unlikely to scale. Our
proof system avoids this problem by being parametric over the
semantics of isolation and consistency levels. This approach is based
on a couple of observations. First, as noted by some
authors~\cite{pldi15,gotsmanconcur15}, semantics of various isolation
and consistency levels can be captured as well-formedness conditions
over execution traces. If an execution trace satisfies specified
well-formedness conditions, then the program is considered to have
experienced the required levels of consistency and isolation in that
execution. Second, well-formedness conditions are often
\emph{prefix-complete}, meaning that every prefix of a well-formed
trace is also well-formed, or conversely, an ill-formed trace cannot
be extended to a well-formed one. Prefix-completeness allows us to
reason about programs operationally, in terms of an imaginary machine
whose primary artifact is an execution trace. The machine starts from
a well-formed trace, executes a read/write/commit operation, and
extends the trace while preserving well-formedness. If well-formedness
cannot be preserved by taking a step, the machine gets ``stuck''. In
executions that run to completion, the machine models a store whose
consistency and isolation guarantees are captured by the trace
well-formedness conditions. Thus, to show that program's invariants
are preserved under weak isolation, it suffices to show that they are
preserved in all executions of the imaginary machine that do not get
stuck. This is indeed how our proof system facilitates program
verification. The ``invalid interference'' described informally before
corresponds to a case of machine getting stuck due to the violation of
trace well-formedness conditions. The proof system inherits the
parametricity of the machine over trace well-formedness conditions,
leading to a generic set of parameterized proof rules that can be
instantiated for various store consistency and transaction isolation
levels.


