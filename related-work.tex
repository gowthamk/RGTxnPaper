\vspace*{-6pt}
\section{Related Work}
\label{sec:relatedwork}

\paragraph{Specifying weak isolation.}
Adya~\cite{adyaphd} specifies several weak isolation levels in terms
of \emph{dependency graphs} between transactions, and the kinds of
dependencies that are forbidden in each case. The operational nature
of Adya's specifications make them suitable for runtime monitoring and
anomaly detection~\cite{kemmevldb,feketesigmod08,pssi2011}, whereas
the declarative nature of our specifications make them suitable for
formal reasoning about program behaviour. Sivaramakrishnan \emph{et
al.}~\cite{pldi15} specify isolation levels declaratively as trace
well-formedness conditions, but their specifications implicitly assume
a complete trace with only committed transactions, making it difficult
to reason about a program as it builds the trace. Cerone \emph{et
al.}~\cite{gotsmanconcur15} specify isolation levels with atomic
visibility, but their specifications are also for complete traces.
Like ours, all the aforementioned specification frameworks use the
vocabulary introduced in~\cite{burckhardt14}. However, none of them
are equipped with a reasoning framework that can use such
specifications to verify programs under weak isolation.

\vspace*{-4pt}
\paragraph{Reasoning under weak isolation} In~\cite{feketessi}, Fekete
\emph{et al.} propose a theory to characterize non-serializable
executions that arise under {\sc si}. Fekete~\cite{fekete2005} also
proposes an algorithm that allocates either {\sc si} or {\sc ser}
isolation levels to transactions while guaranteeing
serializability. In~\cite{gotsmanpodc16}, Cerone \emph{et al.} improve
on Adya's {\sc si} specification and use it to derive a static
analysis that determines the safety of substituting {\sc si} with a
weaker variant called \iso{Parallel Snapshot Isolation}~\cite{psi}.
These efforts focus on establishing the equivalence of executions
between a pair of isolation levels, without taking application
invariants into account.  Bernstein \emph{et al.}~\cite{bern2000}
propose informal semantic conditions to ensure the satisfaction of
application invariants under weaker isolation levels.  All these
techniques are tailor-made for a finite set of well-understood
isolation levels (rooted in~\cite{berenson}) under a pre-defined store
consistency model.

\vspace*{-4pt}
\paragraph{Reasoning under weak consistency} There have been several
recent proposals to reason about programs executing under weak
consistency~\cite{bailisvldb, alvarocalm, gotsmanpopl16,redblueatc,
  redblueosdi, ecinec}. All of them assume a system model that offers
a choice between a \emph{coordination-free} weak consistency level
(\emph{e.g.}, eventual consistency~\cite{redblueosdi, redblueatc,
  ecinec, alvarocalm, bailisvldb}) or causal
consistency~\cite{lbc16,gotsmanpopl16}). All these efforts involve
proving that atomic and fully isolated operations preserve application
invariants when executed under these consistency levels.  In contrast,
we admit weakly-isolated transactions, and our system model accepts
\emph{specifications} of consistency and isolation levels drawn from
an expressive logic.  Gotsman \emph{et al.}~\cite{gotsmanpopl16}
adapt \iso{Parallel Snapshot Isolation} to a transaction-less setting
by interpreting it as a consistency level that serializes writes to
objects; a dedicated proof rule is developed to help prove prove
program invariants hold under this model. By parameterizing our proof
system over a gamut of weak isolation specifications, we avoid the
need to define a separate proof rule for each new isolation level we
may encounter.

\vspace*{-4pt}
\paragraph{Reasoning under relaxed memory} The reasoning mechanisms
used to describe and prove properties about weakly-isolated
transactions bear some resemblance to those used to formalize relaxed
memory behaviour~\cite{battycpp}.  Ridge~\cite{rgtso} generalizes
rely-guarantee reasoning to the x86-TSO memory model.  Likewise,
Vafeiadis \emph{et al.}~\cite{rsl13} generalize concurrent separation
logic (CSL)~\cite{csl} to the C11 relaxed memory model.  Ferreira
\emph{et al.}~\cite{ferreira10} propose a parameterized operational
semantics for relaxed memory models, but the parameterization is over
a relation between relaxed memory programs and related {\sc sc}
programs. Demange \emph{et al.}~\cite{DLZ+13} present a \emph{buffered
  memory model} for Java that defines an axiomatic definition for the
JMM in terms of memory reorderings, and an operational instantiation
consistent with the TSO memory model.

