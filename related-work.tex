\vspace*{-6pt}
\section{Related Work}
\label{sec:relatedwork}

\paragraph{Specifying weak isolation.}
Adya~\cite{adyaphd} specifies several weak isolation levels in terms
of \emph{dependency graphs} between transactions, and the kinds of
dependencies that are forbidden in each case. The operational nature
of Adya's specifications make them suitable for runtime monitoring and
anomaly detection~\cite{kemmevldb,feketesigmod08,pssi2011}, whereas
the declarative nature of our specifications make them suitable for
formal reasoning about program behavior. Sivaramakrishnan \emph{et
al.}~\cite{pldi15} specify isolation levels declaratively as trace
well-formedness conditions, but their specifications implicitly assume
a complete trace with only committed transactions, making it difficult
to reason about a program as it builds the trace. Cerone \emph{et
al.}~\cite{gotsmanconcur15} specify isolation levels with atomic
visibility, but their specifications are also for complete traces.
Like ours, all the aforementioned specification frameworks use the
vocabulary introduced in~\cite{burckhardt14}. However, none of them
are equipped with a reasoning framework that can use such
specifications to verify programs under weak isolation.

Recent work by Crooks \emph{et al.}~\cite{CPA+17} also explores the
use of a state-based interpretation of isolation as we do, and share
our position that enables specification of weak isolation that are not
tied to implementation-specific artifacts.  However, they do not
consider verification (manual or automated) of client programs, and it
is not immediately apparent if their specification formalism is
amenable for use within a verification toolchain. 

\vspace*{-4pt}
\paragraph{Reasoning under weak isolation.} In~\cite{feketessi}, Fekete
\emph{et al.} propose a theory to characterize non-serializable
executions that arise under {\sc si}. Fekete~\cite{fekete2005} also
proposes an algorithm that allocates either {\sc si} or {\sc ser}
isolation levels to transactions while guaranteeing
serializability. In~\cite{gotsmanpodc16}, Cerone \emph{et al.} improve
on Adya's {\sc si} specification and use it to derive a static
analysis that determines the safety of substituting {\sc si} with a
weaker variant called \iso{Parallel Snapshot Isolation}~\cite{psi}.
These efforts focus on establishing the equivalence of executions
between a pair of isolation levels, without taking application
invariants into account.  Bernstein \emph{et al.}~\cite{bern2000}
propose informal semantic conditions to ensure the satisfaction of
application invariants under weaker isolation levels.  All these
techniques are tailor-made for a finite set of well-understood
isolation levels (rooted in~\cite{berenson}) under a pre-defined store
consistency model.

\vspace*{-4pt}
\paragraph{Reasoning under weak consistency.} There have been several
recent proposals to reason about programs executing under weak
consistency~\cite{bailisvldb, alvarocalm, gotsmanpopl16,redblueatc,
redblueosdi, ecinec}. All of them assume a system model that offers a
choice between a \emph{coordination-free} weak consistency level
(\emph{e.g.}, eventual consistency~\cite{redblueosdi, redblueatc,
ecinec, alvarocalm, bailisvldb}) or causal
consistency~\cite{lbc16,gotsmanpopl16}). All these efforts involve
proving that atomic and fully isolated operations preserve application
invariants when executed under these consistency levels.  In contrast,
our focus in on reasoning in the presence of weakly-isolated
transactions under a strongly consistent store.  Gotsman \emph{et
al.}~\cite{gotsmanpopl16} adapt \iso{Parallel Snapshot Isolation} to a
transaction-less setting by interpreting it as a consistency level
that serializes writes to objects; a dedicated proof rule is developed
to help prove prove program invariants hold under this model. By
parameterizing our proof system over a gamut of weak isolation
specifications, we avoid the need to define a separate proof rule for
each new isolation level we may encounter.

%% \vspace*{-4pt}
%% \paragraph{Reasoning under relaxed memory.} The reasoning mechanisms
%% used to describe and prove properties about weakly-isolated
%% transactions bear some resemblance to those used to formalize relaxed
%% memory behaviour~\cite{battycpp}.  Ridge~\cite{rgtso} generalizes
%% rely-guarantee reasoning to the x86-TSO memory model.  Likewise,
%% Vafeiadis \emph{et al.}~\cite{rsl13} generalize concurrent separation
%% logic (CSL)~\cite{csl} to the C11 relaxed memory model.  Ferreira
%% \emph{et al.}~\cite{ferreira10} propose a parameterized operational
%% semantics for relaxed memory models, but the parameterization is over
%% a relation between relaxed memory programs and related {\sc sc}
%% programs. Demange \emph{et al.}~\cite{DLZ+13} present a \emph{buffered
%%   memory model} for Java that defines an axiomatic definition for the
%% JMM in terms of memory reorderings, and an operational instantiation
%% consistent with the TSO memory model.

\vspace*{-4pt}
\paragraph{Inference.}  Prior work by Vafeiadis~\cite{Vaf10,Vaf10a}
describe \emph{action inference}, an inference procedure for computing
rely and guarantee relations in the context of RGSep~\cite{VP07}, an
integration of rely-guarantee and separation logic~\cite{Rey02} that
allows one to precisely reason about local and shared state of a
concurrent program. The ideas underlying action inference have been
used to prove memory safety, linearizability, shape invariant
inference, etc.  of fine-grained concurrent data structures.  While
our motivation is similar (automated inference of intermediate
assertions and local invariants), the context of study (transactions
vs. shared-memory concurrency), the objects being analyzed (relational
database tables vs. concurrent data structures), the properties being
verified (integrity constraints over relational tables vs. memory
safety, or linearizability of concurrent data structure operations)
and the analysis technique used to drive inference (state transformers
vs. abstract interpretation) are quite different.

\section{Conclusions}
\label{sec:conclusions}

To improve performance, modern database systems employ techniques that
weaken the strong isolation guarantees provided by serializable
transactions in favor of alternatives that allow a transaction to
witness the effects of other concurrently executing transactions that
happen commit during its execution.  Typically, it is the
responsibility of the database programmer to determine if an available
weak isolation level would violate a transaction's consistency
constraints.  Although this has proven to be a difficult and
error-prone process, there has heretofore been no attempt to formalize
notions of weak isolation with respect to application semantics, or
consider how we might verify the correctness of database programs that
use weakly-isolated transactions.  In this paper, we provide such a
formalization.  We develop a rely-guarantee proof framework cognizant
of weak isolation semantics, and build on this foundation to devise an
inference procedure that facilitates automated verification of
weakly-isolated transactions, and have applied our ideas on
widely-used database systems to justify their utility.  Our solution
enables database applications to leverage the performance advantages
offered by weak isolation, without compromising safety and
correctness.




