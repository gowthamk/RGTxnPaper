\section{Implementation}
\label{sec:implementation}

\begin{figure}
\begin{ocaml}
type table_name =  District | Order | Order_line | Stock

type district = {d_id: int; d_next_o_id: int}
type order = {o_id: int; o_d_id: int; o_c_id: int; o_ol_cnt: int}
type order_line = {ol_o_id: int; ol_d_id: int; ol_i_id: int; ol_qty: int}
type stock = {s_i_id: int; s_d_id:int; s_qty: int}
\end{ocaml}
\caption{OCaml type definitions corresponding to the TPC-C schema from
Fig.~\ref{fig:schema}}
\label{fig:ocaml-schema}
\vspace*{-10pt}
\end{figure}

We have implemented our DSL to define transactions as monadic
computations in OCaml (modulo some syntactic sugar), and our automatic
reasoning framework as an extra frontend pass (called \thetool) in the
ocamlc 4.03 compiler\footnote{The source code is available at
available at \url{https://github.com/gowthamk/acidifier}}. The input
to \thetool is a program in our DSL that describes the schema of the
database as a collection of OCaml type definitions, and transactions
as OCaml functions, whose top-level expression is an application of
the \C{atomically\_do} combinator. For instance, TPC-C's schema from
Fig.~\ref{fig:schema} can be described via the OCaml type definitions
shown in Fig.~\ref{fig:ocaml-schema}.  \thetool also requires a
specification of the program in the form of a collection of guarantees
($G$), one per transaction, and an invariant $I$ that is a conjunction
of the integrity constraints on the database. An auxiliary DSL that
includes the first-order logic (FOL) combinators has been implemented
for this purpose. \thetool's verification pass follows OCaml's type
checking pass, hence the concrete artifact of verification is OCaml's
typed AST. The tool is already equipped with  an axiomatization of
PostgreSQL and MySQL's isolation levels expressed in our FOL DSL.
Other data stores can be similarly axiomatized. The concrete result of
verification is an assignment of an isolation level of the selected
data store to each transaction in the program.

At the top-level, \thetool runs a loop that picks an unverified
transaction and progressively strengthens its isolation level until it
passes verification. If the selected data store provides a
serializable isolation level, and if the program is sequentially
correct, then the verification is guaranteed to succeed. Within the
loop, \thetool first computes the various rely relations needed for
verification ($R$, $\R_l$, and $\R_c$). It then traverses the AST of a
transaction, applying the inference rules to construct a state
transformer, checks its stability, and weakens it ($\stabilize{\cdot}$)
if it is not stable. The result of traversing the transaction's AST is
therefore a state transformer ($\F$) that is stable w.r.t $\R_l$, which
is also stabilized against $\R_c$ (using $\stabilize{\cdot}$), and
then checked against the transaction's stated guarantee ($G$). If the
check passes, then the guarantee is verified to check if it preserves
the invariant $I$. The successful result from both checks results
in the transaction being certified correct under the current choice of
its isolation level. Successful verification of all transactions
concludes the top-level execution, returning the inferred isolation
levels as its output.  \thetool uses the Z3 SMT solver as its underlying reasoning engine. Each
implication check described above is first encoded in FOL, applying
the translation described in \S\ref{sec:inference} wherever
necessary.

\subsection{Pragmatics}

\textbf{Real-World Isolation Levels} The axiomatization of the
isolation levels presented in \S\ref{sec:isolation} leaves out
certain nuances of their implementations on real data stores, which
need to be taken into account for verification to be effective in
practice.  We take these into account while linking \thetool with
store-specific semantics (isolation specifications, etc.). As an
example, consider how PostgreSQL implements an \C{UPDATE} operation.
\C{UPDATE} first selects the records that meet the search criteria
from the snapshot against which it is executing (the snapshot is
established at the beginning of the transaction if the isolation level
is SI, or at the beginning of the \C{UPDATE} statement if the
isolation level is RC). The selected records are then visited in the
actual database (if they still exist), write locks are obtained, and
the update is performed, provided that each matched record still meets
\C{UPDATE}'s search criteria. If a record no longer meets the
search criteria (due to a concurrent update), it is excluded
from the update, and the write lock is immediately released.
Otherwise, the record remains locked until the transaction commits. 

Clearly, this sequence of events is not atomic, unlike the assumption
made by our formal model because  the implementation admits interference
between the updates of individual records that meet the search
criteria.  Nonetheless, through a series of relatively straightforward
deductions, we can show that PostgreSQL's \C{UPDATE} is in fact
equivalent (in behavior) to a sequential composition of two atomic
operations $c_1;c_2$, where $c_1$ is effectively a \C{SELECT}
operation with the same search criteria as \C{UPDATE}, and $c_2$ is
a slight variation of the original \C{UPDATE} that updates a
record only if a record with the same id is present in the set of records
returned by \C{SELECT}: % This transformation is summarized below:
\begin{smathpar}
\begin{array}{lcl}
\updatee{(\lambda x. ~e_1)}{(\lambda x.~e_2)}
&
\longrightarrow
&
\lete{y}{\selecte{(\lambda x.~e_1})}
     {\updatee{(\lambda x.~e_1 \wedge x.\idf\in\dom(y))}
              {(\lambda x.~e_2})}\\
\end{array}
\end{smathpar}
The intuition behind this translation is the observation that all
interferences possible during the execution of the \C{UPDATE} can be
accommodated between the time the records are selected from the
snapshot, and the time they are actually updated.  \thetool performs this
translation if the selected store is PostgreSQL, allowing it to reason
about \C{UPDATE} operations in a way that is faithful to its semantics
on PostgreSQL. \thetool also admits similar compensatory logic for
certain combinations of isolation levels and operations on MySQL.

\textbf{Set functions} SQL's \C{SELECT} query admits projections of
record fields, and also application of auxiliary functions such as
\C{MAX} and \C{MIN}, e.g., \C{SELECT MAX(ol\_o\_id) FROM
Order\_line WHERE $\ldots$}, etc. We admit such extensions as set functions
in our DSL (e.g., \C{project}, \C{max}, \C{min}), and axiomatize their
behavior. For instance:
\begin{smathpar}
\begin{array}{lcl}
  s_2 \;=\;\C{project}\,s_1\,(\lambda z.~e) & \Leftrightarrow &
  \forall y.~y\in s_2 \Leftrightarrow  \exists(x \in s_1).~y = [x/z]e\\
  x \;=\; \C{max}\,s & \Leftrightarrow & x \in s \conj \forall(y \in
  s).~y\le x\\
\end{array}
\end{smathpar}
There are however certain set functions whose behavior cannot be
completely axiomatized in FOL. These include \C{sum}, \C{count} etc.
For these, we admit imprecise axiomatizations. % expressed as lemmas over these functions.

\textbf{Annotation Burden} \thetool significantly reduces the
annotation burden in verifying a weakly isolated transactions by
eliminating the need to annotate intermediate assertions and loop
invariants.  Guarantees ($G$) and global invariants ($I$), however,
still need to be provided. Alternatively, a weakly isolated
transaction $T$ can be verified against a generic serializability
condition,  eliminating the need for guarantee annotations. In this
mode, \thetool first infers the transformer $\F_{SER}$ of $T$ without
considering any interference, which then becomes its guarantee ($G$).
Doing likewise for every transaction results in a rely relation ($R$)
that includes $\F_{SER}$ of every transaction. Verification now
proceeds by taking interference into account, and verifying that each
transaction still yields the same $F$ as its $F_{SER}$. The result of
this verification is an assignment of (possibly weak) isolation levels
to transactions which nonetheless guarantees behavior equivalent to a
serializable execution.


\section{Evaluation}
\label{sec:case-studies}

In this section, we present our experience in running \thetool on two
different applications: \emph{Courseware}: a course registration
system described by~\cite{gotsmanpopl16}, and the TPC-C benchmark.

\begin{figure}[!t]
\begin{ocaml}
  type table_name = Student | Course | Enrollment
  type student = {s_id: id; s_name: string}
  type course = {c_id: id; c_name: string; c_capacity: int}
  type enrollment = {e_id: id; e_s_id: id; e_c_id: id}

  let enroll_txn sid cid = 
    let crse = SQL.select1 [Course] (fun c -> c.c_id = cid) in
    let s_c_enrs = SQL.select [Enrollment] (fun e -> e.e_s_id = sid && 
                                                     e.e_c_id = cid) in
    if crse.c_capacity > 0 && Set.is_empty s_c_enrs then
      (SQL.insert Enrollment {e_id=new_id (); e_s_id=sid; e_c_id=cid};
       SQL.update Course (fun c -> {c with c_capacity = c.c_capacity - 1})
                         (fun c -> c.c_id = cid))
    else ()

  let deregister_txn sid = 
    let s_enrs = SQL.select [Enrollment] (fun e -> e.e_s_id = sid) in
    if Set.is_empty s_enrs then
      SQL.delete Student (fun s -> s.s_id = sid)
    else ()
\end{ocaml}
\caption{Courseware Application}
\label{fig:courseware_code}
\end{figure}

\textbf{Courseware} The Courseware application allows new courses to be
added (via an \C{add\_course} transaction), and new students to be
registered (via a \C{register} transaction) into a database. A registered
student can enroll (\C{enroll}) in an existing course,
provided that enrollment has not already exceeded the course
capacity (\C{c\_capacity}). A course with no enrollments can be
canceled (\C{cancel\_course}). Likewise, a student who is not enrolled
in any course can be deregistered (\C{deregister}). Besides
\C{Student} and \C{Course} tables, there is also an \C{Enrollment}
table to track the many-to-many enrollment relationship between
courses and students. The simplified code for the Courseware
application with only \C{enroll} and \C{deregister}
transactions is shown in Fig.~\ref{fig:courseware_code}. The
application is required to preserve the following invariants on the
database:

\begin{enumerate}
\item  $I_1$: An enrollment record should always refer to an existing student and an existing course.
\item  $I_2$: The capacity (\C{c\_capacity}) of a course should always be a
  non-negative quantity.
\end{enumerate}
\noindent Both $I_1$ and $I_2$ can be violated under weak isolation.
$I_1$ can be violated, for example, when \C{deregister} runs
concurrently with \C{enroll}, both at RC isolation. While the former
transaction removes the student record after checking that no
enrollments for that student exists, the latter transaction
concurrently adds an enrollment record after checking the student
exists.  Both can succeed concurrently, resulting in an invalid state.
Invariant $I_2$ can be violated by two \C{enroll}s, both reading
\C{c\_capacity}=1, and both (atomically) decrementing it, resulting in
\C{c\_capacity}=-1.  We ran \thetool on the Courseware application
(Fig.~\ref{fig:courseware_code}) after annotating transactions with
their respective guarantees, and asserting $I = I_1 \wedge I_2$ as the
correctness condition. The guarantees $G_e$ and $G_d$ for \C{enroll}
and \C{deregister} transactions, respectively, are shown below:
\begin{smathpar}
  \begin{array}{lcl}
    G_e(\stg,\stg') & \Leftrightarrow & \stg_s'=\stg_s
      \conj \exists\C{cid}.\exists\C{sid}.\\
    & & \hspace*{0.3in} \stg_c' = \stg_c \bind \lambda c.~
        \itel{c.\C{c\_id}=\C{cid}\\
    & & \hspace*{1.15in}}
          {\existsl(c',~c'.\C{id}=c.\C{id} \wedge
              c'.\C{c\_name}=c.\C{c\_name} \\
    & & \hspace*{2.6in}\wedge~ c'.\C{c\_capacity}\ge0, ~\{c'\})\\
    & & \hspace*{1.15in}}
          {\{c\}}\\
    & & \hspace*{0.15in}\conj \stg_e = \stg_e' \bind \lambda e.~ 
        \itel{e.\C{e\_c\_id}=\C{cid} \wedge e.\C{e\_s\_id}=\C{sid}}
          {\emptyset}{\{e\}}\\
%   & & \hspace*{1.15in}} 
%         {\emptyset} {\{e\}}\\
    G_d(\stg,\stg') & \Leftrightarrow & \stg_c' = \stg_c \conj
      \stg_e' = \stg_e \conj \exists \C{sid}.~
      \itel{\forall(e \in \stg_e).~e.\C{e\_s\_id}\neq\C{sid}\\
%   & & \hspace*{0.3in} \itel{\forall(e \in
%           \stg_e).~e.\C{e\_s\_id}\neq\C{sid}\\
    & & \hspace*{1.55in}}
          {\stg_s' = \stg_s \bind \lambda s.~
           \itel{s.\C{id}=\C{sid}}{\emptyset}{\{s\}}\\
    & & \hspace*{1.55in}}
          {\stg_s' = \stg_s}\\
  \end{array}
\end{smathpar}
For the sake of this presentation we split $\stg$ into three disjoint
sets of records, $\stg_s$, $\stg_c$, and $\stg_e$, standing for
\C{Student}, \C{Course}, and \C{Enrollment} tables, respectively.
Observing that the set language $\SL$ (Sec.~\ref{sec:inference}),
besides being useful for automatic verification, also facilitates
succinct expression of transaction semantics, we define $G_e$ and
$G_d$ in a combination of FOL and $\SL$. $G_e$ essentially says that
the \C{enroll} transaction leaves the \C{Student} table unchanged,
while it may update the \C{c\_capacity} field of a \C{Course} record
to a non-negative value (even when it doesn't update, it is the case
that $c'.\C{c\_capacity}\ge0$, because $c'=c$, and $c\in\stg_c$, and
we know that $I_2(\stg_c)$). $G_e$ also conveys that \C{enroll} might
insert a new \C{Enrollment} record by stating that $\stg_e$, the
\C{Enrollment} table in the pre-state, contains all records $e$ from
$\stg_e'$, the table in the post-state, except when $e.\C{e\_c\_id}$ and
$e.\C{e\_s\_id}$ match \C{cid} and \C{sid}, respectively. The
guarantee $G_d$ of \C{deregister} asserts that the
transaction doesn't write to \C{Course} and \C{Enrollment} tables. The
transaction might however delete a \C{Student} record bearing an
\C{id}=\C{sid} (formally, $\stg_s' = \stg_s \bind \lambda s.~
\itel{s.\C{id}=\C{sid}}{\emptyset}{\{s\}}$), for some \C{sid} for
which no corresponding \C{Enrollment} records are present in the
pre-state (in other words, $\forall(e \in
\stg_e).~e.\C{e\_s\_id}\neq\C{sid}$).

With help of the guarantees, such as those described above, \thetool
was able to automatically discover the aforementioned anomalous
executions, and was subsequently able to infer that the anomalies can
be preempted by promoting the isolation level of \C{enroll} and
\C{deregister} to SER (on both MySQL and PostgreSQL), leaving
the isolation levels of remaining transactions at RC. The total time
for inference and verification took less than a minute running on a
conventional laptop.

\begin{table}[]
\centering
\begin{tabular}{l|c|c|c|c|c|}
\cline{2-6}
                                 & \multicolumn{1}{l|}{\C{new\_order}} & \multicolumn{1}{l|}{\C{delivery}} & \multicolumn{1}{l|}{\C{payment}} & \multicolumn{1}{l|}{\C{order\_status}} & \multicolumn{1}{l|}{\C{stock\_level}} \\ \hline
\multicolumn{1}{|l|}{MySQL}      & SER                                   & SER                                 & RC                                 & RC                                       & RC                                      \\ \hline
\multicolumn{1}{|l|}{PostgreSQL} & SI                                    & SI                                  & RC                                 & RC                                       & RC                                      \\ \hline
\end{tabular}
\caption{The discovered isolation levels for TPC-C transactions}
\label{tab:tpcc}
\end{table}

\textbf{TPC-C} The simplified schema of the TPC-C benchmark has been
described in Sec.~\ref{sec:motivation}. In addition to the tables
shown in Fig.~\ref{fig:schema}, the TPC-C schema also has
\C{Warehouse} and \C{New\_order} tables that are relevant for
verification.  To verify TPC-C, we examined four of the twelve
consistency conditions specified by the standard, which we name $I_1$
to $I_4$:

\begin{enumerate}
\item Consistency condition $I_1$  requires that the sales bottom line
of each warehouse equals the sum of the sales bottom lines of all
districts served by the warehouse.

\item Conditions $I_2$ and $I_3$ effectively enforce uniqueness of ids assigned
  to \C{Order} and \C{New\_order} records, respectively, under a district.


\item  Condition $I_4$ requires that the number of order lines under a district
  must match the sum of order line counts of all orders under the district.
% There are five transactions in TPC-C, out of which two are read-only.
% The three remaining transactions are \C{new\_order}, \C{delivery}, and
  % \C{payment}.
\end{enumerate}

Similar to the example discussed in Sec.~\ref{sec:motivation}, there
are a number of ways TPC-C's transactions violate the aforementioned
invariants under weak isolation. \thetool was able to discover all such
violations when verifying the benchmark against $I =
\bigwedge_{i}I_i$, with guarantees of all three transactions
provided. The isolation levels were subsequently strengthened  as
shown in Table.~\ref{tab:tpcc}.  As before, inference and verification
took less than a minute.

%% \begin{table*}[t]\small
%% \centering
%% \begin{tabular}{|c|c|c|c|}
%%   \hline
%% \textbf{Transaction}   & \textbf{Invariant} 
%% & \textbf{MySQL-Isolation} & \textbf{PostgreSQL-Isolation} \\ 
%% \hline
%% \multirow{4}{*}{\C{New\_Order} }  & $I_1$ 
%% & RC &  RC\\ 
%% &  $I_2$ &SER & RR \\
%% &  $I_3$ & SER  &  RR  \\
%% & $I_4$ & RC & RC   \\
%% \hline
%% \multirow{4}{*}{\C{Payment}}  & $I_1 $ 
%% & RC &  RC\\ 
%% &  $I_2$  &RC & RC \\
%% &  $I_3 $ & RC  &  RC  \\
%% & $I_4$  & RC & RC   \\
%% \hline
%% \multirow{4}{*}{\C{Delivery}}  & $I_1$  
%% & RC &  RC \\ 
%% &  $I_2$ &SER & RR \\
%% &  $I_3$ & SER  &  RR \\
%% & $I_4$  & RC & RC   \\
%% \hline
%% \end{tabular}
%% \caption{Various invariant violations witnessed for the TPC-C
%%   benchmark on MySQL and PostgreSQL}
%% \label{tab:tpcc-eval}
%% \end{table*}

To sanity-check the results of \thetool, we conducted experiments with a
high-contention OLTP workload on TPC-C aiming to explore the space of
correct isolation levels for different transactions. The workload
involves a mix of all five TPC-C transactions executing against a
TPC-C database with 10 warehouses. Each warehouse has 10 districts,
and each district serves 3000 customers. There are a total of 5
transactions in TPC-C, and given that MySQL and PostgreSQL support 3
isolation levels each, there are a total of $3^5 = 243$ different
configurations of isolation levels for TPC-C transactions on MySQL and
PostgreSQL. We executed the benchmark with all 243 configurations, and
found 171 of them violated at least one of the four invariants we
considered.  As expected, the isolation levels that \thetool infers for the
TPC-C transactions do not result in invariant violations, either on
MySQL or on PostgreSQL, and were determined to be the weakest safe
assignments possible.


