\section{Implementation}
\label{sec:implementation}

\begin{figure}
\begin{ocaml}
type table_name =  District | Order | Order_line | Stock

type district = {d_id: int; d_next_o_id: int}
type order = {o_id: int; o_d_id: int; o_c_id: int; o_ol_cnt: int}
type order_line = {ol_o_id: int; ol_d_id: int; ol_i_id: int; ol_qty: int}
type stock = {s_i_id: int; s_d_id:int; s_qty: int}
\end{ocaml}
\caption{OCaml type definitions corresponding to the TPC-C schema from
Fig.~\ref{fig:schema}}
\label{fig:ocaml-schema}
\end{figure}

We have implemented our DSL to define transactions as monadic
computations in OCaml (modulo some syntactic sugar), and our automatic
reasoning framework as an extra frontend pass (called \tool) in the
ocamlc 4.03 compiler\footnote{The source code is available at
available at \emph{https://goo.gl/nmUXZK}}. The input to \tool is a
program in our DSL that describes the schema of the database as a
collection of OCaml type definitions, and transactions as OCaml
functions, whose top-level expression is an application of the
\C{atomically\_do} combinator. For instance, TPC-C's schema from
Fig.~\ref{fig:schema} can be described via the OCaml type definitions
shown in Fig.~\ref{fig:ocaml-schema}.  \tool also requires a
specification of the program in the form of a collection of guarantees
($G$), one per transaction, and an invariant $I$ that is a conjunction
of the integrity constraints on the database. An auxiliary DSL that
includes the first-order logic (FOL) combinators has been implemented
for this purpose. \tool's verification pass follows OCaml's type
checking pass, hence the concrete artifact of verification is OCaml's
typed AST. The tool is already equipped with  an axiomatization of
PostgreSQL and MySQL's isolation levels expressed in our FOL DSL.
Other data stores can be similarly axiomatized. The concrete result of
verification is an assignment of an isolation level of the selected
data store to each transaction in the program.

At the top-level, \tool runs a loop that picks an unverified
transaction and progressively strengthens its isolation level until it
passes verification. If the selected data store provides a
serializable isolation level, and if the program is sequentially
correct, then the verification is guaranteed to succeed. Within the
loop, the \tool first computes the various rely relations needed for
verification ($R$, $\R_l$, and $\R_c$). It then traverses the AST of a
transaction, applying the inference rules to construct a state
transformer, check its stability, and weaken it ($\stabilize{\cdot}$)
if it is not stable. The result of traversing the transaction's AST is
therefore a state transformer ($\F$) that is stable w.r.t $\R_l$, which
is also stabilized against $\R_c$ (using $\stabilize{\cdot}$), and
then checked against the transaction's stated guarantee ($G$). If the
check passes, then the guarantee is verified to check if it preserves
the invariant $I$. The successful result from both the checks results
in the transaction being certified correct under the current choice of
its isolation level. Successful verification of all transactions
concludes the top-level execution, returning the inferred isolation
levels as its output.

\tool uses the Z3 SMT solver as its underlying reasoning engine. Each
implication check described above is first encoded in FOL, applying
the translation described in \S\ref{sec:inference} wherever
necessary.

\subsection{Pragmatics}

\textbf{Real-World Isolation Levels} The axiomatization of the
isolation levels presented in \S\ref{sec:isolation} leave out
certain nuances of their implementations on real data stores, which
need to be taken into account for verification to be effective in
practice.  We consider these into account while linking \tool with
store-specific semantics (isolation specifications etc). As an
example, consider how PostgreSQL implements an \C{UPDATE} operation.
\C{UPDATE} first selects the records that meet the search criteria
from the snapshot against which it is executing (the snapshot is
established at the beginning of the transaction if the isolation level
is SI, or at the beginning of the \C{UPDATE} statement if the
isolation level is RC). The selected records are then visited in the
actual database (if they still exist), write locks are obtained, and
the update is performed, provided that the record still meets
\C{UPDATE}'s search criteria. If the record no longer meets the
search criteria (due to a concurrent update) the record is excluded
from the update, and the write lock is immediately released.
Otherwise, the record remains locked until the transaction commits. 

Clearly, this sequence of events is not atomic, unlike the assumption
made by our formal model.  The implementation admits interference
between the updates of individual records that meet the search
criteria.  Nonetheless, through a series of relatively straightforward
deductions, we can show that PostgreSQL's \C{UPDATE} is in fact
equivalent (in behavior) to a sequential composition of two atomic
operations $c_1;c_2$, where $c_1$ is effectively a \C{SELECT}
operation with the same search criteria as \C{UPDATE}, and $c_2$ is
a slight variation of the original \C{UPDATE} that updates a
record only if a record with the same id is present in the set of records
returned by the \C{SELECT}. This transformation is summarized below:
\begin{smathpar}
\begin{array}{lcl}
\updatee{(\lambda x. ~e_1)}{(\lambda x.~e_2)}
&
\longrightarrow
&
\lete{y}{\selecte{(\lambda x.~e_1})}
     {\updatee{(\lambda x.~e_1 \wedge x.\idf\in\dom(y))}
              {(\lambda x.~e_2})}\\
\end{array}
\end{smathpar}
The intuition behind this translation is the observation that all
interferences possible during the execution of the \C{UPDATE} can be
accommodated between the time the records are selected from the
snapshot, and the time they are actually updated.  \tool performs this
translation if the selected store is PostgreSQL, allowing it to reason
about \C{UPDATE} operations in a way that is faithful to its semantics
on PostgreSQL. \tool also admits similar compensatory logic for
certain combinations of isolation levels and operations on MySQL.

\textbf{Set functions} SQL's \C{SELECT} query admits projections of
record fields, and also application of auxiliary functions such as
\C{MAX} and \C{MIN}, e.g., \C{SELECT MAX(ol\_o\_id) FROM
Order\_line WHERE $\ldots$}. We admit such extensions as set functions
in our DSL (e.g., \C{project}, \C{max}, \C{min}), and axiomatize their
behavior. For instance:
\begin{smathpar}
\begin{array}{lcl}
  s_2 \;=\;\C{project}\,s_1\,(\lambda z.~e) & \Leftrightarrow &
  \forall y.~y\in s_2 \Leftrightarrow  \exists(x \in s_1).~y = [x/z]e\\
  x \;=\; \C{max}\,s & \Leftrightarrow & x \in s \conj \forall(y \in
  s).~y\le x\\
\end{array}
\end{smathpar}
There are however certain set functions whose behavior cannot be
completely axiomatized in FOL. These include \C{sum}, \C{count} etc.
For these, we admit imprecise axiomatizations expressed as lemmas
over these functions.

\textbf{Annotation Burden} \tool significantly reduces the annotation
burden in verifying a weakly isolated transactions by eliminating the
need to annotate intermediate assertions and loop invariants.
Guarantees ($G$) and global invariants ($I$), however, still need to be
provided. Alternatively, a weakly isolated transaction $T$ can be
verified against a generic serializability condition,  eliminating
the need for any annotation. In this mode, \tool first infers the
transformer $\F_{SER}$ of $T$ without considering any interference,
which then becomes its guarantee ($G$). Doing likewise for every
transaction results in a rely relation ($R$) that includes $\F_{SER}$
of every transaction. Verification now proceeds by taking interference
into account, and verifying that each transaction still yields the
same $F$ as its $F_{SER}$. The result of this verification is an
assignment of (possibly weak) isolation levels to transactions which
nonetheless guarantees behavior equivalent to a  serializable execution.


\subsection{Evaluation}


To explore the space of correct isolation levels for different transactions, we  also  ran a number  of experiments using  a real transaction OLTP workload   based on TPC-C benchmark.  
 The  workload  involves  a mix of five TPC-C transaction types: 
\C{New\_Order},  \C{Payment},  \C{Delivery}, \C{Stock Level} and  \C{Order Status}, 
operating on top of  a $10$-warehouse database.  
Each warehouse has  $10$ districts, each serving $3000$ customers. 
TPC-C  specification model  requires  maintaining multiple   consistency conditions.  We focus only on   four conditions, named $I_1$, $I_2$, $I_3$, and $I_4$.   
Consistency condition $I_1$  requires that  sales of each  warehouse  equals to the  sum of    its  district's sales.  Conditions $I_2$ and $I_3$  enforce  
 sequentially unique IDs assigned to  orders placed in the same   warehouse and district. Condition $I_4$ requires 
 the  number of  item orders   in each district 
  reflexes  the district's roll-up. 
Given the consistency conditions,  Table  \ref{tab:tpcc-eval} gives the isolation level  for the \C{New\_Order},  \C{Payment},  \C{Delivery} transactions using the  MySql and Postrgres databases   to maintain   each consistency condition. 
Recall that the \C{New\_Order} transaction places a new order for
 items from a particular district in a warehouse.  
The \C{Payment} transaction  processes  a payment for a particular customer and updates the customer, district, and warehouse tables. 
 The  \C{Delivery}  transaction processes  orders for a particular   district at  a warehouse. 
 Our experimental results  shows that  the concurrent executions of   \C{New\_Order}   
    and  \C{Delivery} transactions under a weak isolation level (i.e.,  $RC$) is anomalous, i.e., it may violate the consistency conditions $I_2$ and $I_3$. Depending on the isolation guarantees provided  by  underlying database,   the stronger  isolation level is required. For instance, the isolation level $RR$ in Postgres  is sufficient to maintain the condition while   MySql database  requires  $SER$. 


 	\begin{table*}[t]\small
	\centering
\begin{tabular}{|c|c|c|c|}
		\hline
	\textbf{Transaction}   & \textbf{Invariant} 
	& \textbf{MYSQL-Isolation} & \textbf{Postgres-Isolation} \\ 
	\hline
	\multirow{4}{*}{new\_order }  & $I_1$ 
	& RC &  RC\\ 
	&  $I_2$ &SER & RR \\
	&  $I_3$ & SER  &  RR  \\
	& $I_4$ & RC & RC   \\
	\hline
	\multirow{4}{*}{Payment}  & $I_1 $ 
	& RC &  RC\\ 
	&  $I_2$  &RC & RC \\
	&  $I_3 $ & RC  &  RC  \\
	& $I_4$  & RC & RC   \\
	\hline
	\multirow{4}{*}{Delivery}  & $I_1$  
	& RC &  RC \\ 
	&  $I_2$ &SER & RR \\
	&  $I_3$ & SER  &  RR \\
	& $I_4$  & RC & RC   \\
	\hline
\end{tabular}
	\caption{ Isolation level required by TPC-C transactions. }
	\label{tab:tpcc-eval}
\end{table*}

