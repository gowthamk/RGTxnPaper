% Math mode
%-----------
\newenvironment{nop}{}{}
\newenvironment{smathpar}{
\begin{nop}\small\begin{mathpar}}{
\end{mathpar}\end{nop}\ignorespacesafterend}

% Theorem
%--------
\newtheorem{axiom}{Axiom}[section]
\newtheorem{theorem}{Theorem}[section]
\newtheorem{lemma}[theorem]{Lemma}
\newtheorem{proposition}[theorem]{Proposition}
\newtheorem{corollary}[theorem]{Corollary}

\newenvironment{definition}[1][Definition]{\begin{trivlist}
\item[\hskip \labelsep {\bfseries #1}]}{\end{trivlist}}
\newenvironment{example}[1][Example]{\begin{trivlist}
\item[\hskip \labelsep {\bfseries #1}]}{\end{trivlist}}
\newenvironment{remark}[1][Remark]{\begin{trivlist}
\item[\hskip \labelsep {\bfseries #1}]}{\end{trivlist}}

% Decorations
%-----------
\newenvironment{decoration}
  {\color{blue}\begin{array}{l}}
  {\end{array}}

% Listings
%----------
\newcommand{\lsttxnimp}{\lstset{
      language=c,
      basicstyle=\ttfamily\normalsize,
      flexiblecolumns=false,
			tabsize=2,
      escapechar=',                        
      %basewidth={0.5em,0.45em},
      %aboveskip={3pt},
      %belowskip={3pt},
      keywordstyle=\color{Bittersweet}\bfseries,
      commentstyle=\color{blue}\itshape,
      stringstyle=\color{MidnightBlue},
      morekeywords={transaction,txn,cobegin,from,to},
			classoffset=1,
			upquote=true,
			keywordstyle=\color{Fuchsia}\bfseries,
			classoffset=0,
			mathescape=true
    }}
\lstnewenvironment{txnimpcode}
    { % \centering
      \lsttxnimp
      \lstset{}%
      \csname lst@setfirstlabel\endcsname}
    { %\centering
      \csname lst@savefirstlabel\endcsname}

% Formatting
%---------
\newcommand{\C}[1]{\code{#1}}
\newcommand{\tuplee}[1]{\langle #1 \rangle}
\newcommand*{\rom}[1]{\expandafter\romannumeral #1}

% Formatting commands
% -------------------
\newcommand{\code}[1]{\,{\tt #1}\,}
\newcommand{\spc}[0]{\quad}
\newcommand{\ALT}{~\mid~}
\newcommand{\rel}[1]{{R}_{\mathit{#1}}}
\newcommand{\conj}{~\wedge~}
\newcommand{\disj}{~\vee~}
\newcommand{\rulelabel}[1]{\textrm{\sc {#1}}}
\newcommand{\ilrulelabel}[1]{{\sc #1}}
\newcommand{\RULE}[2]{\frac{\begin{array}{c}#1\end{array}}
                           {\begin{array}{c}#2\end{array}}}
\newcommand{\txnimp}{{\sc Txn}\xspace}
\newcommand{\coloneqq}{::=}
\newcommand{\qqquad}{\quad\quad}
\newcommand{\cskip}{\C{SKIP}}
\newcommand{\ctxn}[1]{\C{Txn}\{#1\}}
\newcommand{\catomic}[1]{\C{ATOMIC}\{#1\}}
\newcommand{\stepsto}{\longrightarrow}
\newcommand{\rstepsto}{\longrightarrow_{R}}
\newcommand{\stepssto}[1]{\longrightarrow^{#1}_{R}}
\newcommand{\rstepssto}[1]{\longrightarrow^{#1}_{R}}
\newcommand{\xstepsto}[1]{\longrightarrow_{#1}}
\newcommand{\xstepssto}[2]{\longrightarrow^{#1}_{#2}}
\newcommand{\tstepsto}{\longrightarrow}
\newcommand{\redsto}{\hookrightarrow}
\newcommand{\xtstepsto}[1]{\hookrightarrow_{#1}}
\newcommand{\rtstepsto}{\hookrightarrow_R}
\newcommand{\rtstepssto}[1]{\hookrightarrow^{#1}_R}
\newcommand{\xtstepssto}[2]{\hookrightarrow^{#1}_{#2}}
\newcommand{\hoare}[3]{\{#1\}\,#2\,\{#3\}}
\newcommand{\defeq}[0]{\overset { def }{ = } }
\newcommand{\rg}[3]{\{#1\}\,#2\,\{#3\}}
%\newcommand{\defeq}[0]{ \triangleq }
\newcommand{\op}{\textsf{op}}
\newcommand{\E}{\textsf{E}}
\newcommand{\A}{\textsf{A}}
\newcommand{\I}{\mathbb{I}}
\newcommand{\F}{\mathcal{F}}
\newcommand{\visZ}{\textsf{vis}}
\newcommand{\soZ}{\textsf{so}}
\newcommand{\hbZ}{\textsf{hb}}
\newcommand{\sameobj}[2]{\textsf{sameobj}(#1,#2)}
\newcommand{\sameobjZ}{\textsf{sameobj}}
\newcommand{\visar}{\xrightarrow{\visZ}}
\newcommand{\hboar}{\xrightarrow{\textsf{hb}}}
\newcommand{\soar}{\xrightarrow{\soZ}}
\newcommand{\visoar}{\xrightarrow{\visZ \,\cup\, \soZ}}
\newcommand{\invisar}{\xrightarrow{\textsf{invis}}}
\newcommand{\etaar}{\xrightarrow{\eta}}
\newcommand{\wrstoar}{\xrightarrow{\textsf{wrsto}}}
\newcommand{\rdsfmar}{\xrightarrow{\textsf{rdsfm}}}
\newcommand{\fresh}{\textsf{fresh}}
\newcommand{\isReadf}{\textsf{isRD}}
\newcommand{\isWritef}{\textsf{isWR}}
\newcommand{\oper}{\textsf{oper}}
\newcommand{\committed}{\textsf{com}}
\newcommand{\txn}{\textsf{txn}}
\newcommand{\id}{\textsf{id}}
\newcommand{\kind}{\textsf{oper}}
\newcommand{\rval}{\textsf{rval}}
\newcommand{\visible}{\textsf{visible}}
\newcommand{\maxId}{\textsf{maxId}}
\newcommand{\aeval}{\textsf{aeval}}
\newcommand{\underE}[1]{\E \Vdash #1}
\newcommand{\underIT}[1]{\;\I,\C{Txn}_i \vdash #1\;}
\newcommand{\underI}[1]{\;\I \vdash #1\;}
\newcommand{\underT}[1]{\;\C{Txn}_i \vdash #1\;}
\newcommand{\stable}{\mathtt{stable}}
\newcommand{\iso}[1]{{\sc #1}}
\newcommand{\writef}{\textsf{Write}}
\newcommand{\readf}{\textsf{Read}}
\newcommand{\commitf}{\textsf{Commit}}
\newcommand{\eg}{\emph{e.g.,}}
\newcommand{\GK}[1]{\textcolor{red}{GK: #1}}

\newcommand{\B}[1]{\small\bf #1}
\newcommand{\txnbox}[1]{\lbrack #1 \rbrack}
\newcommand{\interp}[1]{\llbracket #1 \rrbracket}
\newcommand{\cinterp}[1]{\llbracket #1 \rrbracket_{\C{C}}}
\newcommand{\ectx}{\mathcal{E}}
\newcommand{\isMax}{\textsf{isMax}}
\newcommand{\Prop}{\mathbb{P}}
