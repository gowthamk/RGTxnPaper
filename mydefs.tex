
% Math mode
%-----------
\newenvironment{nop}{}{}
\newenvironment{smathpar}{
\begin{nop}\small\begin{mathpar}}{
\end{mathpar}\end{nop}\ignorespacesafterend}

% Theorem
%--------

% \theoremstyle{plain}
% \newtheorem{axiom}{Axiom}[section]
% \newtheorem{theorem}{Theorem}[section]
% \newtheorem{lemma}[theorem]{Lemma}
% \newtheorem{proposition}[theorem]{Proposition}
% \newtheorem{corollary}[theorem]{Corollary}
% \theoremstyle{definition}
% \newtheorem{definition}[theorem]{Definition}

% \newenvironment{example}[1][Example]{\begin{trivlist}
% \item[\hskip \labelsep {\bfseries #1}]}{\end{trivlist}}
% \newenvironment{remark}[1][Remark]{\begin{trivlist}
% \item[\hskip \labelsep {\bfseries #1}]}{\end{trivlist}}

% Decorations
%-----------
\newenvironment{decoration}
  {\color{blue}\begin{array}{l}}
  {\end{array}}

% New colors
%------------
\definecolor{Bittersweet}{rgb}{1.0, 0.44, 0.37}
\definecolor{MidnightBlue}{rgb}{0.0, 0.2, 0.4}

% Listings
%----------
\newcommand{\lsttxnimp}{\lstset{
      language=c,
      basicstyle=\ttfamily\small,
      flexiblecolumns=false,
			tabsize=2,
      escapechar=',                        
      %basewidth={0.5em,0.45em},
      %aboveskip={3pt},
      %belowskip={3pt},
      keywordstyle=\color{Bittersweet}\bfseries,
      commentstyle=\color{blue}\itshape,
      stringstyle=\color{MidnightBlue},
      morekeywords={transaction,txn,cobegin,from,to,atomic},
			classoffset=1,
			upquote=true,
			keywordstyle=\color{Fuchsia}\bfseries,
			classoffset=0,
			mathescape=true
    }}
\lstnewenvironment{txnimpcode}
    { % \centering
      \lsttxnimp
      \lstset{}%
      \csname lst@setfirstlabel\endcsname}
    { %\centering
      \csname lst@savefirstlabel\endcsname}

% Listings Code
%---------------
\newcommand{\lstml}{
\lstset{ %
language=ML, % choose the language of the code
basicstyle=\footnotesize\ttfamily,       % the size of the fonts that are used for the code
keywordstyle=\color{Bittersweet},
% numbers=left,                   % where to put the line-numbers
numberstyle=\tiny,      % the size of the fonts that are used for the line-numbers
stepnumber=1,                   % the step between two line-numbers. If it is 1 each line will be numbered
numbersep=5pt,                  % how far the line-numbers are from the code
showspaces=false,               % show spaces adding particular underscores
showstringspaces=false,         % underline spaces within strings
showtabs=false,                 % show tabs within strings adding particular underscores
% frame=single,                   % adds a frame around the code
tabsize=2,                      % sets default tabsize to 2 spaces
captionpos=b,                   % sets the caption-position to bottom
breaklines=true,                % sets automatic line breaking
breakatwhitespace=false,        % sets if automatic breaks should only happen at whitespace
commentstyle=\itshape\color{MidnightBlue},
%escapeinside={\%*}{*)},         % if you want to add a comment within your code
mathescape=true,
morekeywords={module, match, when, @@deriving, not, : , txn_do, do, SQL/\\}
}}
\lstnewenvironment{ocaml}
    { % \centering
			\lstml
      \lstset{}%
      \csname lst@setfirstlabel\endcsname}
    { %\centering
      \csname lst@savefirstlabel\endcsname}
\newcommand{\ocamlinline}[1]{\lstinline[language=ML,
                                        basicstyle=\footnotesize\ttfamily, 
                                        keywordstyle=\color{Bittersweet},
                                        mathescape=true]{#1}}

% Formatting
%---------
\newcommand{\C}[1]{\code{#1}}
%\newcommand{\R}[1]{\textsc{#1}}
\newcommand{\tuplee}[1]{\langle #1 \rangle}
\newcommand*{\rom}[1]{\expandafter\romannumeral #1}

% Formatting commands
% -------------------
\newcommand{\code}[1]{{\tt #1}}
\newcommand{\spc}[0]{\quad}
\newcommand{\ALT}{~\mid~}
\newcommand{\rel}[1]{{R}_{\mathit{#1}}}
\newcommand{\conj}{~\wedge~}
\newcommand{\disj}{~\vee~}
\newcommand{\rulelabel}[1]{\textrm{\sc {#1}}}
\newcommand{\ilrulelabel}[1]{{\sc #1}}
\newcommand{\RULE}[2]{\frac{\begin{array}{c}#1\end{array}}
                           {\begin{array}{c}#2\end{array}}}
\newcommand{\txnimp}{\mbox{${\mathcal T}$}}
\newcommand{\coloneqq}{::=}
\newcommand{\qqquad}{\quad\quad}
\newcommand{\cskip}{\C{SKIP}}
\newcommand{\stxn}[3]{\C{TXN}\langle #2 \rangle\{#3\}}
\newcommand{\ctxn}[3]{\C{txn}\langle #2 \rangle\{#3\}}
\newcommand{\catomic}[1]{\C{ATOMIC}\{#1\}}
\newcommand{\stepsto}{\longrightarrow}
\newcommand{\rstepsto}{\longrightarrow_{R}}
\newcommand{\stepssto}[1]{\longrightarrow^{#1}_{R}}
\newcommand{\rstepssto}[1]{\longrightarrow^{#1}_{R}}
\newcommand{\xstepsto}[1]{\longrightarrow_{#1}}
\newcommand{\xstepssto}[2]{\longrightarrow^{#1}_{#2}}
\newcommand{\tstepsto}{\longrightarrow}
\newcommand{\redsto}{\hookrightarrow}
\newcommand{\xtstepsto}[1]{\hookrightarrow_{#1}}
\newcommand{\rtstepsto}{\hookrightarrow_R}
\newcommand{\rtstepssto}[1]{\hookrightarrow^{#1}_R}
\newcommand{\xtstepssto}[2]{\hookrightarrow^{#1}_{#2}}
\newcommand{\hoare}[3]{\{#1\}\,#2\,\{#3\}}
\newcommand{\defeq}[0]{\overset { \mathit{def} }{ = } }
\newcommand{\rg}[3]{\{#1\}\,#2\,\{#3\}}
%\newcommand{\defeq}[0]{ \triangleq }
\newcommand{\op}{\textsf{op}}
\newcommand{\E}{\textsf{E}}
\newcommand{\A}{\textsf{A}}
\newcommand{\I}{\mathbb{I}}
\newcommand{\R}{\mathbb{R}}
\newcommand{\F}{\mathcal{F}}
\newcommand{\visZ}{\textsf{vis}}
\newcommand{\soZ}{\textsf{so}}
\newcommand{\hbZ}{\textsf{hb}}
\newcommand{\sameobj}[2]{\textsf{sameobj}(#1,#2)}
\newcommand{\sameobjZ}{\textsf{sameobj}}
\newcommand{\visar}{\xrightarrow{\visZ}}
\newcommand{\hboar}{\xrightarrow{\textsf{hb}}}
\newcommand{\soar}{\xrightarrow{\soZ}}
\newcommand{\visoar}{\xrightarrow{\visZ \,\cup\, \soZ}}
\newcommand{\invisar}{\xrightarrow{\textsf{invis}}}
\newcommand{\etaar}{\xrightarrow{\eta}}
\newcommand{\wrstoar}{\xrightarrow{\textsf{wrsto}}}
\newcommand{\rdsfmar}{\xrightarrow{\textsf{rdsfm}}}
\newcommand{\usesar}{\xrightarrow{\textsf{uses}}}
\newcommand{\isReadf}{\textsf{isRD}}
\newcommand{\isWritef}{\textsf{isWR}}
\newcommand{\oper}{\textsf{oper}}
\newcommand{\committed}{\textsf{com}}
\newcommand{\txn}{\textsf{txn}}
\newcommand{\id}{\textsf{id}}
\newcommand{\kind}{\textsf{oper}}
\newcommand{\rval}{\textsf{rval}}
\newcommand{\visible}{\textsf{visible}}
\newcommand{\maxId}{\textsf{maxId}}
\newcommand{\eval}{\textsf{eval}}
\newcommand{\dom}{\textsf{dom}}
\newcommand{\underE}[1]{\E \Vdash #1}
\newcommand{\underIT}[1]{\;\I,\C{Txn}_i \vdash #1\;}
\newcommand{\underI}[1]{\;\I \vdash #1\;}
\newcommand{\underT}[1]{\;\C{Txn}_i \vdash #1\;}
\newcommand{\stable}{\mathtt{stable}}
\newcommand{\iso}[1]{\emph{#1}}
\newcommand{\writef}{\textsf{Write}}
\newcommand{\readf}{\textsf{Read}}
\newcommand{\commitf}{\textsf{Commit}}
\newcommand{\eg}{\emph{e.g.,}}
\newcommand{\GK}[1]{\textcolor{red}{GK: #1}}
\newcommand{\SJ}[1]{\textcolor{red}{SJ: #1}}

\newcommand{\B}[1]{\small\bf #1}
\newcommand{\txnbox}[1]{\lbrack #1 \rbrack}
\newcommand{\interp}[1]{\llbracket #1 \rrbracket}
\newcommand{\cinterp}[1]{\llbracket #1 \rrbracket_{\C{C}}}
\newcommand{\ectx}{\mathcal{E}}
\newcommand{\isMax}{\textsf{isMax}}
\newcommand{\Prop}{\mathbb{P}}
\newcommand{\Pow}[1]{\mathcal{P}\left(#1\right)}
\newcommand{\bind}{\gg=}
\newcommand{\ite}[3]{\C{IF}\;#1\;\C{THEN}\;#2\;\C{ELSE}\;#3}
\newcommand{\lete}[3]{\C{LET}\;#1=#2\;\C{IN}\;#3}
\newcommand{\foreache}[2]{\texttt{FOREACH}\;#1\;\texttt{DO}\;#2}
\newcommand{\foreachr}[3]{\textsc{foreach}\langle #1 \rangle\;#2\;\textsc{do}\;#3}
\newcommand{\inserte}[1]{\texttt{INSERT}\;#1}
\newcommand{\selecte}[1]{\texttt{SELECT}\;#1}
\newcommand{\deletee}[1]{\texttt{DELETE}\;#1}
\newcommand{\updatee}[2]{\texttt{UPDATE}\;#1\;#2}
\newcommand{\stl}{\delta}
\newcommand{\stg}{\Delta}
\newcommand{\rec}{r}
\newcommand{\idf}{\texttt{id}}
\newcommand{\delf}{\texttt{del}}
