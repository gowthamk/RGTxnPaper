
\section{Introduction}

Database transactions allow users to group operations on multiple
objects into a single logical unit, equipped with a set of four key
properties - atomicity, consistency, isolation, and durability (ACID).
Concurrency control mechanisms provide specific instantiations of
these properties to yield different ACID variants that characterize
how and when the effects of concurrently executing transactions become
visible to one another.  \emph{Serializability} is a particularly
well-studied instantiation that imposes strong atomicity and isolation
constraints on transaction execution, ensuring that any permissible
concurrent schedule yields results equivalent to a serial one in which
there is no interleaving of actions from different transactions.

The guarantees provided by serializability do not come for free,
however - pessimistic concurrency control methods that enforce
serializability, for example, require databases to provide expensive
techniques such as two-phase locking that incur overhead to deal with
deadlocks, rollbacks, and re-execution~\cite{twopl,ullmanbook}.
Similar criticisms apply to optimistic multi-version concurrency
control methods that must deal with timestamp and version
management~\cite{BG81}.  Regardless of the particular concurrency
control method employed, enforcing serializability in a replicated
data store additionally requires global coordination among
geo-distributed replicas, which makes the store \emph{unavailable} in
the presence of network
partitions~\cite{cap,sernotavlbl,bailishat,bernsigmod13}.

The tension between serializable transactions, which are easy to
reason about but difficult to implement, and more pragmatic variants
that are driven by performance and availability considerations, has
motivated the development of weaker forms of transaction isolation,
beginning as early as 1976~\cite{gray1976}. The ANSI SQL 92 standard
defines three such weak isolation levels which are now implemented in
many relational and NoSQL databases. Not surprisingly, weakly-isolated
transactions have been found to significantly outperform serializable
transactions on benchmark suites, both on single-node databases and
multi-node replicated stores~\cite{dbtuningbook,bailishat,bailisvldb},
leading to their overwhelming adoption. A 2013
study~\cite{bailishotos} of 18 popular ACID and ``NewSQL'' databases
found that only three of them offer serializability by default, and
half, including Oracle 11g, do not offer it at all.  A 2015
study~\cite{bailisferal} of a large corpus of database applications
finds no evidence that applications manifestly change the default
isolation level offered by the database. Taken together, these studies
make clear that weakly-isolated transactions are quite prevalent in
practice, whereas serializable transactions are often eschewed.

Unfortunately, weak isolation admits behaviours that are difficult to
comprehend~\cite{berenson}. To quantify weak isolation anomalies,
Fekete \emph{et al.}~\cite{feketevldb09} devised and experimented
with a microbenchmark suite that executes transactions under a
weakly-isolated \emph{read committed} isolation level - the default
level for 8 of the 18 databases studied in~\cite{bailishotos}, and
found that 25 out of every 1000 rows in the database violate at least
one integrity constraint. Bailis \emph{et al.}~\cite{bailisferal} rely
on Rails' \emph{uniqueness validation} to maintain uniqueness of
records while serving Linkbench's~\cite{linkbench} insertion workload
(6400 records distributed over 1000 keys; 64 concurrent clients), and
report discovering more than 10 duplicate records.  Rails relies on
database transactions to validate uniqueness during insertions, which
is sensible if transactions are serializable, but incorrect under the
\emph{read committed} isolation level used in the experiments. The
same study has found that 13\% of all invariants among 67 open source
Ruby-on-Rails applications are at risk of being violated due to weak
isolation. Indeed, incidents of safety violations due to executing
applications in a weakly-isolated environment have been reported on
web services in production~\cite{starbucksbug, scimedbug}, including
in safety-critical applications such as bitcoin
exchanges~\cite{poloniexbug, bitcoinbug}. While enforcing
serializability for all transactions would be sufficient to avoid
these errors and anomalies, it would likely be an overly conservative
strategy; indeed, 75\% of the invariants studied in~\cite{bailisferal}
were shown to be preserved under some form of weak isolation.  When to
use weak isolation, and in what form, is therefore a prominent
question facing all database programmers.

A major problem with weak isolation is that its semantics in the
context of user programs is not well-understood. The original
proposal~\cite{gray1976} defines multiple ``degrees'' of weak
isolation in terms of implementation details such as the nature and
duration of locks held in each case. The ANSI SQL 92 standard defines
four levels of isolation (including serializability) in terms of
various undesirable \emph{phenomena} (\eg \emph{dirty reads}) each is
required to prevent. While this is an improvement, it requires
programmers to be prescient about the possible ways various
undesirable phenomena might manifest in their applications, and
determine, in each case, if the phenomenon can be allowed without
violating application invariants. This is understandably hard,
especially in the absence of any formal underpinning to define weak
isolation semantics.  Adya~\cite{adyaphd} presents the first formal
definitions of some well-known isolation levels in the context of a
sequentially consistent (SC) database.  However, there has been little
progress relating Adya's system model to a formal operational
semantics or a proof system that can facilitate rigorous correctness
arguments. Consequently, reasoning about weak isolation remains an
error prone endeavour, with major database vendors~\cite{postgresiso,
  mysqliso, oracleiso} continuing to document their isolation levels
primarily in terms of the undesirable phenomena a particular isolation
level may induce, placing the burden on the programmer to determine
application correctness.

%%%SJ: tried to tighten this paragraph; the original now commented seemed overly
%%complicated
Recent results on reasoning about application invariants in the
presence of weak consistency~\cite{burckhardt14, redblueosdi,
redblueatc, ecinec, gotsmanpopl16} address broadly related concerns.
Weak consistency is a phenomenon that manifests on replicated data
stores, where atomic operations are concurrently executed against
different replicas, resulting in an execution order inconsistent with
%% We already say "inconsistent with.. ". 
any sequential order. In contrast, weak isolation is a property of
concurrent transactions interfering with one another resulting in an
execution order that is not serializable. Unlike weak consistency,
weak isolation can manifest even in an unreplicated setting, as
evident from the support for weakly-isolated transactions on
conventional (unreplicated) databases as mentioned above. In the
presence of replication, however, the interaction between weak
isolation and weak consistency can be subtle and non-trivial.
Understanding weak isolation in these varied contexts thus requires
new insights and substantial generalization of existing techniques.

%% Recent results on reasoning about application invariants in the
%% presence of weak consistency~\cite{burckhardt14, redblueosdi,
%% redblueatc, ecinec, gotsmanpopl16} address broadly related concerns.
%% Weak consistency is a phenomenon that usually manifests on replicated
%% data stores, where operations are concurrently executed against
%% different replicas, resulting in an order of execution inconsistent
%% with their serial order. The operations, nonetheless, are atomic and
%% fully isolated, and each operation is required to locally preserve
%% application invariants. In contrast, weak isolation is a property of
%% transactions, which are sets of atomic operations. Weak isolation
%% manifests when successive atomic operations in a transaction witness
%% different contemporary states of the database (or different replicas
%% of the replicated store) which, although consistent individually, may
%% not be obviously reconciled into a consistent global view.
%% Intuitively, the latter problem reduces to the former if all
%% transactions contain a single operation. Furthermore, weak isolation
%% can exist independent of weak consistency, as evident from the
%% presence of weakly-isolated transactions on conventional RDBS. With an
%% added complexity of replication, richer mechanisms are needed to
%% reason about weak isolation in tandem with weak consistency. Thus,
%% frameworks for reasoning about weak isolation will necessarily have to
%% generalize the reasoning frameworks developed for weak consistency in
%% new and important ways.

% The framework should be general enough to reason about the semantics
% of multiple isolation levels, including those proposed after the SQL
% 92 standard, in the context of various stores (\eg sequentially
% consistent store of~\cite{adyaphd}, causally consistent store
% of~\cite{gotsmanpopl16} etc). 

In this paper, we propose a program logic for weakly-isolated
transactions that realizes this goal.  In particular, we develop a set
of syntax-directed compositional proof rules that allow developers to
build correctness proofs for transactional programs in the presence of
a weakly-isolated concurrency control mechanism.  A key novelty of our
approach is that it is parametric over the \emph{isolation semantics}
of transactions in the program, as well as the \emph{consistency
semantics} of the underlying store. In concrete terms, this means
that, unlike recent work focused on reasoning about programs under
weak consistency~\cite{gotsmanpopl16, redblueatc, ecinec}, our system
model does not assume a minimum or predefined set of consistency or
isolation levels. Instead, our operational semantics admits
declarative specifications of transaction isolation and store
consistency, and generates executions that are guaranteed to satisfy
these specifications. The result is a flexible system model that is
general enough to incorporate the semantics of a range of isolation
levels on a variety of stores (\eg~the sequentially consistent store
of~\cite{adyaphd}, or the causally consistent store
of~\cite{gotsmanpopl16}, etc).  Our key technical contribution is thus
a set of proof rules that demonstrate that a program with a given
selection of isolation levels for its transactions preserves its
invariants when executed on a store equipped with certain consistency
guarantees (\eg~causal consistency).

The paper makes the following contributions:

\begin{itemize}
  \item We develop a semantics for a core language equipped with
    weakly-isolated transactions, demonstrating that a general
    parametric isolation semantics can be expressed as
    well-formedness constraints on a program's execution.
  \item We present a compositional proof system for this language
    capable of relating high-level application invariants to the structure
    of traces induced by the operational semantics, in which transactions
    are associated with specific weak isolation levels.
  \item We define a \emph{maximum visibility principle} to reconcile
    store consistency guarantees with transaction isolation
    requirements, and use it to instantiate our operational model and
    the proof system for various kinds of (weakly-consistent) data
    stores.
  \item Case studies modeled after real-world scenarios demonstrate
    the applicability and utility of our proof methodology.
\end{itemize}

\noindent Our results provide the first (to the best of our knowledge)
mechanism that precisely and uniformly relates high-level program
invariants to low-level weak isolation and weak consistency
guarantees, thereby allowing weakly-isolated transactions to enjoy the
same rigorous reasoning capabilities as their strongly-isolated
(serializable) counterparts.

The remainder of the paper is organized as follows. The next section
provides motivation and background on serializable and weakly-isolated
transactions. \S\ref{sec:opsem} presents an operational semantics for
a core language that supports weakly-isolated transactions,
parameterized over different isolation notions. \S\ref{sec:reasoning}
formalizes the proof system that we use to reason about program
invariants, and establishes the soundness of these rules with respect
to the semantics. \S\ref{sec:store-consistency} generalizes the
framework to integrate support for weakly-consistent data stores. We
describe the impact of our reasoning framework in the context of
several real-world case studies in \S\ref{sec:case-studies}.  Related
work and conclusions are given in \S\ref{sec:relatedwork}.
