
\section{Introduction}

Database transactions allow users to group operations on multiple
objects into a coherent unit, equipped with a set of four key
properties - atomicity, consistency, isolation, and durability (ACID).
Concurrency control mechanisms provide specific instantiations of
these properties to yield different ACID variants that characterize
how and when the effects of concurrently executing transactions become
visible to one another.  \emph{Serializability} is a particularly
well-studied instantiation that imposes strong atomicity and isolation
constraints on transaction execution, ensuring that any permissible
concurrent schedule is equivalent to a serial one in which in which no
interleaving of actions from different transactions are allowed.

The guarantees provided by serializability, and strong isolation in
particular, do not come for free, however - pessimistic concurrency
control methods, for example, require databases to provide expensive
techniques such as two-phase locking that incur overhead to deal with
deadlocks, rollbacks, and re-execution~\cite{twopl,ullmanbook}.
Similar criticisms apply to optimistic multi-version concurrency
control methods that must deal with timestamp and version
management~\cite{BG81}.  Regardless of the particular method employed,
enforcing serializability in a replicated data store additionally
requires global coordination among geo-distributed replicas, which
makes the store \emph{unavailable} in the presence of inevitable
network partitions~\cite{cap,sernotavlbl,bailishat,bernsigmod13}.

The tension between serializable transactions, which are easy to
reason about but difficult to implement, and more pragmatic goals
driven by performance and availability considerations, has motivated
the development of multiple weaker forms of transaction isolation,
beginning as early as 1976~\cite{gray1976}. The ANSI SQL 92 standard
defines three such weak isolation levels which are now
implemented in many relational and NoSQL databases. Not surprisingly,
weakly isolated transactions have been found to significantly
outperform serializable transactions on benchmark suites, both on
single-node databases and multi-node replicated
stores~\cite{dbtuningbook,bailishat,bailisvldb}, leading to their
overwhelming adoption. A 2013 study~\cite{bailishotos} of 18 popular
ACID and ``NewSQL'' databases found that only three of them offer
serializability by default, and half, including Oracle 11g, do not
offer it at all.  A 2015 study~\cite{bailisferal} of a large corpus of
database applications finds no evidence that applications manifestly
change the default isolation level offered by the database. Taken
together, these studies make clear that weakly isolated transactions
are quite prevalent in practice, and strongly isolated (serializable)
transactions are often eschewed.

Unfortunately, weak isolation admits behaviors that are difficult to
comprehend~\cite{berenson}. To quantify weak isolation anomalies,
Fekete \emph{et al.}~\cite{feketevldb09} have devised and experimented
with a microbenchmark suite that executes transactions under a weakly
isolated \emph{read committed} isolation level - the default level for
8 of the 18 databases studied in~\cite{bailishotos}, and found that 25
out of every 1000 rows in the database violate at least one integrity
constraint. Bailis \emph{et al.}~\cite{bailisferal} rely on Rails'
\emph{uniqueness validation} to maintain uniqueness of records while
serving Linkbench's~\cite{linkbench} insertion workload (6400 records
distributed over 1000 keys; 64 concurrent clients), and end up with
more than 10 duplicate records. Rails relies on database transactions
to validate uniqueness during insertions, which is sensible if
transactions are serializable, but incorrect under the \emph{read
  committed} isolation level used in the experiments. The same study
has found that 13\% of all invariants among 67 open source
Ruby-on-Rails applications are at risk of being violated due to weak
isolation. Indeed, incidents of safety violations due to executing
applications in a weakly isolated environment have been reported on
web services in production~\cite{starbucksbug, scimedbug}, including
in safety-critical applications such as bitcoin
exchanges~\cite{poloniexbug, bitcoinbug}. While enforcing
serializability for all transactions would be sufficient to avoid
these errors and anomalies, it would likely be an overly conservative
strategy; indeed, 75\% of the invariants studied in~\cite{bailisferal}
were shown to be preserved under some form of weak isolation.  When to
use weak isolation, and in what form, is therefore a prominent
question facing all database programmers.

A major problem with weak isolation is that its semantics in the
context of user programs is not well-understood. The original
proposal~\cite{gray1976} defines multiple ``degrees'' of weak
isolation in terms of implementation details such as the nature and
duration of locks held in each case. The ANSI SQL 92 standard four
levels of isolation (including serializability) in terms of various
undesirable \emph{phenomena} (\eg \emph{dirty reads}) each is required
to prevent. While this is an improvement, it requires programmers to
be prescient about the possible ways various undesirable phenomena
might manifest in their applications, and determine, in each case, if
the phenomenon can be allowed without violating application
invariants. This is understandably hard. Adya~\cite{adyaphd} presents
the first formal definitions of some of well-known isolation levels in
the context of a sequentially consistent (SC) database.  However,
there has been little progress relating Adya's system model to a
formal operational semantics or a proof system that can facilitate
rigorous correctness arguments.  Consequently, reasoning about weak
isolation remains an error prone endeavor, with major database vendors
continuing to document their isolation levels primarily in terms of
the undesirable phenomena~\cite{postgresiso, mysqliso, oracleiso} a
particular isolation level may induce; it thus squarely becomes the
programmer's responsibility to determine if their application can
yield an execution which can trigger such phenomena.

Recent results on reasoning about application invariants in the
presence of weak \emph{consistency}~\cite{burckhardt14, redblueosdi,
  redblueatc, ecinec, gotsmanpopl16} provide insights on possible ways
we might overcome some of the difficulties in understanding weak
isolation, even though the latter is strictly more general than the
former.  At a high level, reasoning about weak consistency involves
reasoning about the concurrent execution of strongly atomic and
isolated operations against different but consistent states of the
database or data store, where each operation locally preserves
application invariants. In contrast, reasoning about weak isolation
involves reasoning about the concurrent execution of transactions,
whose operations witness different states which, although consistent
individually, may not be obviously reconciled into a consistent global
view.  Moreover, operations may violate invariants as long as the
transaction as a whole preserves them.  Intuitively, the latter
problem reduces to the former if all transactions contain a single
operation. Furthermore, weak isolation can exist independent of weak
consistency, as evident from the presence of weakly isolated
transactions on sequentially consistent relational databases; such
systems existed long before the advent of weakly consistent NoSQL
stores.  Thus, frameworks for reasoning about weak isolation will
necessarily have to generalize the reasoning frameworks developed for
weak consistency in new and important ways.

% The framework should be general enough to reason about the semantics
% of multiple isolation levels, including those proposed after the SQL
% 92 standard, in the context of various stores (\eg sequentially
% consistent store of~\cite{adyaphd}, causally consistent store
% of~\cite{gotsmanpopl16} etc). 

In this paper, we propose a program logic for weakly isolated
transactions that realizes this goal.  In particular, we develop a set
of syntax-directed compositional proof rules that allow developers to
build correctness proofs for transactional programs in the presence of
a weakly isolated concurrency control mechanism  A key novelty of our
approach is that it is parametric over the \emph{isolation semantics}
of transactions in the program, as well as the \emph{consistency
  semantics} of the underlying store. In concrete terms, this means
that, unlike recent work focussed on formalizing the semantics of weak
consistency~\cite{gotsmanpopl16, redblueatc, ecinec}, our system model
does not assume a minimum consistency or isolation level, nor does it
assume a predefined set of such levels.  Instead, our operational
semantics admits declarative specifications of transaction isolation
and store consistency levels (\eg~\cite{pldi15,gotsmanconcur15}), and
generates executions that are guaranteed to satisfy these
specifications. The result is a flexible system model that is general
enough to incorporate the semantics of a range of isolation levels on
a variety of stores (\eg the sequentially consistent store
of~\cite{adyaphd}, or the causally consistent store
of~\cite{gotsmanpopl16}, etc).  Our key technical contribution is thus
a set of proof rules that demonstrate that a program with a given
selection of isolation levels for its transactions preserves its
invariants when executed on a store satisfying certain consistency
guarantees (e.g., causal visibility).

The paper makes the following contributions:
\begin{itemize}
  \item We develop a semantics for a core language equipped with
    weakly isolated transactions, demonstrating that that a general
    isolation semantics can be expressed as well-formedness constraints
    on a program's execution.
  \item We propose a compositional proof system for this language
    capable of relating high-level application invariants to the structure
    of traces induced by the operational semantics, in which transactions
    are parameterized with specific weak isolation levels.
  \item We define a \emph{maximum visibility principle} to reconcile
    store consistency guarantees with transaction isolation
    requirements, and use it to instantiate our operational model and
    the proof system for various kinds of (weakly consistent) data
    stores.
  \item Case studies modeled after real-world scenarios demonstrate
    the applicability and utility of our proof methodology.
\end{itemize}

Our results thus provide the first (to the best of our knowledge)
mechanism that precisely relates high-level program invariants to
low-level store consistency and weak isolation guarantees, thereby
allowing weakly isolated transactions to enjoy the same rigorous
reasoning capabilities as their strongly isolated (serializable)
counterparts.


