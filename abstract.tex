\begin{abstract}

  Serializability is a well-understood correctness criterion that
  simplifies reasoning about the behavior of concurrent transactions
  by ensuring that transactions are \emph{isolated} from each other
  while they execute.  However, enforcing serializable isolation comes
  at a steep cost in performance because it necessarily restricts
  opportunities to exploit concurrency even when such opportunities
  would not violate application-specific invariants. As a result,
  database systems in practice support, and often encourage,
  developers to use weaker alternatives. These alternatives break the
  strong isolation guarantees offered by serializable transactions to
  permit greater concurrency. Unfortunately, the semantics of weak
  isolation is poorly understood, and usually explained only
  informally in terms of low-level implementation
  artifacts. Consequently, verifying high-level correctness properties
  in such environments remains a challenging problem.

  To address this issue, we present a program logic that enables
  compositional reasoning about the behavior of concurrently executing
  weakly-isolated transactions. Notably, our development is parametric
  over a transaction's specific isolation semantics, and the
  consistency guarantees provided by the underlying data store,
  allowing it to be applicable over a range of concurrency control
  mechanisms.  Case studies and experiments on real-world applications
  demonstrate the utility of our approach, and provide strong evidence
  that weakly-isolated transactions can be placed on the same formal
  footing as their strongly-isolated serializable counterparts.

\end{abstract}
