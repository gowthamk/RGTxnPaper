\begin{abstract}

  Serializability is a well-understood correctness criterion that
  simplifies reasoning about the behaviour of concurrent transactions
  by ensuring they are \emph{isolated} from each other while they
  execute.  However, enforcing serializable isolation comes at a steep
  cost in performance because it necessarily restricts opportunities
  to exploit concurrency even when such opportunities would not
  violate application-specific invariants. As a result, database
  systems in practice support, and often encourage, developers to
  implement transactions using weaker alternatives. These alternatives
  break the strong isolation guarantees offered by serializable
  transactions to permit greater concurrency. Unfortunately, the
  semantics of weak isolation is poorly understood, and usually
  explained only informally in terms of low-level implementation
  artifacts. Consequently, verifying high-level correctness properties
  in such environments remains a challenging problem.

  To address this issue, we present an automated verification
  methodology based on a novel program logic that enables
  compositional reasoning about the behavior of concurrently executing
  weakly-isolated transactions. Notably, our development is parametric
  over a transaction's specific isolation semantics, allowing it to be
  applicable over a range of concurrency control mechanisms.  Given a
  a collection of concurrently executing transactions, our analysis
  verifies whether assertion ${\mathcal{A}}$ associated with
  transaction $T$ in this collection, holds under a weak isolation
  level ${\mathcal{I}}$.  Case studies and experiments on real-world
  applications demonstrate the utility of our approach, and provide
  strong evidence that verification of weakly-isolated transactions
  can be placed on the same formal footing as their strongly-isolated
  serializable counterparts.

\end{abstract}
