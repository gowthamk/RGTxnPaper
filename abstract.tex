\begin{abstract}

  Serializability is a well-understood mechanism that simplifies
  reasoning about the behavior of concurrent transactions.  However,
  its two key properties, \emph{atomicity} - a transaction's effects
  are seen only when it commits, and \emph{isolation} - a transaction
  never witnesses the effects of other committed transactions while it
  is executing, impose significant constraints on implementations, and
  can negatively impact database performance by restricting
  opportunities to exploit concurrency.
  
  As a result, database systems (in practice) support, and often
  encourage, developers to use weaker alternatives.  Generally, these
  alternatives break the strong isolation guarantees provided by
  serializable transactions to permit greater concurrency.
  Unfortunately, the semantics of weak isolation is often poorly
  understood, and usually explained informally in terms of low-level
  implementation artifacts.  Consequently, verifying high-level
  correctness properties in such environments can be problematic.

  To address this issue, this paper presents a novel program logic
  that enables compositional reasoning about the behavior of
  concurrently executing weakly-isolated transactions.  Notably, our
  development is parametric over a transaction's specific isolation
  semantics, and the consistency guarantees provided by the underlying
  data store, allowing it to be applicable over a large range of
  different isolation and consistency models.  Case studies and
  experiments on real-world applications demonstrate the utility of
  our approach, and provide strong evidence that weakly-isolated
  transactions can be placed on the same formal footing as their
  strongly-isolated serializable counterparts.

\end{abstract}
