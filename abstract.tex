\begin{abstract}

  Serializability is a well-understood mechanism that simplifies
  reasoning about the behavior of concurrent transactions by ensuring
  transactions are \emph{isolated} from each other's effects once they
  begin executing.  However, enforcing serializability comes at a
  steep cost in performance because it necessarily restricts
  opportunities to exploit concurrency even when such opportunities
  would not violate application-specific invariants.
  
  As a result, database systems (in practice) support, and often
  encourage, developers to use weaker alternatives.  These
  alternatives break the strong isolation guarantees offered by
  serializable transactions to permit greater concurrency.
  Unfortunately, the semantics of weak isolation is often poorly
  understood, and usually explained only informally in terms of
  low-level implementation artifacts.  Consequently, verifying
  high-level correctness properties in such environments is
  problematic.

  To address this issue, we present a program logic that enables
  compositional reasoning about the behavior of concurrently executing
  weakly-isolated transactions.  Notably, our development is
  parametric over a transaction's specific isolation semantics, and
  the consistency guarantees provided by the underlying data store,
  allowing it to be applicable over a large range of concurrency
  control mechanisms.  Case studies and experiments on real-world
  applications demonstrate the utility of our approach, and provide
  strong evidence that weakly-isolated transactions can be placed on
  the same formal footing as their strongly-isolated serializable
  counterparts.

\end{abstract}
