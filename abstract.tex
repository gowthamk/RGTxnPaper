\begin{abstract}
ACID model of database transactions advocates serializability as the
correctness criterion for concurrent transactions. 
% Serializability is an easily-stated property,
% providing database programmers with a simple
% implementation-independent model to reason about program invariants in
% presence of concurrent transactions. Unfortunately, 
While serializability is an easily-stated high-level property,
enforcing it requires the database to severely restrict
concurrency.
% , thus limiting the availability and performance
% (latency/throughput) of the data store.  In the database parlance,
% serializibility is \emph{not highly available}~\cite{bailishotos13},
% making it particularly unsuitable for modern-day distributed web
% applications that aspire to be \emph{always on}.
Pragmatic concerns led to the development of various weaker
alternatives to serializability, which are overwhelmingly preferred in
practice.
% some of which have been
% standardized by the community as \emph{ANSI SQL isolation
% levels}~\cite{berenson}. Indeed, weakly isolated transactions are
% overwhelmingly preferred to serializability in practice, to the extent
% that the default isolation level in popular industrial-strength
% databases is often weaker than serializability.
However, unlike serializability, the semantics of weaker isolation
levels are often described informally in terms of an implementation.
While there have been some attempts at formalization, there
nonetheless exists a wide chasm between formal models of weak
isolation and the operational models of programs, hindering uniform
reasoning about program invariants in presence of weakly isolated
transactions.

In this paper, we propose an operational model and a compositional
reasoning framework for concurrent imperative programs with weakly
isolated transactions (\txnimp programs, as we call them).  Our
operational model captures all the behaviors exhibited by a program
under weak isolation without referring to any specific implementation
of isolation. Our reasoning framework enables formal reasoning about
program invariants while taking into account the fine-grained
isolation semantics of transactions. We use a combination of simple
static analysis and manual proofs in our reasoning framework to
formally establish the correctness for several benchmarks, which were
hitherto known to be correct only through experiments.
\end{abstract}
