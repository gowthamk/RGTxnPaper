\subsection{Isolation Specifications}
\label{sec:isolation}

Databases implement isolation levels through a combination of various
concurrency control mechanisms, which are absent from
Fig.~\ref{fig:txnimp}. In this section, we first axiomatize the
semantics of various concurrency control constructs, and then define
the precise semantics of the isolation levels as they are implemented
on two real-world databases -  PostgreSQL and MySQL.
% Any additional concurrency control provided by the database
% implementation, beyond that which is already incuded in the
% isolation specification, also needs to be axiomatized in $\I$. We
% first describe such axiomatizations, and then present the
% specifications of the standard isolation levels as implemented by
% popular off-the-shelf databases.

\textbf{Unique Ids}. Most relational databases associate every record
in a table with a unique identifier. Often, auto-generated unique
identifiers are part of the schema itself (via SQL's \C{UNIQUE} and
\C{AUTO\_INCREMENT} keywords), but if not, database adds one (e.g.,
MySQL's InnoDB engine automatically adds a 6-byte identifier if none
exists). Enforcing global uniqueness of identifiers requires exclusive
locks, which are absent from Fig.~\ref{fig:txnimp}. The
\rulelabel{E-Insert} rule assigns an id to the newly added record
after checking that it is not already present in $\stg$ and $\stl$.
But the new record is only added to $\stl$, which is invisible outside
the transaction. Thus, it is possible that the uniqueness check for the same id may
also succeed in a concurrent transaction, resulting in duplicate ids
after both commits. To prevent this, we need the following proposition
in $\I$ of every transaction, regardless of whether it is executing
or committing:
\begin{smathpar}
\begin{array}{lcl}
  \I_{id}(\stl,\stg,\stg') & = & \forall(r\in\stl).~
      r.\idf\notin \dom(\stg) \Rightarrow r.\idf\notin \dom(\stg').
\end{array}
\end{smathpar}
$\I_{id}$ ensures that a globally unique id generated during a
transaction's execution remains globally unique until it commits, prohibiting
interference from a concurrent transaction that adds the same id.

\textbf{Write Locks}. Transactions in MySQL and PostgreSQL (effectively)
obtain exclusive locks on records they update, and do not release them
until they commit, thus preventing write-write (ww) conflicts. To
obtain this behavior on the abstract machine of Fig.~\ref{fig:txnimp},
the following proposition must hold for every transaction:
\begin{smathpar}
\begin{array}{lcl}
  \I_{ww}(\stl,\stg,\stg') & = & \forall(r'\in\stl)(r \in \stg).~
      r.\idf = r'.\idf  \Rightarrow r\in\stg'.
\end{array}
\end{smathpar}
That is, given a record $r'\in\stl$, if there exists an $r\in\stg$
with the same id (i.e., $r'$ is an updated version of $r$), then $r$
must be present unmodified in $\stg'$. This prevents a concurrent
transaction from changing $r$, thus simulating the behavior of an
exclusive lock. There is however a caveat: if we assume extensional
equality over records, a write by a concurrent transaction that
doesn't change $r$'s contents is still allowed. While this in itself
is not a problem, the concurrent transaction may modify other records,
which then become visible in the current transaction. A write lock
prevents this behavior, whereas our axiomatization ($\I_{ww}$) allows
it. An easy fix for this is to add version timestamps to records that
effectively intensionalizes equality. Nonetheless, imprecise
axiomatization of write locks hasn't been a problem in practice.

\textbf{Snapshots} Both MySQL and PostgresSQL implement isolation levels
that execute transactions against a static snapshot of the database.
The semantics of a snapshot be axiomatized as following:
\begin{smathpar}
\begin{array}{lcl}
  \I_{ss}(\stl,\stg,\stg') & = & \stg' = \stg.
\end{array}
\end{smathpar}
% The snapshot semantics however may not be applicable throughout the
% lifetime of the transaction, like in the case of Repeatable Read (RR).

\textbf{Rollbacks} Rollbacks triggered by the post-facto detection of
write-write conflicts can be axiomatized as following:
\begin{smathpar}
\begin{array}{lcl}
  \I_{rb}(\stl,\stg,\stg') & = & \forall(r\in\stl)(r'\in\stg).~
  r'.\idf = r.\idf \Rightarrow r'\in\stg'.
\end{array}
\end{smathpar}

\textbf{Read-Only Transactions}. Certain databases implement special
previleges for read-only transactions. Read-only behavior can be
enforced on a transaction by including the following proposition in
its $\I$:
\begin{smathpar}
\begin{array}{lcl}
  \I_{ro}(\stl,\stg,\stg') & = & \stl = \emptyset\\
\end{array}
\end{smathpar}
If a transaction declared as read-only performs a write, then its
$\stl\neq \emptyset$, and the transaction never commits.

%\usepackage[table,xcdraw]{xcolor}
\begin{table}[]
\centering
\begin{tabular}{l|l|l|l|l|l|l|l|l|l|l|l|l|}
\cline{2-13}
                                & \multicolumn{6}{c|}{{\color[HTML]{333333} PostgreSQL}}                         & \multicolumn{6}{c|}{MySql}                                                   \\ \cline{2-13} 
                                & \multicolumn{2}{c|}{RC} & \multicolumn{2}{c|}{RR} & \multicolumn{2}{c|}{SER} & \multicolumn{2}{c|}{RC} & \multicolumn{2}{c|}{RR} & \multicolumn{2}{c|}{SER} \\ \cline{2-13} 
                                & T          & F          & T          & F          & T           & F          & T          & F          & T          & F          & T           & F          \\ \hline
\multicolumn{1}{|l|}{$\I_{ss}$} &            &            & \checkmark &            & \checkmark  & \checkmark &            &            & \checkmark &            & \checkmark  & \checkmark \\ \hline
\multicolumn{1}{|l|}{$\I_{rb}$} &            &            &            & \checkmark &             &            &            &            &            &            &             &            \\ \hline
\multicolumn{1}{|l|}{$\I_{ro}$} &            &            &            &            &             &            &            &            & \checkmark & \checkmark &             &            \\ \hline
\multicolumn{1}{|l|}{$\I_{id}$} & \checkmark & \checkmark & \checkmark & \checkmark & \checkmark  & \checkmark & \checkmark & \checkmark & \checkmark & \checkmark & \checkmark  & \checkmark \\ \hline
\multicolumn{1}{|l|}{$\I_{ww}$} & \checkmark & \checkmark & \checkmark & \checkmark & \checkmark  & \checkmark & \checkmark & \checkmark & \checkmark & \checkmark & \checkmark  & \checkmark \\ \hline
\end{tabular}

\caption{The semantics of isolation levels on PostgreSQL and MySQL}
\label{fig:iso-table}
\end{table}

With help of the above axiomatizations, we can now define the precise
semantics of various isolation levels on PostgreSQL and MySQL. The table
in Table~\ref{fig:iso-table} captures the summary.
