\subsection{Isolation Specifications}
\label{sec:isolation}

A distinctive characteristic of our development is that it is
parameterized on the isolation specification $\I$. $\I$ can be
instantiated with the declarative characterization of an isolation
guarantee or a concurrency control mechanism, regardless of its actual
implementation. This allows us to model a range of isolation
properties that are relevant to the theory and practice of realistic
transaction processing systems, some of the most well-known of which are
given below:

% Databases implement isolation levels through a combination of various
% concurrency control mechanisms, which are absent from
% Fig.~\ref{fig:txnimp}. In this section, we first axiomatize the
% semantics of various concurrency control constructs, and then define
% the precise semantics of the isolation levels as they are implemented
% on two real-world databases -  PostgreSQL and MySQL.
% Any additional concurrency control provided by the database
% implementation, beyond that which is already incuded in the
% isolation specification, also needs to be axiomatized in $\I$. We
% first describe such axiomatizations, and then present the
% specifications of the standard isolation levels as implemented by
% popular off-the-shelf databases.

\textbf{Unique Ids}. As the \C{new\_order} example
(Sec.~\ref{sec:motivation}) demonstrates, enforcing global uniqueness
of ordered identifiers requires stronger isolation than possible using
a weak read committed isolation level. Alternatively, globally unique
sequence numbers, regardless of the isolation level, can be requested
from a relational databases via SQL's \C{UNIQUE} and
\C{AUTO\_INCREMENT} keywords. Our development crucially relies on the
uniqueness of record identifiers\footnote{This has parallels in
  real-world. For e.g., MySQL's InnoDB engine automatically adds a
  6-byte unique identifier if none exists for a record.}, which are
generated locally by the \rulelabel{E-Insert} rule.  The global
uniqueness of locally generated identifiers can be axiomatized as an
isolation property as following:
\begin{smathpar}
\begin{array}{lcl}
  \I_{id}(\stl,\stg,\stg') & = & \forall(r\in\stl).~
      r.\idf\notin \dom(\stg) \Rightarrow r.\idf\notin \dom(\stg').
\end{array}
\end{smathpar}
$\I_{id}$ ensures that a globally unique id generated during a
transaction's execution remains globally unique until it commits,
prohibiting interference from a concurrent transaction that adds the
same id. The axiom thus simulates a global counter protected by an
exclusive lock without explicitly appealing to specific
implementation artifacts.

\textbf{Write-Write Conflicts}. Databases often employ a combination
of concurrency control methods, both optimistic (e.g., speculation and
rollback) and pessimistic (e.g., various degrees of locking), to
eliminate write-write (ww) conflicts among concurrent
transactions. We can specify the  absence of such conflicts using our
tri-state formulation thus:
\begin{smathpar}
\begin{array}{lcl}
  \I_{ww}(\stl,\stg,\stg') & = & \forall(r'\in\stl)(r \in \stg).~
      r.\idf = r'.\idf  \Rightarrow r\in\stg'.
\end{array}
\end{smathpar}
That is, given a record $r'\in\stl$, if there exists an $r\in\stg$
with the same id (i.e., $r'$ is an updated version of $r$), then $r$
must be present unmodified in $\stg'$. This prevents a concurrent
transaction from changing $r$, thus simulating the behavior of an
exclusive lock or a speculative execution that succeeded (Note: a
transaction writing to $r$ always changes $r$ because its $\txnf$
field is updated). 
% There is however a caveat: if we assume extensional
% equality over records, a write by a concurrent transaction that
% doesn't change $r$'s contents is still allowed. While this in itself
% is not a problem, the concurrent transaction may modify other records,
% which then become visible in the current transaction. A write lock
% prevents this behavior, whereas our axiomatization ($\I_{ww}$) allows
% it. An easy fix for this is to add version timestamps to records that
% effectively intensionalizes equality. Nonetheless, imprecise
% axiomatization of write locks hasn't been a problem in practice.

\textbf{Snapshots} Almost all major relational databases implement
isolation levels that execute transactions against a static snapshot
of the database that can be axiomatized thus:
\begin{smathpar}
\begin{array}{lcl}
  \I_{ss}(\stl,\stg,\stg') & = & \stg' = \stg.
\end{array}
\end{smathpar}
% The snapshot semantics however may not be applicable throughout the
% lifetime of the transaction, like in the case of Repeatable Read (RR).

\textbf{Read-Only Transactions}. Certain databases implement special
privileges for read-only transactions. Read-only behavior can be
enforced on a transaction by including the following proposition as
part of its isolation invariant:
\begin{smathpar}
\begin{array}{lcl}
  \I_{ro}(\stl,\stg,\stg') & = & \stl = \emptyset\\
\end{array}
\end{smathpar}
%% If a transaction declared as read-only performs a write, then its
%% $\stl\neq \emptyset$, and the transaction never commits.

In addition to these properties, there are various isolation
guarantees proposed in the database or distributed systems literature,
or implemented by commercial vendors that can also be specified within
this framework:

\textbf{Read Committed (RC) and Monotonic Atomic View (MAV)} RC
isolation allows a transaction to witness writes of committed
transactions at any point during the transaction's execution.
Although it offers only weak isolation guarantees, it nonetheless
prevents witnessing \emph{dirty writes} (i.e., writes performed by
uncommitted transactions).  Monotonic Atomic View
(MAV)~\cite{bailishat} is an extension to RC that guarantees the
continuous visibility of a committed transaction's writes once they
become visible in the current transaction. That is, a MAV transaction
does not witness \emph{disappearing writes}, which can happen on a
weakly consistent machine. Due to the SC nature of our abstract
machine (there is always a single global database state $\stg$; not a
vector of states indexed by vector clocks), and our choice to never
violate atomicity of a transaction's writes, both RC and MAV are
already guaranteed by our semantics.  Thus, defining $\I_e$ and $\I_c$
to \emph{true} ensures RC and MAV behavior under our semantics.

\textbf{Repeatable Read (RR)} By definition, a read to a transactional
variable in a Repeatable Read transaction is required to return the
same value whenever accessed.  RR is often implemented (for e.g., in
~\cite{mysqliso,bailishat}) by executing the transaction against a
(conceptual) snapshot of the database, but committing its writes to
the actual database. This implementation of RR can be axiomatized as
$\I_e = \I_{ss}$ and $\I_{c}=true$. However, this specification of RR
is stronger than the ANSI SQL specification, which requires no more
than the invariance of already read records. In particular, ANSI SQL
RR allows \emph{phantom reads}, a phenonmenon in which a repeated
\C{SELECT} query might return newly inserted records that were not
previously returned. This specification is implemented, for e.g., in
Microsoft's SQL server, using record-level exclusive read locks, that
prevent a record from being modified while it is read by an
uncommitted transaction.  The ANSI SQL RR specification can be
axiomatized in our framework, but it requires a minor extension to our
operational semantics to track a transaction's reads. In particular,
the records returned by \C{SELECT} should be added to the local
database $\stl$, but without changing their transaction identifiers
($\txnf$ fields), and flush ($\gg$) should only flush the records that
bear the current transaction's identifier. With this extension, ANSI
SQL RR can be axiomatized thus:
\begin{smathpar}
\begin{array}{lcl}
  \I_e(\stl,\stg,\stg') & \Leftrightarrow & \forall(r\in\stl).
      r \in \Delta \Rightarrow r \in \Delta'\\
  \I_c(\stl,\stg,\stg') & \Leftrightarrow & true\\
\end{array}
\end{smathpar}
If a record $r$ belongs to both $\stl$ and $\stg$, then it must be a
record written by a different transaction and read by the current
transaction. By requiring $r\in\stg'$, $\I_e$ guarantees the
invariance of $r$, thus the repeatability of the read. 

\textbf{Snapshot Isolation (SI)} The concept of executing a
transaction against a consistent snapshot of the database was first
proposed as Snapshot Isolation in~\cite{berenson}. SI doesn't admit
write-write conflicts, and the original proposal, which is implemented
in Microsoft SQL Server, required the database to roll-back an SI
transaction if conflicts are detected during the commit. This behavior
can be axiomatized thus:
\begin{smathpar}
\begin{array}{lcl}
\I_e\,\,(\stl,\stg,\stg') & = & \stg' = \stg\\
\I_c\,\,(\stl,\stg,\stg') & = & \forall(r\in\stl)(r'\in\stg).~ r'.\idf = r.\idf \Rightarrow r'\in\stg'.
\end{array}
\end{smathpar}
Observe that the same axiomatization applies to Postgres's RR,
although its implementation (described in Sec.~\ref{sec:motivation})
differs considerably from the original proposal. Thus, reasoning done
for an SI transaction on MS SQL server carries over to Postgres's RR
and vice-versa, demonstrating the benefits of reasoning axiomatically
about isolation properties.

\textbf{Serializability (SER)} Axiomatization of serializability is
straightforward:
\begin{smathpar}
\begin{array}{lcl}
\I_e\,\,(\stl,\stg,\stg') & = & \stg' = \stg\\
\I_c\,\,(\stl,\stg,\stg') & = & \stg' = \stg.
\end{array}
\end{smathpar}



%% %\usepackage[table,xcdraw]{xcolor}
%% \begin{table}[]
%% \centering
%% \begin{tabular}{l|l|l|l|l|l|l|l|l|l|l|l|l|}
%% \cline{2-13}
%%                                 & \multicolumn{6}{c|}{{\color[HTML]{333333} PostgreSQL}}                         & \multicolumn{6}{c|}{MySql}                                                   \\ \cline{2-13} 
%%                                 & \multicolumn{2}{c|}{RC} & \multicolumn{2}{c|}{RR} & \multicolumn{2}{c|}{SER} & \multicolumn{2}{c|}{RC} & \multicolumn{2}{c|}{RR} & \multicolumn{2}{c|}{SER} \\ \cline{2-13} 
%%                                 & T          & F          & T          & F          & T           & F          & T          & F          & T          & F          & T           & F          \\ \hline
%% \multicolumn{1}{|l|}{$\I_{ss}$} &            &            & \checkmark &            & \checkmark  & \checkmark &            &            & \checkmark &            & \checkmark  & \checkmark \\ \hline
%% \multicolumn{1}{|l|}{$\I_{rb}$} &            &            &            & \checkmark &             &            &            &            &            &            &             &            \\ \hline
%% \multicolumn{1}{|l|}{$\I_{ro}$} &            &            &            &            &             &            &            &            & \checkmark & \checkmark &             &            \\ \hline
%% \multicolumn{1}{|l|}{$\I_{id}$} & \checkmark & \checkmark & \checkmark & \checkmark & \checkmark  & \checkmark & \checkmark & \checkmark & \checkmark & \checkmark & \checkmark  & \checkmark \\ \hline
%% \multicolumn{1}{|l|}{$\I_{ww}$} & \checkmark & \checkmark & \checkmark & \checkmark & \checkmark  & \checkmark & \checkmark & \checkmark & \checkmark & \checkmark & \checkmark  & \checkmark \\ \hline
%% \end{tabular}

%% \caption{The semantics of isolation levels on PostgreSQL and MySQL}
%% \label{fig:iso-table}
%% \end{table}

%% With help of the above axiomatizations, we can now define the precise
%% semantics of various isolation levels on PostgreSQL and MySQL. The table
%% in Table~\ref{fig:iso-table} captures the summary.
