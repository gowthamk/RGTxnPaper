The second observation that informs our approach is one that pertains
to automation. Program verification, even when machine-aided, often
entails significant annotation burden in the form of intermediary
assertions and loop invariants required to prove a program correct.
Even with concurrent program logics, such as Rely-Gurantee, which
extend Hoare logic with additional artifacts, (stable) intermediary
assertions and loop invariants remain a major source of annotation
burden. Since database transactions inherently deal with the mutable
state of the database, it is natural to formulate them as concurrent
imperative programs. An advantage with this formulation is that it
lets us immediately apply existing concurrent program logics to
database transactions, albeit with necessary customizations to deal
with weak isolation. The downside, however, is that there hardly is
any automation support to apply such program logics on concurrent
imperative programs. Attemps at automation face the problem of
inferring stable and inductive loop invariants in its full generality,
which, while being undecidable in theory, and hard in practice, also
takes our focus away from weak isolation.

In the light of the above observations, we refrain from thinking of
database transactions as concurrent imperative programs insofar as
automation is concerned.  Instead, we see value in viewing them as
essentially functional programs that manage the database state via a
monad akin to the \C{State} monad~\cite{statemonad}. We find it useful
to reason about statements that mutate the database state, not in
terms of a pre- and post-condition pair, but in terms of a state
transformer that relates the pre- and post-states of a statement. The
state transformer semantics is defined algorithmically, just like the
predicate transformer semantics (e.g., strongest post-condition), but
the semantic domain doesn't admit arbitrary logical formulas. Instead,
it is the domain of sets equipped with a (set) monad bind ($\bind$)
operator.  Thus, a state transformer interprets a SQL statement in the
set domain, taking advantage of the fact that a a database is
essentially a set of tuples, and a SQL statement a set transformer.
The benefit of this approach is that loops can now be substituted with
higher-order combinators that automatically lift the state transformer
of its higher-order argument (i.e., the loop body) to the state
transformer of the combined expression (i.e., the loop).  Thus the
semantics of a \C{foreach} loop, for instance, can be captured as a
state transformer, where the state is a set, and the transformation
involves a bind operation. We illustrate this intuition on a simple
example.


% \begin{figure}[!t]
% \centering
% %
% \begin{subfigure}[b]{0.46\textwidth}
% \begin{ocaml}
% Set s' = $\emptyset$;
% foreach x in s {
%   s'.add(f(x)); 
% }
% \end{ocaml}
% \caption{}
% \end{subfigure}
% %
% \begin{subfigure}[b]{0.5\textwidth}
% \begin{ocaml}
% let s' = ref [];
% foreach s (fun x -> s' := x::(!s'));
% \end{ocaml}
% \caption{Lorem ipsum, lorem ipsum,Lorem ipsum, lorem ipsum,Lorem ipsum}
% \end{subfigure}
% \caption{Caption place holder}
% %
% \caption{New versions are created from existing versions either
% through \C{push} or \C{merge}.}
% \label{fig:syntactic-ancestors}
% \end{figure}
% let s' = ref (Set.empty) in
% foreach xs @@ fun x -> 
%   begin
%     s' := Set.union !s' @@ 
%             Set.map_selected s (fun y -> y<x)
%                                (fun y -> y+x);
%     s' := Set.add !s' x;
%   end
\begin{figure}[!h]
\begin{ocaml}
foreach item_reqs @@ fun item_req -> do
  SQL.update Stock (fun s -> {s with s_qty = k1}) 
                   (fun s -> s.s_i_id = item_req.ol_i_id);
  SQL.insert Order_line {ol_o_id=k2; ol_d_id=k3; 
                         ol_i_id=item_req.ol_i_id; ol_qty=item_req.ol_qty}
\end{ocaml}
\caption{Foreach loop from Fig.~\ref{fig:new_order_code}}
\label{fig:foreach_code}
\end{figure}

Fig.~\ref{fig:foreach_code} shows a (simplified) snippet of code taken
from Fig.~\ref{fig:new_order_code}. Few irrelevant expressions have
been replaced with constants (\C{k1} to \C{k3}). Body of the loop
executes a SQL update followed by an insert. Recall that a transaction
reads from the global database ($\Delta$), and writes to a
transaction-local database ($\delta$). An update
statement filters the tuples that match the search criteria from the
global database ($\Delta$), and computes the updated tuples to be
added to the local database. Thus, its state transformer (call it
$T_U$) is the following function on sets (Ocaml syntax is used when
convenient; $\bind$ has higher precedence than $\cup$):
\begin{ocaml}
  fun ($\Delta$,$\delta$) -> $\delta$ $\cup$ $\Delta$$\bind$(fun s -> if table(s) = stock && s.s_i_id = item_req.ol_i_id 
                                 then {{s with s_qty = k1}} (* a singleton set *)
                                 else $\emptyset$)
\end{ocaml}
% \begin{smathpar}
% \begin{array}{c}
%   \lambda\Delta.\lambda\delta.\, \delta \;\cup\; 
%       \Delta \,\bind\, (\lambda s.\, 
%           \C{if}\;{\C{stock}(s) \conj 
%                    \C{s.s\_i\_id} = \C{item\_req.ol\_i\_id}}\\
%            \hspace*{0.9in}
%            \C{then}\;{\{\{s \;\C{with}\; \C{s\_qty} = \C{k1}\}\}}\;
%            \C{else}\;{\emptyset})
% \end{array}
% \end{smathpar}
% \begin{verbatim}
%   Rmem(σ'(s')) = Rmem(σ(s')) U 
%                  Rmem[\x. Rmem[\y. if y<x then {y+x} else {}](σ(s)) U {x}](xs)
% \end{verbatim}
The transformer maps the current global ($\Delta$) and local
($\delta$) databases to the new local database. Observe that
$T_U(\Delta,\delta)$ is of the form $\delta \,\cup\,
F_U(\Delta)$. The transformer ($T_I(\Delta,\delta)$)
for the subsequent \C{insert} statement can be similarly calculated to
be of the form $\delta \cup F_I(\Delta)$.
Their composition gives the state transformer of the loop body:
\begin{ocaml}
  fun ($\Delta$,$\delta$) -> $\delta$ $\cup$ $F_U(\Delta) \,\cup\, F_I(\Delta)$
\end{ocaml}
Let $F_{body}(\Delta) = F_U(\Delta) \,\cup\, F_I(\Delta)$.  Observe
that the loop variable \C{item\_req} occurs free in $F_{body}$. The
transformer for the \C{foreach} loop can now be computed as following:
\begin{ocaml}
  fun ($\Delta$,$\delta$) -> $\delta$ $\cup$ item_reqs$\bind$(fun item_req -> $F_{body}(\Delta)$)
\end{ocaml}
Observe that the transformer captures the precise semantics of the
\C{foeach} loop as a formula in the domain of sets extended with a
bind function. The advantage of doing so is that we can now make use
of a  semantics-preserving translation from this domain to the
first-order logic, allowing us to leverage SMT solvers for automatic
proofs without having to infer a loop invariant.
Sec.~\ref{sec:automation} describes this translation. In the
exposition, we assumed the global database state ($\Delta$) to remain
invaraint, which is clearly not the case when we admit concurrency.
The concurrency extension of the state transformer semantics is also
covered in Sec.~\ref{sec:automation}. The immediate two sections,
however, focus on laying theoretical foundations for reasoning about
weak isolation.
