%-----------------------------------------------------------------------------
%
%               Template for sigplanconf LaTeX Class
%
% Name:         sigplanconf-template.tex
%
% Purpose:      A template for sigplanconf.cls, which is a LaTeX 2e class
%               file for SIGPLAN conference proceedings.
%
% Guide:        Refer to "Author's Guide to the ACM SIGPLAN Class,"
%               sigplanconf-guide.pdf
%
% Author:       Paul C. Anagnostopoulos
%               Windfall Software
%               978 371-2316
%               paul@windfall.com
%
% Created:      15 February 2005
%
%-----------------------------------------------------------------------------


\documentclass[acmsmall,screen]{acmart}

% The following \documentclass options may be useful:

% preprint      Remove this option only once the paper is in final form.
% 10pt          To set in 10-point type instead of 9-point.
% 11pt          To set in 11-point type instead of 9-point.
% numbers       To obtain numeric citation style instead of author/year.

\usepackage{amssymb}
\usepackage{amsmath}
\usepackage{amsfonts}
\usepackage{caption}
\usepackage{subcaption}
\usepackage{xspace}
\usepackage{mathtools}
\usepackage{mathpartir}
\usepackage{ifpdf}
\usepackage{graphicx}
%\usepackage[usenames,dvipsnames]{color}
\usepackage{stmaryrd}
%\usepackage[numbers]{natbib}
\usepackage{amsthm}
\usepackage{listings}          % format code
\usepackage{wrapfig}
\usepackage{textcomp}
\usepackage{tabularx}
\usepackage{color}
\usepackage{url}
\usepackage{tikz}
\usepackage{multirow,array}
\usepackage[utf8]{inputenc}
\usepackage[T1]{fontenc}
\usepackage{microtype}

% Math mode
%-----------
\newenvironment{nop}{}{}
\newenvironment{smathpar}{
\begin{nop}\small\begin{mathpar}}{
\end{mathpar}\end{nop}\ignorespacesafterend}

% Theorem
%--------
\newtheorem{axiom}{Axiom}[section]
\newtheorem{theorem}{Theorem}[section]
\newtheorem{lemma}[theorem]{Lemma}
\newtheorem{proposition}[theorem]{Proposition}
\newtheorem{corollary}[theorem]{Corollary}

\newenvironment{definition}[1][Definition]{\begin{trivlist}
\item[\hskip \labelsep {\bfseries #1}]}{\end{trivlist}}
\newenvironment{example}[1][Example]{\begin{trivlist}
\item[\hskip \labelsep {\bfseries #1}]}{\end{trivlist}}
\newenvironment{remark}[1][Remark]{\begin{trivlist}
\item[\hskip \labelsep {\bfseries #1}]}{\end{trivlist}}

% Decorations
%-----------
\newenvironment{decoration}
  {\color{blue}\begin{array}{l}}
  {\end{array}}

% Listings
%----------
\newcommand{\lsttxnimp}{\lstset{
      language=c,
      basicstyle=\ttfamily\normalsize,
      flexiblecolumns=false,
			tabsize=2,
      escapechar=',                        
      %basewidth={0.5em,0.45em},
      %aboveskip={3pt},
      %belowskip={3pt},
      keywordstyle=\color{Bittersweet}\bfseries,
      commentstyle=\color{blue}\itshape,
      stringstyle=\color{MidnightBlue},
      morekeywords={transaction,txn,cobegin,from,to},
			classoffset=1,
			upquote=true,
			keywordstyle=\color{Fuchsia}\bfseries,
			classoffset=0,
			mathescape=true
    }}
\lstnewenvironment{txnimpcode}
    { % \centering
      \lsttxnimp
      \lstset{}%
      \csname lst@setfirstlabel\endcsname}
    { %\centering
      \csname lst@savefirstlabel\endcsname}

% Formatting
%---------
\newcommand{\C}[1]{\code{#1}}
\newcommand{\tuplee}[1]{\langle #1 \rangle}
\newcommand*{\rom}[1]{\expandafter\romannumeral #1}

% Formatting commands
% -------------------
\newcommand{\code}[1]{\,{\tt #1}\,}
\newcommand{\spc}[0]{\quad}
\newcommand{\ALT}{~\mid~}
\newcommand{\rel}[1]{{R}_{\mathit{#1}}}
\newcommand{\conj}{~\wedge~}
\newcommand{\disj}{~\vee~}
\newcommand{\rulelabel}[1]{\textrm{\sc {#1}}}
\newcommand{\ilrulelabel}[1]{{\sc #1}}
\newcommand{\RULE}[2]{\frac{\begin{array}{c}#1\end{array}}
                           {\begin{array}{c}#2\end{array}}}
\newcommand{\txnimp}{{\sc TxnImp}\xspace}
\newcommand{\coloneqq}{::=}
\newcommand{\qqquad}{\quad\quad}
\newcommand{\cskip}{\C{SKIP}}
\newcommand{\ctxn}[1]{\C{Txn}\{#1\}}
\newcommand{\catomic}[1]{\C{ATOMIC}\{#1\}}
\newcommand{\stepsto}{\longrightarrow}
\newcommand{\rstepsto}{\longrightarrow_{R}}
\newcommand{\stepssto}{\longrightarrow^{*}}
\newcommand{\rstepssto}[1]{\longrightarrow^{#1}_{R}}
\newcommand{\xstepsto}[1]{\longrightarrow_{#1}}
\newcommand{\xstepssto}[2]{\longrightarrow^{#1}_{#2}}
\newcommand{\tstepsto}{\longrightarrow}
\newcommand{\redsto}{\hookrightarrow}
\newcommand{\xtstepsto}[1]{\hookrightarrow_{#1}}
\newcommand{\rtstepsto}{\hookrightarrow_R}
\newcommand{\rtstepssto}[1]{\hookrightarrow^{#1}_R}
\newcommand{\xtstepssto}[2]{\hookrightarrow^{#1}_{#2}}
\newcommand{\hoare}[3]{\{#1\}\,#2\,\{#3\}}
\newcommand{\defeq}[0]{\overset { def }{ = } }
\newcommand{\rg}[3]{\{#1\}\,#2\,\{#3\}}
%\newcommand{\defeq}[0]{ \triangleq }
\newcommand{\op}{\textsf{op}}
\newcommand{\E}{\textsf{E}}
\newcommand{\A}{\textsf{A}}
\newcommand{\I}{\mathbb{I}}
\newcommand{\F}{\mathcal{F}}
\newcommand{\visZ}{\textsf{vis}}
\newcommand{\soZ}{\textsf{so}}
\newcommand{\hbZ}{\textsf{hb}}
\newcommand{\sameobj}[2]{\textsf{sameobj}(#1,#2)}
\newcommand{\sameobjZ}{\textsf{sameobj}}
\newcommand{\visar}{\xrightarrow{\visZ}}
\newcommand{\hboar}{\xrightarrow{\textsf{hb}}}
\newcommand{\soar}{\xrightarrow{\soZ}}
\newcommand{\invisar}{\xrightarrow{\textsf{invis}}}
\newcommand{\etaar}{\xrightarrow{\eta}}
\newcommand{\wrstoar}{\xrightarrow{\textsf{wrsto}}}
\newcommand{\rdsfmar}{\xrightarrow{\textsf{rdsfm}}}
\newcommand{\fresh}{\textsf{fresh}}
\newcommand{\isReadf}{\textsf{isRD}}
\newcommand{\isWritef}{\textsf{isWR}}
\newcommand{\oper}{\textsf{oper}}
\newcommand{\committed}{\textsf{com}}
\newcommand{\txn}{\textsf{txn}}
\newcommand{\id}{\textsf{id}}
\newcommand{\kind}{\textsf{oper}}
\newcommand{\rval}{\textsf{rval}}
\newcommand{\visible}{\textsf{visible}}
\newcommand{\maxId}{\textsf{maxId}}
\newcommand{\aeval}{\textsf{aeval}}
\newcommand{\underE}[1]{\E \Vdash #1}
\newcommand{\underIT}[1]{\;\I,\C{Txn}_i \vdash #1\;}
\newcommand{\underI}[1]{\;\I \vdash #1\;}
\newcommand{\underT}[1]{\;\C{Txn}_i \vdash #1\;}
\newcommand{\stable}{\mathtt{stable}}
\newcommand{\iso}[1]{{\sc #1}}
\newcommand{\writef}{\textsf{Write}}
\newcommand{\readf}{\textsf{Read}}
\newcommand{\commitf}{\textsf{Commit}}
\newcommand{\eg}{\emph{e.g.,}}
\newcommand{\GK}[1]{\textcolor{red}{GK: #1}}

\newcommand{\B}[1]{\small\bf #1}
\newcommand{\txnbox}[1]{\lbrack #1 \rbrack}
\newcommand{\interp}[1]{\llbracket #1 \rrbracket}
\newcommand{\cinterp}[1]{\llbracket #1 \rrbracket_{\C{C}}}
\newcommand{\ectx}{\mathcal{E}}
\newcommand{\isMax}{\textsf{isMax}}


%% Journal information (used by PACMPL format)
%% Supplied to authors by publisher for camera-ready submission
\setcopyright{rightsretained} 
\acmJournal{PACMPL}
\acmYear{2018} 
\acmVolume{2} 
\acmNumber{POPL} 
\acmArticle{27}
\acmMonth{1} 
\acmPrice{}
\acmDOI{10.1145/3158115}

% To avoid orphans and widows
\clubpenalty = 10000
\widowpenalty = 10000
\displaywidowpenalty = 10000

\begin{document}

\special{papersize=8.5in,11in}
\setlength{\pdfpageheight}{\paperheight}
\setlength{\pdfpagewidth}{\paperwidth}


\makeatletter\if@ACM@journal\makeatother
\startPage{1}
\else\makeatother
% %% Conference information (used by SIGPLAN proceedings format)
% %% Supplied to authors by publisher for camera-ready submission
% \GK{---- We did not get any ----}
% \acmConference[PL'17]{ACM SIGPLAN Conference on Programming Languages}{January 01--03, 2017}{New York, NY, USA}
% \acmYear{2017}
% \acmISBN{978-x-xxxx-xxxx-x/YY/MM}
% \acmDOI{10.1145/nnnnnnn.nnnnnnn}
% \startPage{1}
\fi

%% Bibliography style
\bibliographystyle{ACM-Reference-Format}
% Uncomment the publication rights you want to use.
%\publicationrights{transferred}
%\publicationrights{licensed}     % this is the default
%\publicationrights{author-pays}
%% Citation style
%% Note: author/year citations are required for papers published as an
%% issue of PACMPL.
\citestyle{acmauthoryear}   %% For author/year citations


\title{Alone Together:
  Compositional Reasoning and Inference for Weak Isolation}

%% Author information
%% Contents and number of authors suppressed with 'anonymous'.
%% Each author should be introduced by \author, followed by
%% \authornote (optional), \orcid (optional), \affiliation, and
%% \email.
%% An author may have multiple affiliations and/or emails; repeat the
%% appropriate command.
%% Many elements are not rendered, but should be provided for metadata
%% extraction tools.

%% Author with single affiliation.
\author{Gowtham Kaki}
%\authornote{with author1 note}          %% \authornote is optional;
                                        %% can be repeated if necessary
%\orcid{nnnn-nnnn-nnnn-nnnn}             %% \orcid is optional
\affiliation{
% \position{Position1}
% \department{Department1}              %% \department is recommended
  \institution{Purdue University}            %% \institution is required
% \streetaddress{Street1 Address1}
% \city{City1}
% \state{State1}
% \postcode{Post-Code1}
  \country{USA}
}
\email{gkaki@purdue.edu}          %% \email is recommended
%
\author{Kartik Nagar}
\affiliation{
  \institution{Purdue University}            %% \institution is required
  \country{USA}
}
\email{nagark@purdue.edu}          %% \email is recommended
%
\author{Mahsa Najafzadeh}
\affiliation{
  \institution{Purdue University}            %% \institution is required
  \country{USA}
}
\email{mnajafza@purdue.edu}          %% \email is recommended
%
\author{Suresh Jagannathan}
\affiliation{
  \institution{Purdue University}            %% \institution is required
  \country{USA}
}
\email{suresh@cs.purdue.edu}          %% \email is recommended

% %% Author with two affiliations and emails.
% \author{First2 Last2}
% \authornote{with author2 note}          %% \authornote is optional;
%                                         %% can be repeated if necessary
% \orcid{nnnn-nnnn-nnnn-nnnn}             %% \orcid is optional
% \affiliation{
%   \position{Position2a}
%   \department{Department2a}             %% \department is recommended
%   \institution{Institution2a}           %% \institution is required
%   \streetaddress{Street2a Address2a}
%   \city{City2a}
%   \state{State2a}
%   \postcode{Post-Code2a}
%   \country{Country2a}
% }
% \email{first2.last2@inst2a.com}         %% \email is recommended
% \affiliation{
%   \position{Position2b}
%   \department{Department2b}             %% \department is recommended
%   \institution{Institution2b}           %% \institution is required
%   \streetaddress{Street3b Address2b}
%   \city{City2b}
%   \state{State2b}
%   \postcode{Post-Code2b}
%   \country{Country2b}
% }
% \email{first2.last2@inst2b.org}         %% \email is recommended


%% Abstract
%% Note: \begin{abstract}...\end{abstract} environment must come
%% before \maketitle command
\begin{abstract}

  Serializability is a well-understood correctness criterion that
  simplifies reasoning about the behavior of concurrent transactions
  by ensuring that transactions are \emph{isolated} from each other
  while they execute.  However, enforcing serializable isolation comes
  at a steep cost in performance because it necessarily restricts
  opportunities to exploit concurrency even when such opportunities
  would not violate application-specific invariants. As a result,
  database systems in practice support, and often encourage,
  developers to use weaker alternatives. These alternatives break the
  strong isolation guarantees offered by serializable transactions to
  permit greater concurrency. Unfortunately, the semantics of weak
  isolation is poorly understood, and usually explained only
  informally in terms of low-level implementation
  artifacts. Consequently, verifying high-level correctness properties
  in such environments remains a challenging problem.

  To address this issue, we present a program logic that enables
  compositional reasoning about the behavior of concurrently executing
  weakly-isolated transactions. Notably, our development is parametric
  over a transaction's specific isolation semantics, and the
  consistency guarantees provided by the underlying data store,
  allowing it to be applicable over a range of concurrency control
  mechanisms.  Case studies and experiments on real-world applications
  demonstrate the utility of our approach, and provide strong evidence
  that weakly-isolated transactions can be placed on the same formal
  footing as their strongly-isolated serializable counterparts.

\end{abstract}


% Classification
\begin{CCSXML}
  <ccs2012>
    <concept>
      <concept_id>10011007.10011074.10011099.10011692</concept_id>
      <concept_desc>Software and its engineering~Formal software verification</concept_desc>
      <concept_significance>500</concept_significance>
    </concept>
    <concept>
      <concept_id>10002951.10002952.10003190.10003206</concept_id>
      <concept_desc>Information systems~Integrity checking</concept_desc>
      <concept_significance>500</concept_significance>
    </concept>
    <concept>
      <concept_id>10002951.10002952.10002953.10002955</concept_id>
      <concept_desc>Information systems~Relational database model</concept_desc>
      <concept_significance>300</concept_significance>
    </concept>
  </ccs2012>
\end{CCSXML}

\ccsdesc[500]{Software and its engineering~Formal software verification}
\ccsdesc[500]{Information systems~Integrity checking}
\ccsdesc[300]{Information systems~Relational database model}

% Keywords
\keywords{
Transactions, Weak Isolation, Concurrency, Rely-Guarantee,
Verification}

\maketitle

\section{Introduction}

Database transactions allow users to group operations on multiple
objects into a coherent unit with useful properties including
atomicity, consistency, (serializable) isolation and durability
(ACID). ACID has been the gold standard for database transactions for
many decades, providing programmers with a simple
implementation-independent model to reason about concurrent
transactions. However, serializability, the ACID-prescribed standard for
isolation, does not come for free; it requires databases to rely on
expensive techniques such as two-phase locking~\cite{twopl,ullmanbook}
that severely restrict concurrency while incurring an additional
overhead of deadlock detection, rollback and re-execution. Enforcing
serializability in a replicated data store requires global
coordination among geo-distributed replicas, which makes the store
\emph{unavailable} in presence of the inevitable network
partitions~\cite{cap,sernotavlbl,bailishat,bernsigmod13}. The
conflict between serializability and pragmatic goals of high
peformance and availability has motivated the development of multiple
weaker forms of transaction isolation beginning as early as
1976~\cite{gray1976}. ANSI SQL 92 standard defines three such weak
isolation \emph{levels} which are now implemented in
many relational and NoSQL databases. Weakly isolated transactions have
been found to significantly outperform serializable transactions on
benchmark suites, both on single-node~\cite{dbtuningbook} and
multi-node replicated stores~\cite{bailishat,bailisvldb}, leading to
their overwhelming adoption. A 2013 study~\cite{bailishotos} of 18
popular ACID and ``NewSQL'' databases found that only 3 of them offer
serializability by default, and half, including Oracle 11g, do not
offer it at all.  A 2015 study~\cite{bailisferal} of a large corpus of
database applications finds no evidence of any of them changing the
default isolation level offered by the database. Together, the studies
establish the prevalance of weak isolation in practice.

Unfortunately, weak isolation introduces new set of behaviors into
programs that often give rise to anomalous executions and
incomprehensible results~\cite{pldi15}. To quantify weak isolation
anomalies, Fekete et al~\cite{feketevldb09} have devised and
experimented with a microbenchmark suite that executes transactions
under \iso{Read Committed} isolation level - the default level for 8
of the 18 databases studied in~\cite{bailishotos}, and found that 25
out of every 1000 rows in the database violate at least one integrity
constraint. Bailis et al~\cite{bailisferal} rely on Rails's
\emph{uniqueness validation} to maintain uniqueness of records while
serving Linkbench's~\cite{linkbench} insertion workload (6400 records
distributed over 1000 keys; 64 concurrent clients), and end up with
more than 10 duplicate records. Rails relies on database transactions
to validate uniqueness during insertions, which is ok if transactions
are serializable, but incorrect under \iso{Read Committed} isolation
used in the experiments. The same study has found that 13\% of all
invariants among 67 open source Ruby-on-Rails applications are the
risk of being violated due to weak isolation. The ill effects of weak
isolation are therefore as prevalent as weak isolation itself.

The root cause of problems with weak isolation is that its semantics
in context of user programs is not well-understood. The original
proposal~\cite{gray1976} defines multiple ``degrees'' of weak
isolation in terms of implementation details, such as the nature and
duration of locks held in each case. ANSI SQL 92 standard four levels
of isolation (including serializability) in terms of sevaral
undesirable \emph{phenomena} (\eg \emph{dirty reads}) each is required
to prevent. While this is an improvement, it is hard to reason about
application semantics in terms of the undesirable phenomena. Moreover,
a later critique~\cite{berenson} uncovers many ambiguities and
shortcomings in the standard. Adya's thesis~\cite{adyaphd} includes
first formal definitions of some of the well-known isolation levels in
the context of a system model of sequentially consistent database
complete with timestamps and version histories. However, Adya's model,
which predates the proliferation of weakly consistent replicated data
stores, is not enough to express the semantics of a host of new
isolation levels~\cite{psi,nmsi} tailor-made for such stores, or the
interplay between weak consistency and isolation. Nonetheless,
regardless of the specification methodology employed, the major
missing piece in all the aforementioned approaches is the bridge
between the system model of specifications and the operational model
of programs that enables formal reasoning about program invariants in
presence of weak isolation.

There have been encouraging developments in Programming Languages
research towards addressing the problem of reasoning about application
invariants on weakly consistent replicated stores~\cite{burckhardt14,
redblueosdi, redblueatc, ecinec, gotsmanpopl16}. Weak isolation can
make use of some of the reasoning inventory built for weak consistency
(as we demonstrate in later sections), but the latter does not subsume
the former. While weak consistency (as considered by the
aforementioned research) deals with lazily propagating effects of
atomic and fully isolated operations, reasoning about weakly isolated
transactions involves reasoning about the effects of groups of
operations that are not necessarily atomic or isolated. Indeed, weakly
isolated transactions have existed on sequentially consistent
relational databases long before the advent of weakly consistent
replicated stores, indicating the subtle interplay between consistency
and isolation. Moreover, a common theme among the aforementioned line
of work is the assumption of a system with a predefined set of
consistency levels, and a minimum consistency level that is often
stronger than eventual consistency. For instance, the reasoning
framework proposed in~\cite{gotsmanpopl16} crucially relies on the
assumption of causal consistency with only admissable strengthening
being strong consistency. None of the ANSI SQL weak isolation levels
guarantee causal consistency. Moreover, there exist real-world
distributed stores~\cite{spanner} that offer serializable transactions
without causal consistency. Evidently, the reality is significantly
more diverse~\cite{zoo}, with more forms of weak consistency and
isolation being invented to meet the changing needs of applications.
What is needed in this context is a reasoning framework parameterized
over the semantics of weak isolation, that can answer the following
question:
\begin{center}
\emph{Given a program annotated with desired invaraints, and a
selection of isolation levels for transactions in the program, are the
invariants guaranteed to hold?}
\end{center}

In this paper, we propose a reasoning framework that attempts to
answer this question.



\section{Motivation}
\label{sec:motivation}

We present our ideas in the context of our DSL embedded in OCaml,
which offers a \C{DB} monad to define and compose database
computations given in terms of SQL operations.  Besides the usual
\C{bind} and \C{return} operators, the monad offers an \C{atomically}
combinator that executes a database computation as an atomic
transaction and returns the result.

\begin{figure}
\centering
\begin{ocaml}
let new_order (d_id, c_id, item_reqs) = atomically do
  dist <- SQL.select1 District (fun d -> d.d_id = d_id);
  let o_id = dist.d_next_o_id;
  SQL.update(* UPDATE *) District 
            (* SET *)(fun d -> {d with d_next_o_id =d_next_o_id + 1})
            (* WHERE *)(fun d -> d.d_id = d_id );
  SQL.insert(* INSERT INTO *) Order (* VALUES *){o_id=o_id;  
            o_d_id=d_id; o_c_id=c_id; o_ol_cnt=S.size item_reqs; };
  foreach item_reqs @@ fun item_req -> do
    stk <- SQL.select1(* SELECT * FROM *) Stock 
              (* WHERE *)(fun s -> s.s_i_id = item_req.ol_i_id 
                                  && s.s_d_id = d_id)(* LIMIT 1 *); 
    let s_qty' = if stk.s_qty >= item_req.ol_qty + 10 
                then stk.s_qty - item_req.ol_qty 
                else stk.s_qty - item_req.ol_qty + 91;
    SQL.update Stock (fun s -> {s with s_qty = s_qty'}) 
                     (fun s -> s.s_i_id = item_req.ol_i_id);
    SQL.insert Order_line {ol_o_id=o_id; ol_d_id=d_id; 
                           ol_i_id=item_req.ol_i_id; ol_qty=item_req.ol_qty}
 
\end{ocaml}
\caption{TPC-C \C{new\_order} transaction}
\label{fig:new_order_code}
\vspace*{-10pt}
\end{figure}

Fig.~\ref{fig:new_order_code} shows a simplified version of the TPC-C
\C{new\_order} transaction written in this language. TPC-C is a
widely-used Online Transaction Processing (OLTP) benchmark that models
an order-processing system for a wholesale parts supply business. The
business logic is captured in 5 database transactions that operate on
9 tables; \C{new\_order} is one such transaction that uses
\C{District}, \C{Order}, \C{New\_order}, \C{Stock}, and
\C{Order\_line} tables. The transaction acts on the behalf of a
customer, whose id is \C{c\_id}, to place a new order for a given
set of items (\C{item\_reqs}), to be served by a warehouse under the
district identified by \C{d\_id}.  Fig.~\ref{fig:schema} illustrates
the relationship among these different tables.

The transaction manages order placement by invoking appropriate SQL
functionality, captured by various calls to functions defined by the
\C{SQL} module. All \C{SQL} functions take the table name (a nullary
constructor) as their first argument. The higher-order \C{SQL.select1}
function accepts a boolean function that describes the selection
criteria, and returns any record that meets the criteria (it models
the SQL query \C{SELECT \ldots\xspace LIMIT 1}). \C{SQL.update} also
accepts a Boolean function (its 3$^{rd}$ argument) to select the records to be
updated. Its 2$^{nd}$ argument is a function that maps each selected
record to a new (updated) record. \C{SQL.insert} inserts a given
record into the specified table in the database.

%%%SJ: Reviewers may not understand what primary and foreign keys are,
%%%or why they are important.  There are also some fields (e.g., ol_i_id)
%%%that are not described either in the caption or the text.

% \begin{figure}[!t]
% \includegraphics[scale=0.5]{Figures/schema}
% \caption{Database schema of TPC-C's order management system.
%   Columns against highlighted background are primary keys. Arrows denote
%   foreign key relationships.}
% \label{fig:schema}
% \end{figure}

\begin{figure}[t]
  \centering
	\begin{subfigure}{0.48\textwidth}
		\includegraphics[width=0.9\textwidth]{Figures/schema1}
    \caption{A valid TPC-C database. The only existing order belongs
      to the district with \C{d\_id}=11. It's id (\C{o\_id}) is one
      less than the district's \C{d\_next\_o\_id}, and it's order
      count (\C{o\_ol\_cnt}) is equal to the number of order line
      records whose \C{ol\_o\_id} is equal to the order's id.  }
		\label{fig:tpcc_db1}
	\end{subfigure}
	\begin{subfigure}{0.48\textwidth}
		\includegraphics[width=0.9\textwidth]{Figures/schema2}
    \caption{The database in Fig.~\ref{fig:tpcc_db1} after correctly
      executing a \C{new\_order} transaction. A new order record is
      added whose \C{o\_id} is equal to the \C{d\_next\_o\_id} from
      Fig.~\ref{fig:tpcc_db1}. The district's \C{d\_next\_o\_id} is
      incremented. The order's \C{o\_ol\_cnt} is 2, reflecting the
      actual number of order line records whose \C{ol\_o\_id} is equal
      to the order's id (2).}
		\label{fig:tpcc_db2}
	\end{subfigure}

\caption{Database schema of TPC-C's order management system.
  The naming
  convention indicates primary keys and foreign keys. For e.g.,
  \C{ol\_id} is the primary key column of the order line table,
  whereas \C{ol\_o\_id} is a foreign key that refers to the \C{o\_id}
  column of the order table.}
\label{fig:schema}
\end{figure}

The \C{new\_order} transaction inserts a new \C{Order} record, whose
id is the sequence number of the next order under the given district
(\C{d\_id}). The sequence number is stored in the corresponding
\C{District} record, and updated each time a new order is added to the
system. Since each order may request multiple items (\C{item\_reqs}),
an \C{Order\_line} record is created for each requested item to relate
the order with the item. Each item has a corresponding record in the
\C{Stock} table, which keeps track of the quantity of the item left in
stock (\C{s\_qty}). The quantity is updated by the transaction to
reflect the processing of new orders (if the stock quantity falls below
10, it is automatically replenished by 91).

TPC-C defines multiple invariants, called \emph{consistency
conditions}, over the state of the application in the database. One
such consistency condition is the requirement that for a given order
\C{o}, the \emph{order-line-count} field (\C{o.o\_ol\_cnt}) should
reflect the number of order lines under the order; this is the number
of \C{Order\_line} records whose \C{ol\_o\_id} field is the same as
\C{o.o\_id}.  In a sequential execution, it is easy to see how this
condition is preserved.  A new \C{Order} record is added with its
\C{o\_id} distinct from existing order ids, and its \C{o\_ol\_cnt} is
set to be equal to the size of the \C{item\_reqs} set. The \C{foreach}
loop runs once for each \C{item\_req}, adding a new \C{Order\_line}
record for each requested item, with its \C{ol\_o\_id} field set to
\C{o\_id}. Thus, at the end of the loop, the number of \C{Order\_line}
records in the database, whose \C{ol\_o\_id} is equal to \C{o\_id} is
equal to the size of the \C{item\_req} set, which in turn is equal to
the \C{Order} record's \C{o\_ol\_cnt} field, thus preserving the
consistency condition.

Because the aforementioned reasoning is reasonably simple to perform
manually, verifying the soundess of TPC-C's consistency conditions
would appear to be feasible.  Serializability aids the tractability of
verification by preventing any interference among concurrently
executing transactions while the \C{new\_order} transaction executes.
Under weak isolation\footnote{Weak isolation does not violate
  atomicity as long as the witnessed effects are those of committed
  transactions}, however, interferences of various kinds are
permitted.  Although the verification problem for weakly isolated
transactions would appear to be superficially similar to the
verification of (racy) concurrent programs (e.g., garbage
collectors~\cite{JLP+14,GHE15,HPQ+15}), weak isolation introduces new
challenges arising from the use of transactions, and new opportunities
arising from the fact that the store abstraction used by transactions
is a relational database, not low-level memory.

To illustrate some of these challenges, consider the behavior of the
\C{new\_order} transaction when executed with a \emph{Read Committed}
(RC) isolation level, the default isolation level in 8 of the 18
databases studied in~\ref{bailishotos}.  An RC transaction is
isolated from \emph{dirty writes}, i.e., writes of uncommitted
transactions, but is allowed to witness the writes of concurrent
transactions as soon as they are committed. Thus, with two concurrent
instances of the \C{new\_order} transaction (call them $T_1$ and
$T_2$), both concurrently placing new orders for different customers
under the same district (\C{d\_id}), RC isolation allows the
execution shown in Fig.~\ref{fig:new_order_execs}.

\begin{figure}[!t]
\includegraphics[scale=0.45]{Figures/motiv-eg-1-b}
\caption{\small An RC execution involving two instances ($T_1$ and
  $T_2$) of the \C{new\_order} transaction depicted in
  Fig.~\ref{fig:new_order_code}. 
  Both instances read the \C{d\_id} \C{District} record concurrently,
  because neither transaction is committed when the reads are
  executed.  The subsequent operations are effectively sequentialized,
  since $T_2$ commits. Nonetheless, reading the same value for
  \C{d\_next\_o\_id} prompts both instances to add \C{Order} records
  with same ids, which inturn triggers the the violation of TPC-C's
  consistency condition.}
\label{fig:new_order_exec}
\end{figure}

% \begin{figure}[!h]
% \centering
% \subcaptionbox {
%   {\sc rc} Execution 1
%   \label{fig:motiv-eg-1-b}
% } [
%   0.55\columnwidth
% ] {
%   \includegraphics[scale=0.45]{Figures/motiv-eg-1-b}
% }
% %\hspace*{0.5in}
% \subcaptionbox {
%   {\sc rc} Execution 2
%   \label{fig:motiv-eg-1-a}
% }{
%   \includegraphics[scale=0.45]{Figures/motiv-eg-1-a}
% }
%   \caption{\small RC executions involving two instances ($T_1$ and
%   $T_2$) of the \C{new\_order} transaction depicted in
%   Fig.~\ref{fig:new_order_code}. 
%   Each instance reads the \C{d\_id} \C{District} record twice, the second
%   time to (atomically) update the \C{d\_next\_o\_id} field.}
% \label{fig:new_order_execs}
% \end{figure}

The figure depicts an execution as a series of SQL 
operations. In the execution, the \C{new\_order} instance
$T_1$ (green) reads the \C{d\_next\_o\_id} field of the district
record for \C{d\_id}, but before it increments the field, another
\C{new\_order} instance ($T_2$) begins its execution and commits. Note
that $T_2$ reads the same \C{d\_next\_o\_id} value as $T_1$, and
inserts new \C{Order} and \C{Order\_line} records with their \C{o\_id}
and \C{ol\_o\_id} fields (resp.) equal to \C{d\_next\_o\_id}. $T_2$
also increments the \C{d\_next\_o\_id} field, which $T_1$ has already
acccessed. This is allowed because reads do not obtain a mutually
exclusive lock on most databases. After $T_2$'s commit, $T_1$ resumes
execution and adds new \C{Order} and \C{Order\_line} fields with the
same order id as $T_1$. Thus, by the end of the execution,
\C{Order\_line} records inserted by $T_1$ and $T_2$ all bear the same
order id. There are also two \C{Order} records with the same district
id (\C{d\_id}) and order id, none of whose \C{o\_ol\_cnt} reflects the
actual number of \C{Order\_line} records inserted with that order id.
This clearly violates TPC-C's consistency condition. 

Notably, this example does not exhibit any of the anomalies that
characterize RC isolation~\cite{berenson}. For instance, there are no
\emph{lost writes}; both concurrent transactions' writes are present
in the final state of the database. Thus, program analyses that aim to
determine appropriate isolation by checking for possible
manifestations of these anomalies would fail to identify grounds for
promoting the isolation level of \C{new\_order} to something stronger.
Yet, if we take the semantics of the application into account, it is
quite clear that RC is not an appropriate isolation level for
\C{new\_order}.

While reasoning in terms of anomalies is cumbersome and inadequate,
reasoning about weak isolation in terms of low-level
traces~\cite{adyaphd,gotsmanconcur15} on memory read and write actions
complicates high-level reasoning.  A possible alternative would
interleave weak isolation implementation details within the program,
yielding a (more-or-less) conventional concurrent program that can be
then subject to classical concurrent verification methods.
Considering the size and complexity of real-world transaction systems,
this strategy is unlikely to scale.

In this paper, we adopt a different approach that \emph{lifts}
isolation semantics (\emph{not} their implementations) to the
application layer, providing a principled framework to simultaneously
reason about application invariants and isolation properties.  To
illustrate this idea informally, consider how we might verify that
\C{new\_order} is sound when executed under \emph{Snapshot Isolation}
(SI), and isolation level stronger than RC. Snapshot isolation
allows transactions to be executed against their respective private
snapshots of the database, thus admitting concurrency, but it also
requires that there not be any write-writeconflicts among
concurrent transactions. Write-write conflicts can be eliminated in various
ways; either through detection followed by a rollback, or through
exclusive locks, or a combination of both, leading to different ways
in which SI can be realized. For instance, one possible implementation
of SI, close to the one use by PostgreSQL~\cite{postgres-ssi},
executes a transaction against its private snapshot of the database,
but obtains exclusive locks on the actual records in the database
before performing writes. A write is performed only if the record
hasn't already been updated by a concurrent transaction, i.e,  when it
doesn't result in a write-write conflict.  Otherwise, the transaction is
rolledback. This implementation is demonstrated for a simple database
with three records, $a$, $b$, and $c$ in Fig.~\ref{fig:RR-postgres}.

\begin{figure}[t]
\includegraphics[scale=0.5]{Figures/RR-postgres}
\caption{Database state transitions corresponding to an execution of
  an SI transaction $T_1$ on a PostgreSQL-like store. $T_2$ and $T_3$
  are concurrent transactions. }
\label{fig:rr-postgres}
\end{figure}
\begin{figure}[t]
\includegraphics[scale=0.5]{Figures/RR-abstract}
  \caption{An abstract execution that includes the concrete execution
  shown in Fig.~\ref{fig:rr-postgres}. It has no locks or snapshots,
  admits fewer interleavings, yet results in the same post-state.  }
\label{fig:rr-abstract}
\end{figure}

Implementations of SI on real databases (e.g., PostgreSQL) are
complicated, often running into thousands of lines of code.
Nonetheless, the semantics of Snapshot Isolation, in terms of how it
effects the transitions on the database state, can be captured in a
fairly simple model. Firstly, although SI admits concurrent
transactions, their effects are  not witnessed in the current
transaction (call it $T$), during $T$'s execution. Thus, insofar as
$T$ is concerned, the global state doesn't change as long as it
executes. More formally, for every consecutive pair of global states
$(\stg,\stg')$ witnessed by $T$ during its execution, SI requires
$\stg'=\stg$.  After $T$ finishes execution, it commits to the actual
database, which incorporates the effects of concurrent transactions.
In executions where $T$ successfully commits, concurrent transactions
are guaranteed to not be in write-write conflict with $T$. Thus, if
$\stg$ is the global state that $T$ witnessed when it finished
execution (which would be same as the snapshot state), and $\stg'$ is
the state to which $T$ commits, then the diff between $\stg$ and
$\stg'$ shouldn't be in a write-write conflict with $T$. To concretize
this notion, let the database state be a map from transaction
variables to values, and let $\stl$ denote a transaction-local log
that maps the variables being written to their updated values. Then
the absence of write-write conflicts between $T$ and the diff between
$\stg$ and $\stg'$ can be defined as: $\forall
x\in\mathit{dom}(\delta)$, $\stg'(x) = \stg(x)$.  To summarize, the
semantics of SI can be captured as an axiomatization over transitions
of the database state ($\Delta \longrightarrow \Delta'$) during the
lifetime an SI transaction ($T$):
\begin{itemize}
  \item While $T$ executes, $\Delta' = \Delta$.
  \item After $T$ finishes execution, but before it commits its local
    state $\delta$, $\forall(x\in\delta).~\Delta'(x) = \Delta(x)$.
\end{itemize}

% nor $T$'s effects are
% witnessed by concurrent transactions, during $T$'s execution. Thus,
% $T$'s execution can be thought of as a series of transitions of a
% state local to $T$ (call it $\stl$), that has no bearing outside $T$. Conversely, any
% transitions performed by the concurrent transactions on the global
% state (call it $\stg$) has no
% bearing on $T$ since $T$ reads off a static snapshot established at
% the beginning of the transaction. Consequently, any
% state transitions performed by the concurrent transactions can be
% \emph{moved past} $T$'s local state transitions. $T$'s commit,
% however, is a global state transition that is enabled only if
% there are no WW conflicts between $T$ and concurrent transactions.
% Thus, concurrent transactions' state transitions cannot be moved past
% $T$'s commit, and $T$ can be thought of witnessing the effects of
% concurrent transactions just before its commit. 
% therefore treat Since a
% transaction's reads are always served from a snapshot, no state
% changes are witnessed while the execution is in progress. Thus,
% insofar as an RR transaction is concerned, the database state does not
% change during the execution.  Uncommitted writes are recorded in a
% transaction-local state.  When the transaction commits, the local
% state is atomically written to the global state to yield a global
% state that reflects the transaction's updates.  However, unlike a
% strongly isolated serializable transaction, the commit operation is
% performed against the current state of the database, not the
% snapshot. Thus, after the transaction finishes execution, but before
% it commits, the transaction is able to witness the effects of all
% concurrent transactions that committed before it.  The PostgreSQL RR
% implementation effectively constrains this transition.

% We can axiomatize this operational description by observing that (a)
% due to the version check, the current transaction cannot update a data
% item that was already updated by a concurrent transaction, and (b) due
% to the use of exclusive write locks, a data item updated by the
% current transaction cannot be overwritten by a concurrent transaction.
% If $\Delta$ is the state of the database when an RR transaction
% finishes, and $\Delta_c$ is the state visible to the transaction at
% the point of commit, we know the transition from $\Delta$ to
% $\Delta_c$ (written $\Delta \longrightarrow \Delta_c$) cannot exhibit
% effects from any concurrent transactions that write to the same data
% items as the current RR transaction.  Similarly, if $\delta$ denotes
% a local log that relates transaction variables being written with their updated values,
% then $\forall x\in\mathit{dom}(\delta)$, $\Delta_c(x) =
% \Delta(x)$. To summarize, the operational semantics of PostgreSQL's RR
% implementation can be captured as an axiomatization over transitions of
% the database state ($\Delta \longrightarrow \Delta'$) during the
% lifetime an RR transaction ($T$):
% \begin{itemize}
%   \item While $T$ executes, $\Delta' = \Delta$.
%   \item After $T$ finishes execution, but before it commits its local
%     state $\delta$, $\forall(x\in\delta).~\Delta'(x) = \Delta(x)$.
% \end{itemize}

This simple characterization of SI isolation allows us to verify the
consistency conditions associated with the \C{new\_order} transaction.
First, since the database doesn't change ($\Delta' = \Delta$) during
the execution of the transaction's body, we can reason about
\C{new\_order} as though it executed in complete isolation until its
commit point, leading to a verification process similar to what would
been applied when reasoning about serializability.  When
\C{new\_order} finishes execution, but before it commits, the SI
axiomatization shown above requires us to consider global state
transitions $\stg \longrightarrow \stg'$ that do not include changes
to the records ($\stl$) written by \C{new\_order}, i.e.,
$\forall(x\in\delta).~\Delta'(x) = \Delta(x)$. Since concurrent
\C{new\_order} transactions that write to the same \C{District} record
modify the record by incrementing its \C{d\_next\_o\_id} field, their
transitions do not satisfy the necessary condition imposed by the SI
axiomatization, allowing us to rightfully ignore their interference.
This leaves the interference due to concurrent \C{new\_order}
transactions that write to a different district record.  Fortunately,
such interference does not prompt \C{new\_order} to violate the
invariant, which can be established by adopting a reasoning similar to
the one used on concurrent program logics, such as the Rely-Guarantee
logic, customized for the database programs. Our approach generalizes
this style of reasoning to a range of isolation levels, concurrency
control techniques, and their combinations used frequently in practice.

% and becomes ready to commit, a
% transition transfers the transaction's local writes ($\delta$) to the
% unchanged database state ($\Delta$).  However, we are required to
% consider the interference from concurrent transitions at this point,
% which might change the database state from $\Delta$ to $\Delta_c$. If
% this interference includes the effects of a concurrent \C{new\_order}
% transaction (with the same \C{d\_id}), then verification fails as
% described previously.  Fortunately, sequential reasoning shows that
% this is impossible - RR prevents a concurrent \C{new\_order}
% transaction that modifies the same \C{District} record as the current
% transaction (concretely, since the record is already present in the
% current transaction's local log, any transition from $\Delta$ to
% $\Delta_c$ cannot change this record).  Applying such axiomatic
% reasoning on \C{new\_order} allows us to prove that the TPC-C
% invariant holds when the transaction is executed under PostgreSQL's RR
% isolation.  Our proof framework generalizes this style of reasoning to
% various isolation levels on databases.

The second observation that informs our approach is one that pertains
to automation. Program verification, even when machine-aided, often
entails significant annotation burden in the form of intermediary
assertions and loop invariants required to prove a program correct.
This is certainly true for concurrent program logics, such as
Rely-Gurantee, which extend Hoare logic with additional artifacts and
where (stable) intermediary assertions and loop invariants remain a
major source of annotation burden.  However, a relational database is
a significantly simpler abstraction than shared memory. There are no
pointers, linked data structures, or aliasing.  Although a database
essentially abstracts a mutable state, the state is mutated through a
well-defined fixed number of interfaces (SQL statements), each tagged
with a logical formula describing what records are accessed and
updated.

This observation leads us away from thinking of database transactions
as concurrent imperative programs.  Instead, we see value in viewing
them as essentially functional computations that manage database state
monadically. We find it useful to reason about
statements that mutate the database state, not in terms of a pre- and
post-condition pair, but in terms of a state transformer that relates
the pre- and post-states of a statement. This state transformer
semantics can be defined algorithmically, just like predicate
transformer semantics (e.g., strongest post-condition).  Here, a state
transformer interprets a SQL statement in the set domain, taking
advantage of the fact that the database is essentially a set of
tuples, and a SQL statement is a transformer over these sets.  The
benefit of this approach is that low-level loops can now be
substituted with higher-order combinators that automatically lift the
state transformer of its higher-order argument (i.e., the loop body)
to the state transformer of the combined expression (i.e., the loop).
Thus the semantics of a \C{foreach} loop, for instance, can be
captured as a state transformer, where the state is a set, and the
transformation defines a bind operation. We illustrate this intuition
on a simple example.

% \begin{figure}[!t]
% \centering
% %
% \begin{subfigure}[b]{0.46\textwidth}
% \begin{ocaml}
% Set s' = $\emptyset$;
% foreach x in s {
%   s'.add(f(x)); 
% }
% \end{ocaml}
% \caption{}
% \end{subfigure}
% %
% \begin{subfigure}[b]{0.5\textwidth}
% \begin{ocaml}
% let s' = ref [];
% foreach s (fun x -> s' := x::(!s'));
% \end{ocaml}
% \caption{Lorem ipsum, lorem ipsum,Lorem ipsum, lorem ipsum,Lorem ipsum}
% \end{subfigure}
% \caption{Caption place holder}
% %
% \caption{New versions are created from existing versions either
% through \C{push} or \C{merge}.}
% \label{fig:syntactic-ancestors}
% \end{figure}
% let s' = ref (Set.empty) in
% foreach xs @@ fun x -> 
%   begin
%     s' := Set.union !s' @@ 
%             Set.map_selected s (fun y -> y<x)
%                                (fun y -> y+x);
%     s' := Set.add !s' x;
%   end
\begin{figure}[!h]
\begin{ocaml}
foreach item_reqs @@ fun item_req -> do
  SQL.update Stock (fun s -> {s with s_qty = k1}) 
                   (fun s -> s.s_i_id = item_req.ol_i_id);
  SQL.insert Order_line {ol_o_id=k2; ol_d_id=k3; 
                         ol_i_id=item_req.ol_i_id; ol_qty=item_req.ol_qty}
\end{ocaml}
\caption{Foreach loop from Fig.~\ref{fig:new_order_code}}
\label{fig:foreach_code}
\end{figure}

Fig.~\ref{fig:foreach_code} shows a (simplified) snippet of code taken
from Fig.~\ref{fig:new_order_code}. Some irrelevant expressions have
been replaced with constants (\C{k1} to \C{k3}).  The body of the loop
executes a SQL update followed by an insert.  Recall that a transaction
reads from the global database ($\Delta$), and writes to a
transaction-local database ($\delta$) before committing these updates. An update
statement filters the tuples that match the search criteria from $\Delta$
and computes the updated tuples that are to be
added to the local database. Thus, its state transformer (call it
$\T_U$) is the following function on sets\footnote{$\bind$ has higher
precedence than $\cup$.}:
\begin{smathpar}
\begin{array}{l}
  \lambda(\stl,\stg).~ \stl \cup \stg \bind(\lambda s. 
      \itel{{\sf table}(s) = \C{Stock} \conj 
                s.\C{s\_i\_id} = \C{item\_req.ol\_i\_id}\\\hspace*{1.3in}} 
           {\{\{s \with \C{s\_qty} = \C{k1}\}\}\\\hspace*{1.3in}}
           {\emptyset}
\end{array}
\end{smathpar}
% \begin{smathpar}
% \begin{array}{c}
%   \lambda\Delta.\lambda\delta.\, \delta \;\cup\; 
%       \Delta \,\bind\, (\lambda s.\, 
%           \C{if}\;{\C{stock}(s) \conj 
%                    \C{s.s\_i\_id} = \C{item\_req.ol\_i\_id}}\\
%            \hspace*{0.9in}
%            \C{then}\;{\{\{s \;\C{with}\; \C{s\_qty} = \C{k1}\}\}}\;
%            \C{else}\;{\emptyset})
% \end{array}
% \end{smathpar}
% \begin{verbatim}
%   Rmem(σ'(s')) = Rmem(σ(s')) U 
%                  Rmem[\x. Rmem[\y. if y<x then {y+x} else {}](σ(s)) U {x}](xs)
% \end{verbatim}
% Observe that
% $\T_U(\Delta,\delta)$ is of the form $\delta \,\cup\,
% F_U(\Delta)$. 
The transformer ($\T_I(\Delta,\delta)$) for the subsequent \C{insert}
statement can be similarly calculated as the following:
\begin{smathpar}
\begin{array}{l}
  \lambda(\stl,\stg).~ \stl \cup \{\C{ol\_o\_id}=\C{k2};\, 
        \C{ol\_d\_id}=\C{k3};\, \C{ol\_i\_id}=\C{item\_req.ol\_i\_id};\, 
        \C{ol\_qty}=\C{item\_req.ol\_qty}\}
\end{array}
\end{smathpar}
Observe that both transformers are of the form $T(\Delta,\delta) =
\stl \cup F(\stg)$, where $F$ is a function that returns the set of
records added by the transformer to the transaction-local database
($\stl$). Let $\F_U$ and $\F_I$ be the corresponding functions for $\T_U$
and $\T_I$ shown above. The state transformed of the loop body in
Fig.~\ref{fig:new} can be expressed as the following composition of
$\F_U$ and $\F_I$:
\begin{smathpar}
\begin{array}{l}
  \lambda(\stl,\stg).~ \stl \cup \F_U(\stg) \cup \F_I(\stg)
\end{array}
\end{smathpar}
The transformer for the loop itself can now be computed as following:
\begin{smathpar}
\begin{array}{l}
  \lambda(\stl,\stg).~ \stl \cup \C{item\_reqs}\bind
      (\lambda\C{item\_req}.~ \F_U(\stg) \cup \F_I(\stg))
\end{array}
\end{smathpar}
Observe that the structure of the transformer mirrors the structure of
the program itself. In particular, SQL statements become set
operations, and the \C{DB} monad's $\bind$ becoms set monadic $\bind$.
Thus, state transformer inference is completely mechanical. The
advantage of inferring such transformers is that we can now make use
of a semantics-preserving translation from the domain of sets equipped
with $\bind$ to the first-order logic, allowing us to leverage SMT
solvers for automatic proofs without having to infer loop invariants,
or provide intermediate assertions.  Sec.~\ref{sec:inference}
describes this translation. In the exposition thus far, we assumed
$\Delta$ remains invaraint, which is clearly not the case when we
admit concurrency.  Necessary concurrency extensions of the state
transformer semantics is also covered in Sec.~\ref{sec:inference}.
The following two sections, however, first focus on laying the
theoretical foundations for reasoning about weak isolation.


\section{\txnimp: Syntax and Semantics}
\label{sec:opsem}

\label{sec:syntax}

%\renewcommand{\ctxn}[3]{\C{TXN}_{#1}\langle #2 \rangle\{#3\}}
\begin{figure*}[!ht]
\raggedright
%
\textbf{Syntax}\\
%
\begin{smathpar}
\renewcommand{\arraystretch}{1.2}
\begin{array}{lclcl}
\multicolumn{5}{c} {
  {x,y} \in \mathtt{Variables}\qquad
  {f} \in \mathtt{Field\;Names} \qquad
% {\tau} \in \mathtt{Table\;Names}
  {i,j} \in \mathbb{N} \qquad
  {\odot} \in \{+,-,\le,\ge,=\}\qquad
  {k} \in \mathbb{Z}\cup\mathbb{B} \qquad
  {\rec} \in \{\bar{f}=\bar{k}\}\
}\\
{\stl,\stg,s} & \in & \mathtt{State} & \coloneqq &  \Pow{\{\bar{f}=\bar{k}\}} \\
{\I_e, \I_c }  & \in & \mathtt{Isolation Spec} & \coloneqq & (\stl,\stg,\stg') \rightarrow \Prop\\
v & \in & \mathtt{Values} & \coloneqq & k \ALT \rec \ALT s\\
e & \in & \mathtt{Expressions} & \coloneqq & k \ALT x \ALT x.f 
    \ALT \{\bar{f}=\bar{e}\} \ALT e_1 \odot e_2\\ 
c & \in & \mathtt{Commands} & \coloneqq & \cskip \ALT \lete{x}{e}{c}
    \ALT \ite{e}{c_1}{c_2}\ALT c_1;c_2 \ALT \inserte{x}  \\
&&&&\ALT \deletee{\lambda x.e}
    \ALT \lete{y}{\selecte{\lambda x.e}}{c}
    \ALT \updatee{\lambda x.e_1}{\lambda x.e_2}\\
&&&&\ALT \foreache{x}{\lambda y.\lambda z. c} 
    \ALT \foreachr{s_1}{s_2}{\lambda x.\lambda y. e}\\
&&&&\ALT \ctxn{i}{\I}{ c } \ALT \ctxnr{i}{\I,\stl,\stg}{c} \ALT c1 || c2\\
% t & \in & \mathtt{Terms} & \coloneqq & e \ALT c\\
\ectx & \in & \mathtt{Eval\;Ctx} & ::= & \bullet \ALT  
  \bullet || c_2 \ALT c_1 || \bullet \ALT \bullet;\,c_2 
  \ALT \ctxnr{i}{\I,\stl,\stg}{\bullet} \\
\end{array}
\end{smathpar}
%
\bigskip

\renewcommand{\arraystretch}{1.2}

%
\textbf{Local Reduction} \quad 
\fbox {\(\stg \vdash (\tbox{c}_i,\stl) \stepsto (\tbox{c'}_i,\stl')\)}\\
%
\begin{minipage}{2.8in}
\rulelabel{E-Insert}
\begin{smathpar}
\begin{array}{c}
\RULE
{
  r.\idf \not\in \dom(\stl \cup \stg)\\
  r' = \{r \;\C{with}\; \txnf=i;\,\delf=\C{false}\}
}
{
  \stg \vdash (\tbox{\inserte{r}}_i,\stl) \stepsto
  (\tbox{\cskip}_i,\stl \cup \{r'\})
}
\end{array}
\end{smathpar}
\end{minipage}
%
%
\begin{minipage}{2.8in}
\rulelabel{E-Delete}
\begin{smathpar}
\begin{array}{c}
\RULE
{
  \hspace*{-1in}
  s = \{r' \,|\, \exists(r\in\Delta).~ \eval([r/x]e)=\C{true} \\
      \hspace*{0.1in}\conj r'=\{r \with \delf=\C{true};\, \txnf=i\}\}\spc
  \dom(\stl)\cap\dom(s) = \emptyset \spc
% \dom(s) \cap \dom(\delta) = \emptyset
}
{
  \stg \vdash (\tbox{\deletee{\lambda x.e}}_i,\stl) \stepsto 
  (\tbox{\cskip}_i,\stl \cup s)
}
\end{array}
\end{smathpar}
\end{minipage}
%
\bigskip

%
\begin{minipage}{2.8in}
\rulelabel{E-Select}
\begin{smathpar}
\begin{array}{c}
\RULE
{
  \\
  s = \{r\in\Delta \,|\, \eval([r/x]e)=\C{true}\}\spc
  c' = [s/y]c
}
{
  \stg \vdash (\tbox{\lete{y}{\selecte{\lambda x.e}}{c}}_i, \stl) \stepsto 
              (\tbox{c'}_i,\stl)
}
\end{array}
\end{smathpar}
\end{minipage}
%
%
\begin{minipage}{2.8in}
\rulelabel{E-Update}
\begin{smathpar}
\begin{array}{c}
\RULE
{
  \hspace*{-0.2in}s = \{r' \,|\, \exists(r\in\Delta).~ 
    \eval([r/x]e_2)=\C{true} \conj r'= \{ [r/x]e_1 \;\C{with}\\
    \idf=r.\idf;\,\txnf=i;\,\delf = r.\delf \}\} \spc
  \dom(\stl) \cap \dom(s) = \emptyset
}
{
  \stg \vdash (\tbox{\updatee{\lambda x.e_1}{\lambda x.e_2}}_i,\stl) 
      \stepsto (\tbox{\cskip}_i,\stl \cup s)
}
\end{array}
\end{smathpar}
\end{minipage}
%

\begin{smathpar}
\begin{array}{ll}
  \rulelabel{E-Foreach1} & \stg \vdash (\tbox{\foreache{s}{\lambda y.\lambda
    z.c}}_i,\stl) \stepsto (\tbox{\foreachr{\emptyset}{s}{\lambda
    y.\lambda z. c}}_i,\stl)\\
  \rulelabel{E-Foreach2} & \stg \vdash (\tbox{\foreachr{s_1}{\{r\} \uplus s_2}
    {\lambda y.\lambda z.c}}_i,\stl) \stepsto (\tbox{[r/z][s_1/y]c;\,
    \foreachr{s_1 \cup \{r\}}{s_2}{\lambda y.\lambda z. c}}_i,\stl)\\
  \rulelabel{E-Foreach3} & \stg \vdash (\tbox{\foreachr{s}{\emptyset}
    {\lambda y.\lambda z.c}}_i,\stl) \stepsto (\tbox{\cskip}_i,\stl)\\
\end{array}
\end{smathpar}
%
\bigskip

%
\textbf{Top-Level Reduction} \quad 
\fbox {\((c,\stg) \stepsto (c',\stg')\)}\\
%
\begin{minipage}{3in}
  \rulelabel{E-Txn-Start}
  \begin{smathpar}
  \begin{array}{c}
    \RULE{}
         {(\ctxn{i}{\I}{c},\stg) \stepsto (\ctxnr{i}{\I,\emptyset,\stg}{c},\stg)}
  \end{array}
  \end{smathpar}
\end{minipage}%
\hfill
\begin{minipage}{3in}
\rulelabel{E-Txn}
\begin{smathpar}
\begin{array}{c}
\RULE
{
  \I_e\,\,(\stl,\stg,\stg')\spc
  \stg \vdash (\tbox{c}_i,\stl) \stepsto (\tbox{c'}_i,\stl')
}
{
  (\ctxnr{i}{\I,\stl,\stg}{c},\stg') \stepsto
  (\ctxnr{i}{\I,\stl',\stg'}{c'},\stg')
}
\end{array}
\end{smathpar}
\end{minipage}\\

\begin{center}
\begin{minipage}{3in}
\rulelabel{E-Commit}
\begin{smathpar}
\begin{array}{c}
\RULE
{
  \I_c\,\,(\stl,\stg,\stg')
}
{
  (\ctxnr{i}{\I,\stl,\stg}{\cskip},\stg') \stepsto (\cskip,\stl\gg\stg')
}
\end{array}
\end{smathpar}
\end{minipage}
\end{center}
\hfill
%
\caption{\small \txnimp: Syntax and Small-step semantics}
\label{fig:txnimp}
\end{figure*}



Fig.~\ref{fig:txnimp} shows the syntax and small-step semantics of
\txnimp, a core language that we use to formalize our intuitions about
reasoning under weak isolation. Variables ($x$), Integer and boolean
constants ($k$), records ($r$) of named constants, sets ($s$) of such
records, arithmetic and boolean expressions ($e_1 \odot e_2$), and
record expressions ($\{\bar{f}=\bar{e}\}$) constitute the syntactic
class of expressions ($e$). Commands ($c$) include $\cskip$,
conditional statements, \C{LET} constructs to bind names, \C{FOREACH}
loops, SQL statements, their sequential composition ($c_1;c_2$), and
transactions ($\ctxn{i}{\I}{c}$) and their parallel composition
($c_1\,||\,c_2$). The $\I$ in \C{TXN} block syntax is the
transaction's isolation specification, which will be explained later.
Certain terms that only appear at run-time are also present in $c$.
These include a \C{TXN} block tagged with sets ($\stl$ and $\stg$) of
records that assume special meaning in operational semantics, and a
\C{FOREACH} loop that keeps track of the set ($s_1$) of items already
iterated, and the set ($s_2$) of items yet to be iterated. Note that
even the surface-level \C{FOREACH} shown here is a little different
from the one used in previous sections; its higher-order argument has
two arguments, $y$ and $z$, which are invoked (during the reduction)
with the set of already-iterated items, and the current item,
respectively. This form of \C{FOREACH} lends itself to inductive
reasoning, facilitating inductive proofs (Sec.~\ref{sec:reasoning}).

% also We let $T_i$ for $i \in \mathbb{N}$
% range over transaction identifiers. When it is evident we are
% referring to a transaction, we use the number $i$ instead of $T_i$ for
% identification (\eg in $\C{txn}\langle i \rangle$). Like variables,
% transaction identifiers are globally accessible. For notational
% convenience, we let $t$ range over both expressions and commands.

We define a small-step operational semantics for this language in
terms of an abstract machine that executes a command, and updates
either a transaction-local database ($\stl$), or the global database
($\stg$). Database ($\stg$) is a modeled as a set of records of a
pre-defined type, i.e., they all belong to a single table.
Generalization to multiple tables is straightforward. Records in
$\stg$ are uniquely identifiable through their $\idf$ field, which is
auto-generated and does not belong to the surface language, i.e.,
$\C{id}\notin f$. For a set $S$ of records, we define $\dom(S)$ as the
set of unique ids of all the records in $S$. Thus $|\dom(\stg)| =
|\stg|$. During its execution, a transaction may write to multiple
records in $\stg$. Atomicity dictates that such writes shouldn't be
available in $\stg$ until the transaction commits. We therefore
associate each transaction with a local database ($\stl$) that stores
\emph{only} the uncommitted records\footnote{While SQL's \C{UPDATE}
admits writes at the granularity of record fields, databases, in
reality, enforce record-level locking, allowing us to think of
``uncommitted writes'' as ``uncommitted records''. }. Uncommitted
records include deleted records, which are identified by a hidden
$\delf$ field set to \C{true}. When the transaction commits, its local
database is atomically \emph{flushed} to the global database,
committing the uncommitted records. The flush operation ($\gg$) is defined as
following:
\begin{smathpar}
\begin{array}{c}
\forall r.~ r \in (\stl\gg\stg) ~\Leftrightarrow~ 
  (r.\idf \notin \dom(\stl) \conj r \in \stg)
\disj (r \in \stl \conj \neg r.\delf) 
\end{array}
\end{smathpar}
Let $\stg' = \stl\gg\stg$. A record $r$ belongs to $\stg'$ iff it
belongs to $\stg$ and not been updated in $\stl$, i.e., $r.\idf \notin
\dom(\stl)$, or it belongs to $\stl$, i.e., it is either a new record,
or an updated version of an old record, but the updatation is not a
deletion ($\neg r.\delf$). Thus, flush defines the result of
atomically applying transaction's local writes to the global database.
Besides the commit, flush also helps a transaction read its own
writes. Intuitively, the result of a read operation inside a
transaction must be computed on the database resulting from flushing
the current local state ($\stl$) to the global state ($\stg$). The
abstract machine of Fig.~\ref{fig:txnimp}, however, doesn't let a
transaction read its own writes. This reduces verbosity and simplifies
semantics, without losing any generality; substituting $\stl\gg\stg$
for $\stg$ at select places in reduction rules recovers the real

Small-step semantics is stratified into a transaction-local reduction
relation, and a top-level reduction relation. Transaction-local
relation ($\stg \vdash (c,\stl) \stepsto (c',\stl')$) defines a
small-step reduction for a command inside a transaction, when the
database state is $\stg$; the command $c$ reduces to $c'$, while
updating the transaction-local database $\stl$ to $\stl'$. The
definition assumes a meta function $\eval$ that evaluates expressions
with no free variables to values. The reduction relation for SQL
statements is defined straightforwardly.  \C{INSERT} adds a new record
to $\stl$ after adding a unique identifier (more discussion on
uniqueness later). \C{DELETE} finds the records in $\stg$ that match
the search criteria defined by its boolean function argument, and adds
the records to $\stl$ after marking them for deletion. \C{SELECT}
bounds the name introduced by \C{LET} to the set of records from
$\stg$ that match the search criteria, and then executes the bound
command $c$. \C{UPDATE} uses its first function argument to compute
the updated version of the records that match the search criteria
defined by its second function argument. Updated records are added to
$\stl$. 

The reduction of \C{FOREACH} starts by first converting it to its
run-time form that also keeps track of the iterated items ($s_1$),
besides the yet-to-be-iterated items ($s_2$). Initially, $s_1$ is
empty. As the elements are iterated, they are removed from $s_2$ and
added to $s_1$ . Iteration involves invoking the higher-order function
with $s_1$ and the current element $x$ (note: $\uplus$ in $\{x\}
\uplus s_2$ denotes a disjoint union). The reduction ends when the
$s_2$ becomes empty. The reduction rules for conditionals, \C{LET}
binders, and sequences are straightforward, hence ommitted.

The top-level reduction relation defines small-step semantics of
transactions, and their parallel composition. A transaction always
comes tagged with an isolation specification $\I$ that dictates the
timing and nature of iterferences that the transaction can witness.
Formally, $\I$ is a function that maps a boolean ($\mathbb{B}$) and a
triplet of database states (three sets of records:
$\Pow{r}\times\Pow{r}\times\Pow{r}$) to a logical proposition
($\Prop$). The abstract machine calls $\I$ while reducing
$\ctxn{i}{\I}{c}$. The boolean argument identifies if the transaction
is still executing (\C{true}), or if it has finished execution and
ready to commit (\C{false}).  Among the triplet of sets (the second
argument), first is the (current) transaction-local database state
($\stl$), second is the state ($\stg$) of the global database when the
transaction last took a step, and third is the current state ($\stg'$)
of the global database.  Intuitively, $\stg'\neq\stg$ indicates an
interference from a concurrent transaction, and the $\I$ decides if
this interference is allowed or not, considering the local database
state ($\stl$). For instance, as described in
(Sec.~\ref{sec:motivation}), an RR transaction on Postgres defines
$\I$ as following:
\begin{smathpar}
\begin{array}{lcl}
\I\,\C{true}\,(\stl,\stg,\stg') & = & \stg' = \stg\\
\I\,\C{false}\,(\stl,\stg,\stg') & = & \forall(r\in\stl)(r'\in\stg).~
  r'.\idf = r.\idf \Rightarrow r'\in\stg'.
\end{array}
\end{smathpar}

The reduction of a $\ctxn{i}{\I}{c}$ begins by first converting it to
its run-time form $\ctxn{i}{\I,\stl,\stg}{c}$, where $\stl =
\emptyset$, and $\stg$ the current (global) database. The top-level
reduction rules in Fig.~\ref{fig:txnimp} assume this form. Rule
\rulelabel{E-Txn} reduces $\ctxn{i}{\I,\stl,\stg}{c}$ under a database
state ($\stg'$), only if the isolation specification ($\I$) allows the
interference between $\stg$ and $\stg'$ during the execution. Rule
\rulelabel{E-Commit} commits the transaction
$\ctxn{i}{\I,\stl,\stg}{c}$ by flushing its uncommitted records to the
database. This is done only if the interference between $\stg$ and
$\stg'$ is allowed during the commit by the isolation specification.

\subsection{Isolation Specifications}
\label{sec:isolation}

Operational semantics of Fig.~\ref{fig:txnimp} does not employ
concurrency control constructs in any form. Any additional concurrency
control provided by the database implementation, beyond that which is
already incuded in the isolation specification, also needs to be
axiomatized in $\I$. We first describe such axiomatizations, and then
present the specifications of the standard isolation levels as
implemented by popular off-the-shelf databases.

\textbf{Unique Ids}. Most relational databases associate every record
in a table with a unique identifier. Often, auto-generated unique
identifiers are part of the schema itself (via SQL's \C{UNIQUE} and
\C{AUTO\_INCREMENT} keywords), but if not, database adds one (e.g.,
MySQL's InnoDB engine automatically adds a 6-byte identifier if none
exists). Enforcing global uniqueness of identifiers requires exclusive
locks, which are absent from Fig.~\ref{fig:txnimp}. The
\rulelabel{E-Insert} rule assigns an id to the newly added record
after checking that it is not already present in $\stg$ and $\stl$.
But the new record is only added to $\stl$, which is invisible outside
the transaction. Thus, the uniqueness check for the same id also
passes in a concurrent transaction, resulting in duplicate ids
after both commits. To prevent this, we need the following proposition
in $\I$ of every transaction, regardless of whether it is executing
or committing:
\begin{smathpar}
\begin{array}{lcl}
  \I_{id}(\stl,\stg,\stg') & = & \forall(r\in\stl).~
      r.\idf\notin \dom(\stg) \Rightarrow r.\idf\notin \dom(\stg').
\end{array}
\end{smathpar}
$\I_{id}$ makes sure that a globally unique id generated during a
transaction's execution remains globally unique until it commits. An
interference from a concurrent transaction that adds the same id to
$\stg$ is prohibited.

\textbf{Write Locks}. Transactions in MySQL and Postgres (effectively)
obtain exclusive locks on records they update, and do not release them
until they commit, thus preventing write-write (ww) conflicts. To
obtain this behavior on the abstract machine of Fig.~\ref{fig:txnimp},
the following proposition needs to be in $\I$ for every transaction:
\begin{smathpar}
\begin{array}{lcl}
  \I_{ww}(\stl,\stg,\stg') & = & \forall(r'\in\stl)(r \in \stg).~
      r.\idf = r'.\idf  \Rightarrow r\in\stg'.
\end{array}
\end{smathpar}
That is, given a record $r'\in\stl$, if there exists an $r\in\stg$
with the same id (i.e., $r'$ is an updated version of $r$), then $r$
must be present unmodified in $\stg'$. This prevents a concurrent
transaction from changing $r$, thus simulating the behavior of an
exclusive lock. There is however a caveat: if we assume extensional
equality over records, a write by a concurrent transaction that
doesn't change $r$'s contents is still allowed. While this in itself
is not a problem, the concurrent transaction may modify other records,
which then become visible in the current transaction. A write lock
prevents this behavior, whereas our axiomatization ($\I_{ww}$) allows
it. An easy fix for this is to add version timestamps to records that
effectively intensionalizes equality. Nonetheless, imprecise
axiomatization of write locks hasn't been a problem in practice.

\textbf{Read-Only Transactions}. Certain databases implement special
previleges for read-only transactions. Read-only behavior can be
enforced on a transaction by including the following proposition in
its $\I$:
\begin{smathpar}
\begin{array}{lcl}
  \I_{ro}(\stl,\stg,\stg') & = & \stl = \emptyset\\
\end{array}
\end{smathpar}
If a transaction declared as read-only performs a write, then its
$\stl\neq \emptyset$, and the transaction never commits.






\section{The Reasoning Framework}
\label{sec:reasoning}

We now describe a proof system that lets us prove the correctness of a \txnimp
program $c$ w.r.t its high-level invariant $I$, on a machine that
satisfies the isolation specifications ($\I$) of its
transactions\footnote{Note the difference between $I$ and $\I$. The
former constitute \emph{proof} \emph{obligations} for the programmer,
whereas the latter describes a transaction's \emph{assumptions} about
operational characteristics of the underlying machine.}.  Our proof
system is essentially an adapatation of the rely-guarantee reasoning
framework to the setting of weakly isolated database transactions. It
deals with the additional complexity introduced by this setting, while
also taking advantage of the opportunities it provides to simplify the
reasoning. The primary challenge is the weak isolation; how do we
relate a transaction's isolation specification ($\I$) to its rely
relation ($R$) that describes the environment, so that the
interference is considered only insofar as the isolation level allows
it? Our proof system shows how to compute a rely relation ($R$) modulo
the isolation specification ($\I$) of a transaction. Another
characteristic of the transaction setting is the atomicity of a
trasaction's aggregate writes; a transaction's writes are not visible
to its peers until it commits. In the context of rely-guarantee, this
means that the transactions guarantee ($G$) should capture the
aggregate effect of a transaction, and not its individual writes.
While \C{atomic} blocks are also present in the shared-memory
programs, the fact that a transactions are weakly isolated introduces
complexity.  Unlike an \C{atomic} block, the effect of a transaction
is \emph{not} a sequential composition of the effects of its
statements because each statement witnesses a potentially different
version of the state. This is illustrated in
Fig.~\ref{fig:atomic-vs-transaction}, where a series of operations on
\C{x} can be composed to describe the effect of an \C{atomic} block,
while same cannot be done in a transaction. Thus, weak isolation needs
to be taken into account when defining and verifying guarantees, and
our proof system shows how. 

\subsection{The Rely-Guarantee Judgment}
\label{sec:rely-guarantee}

\begin{figure}[t]
\raggedright
%
\fbox {\( \R \vdash \hoare{P}{c}{Q} \)} 
\quad \fbox {\( \rg{I,R}{c}{G,I} \)}\\[4pt]
%\fbox{\( \rg{\mathbb{I},P,R}{\txnbox{c}_i}{G,Q}\)} \quad
%\fbox{\( \rg{\mathbb{I},P,R}{c}{G,Q}\)} \quad\\
%
\begin{minipage}{3.4in}
\rulelabel{RG-Update}
\begin{smathpar}
\begin{array}{c}
\RULE
{
  \stable(\R,P)\\
  \hspace*{-0.4in} \forall\stl,\stl',\stg.~P(\stl,\stg) \conj 
  \stl' = \stl \cup \{r' \,|\, \exists(r\in\Delta).[r/x]e_2 \conj\\
   r'=\langle[r/x]e_1 \with \idf=r.\idf;\,\txnf=i;\,\delf=\C{false}\rangle\} \Rightarrow   Q(\stl',\stg)
}
{
  \R \vdash \hoare{P}{\updatee{\lambda x.e_1}{\lambda x.e_2}}{Q}
}
\end{array}
\end{smathpar}
\end{minipage}
%
%
\begin{minipage}{2.6in}
\rulelabel{RG-Select}
\begin{smathpar}
\begin{array}{c}
\RULE
{
  \\
  \R \vdash \hoare{P'}{c}{Q}\spc
  \stable(\R,P)\\
  \hspace*{-0.3in} P'(\stl,\stg) \Leftrightarrow P(\stl,\stg) \\
  \hspace*{0.6in}\conj
  x = \{r \,|\, r\in\Delta \wedge [r/y]e\} \\
}
{
  \R \vdash \hoare{P}{\lete{x}{\selecte{\lambda y.e}}{c}}{Q}
}
\end{array}
\end{smathpar}
\end{minipage}
%
\bigskip

%
\begin{minipage}{3.2in}
\rulelabel{RG-Delete}
\begin{smathpar}
\begin{array}{c}
\RULE
{
  \stable(\R,P)\\
  \forall\stl,\stl',\stg.~P(\stl,\stg) \conj 
  \stl' = \stl \cup \{r' \,|\, \exists(r\in\Delta).~ [r/x]e
        \conj r'=\langle r \with \txnf=i; \delf=\C{true}\rangle\}
  \Rightarrow 
  Q(\stl',\stg)
}
{
  \R \vdash \hoare{P}{\deletee{\lambda x.e}}{Q}
}
\end{array}
\end{smathpar}
\end{minipage}
%
\bigskip

%
\begin{minipage}{3.2in}
\rulelabel{RG-Insert}
\begin{smathpar}
\begin{array}{c}
\RULE
{
  \stable(\R,P)\\
  \forall\stl,\stl',\stg,i.~P(\stl,\stg) \conj j \not\in
  \dom(\stl\cup\stg) \conj \\
  \stl'=\stl \cup 
  \{\langle x \with \idf=j;\,\txnf=i;\,\delf=\C{false}\rangle\}\\ \Rightarrow Q(\stl',\stg)
}
{
  \R \vdash \hoare{P}{\inserte{x}}{Q}
}
\end{array}
\end{smathpar}
\end{minipage}
%
%
\begin{minipage}{3in}
\rulelabel{RG-Foreach}
\begin{smathpar}
\begin{array}{c}
\RULE
{
  \stable(\R,Q)\spc
  \stable(\R,\psi) \spc \stable(\R, P)\\
  P \Rightarrow [\emptyset/y]\psi\spc
  \R \vdash \hoare{\psi \wedge z\in x}{c}{Q_c}\\
  Q_c \Rightarrow [y \cup \{z\}/y]\psi\spc
  [x/y]\psi \Rightarrow Q
}
{
  \R \vdash \hoare{P}{\foreache{x}{\lambda y.\lambda z.c}}{Q}
}
\end{array}
\end{smathpar}
\end{minipage}
%
\bigskip

%
\begin{minipage}{3.9in}
\rulelabel{RG-Txn}
\begin{smathpar}
\begin{array}{c}
\RULE
{
  \stable(R,\I)\spc
  \stable(R,I)\spc
  P(\stl,\stg) \Leftrightarrow \stl=\emptyset \wedge I(\stg)\\
  \R_e = R \backslash \I_e \spc \R_c = R \backslash \I_c \spc 
   \R_e \vdash \rg{P}{c}{Q} \spc \stable(\R_c,Q) \\ 
  \forall \stl,\stg.~ Q(\stl,\stg) \Rightarrow 
    G(\stg, \stl \rhd \stg)\spc
  \forall \stg,\stg'.~I(\stg) \wedge G(\stg,\stg') \Rightarrow I(\stg')\\
}
{
  \rg{I,R}{\ctxn{i}{\I}{c}}{G,I}
}
\end{array}
\end{smathpar}
\end{minipage}
%
%
\begin{minipage}{2in}
\rulelabel{RG-Conseq}
\begin{smathpar}
\begin{array}{c}
\RULE
{
  \rg{I,R}{\ctxn{i}{\I}{c}}{G,I}\\
  \I' \Rightarrow \I \spc 
  R' \subseteq R \\
  \stable(R',\I')\spc
  G \subseteq G' \\
  \forall \stg,\stg'.~I(\stg) \wedge G'(\stg,\stg') \Rightarrow I(\stg')\\
}
{
  \rg{I,R'}{\ctxn{i}{\I'}{c}}{G',I}
}
\end{array}
\end{smathpar}
\end{minipage}
%

\caption{\small \txnimp: Rely-Guarantee Rules}
\label{fig:rg-rules}
\vspace*{-12pt}
\end{figure}


Fig.~\ref{fig:rg-rules} shows an illustrative subset of the
rely-guarantee (RG) reasoning rules for $\txnimp$. We define two RG
judgments: top-level ($\rg{I,R}c{G,I}$), and transaction-local ($\R
\vdash \hoare{P}c{Q}$).  Recall that the standard RG judgment is the
quintuple $\rg{P,R}{c}{G,Q}$. Instead of a separate $P$ and $Q$, our
top-level judgment has $I$ for both pre- and post-conditions, because
our focus is on verifying that a \txnimp program \emph{preserves}
database's consistency conditions\footnote{The terms \emph{consistency
condition}, \emph{high-level invaraint}, and \emph{integrity
constraint} all mean the same.}. Transaction-local RG judgment doesn't
include a guarantee relation because transaction-local effects are not
visible outside a transaction. Also, the rely relation ($\R$) of the
transaction-local judgment is not same as the top-level rely relation
($R$) as it takes into account the transaction's isolation
specification ($\I$). Intuitively, $\R$ is $R$ modulo $\I$; formal
definition will be given shortly. Recall that a transaction writes to
its local database ($\stl$), which is then flushed when the
transaction commits. Thus, the guarantee of a transaction depends on
the state of its local database at the commit point. The pre- and
post-condition assertions ($P$ and $Q$) in the local judgment
facilitate tracking the changes to the transaction-local state, which
eventually helps us prove the validity of the transaction's guarantee.
Note that $P$ and $Q$ are bi-state assertions; they relate
transaction-local database state ($\stl$) to the global database state
($\stg$). Thus, the transaction-local judgment effectively tracks how
transaction-local and global states change in relation to each other.

The quintessential aspect of a rely-guarantee judgment is the
stability condition, which, intuitively, requires the validity of an
assertion $\phi$ to be unaffacted by the interference, i.e., the rely
relation $R$. In conventional RG, stability is defined as following
($\sigma$ denotes a state):
\begin{smathpar}
\begin{array}{lcl}
\stable(R,\phi) & \Leftrightarrow & \forall \sigma,\sigma'.~
\phi(\sigma) \conj R(\sigma,\sigma') \Rightarrow \phi(\sigma')\\
\end{array}
\end{smathpar}
Due to the presence of local and global database states, and isolation
specification, we use multiple definitions of stability in
Fig.~\ref{fig:rg-rules}, but they all convey the same intuition as
above. We will introduce them as they are encountered.

\rulelabel{RG-Txn} is the top-level rule that lets us prove a
transaction preserves the high-level invariant $I$ when executed under
the required isolation as specified by $\I$. It relies on the
transaction-local judgment to verify the the transaction body ($c$).
The precondition $P$ of $c$ must follow from the fact that the
transaction-local database ($\stl$) is initially empty, and the global
database satisfies the high-level invariant $I$. The rely relation
($\R_l$) for the local judgment is a ternary relation computed as $R$
modulo $\I$ as following:
\begin{smathpar}
\begin{array}{lcl}
\R_l(\stl,\stg,\stg') & \Leftrightarrow & R(\stg,\stg') \conj
\I\,\C{true}\,(\stl,\stg,\stg')
\end{array}
\end{smathpar}
Thus, $\R_l$ allows an interference only if it doesn't violate the
execution-time isolation specification of the transaction. If a
certain interference violates the isolation spec, i.e., $R(\stg,\stg')
\Rightarrow \neg(\I\,\C{true}\,(\stl,\stg,\stg'))$, then
$\R_l(\stl,\stg,\stg') \Leftrightarrow false$, and any assertion is
trivially stable w.r.t that interference. This is sensible considering
such interference is pre-empted in the operational semantics. 

Recall that $P$ and $Q$ of the transaction-local RG judgment are
binary assertions; they relate local and global database states. The
local judgment rules require one or both of them to be stable. The
stability of a binary assertion $Q$ w.r.t a ternary rely relation $\R$
is defined as following:
\begin{smathpar}
\begin{array}{c}
\forall \stl,\stg,\stg'.~ Q(\stl,\stg) \conj \R(\stl,\stg,\stg')
\Rightarrow Q(\stl,\stg')
\end{array}
\end{smathpar}
That is, if $Q$ relates $\stl$ to $\stg$, and an interference allowed
by the isolation specfication (which implicitly considers the local
state $\stl$) takes $\stg$ to $\stg'$, the $Q$ must also relate $\stl$
to $\stg'$.

For the guarantee $G$ of the transaction to be valid, it must follow
from the post-condition $Q$ of the body, provided that $Q$ is stable
w.r.t the commit-time interference captured by $R_c$. $R_c$, like
$R_l$, is computed as rely relation modulo isolation, except that
commit-time isolation ($\I\,\C{false}$) is considered. The validity of
$G$ is captured by the following implication:
\begin{smathpar}
\begin{array}{c}
  \forall \stl,\stg.~ Q(\stl,\stg) \Rightarrow G(\stg, \stl \gg \stg)\spc
\end{array}
\end{smathpar}
In other words, if $Q$ relates the transaction-local database state
($\stl$) to the state of the global database ($\stg$) before the
commit, then $G$ must relate the states of the global database before
and after the commit. The act of commit captured by the flush
operation ($\stl\gg\stg$). Once we establish the validity of $G$ as a
faithful representative of the transaction, we can verify that the
transaction preserves the high-level invariant $I$ by checking the
stability of $I$ w.r.t $G$, i.e., $\forall \stg,\stg'.~I(\stg) \wedge
G(\stg,\stg') \Rightarrow I(\stg')$.

A characteristic of RG reasoning is that stability of an assertion is
always proven w.r.t to $R$, and not $R^{*}$, although interference may
include multiple environment steps, and $R$ only captures a single
step. This is nonetheless sound due to the the inductive reasoning: if
$Q$ is preserved by every step of $R$, then $Q$ is preserved by
$R^{*}$, and vice-versa.  However, such reasoning does not extend
naturally to isolation-constrained interference because $R^{*}$ modulo
$\I$ is not same as $\R^{*}$; the former is a transitive relation
constrained by $\I$, whereas the latter is the transitive closure of a
$\I$-constrained relation. We therefore introduce a side-condition on
$\I$ that restores the equality. The condition requires $\I$ to allow
an intereference $R^{*}(\stg,\stg'')$, for two database states $\stg$
and $\stg''$, only if it also allows interference for every prefix of
$R^{*}(\stg,\stg'')$. In other words, if $\I$ disallows interference
from $\stg$ to $\stg'$, then an $R$-step from $\stg'$ to $\stg''$
shouldn't make the interference from $\stg$ to $\stg''$ valid. We call
this the stability condition on $\I$, defined as below:
\begin{smathpar}
\begin{array}{lcl}
  \stable(R,\I) & \Leftrightarrow & \forall \stl,\stg,\stg',\stg''.~
  \neg\I(\stl,\stg,\stg') \conj R(\stg',\stg'') \Rightarrow
  \neg\I(\stl,\stg,\stg'')
\end{array}
\end{smathpar}
It can be easily verified that the above stability condition is
satisfied by the isolation axioms from Sec.~\ref{sec:isolation}. For
instance, $\I_{ss}$, the snapshot axiom, is stable because if
$\I_{ss}$ is invalid, then an interference has already modified a
record, and no further interference will restore the original record,
because original record bears the id of a transaction that has long
committed. Thus, $\I_{ss}$ remains invalid.

The \rulelabel{RG-Conseq} rule lets us safely strengthen the guarantee
$G$, or weaken the rely $R$ of a transaction. Importantly, it also
allows its isolation specification $\I$ to be strengthened. This means
that a transaction proven correct under a weaker isolation level is
also correct under a stronger level. Parametricity over the isolation
specification $\I$, combined with the ability to strengthen $\I$ as
needed, admits a flexible proof strategy to prove database programs
correct. For example, programmers can declare isolation requirements
of their choice through $\I$, and then prove programs correct assuming
the guarantees hold. The soundness of strengthening $\I$ ensures that
a program can be safely executed on any system that offers isolation
guarantees at least as strong as those assumed.

Two rules of the transaction-local RG judgment are shown in
Fig.~\ref{fig:rg-rules}. The rule \rulelabel{RG-Update} is
illustrative of the RG rules for SQL statements; they basically
reflect the structure of the corresponding reduction rule from
Fig.~\ref{fig:txnimp}, and require no further explanation. The rule
\rulelabel{RG-Foreach} defines the RG judgment for a \C{FOREACH} loop.
As is characteristic of loops, the reasoning is pivoted on a loop
invariant $\I$ (not same as the high-level invariant), that needs to
be stable w.r.t $\R$. $I$ must follow $P$, the pre-condition of
\C{FOREACH}, when no elements have been iterated, i.e, when
$y=\emptyset$. The body of the loop can assume the loop invariant, and
the fact that $z$ is an element from the set $x$ being iterated, to
prove its post-condition $Q_c$. Operational semantics ensures that $z$
is added to $y$ at the end of the iteration, hence $Q_c \conj z\in y$
is valid at the end of the iteration, from which $I$ must follow. When
\C{FOREACH} has finished execution, $y$, the set of iterated items, is
the entire set $x$. Thus $I \conj y=x$ must imply the post-condition
$Q$, which also needs to be stable. The rules for conditionals,
sequencing etc., are more-or-less standard, hence elided.

\subsection{Semantics and Soundness}

\begin{definition}[\bfseries Step-indexed reflexive transitive closure]
For all $A:\text{Type}$, $R: A \rightarrow A \rightarrow \mathbb{P}$, and $n :
\mathbb{N}$, the step-indexed reflexive transitive closure $R^n$ of $R$ is
the smallest relation satisfying the following
properties:
\begin{itemize}
\item $\forall (x:A).\, R^0 (x,x)$
\item $\forall (n:\mathbb{N})(x,y,z : A).\, R(x,y) \conj R^n(y,z) \Rightarrow
R^{n+1}(x,z)$
\end{itemize}
\end{definition}

% \begin{definition}[\bfseries Interleaved step and multi-step relations]
% An interleaved step relation interleaves transaction-local reductions with
% legal interference from concurrent transactions as defined by the
% $\I$-constrained rely relation ($\R$).  It is defined thus:
% \begin{smathpar}
% \begin{array}{lcl}
% (c,\stl,\stg) \rstepsto (c',\stl',\stg') & \defeq & \stg \vdash 
%   (c,\stl) \stepsto (c',\stl') \conj \stg'=\stg \\
%   &   & \disj (c' = c \conj \stl'=\stl \conj \R(\stl, \stg, \stg'))\\
% \end{array}
% \end{smathpar}

\begin{definition}[\bfseries Interleaved step and multi-step relations]
An interleaved step relation interleaves a transaction's reduction with
interference from concurrent transactions as captured by the rely
relation. It is defined thus:
\begin{smathpar}
\begin{array}{lcl}
(c,\stg) \rstepsto (c',\stg') & \defeq &  
  (c,\stg) \stepsto (c',\stg') \disj (c' = c \conj R(\stg, \stg'))\\
\end{array}
\end{smathpar}

\noindent The interleaved step relation for transaction bound expressions
($\txnbox{e}_i$) and commands ($\txnbox{c}_i$) is defined similarly.
An interleaved multi-step relation ($\stepssto{n}$) is the step-indexed
reflexive transitive closure of the interleaved step relation.
\end{definition}

\begin{definition}[\bfseries Semantics of the RG judgment]
\label{def:rg-semantics}
The semantics of the RG sextuple $\rg{\mathbb{I},P,R}{c}{G,Q}$ is defined
in terms of the interleaved step relation thus:\vspace*{-10pt}

\begin{smathpar}
\begin{array}{l}
\hspace*{-0.3in}
\rg{\mathbb{I},P,R}{c}{G,Q} \;\defeq\; \forall \E.\, P(\E)
  \wedge \mathbb{I}(\E) \\
\hspace*{0.4in}\Rightarrow (\forall n,\E'.\; \I \vdash (c,\E) 
    \rstepssto{n} (\cskip,\E') \Rightarrow Q(\E')) \\
\hspace*{0.5in}\conj \texttt{step-guaranteed}(\I,R,G,c,\E)\\
\end{array}
\end{smathpar}

\noindent The first conjunct in the consequent is called the \emph{Hoare
consequent} since it ascribes Hoare triple semantics to an RG sextuple.
The second conjunct, called the \emph{guarantee consequent}, uses the
$\texttt{step-guaranteed}$ predicate defined below:\vspace*{-10pt}

\begin{smathpar}
\begin{array}{l}
\texttt{step-guaranteed}(\I,R,G,c,\E) \;\defeq\; \forall n,\E',c'',\E''.\\
\hspace*{0.2in}\I \vdash (c,\E) \rstepssto{n} (c',\E') \conj \I \vdash (c',\E') \stepsto
  (c'',\E'') \Rightarrow G(\E',\E'')\\
\end{array}
\end{smathpar}

\noindent The guarantee consequent requires $G$ to capture the trace effect of
every small-step of $c$, where the reduction can be interleaved by the
interference ($R$) from concurrent threads. The semantics of the RG
sextuple for transaction-bound commands ($\txnbox{c}_i$) is defined
similarly. Expressions, unlike commands, evaluate to a value $v$, and
the semantics of their RG septuple ($\rg{\I,P,R}{\txnbox{e}_i}{G,C,Q}$) differs slightly in that its
Hoare consequent requires the value $v$ to satisfy the assertion $C$. 
\end{definition}

Note that the semantics of all RG judgments, including the judgments
for transaction-bound terms, make similar demands of the guarantee
relation. Given that transactions are atomic (though not isolated), it
is not immediately apparent why a transaction's guarantee is required
to make explicit every step of its reduction. This requirement is
justified however because, in reality, a transaction's atomicity is
predicated on the isolation settings of the observer. A \iso{Read
  Uncommitted} transaction, for example, is permitted to observe the
internal state of a transaction $T$ even if $T$ is claimed to execute
atomically.  In the interest of modular verification, the transaction
must therefore make its internal state available via its guarantee
relation.

\begin{theorem}[\bfseries Soundness] 
The rely-guarantee judgments defined by the rules in
Fig.~\ref{fig:rg-rules} are sound with respect to the semantics of
Definition~\ref{def:rg-semantics}.\footnote{Formal proof of soundness
is provided in the supplementary material.}
\end{theorem}

\noindent In particular, if $\rg{\I,P,R}{c}{G,Q}$ can be derived using
the rules of Fig.~\ref{fig:rg-rules}, then (a) every interleaved
multi-step reduction of $c$ starting from a trace that satisfies $P$
and $\I$, results in a trace that satisfies $Q$, and (b) the effect
that every small-step of $c$ has on the trace is contained in $G$.
Soundness of the RG judgment for transaction-bound commands
($\txnbox{c}_i$) is stated similarly.  For expressions, soundness of
the judgment $\rg{\I,P,R}{\txnbox{e}_i}{G,C,Q}$ also proves that $e$
is always evaluated to a value that satisfies $C$.



\section{Inference}
\label{sec:inference}

The Rely-Guarantee framework presented in the previous section
facilitates modular proofs for weakly-isolated transactions, but
imposes a non-trivial annotation burden.  In particular, it requires
each statement ($c$) of the transaction to be annotated with a stable
pre- ($P$) and post-condition ($Q$), and loops to be annotated with
stable inductive invariants ($\psi$). While weakest pre-condition
style predicate transformers can help in inferring intermediate
assertions for regular statements, loop invariant inference remains
challenging, even for the simple form of loops considered here.  As an
alternative, we present an inference algorithm based on state
transformers.  The idea is to infer the logical effect that each
statement has on the transaction-local database state $\stl$ (i.e.,
how it transforms $\stl$), and compose multiple such effects together
to describe the effect of the transaction as a whole.  Importantly,
this approach generalizes to loops, where the effect of a loop can be
inferred as a well-defined function of the effect of its body, thanks
to certain pleasant properties enjoyed by the database programs
modeled by $\txnimp$.  Interpreting database semantics as functional
transformations on sets (described in terms of their logical effects)
enables an inference mechanism that can leverage off-the-shelf SMT
solvers for automated verification.

\begin{figure}[h]
\begin{smathpar}
\renewcommand{\arraystretch}{1.2}
\begin{array}{lcl}
\multicolumn{3}{c}{
  x,y,\stl,\stg \in \texttt{Variables}\spc
  \varphi \in \Prop^{0}\spc
  \phi \in \Prop^{1}
}\\
s & \coloneqq & x \ALT \stl \ALT \stg \ALT \{x \,|\, \varphi\} 
  \ALT \existsl(\stg,\phi,s) 
  \ALT s_1 \bind \lambda x. s_2 \ALT \itel{\varphi}{s_1}{s_2} 
  \ALT s_1 \cup s_2 \\
\end{array}
\end{smathpar}
%
\caption{Syntax of the set language $\SL$}
\label{fig:logic-syntax}
\end{figure}



At the core of our approach is a simple language ($\SL$) to express
set transformations (see Fig.~\ref{fig:logic-syntax}). The language
admits only set expressions that include variables ($x$), literals of
the form $\{x \,|\, \varphi\}$ where $\varphi$ is a propositional
(quantifier-free) formula on $x$, a restricted form of existential
quantification that binds a set $x$ satisfying proposition $\phi$ in a
set expression $s$, a monadic composition of two set expressions
($s_1$ and $s_2$) composed using a bind ($\bind$) operation, a
conditional set expression where the condition is a propositional
formula, and a union of two set expressions. Symbols $\stl$ and $\stg$
are also variables in $\SL$, but are used to denote local and database
states, respectively.  The language is carefully chosen to be
expressive enough to capture the semantics of $\txnimp$ statements (as
well as SQL operations more generally), yet simple enough to have a
semantics-preserving translation to a decidable fragment of
first-order logic.

\begin{figure}
\raggedright
%
\fbox {\(c \elabsto \F \)}\\
%

\renewcommand{\arraystretch}{1.2}
\begin{smathpar}
\begin{array}{lcl}
\inserte{x} & \elabsto & \stabilize{\lambda(\stl,\stg).~
  \{ r \,|\, r = \{x \with \idf = \uid(x);\,r.\delf=\C{false};\,
                \txnf = T_i\} \conj \uid(x) \notin \dom(\stl\cup\stg) \}}\\ 
%
\updatee{\lambda x.e_1}{\lambda x.e_2} & \elabsto & \stabilize{
  \lambda(\stl,\stg).~ \stg \bind (\lambda r.~ 
  \{r' \,|\, [r/x]e_2 \conj r' = \{[r/x]e_1 \with 
      \idf=r.\idf;\,\delf = r.\delf;\,\txnf=T_i \}) \} }\\ 
%
\deletee{\lambda x.e} & \elabsto & \stabilize{
  \lambda(\stl,\stg).~ \stg \bind (\lambda r.~ 
  \{r' \,|\, [r/x]e \conj r'=\{r \with \delf=\C{true}\})\} }\\ 
%
\end{array}
\end{smathpar}

%
\begin{minipage}{2.7in}
\begin{smathpar}
\begin{array}{c}
\RULE
{
  c \elabsto \F \spc
}
{
  \lete{x}{e}{c} ~\elabsto~  
    \stabilize{\lambda(\stl,\stg).~ [e/x]\,\F(\stl,\stg)}\\
}
\end{array}
\end{smathpar}
\end{minipage}
%
%
\begin{minipage}{3in}
\begin{smathpar}
\begin{array}{c}
\RULE
{
  c \elabsto \F \spc
  G = \lambda r.~ \itel{[r/x]e}{\{r\}}{\emptyset}
}
{
  \lete{y}{\selecte{\lambda x.e}}{c} ~\elabsto~  
    \stabilize{\lambda (\stl,\stg).~ [(\stg \bind G)/y]\,\F(\stl,\stg)}\\
}
\end{array}
\end{smathpar}
\end{minipage}
%

%
\begin{minipage}{3in}
\begin{smathpar}
\begin{array}{c}
\RULE
{
  c_1 \elabsto \F_1 \spc
  c_2 \elabsto \F_2 
}
{
  \ite{e}{c_1}{c_2} ~\elabsto~  \itel{e}{\F_1}{\F_2}\\
}
\end{array}
\end{smathpar}
\end{minipage}
%
%
\begin{minipage}{3in}
\begin{smathpar}
\begin{array}{c}
\RULE
{
  c_1 \elabsto \F_1 \spc
  c_2 \elabsto \F_2 
}
{
  c_1;c_2 ~\elabsto~  \lambda(\stl,\stg).~\F_1 \cup \F_2(\stl \cup \F_1(\stl,\stg),\stg)
}
\end{array}
\end{smathpar}
\end{minipage}
%

%
\begin{minipage}{3in}
\begin{smathpar}
\begin{array}{c}
\RULE
{
  c \elabsto \F \spc
}
{
  \foreache{x}{\lambda y.\lambda z.~c} ~\elabsto~
  \lambda(\stl,\stg).~ x\bind(\lambda z.~\F(\stl,\stg))
}
\end{array}
\end{smathpar}
\end{minipage}
%

\caption{\txnimp: Effect inference rules }
\label{fig:inference-rules}
\end{figure}


Fig.~\ref{fig:inference-rules} shows the syntax-directed state
transformer inference rules for $\txnimp$ commands inside a
transaction $\C{TXN}_i$.  Inference depends on the $\I$-constrained
rely relation $\R$, and the high-level invariant $I$ for the reasons
that will become clear shortly.  Inference rules compute, for each
command $c$, a (meta) function $\F$ that returns a set as an
expression in $\SL$, given a pair of sets ($\stl$ and $\stg$) that
describe local and a global database states, respectively. The
expression returned by $\F(\stl,\stg)$ (abstractly) describes the set
of records that get added to $\stl$ as a result of executing $c$ under
$\stl$ and $\stg$.  Thus, $\F$ captures the \emph{effect} part of the
state transformer of $c$, which is the function
$\lambda(\stl,\stg).~\stl \cup \F(\stl,\stg)$\footnote{Recall that the
  operational semantics treats deletion of records as the addition of
  the deleted record with its \C{del} field set to true in the local
  store.}. For $\F$ to be useful in RG verification, it needs to be
stable w.r.t the rely relation $\R$. The stability condition on
effects can be defined thus:
\begin{smathpar}
\begin{array}{lcl}
  \stable(\R,\F) & \Leftrightarrow & \forall \stl,\stg,\stg',\bar{v}.~
  \R(\stl,\stg,\stg') \Rightarrow \F(\stl,\stg) \equiv \F(\stl,\stg')
\end{array}
\end{smathpar}
where, $\bar{v}$ are the variables that occur free in $\F$; this is
possible because of how the inference rules are structured.
Intuitively, the stability condition requires an effect to describe
the same set of records before and after the interference. The
equivalence in $\SL$ translates to the equality in first-order logic,
as we describe below. In the inference rules, stability is
enforced constructively by a meta function $\stabilize{\cdot}$, which
accepts an effect and returns a new effect that is guaranteed to be
stable under $\R$.  $\stabilize{\cdot}$ achieves the stability
guarantee by abstracting away the bound $\stg$ in an unstable $\F$ to
an existentially bound $\stg'$ as described below:
\begin{smathpar}
\begin{array}{lcll}
  \stabilize{\F} & = & \F & \texttt{if } \stable(\R,\F).\\
  & = & \lambda (\stl,\stg).~\existsl(\stg',I(\stg'),\F(\stl,\stg')) 
      & \texttt{otherwise. }\stg'\texttt{ is a fresh name.}\\
\end{array}
\end{smathpar}
Observe that when $\F$ is not stable, then $\stabilize{\F}$ returns a
transformer $\F'$ that simply ignores its $\stg$ argument in favor of a generic
$\stg'$, making $\F'$ trivially stable. It is safe to assume
$I(\stg')$ because all verified transactions preserve the invariant,
and hence only valid database states will ever be witnessed. From the
perspective of RG reasoning, $\stabilize{\cdot}$ effectively weakens
the post-condition of a statement, as done by the
\rulelabel{RG-Conseq} rule for transaction-bound commands.  The weakening semantics chosen by
$\stabilize{\cdot}$, while being simple, is nonetheless useful because
of the $I(\stg')$ assumption on the existentially bound $\stg'$. The
example in Fig.~\ref{fig:weakening-example} demonstrates. 
\begin{figure}[h]
\begin{ocaml}
let add_interest acc_id pc = atomically @@ do
  let a = SQL.select1 BankAccount (fun acc -> acc.id = acc_id);
  let y = a.bal + pc*a.bal;
  SQL.update BankAccount (fun acc -> {acc with bal = acc.bal + y})
                         (fun acc -> acc.id = acc_id)
\end{ocaml}
\caption{A transaction that deposits an interest to a bank account.}
\label{fig:weakening-example}
\end{figure}
Here, an \C{add\_interest} transaction adds a positive interest
(\C{pc}) to the balance of a bank account, which is required to be
positive ($I(\stg) \Leftrightarrow \forall(r\in\stg).~r.\C{bal}\ge
0$). The transaction starts by issuing a \C{select1} query, whose
effect implicitly asserts that there exists a record in the bank
account database that is equal to $a$
($\exists(r\in\stg).~r=a$). However, this assertion is unstable
because a concurrent \C{withdraw} or \C{deposit} transactions might
update the account balance, so that such a record no longer exists in
the database.  Fortunately, the weakening operator
($\stabilize{\cdot}$) allows us to weaken the assertion to $\exists
\stg'.I(\stg') \conj \exists(r\in\stg').~a=r$ (Fig.~\ref{fig:logic}
formalizes the encoding of {\sf exists} to logic), which is enough to
prove that $\C{a.bal + pc*a.bal}\ge 0$, and verify that
\C{add\_interest} preserves the positive balance invariant.

The state transformer rules, like the earlier RG rules, closely follow
the corresponding reduction rules in Fig.~\ref{fig:txnimp}, except
that their language of expression is $\SL$. For instance, while the
reduction rule for \C{UPDATE} declaratively specifies the set of
updated records, the state transformer rule uses $\SL$'s bind
operation to \emph{compute} the set. Other SQL rules do likewise. The
rules for \C{LET} binders, conditionals, and sequences compose the
effects inferred for their subcommands. For instance, the effect of a
sequence of commands $c_1;c_2$ is the union of effects $F_1$ and $F_2$
of $c_1$ and $c_2$, respectively, except that $F_2$ is applied to a
state ($\stl$) to which $F_1$ has already been applied, reflecting
their order of reduction. The inference rule for \C{FOREACH} takes
advantage of the $\SL$'s bind operator to lift the effect inferred for
the loop body to the level of the loop. Since records added to $\stl$
in each iteration of \C{FOREACH} are independent of the previous
iteration, sequential composition of the effects of different
iterations is same as their parallel composition. Since the loop body
is executed once per each $z\in x$, the effect of the the loop is the
union of effects ($\F$) for all $z\in x$, all applied to the same
state ($\stl$ and $\stg$). That is, $\F_{loop}(\stl,\stg) =
\bigcup_{z\in x}\F_{body}(\stl,\stg)$. From the definition of the set
monad's bind operator, $\F_{loop}(\stl,\stg) = x \bind (\lambda
z.~F_{body}(\stl,\stg))$, which mirrors the definition of the rule.

% \begin{figure}[h]
% \begin{smathpar}
% \begin{array}{l}
%   \foreache{xs}{(\lambda y.\lambda x.~ 
%         \updatee{(\lambda z.~ \{\idf=z.\idf;\,\C{name}=z.\C{name};\,
%                                \C{s\_id}=x.\idf\})}
%                 {(\lambda z.~ z.\idf = x.\C{s\_id})};\\
%         \hspace*{1.27in} \inserte{x})}
% \end{array}
% \end{smathpar}
% \caption{A $\txnimp$ program that populates a database
%   of couples.}
% \label{fig:people_code}
% \end{figure}

%% \begin{example}
%% Consider a database of couples that tracks the spouse
%% relationship. Each record has an $\idf$ field, a \C{name} field, and a
%% spouse id (\C{s\_id}) field. A $\txnimp$ transaction bearing id $i$
%% that populates the database with a given set ($xs$) of couples will
%% include the following code snippet:
%% \begin{smathpar}
%% \begin{array}{l}
%%   \foreache{xs}{(\lambda y.\lambda x.~ 
%%         \updatee{(\lambda z.~ \{\idf=z.\idf;\,\C{name}=z.\C{name};\,
%%                                \C{s\_id}=x.\idf\})}
%%                 {(\lambda z.~ z.\idf = x.\C{s\_id})};\\
%%         \hspace*{1.27in} \inserte{x})}
%% \end{array}
%% \end{smathpar}
%% The effect captured by a state transformer inferred for the program, assuming no interference would be:
%% \begin{smathpar}
%% \begin{array}{l}
%%   \lambda(\stl,\stg).~xs \bind 
%%     (\lambda x. \stg \bind 
%%       (\lambda z. \itel{z.\idf = x.\C{s\_id}}
%%                        {\{\{\idf=z.\idf;\,\C{name}=z.\C{name};\,
%%                                \C{s\_id}=x.\idf\}\}}
%%                        {\emptyset}) \\
%%       \hspace*{1cm} ~\cup~ \{y \,|\, y = \{x \with \delf=\mathit{false}; \}\})
%% \end{array}
%% \end{smathpar}
%% Under the possibility of an interference affecting the stability, the
%% following stable effect is inferred ($I$ is the database invariant):
%% \begin{smathpar}
%% \begin{array}{l}
%%   \lambda(\stl,\stg).~xs \bind 
%%     (\lambda x.~\existsl(\stg',I,\stg' \bind 
%%       (\lambda z. \itel{z.\idf = x.\C{s\_id}}
%%                        {\{\{\idf=z.\idf;\,\C{name}=z.\C{name};\,
%%                                \C{s\_id}=x.\idf\}\}}
%%                        {\emptyset})) \\
%%       \hspace*{1cm} ~\cup~ \{y \,|\, y = \{x \with \delf=false; \}\})
%% \end{array}
%% \end{smathpar}
%% \end{example}

\begin{figure}
\raggedright
\begin{smathpar}
\renewcommand{\arraystretch}{1.2}
\begin{array}{lcl}
\multicolumn{3}{c}{
  x \in \texttt{Variables}\spc
  \phi^0 \in \Prop^{0}
}\\
s & \coloneqq & x \ALT \stl \ALT \stg \ALT \{x \,|\, \phi^0\} 
  \ALT \existsl(x,\phi^0,s) 
  \ALT s_1 \bind \lambda x. s_2 \ALT \itel{\phi^0}{s_1}{s_2} 
  \ALT s_1 \cup s_2 \\
\end{array}
\end{smathpar}
%
\bigskip

%
\begin{smathpar}
\begin{array}{lclcr}
\semof{x \ALT \stl \ALT \stg} & = & (true,x) \ALT (true,\stl) \ALT
(true,\stg) & \\
%
\semof{\{a \,|\, \phi^0\}} & = & (\forall a.~ g(a) \Leftrightarrow \phi^0,g) 
  & \texttt{  } & \fresh(g)\\
%
\semof{\existsl(a,\phi^0,s)} & = &  (\phi^1 \conj 
  \exists a.\forall b.~ g(b) \Leftrightarrow \phi^0 \wedge g_1(b), ~g)
  & \texttt{  } & (\phi^1,g_1)=\semof{s} \spc \fresh(g)\\
%
\semof{s_1 \bind \lambda x. s_2} & = & (\phi^1_1 \conj \phi^1_2 \conj
  \forall a.\exists b.~ g(a) \Rightarrow g_2(b) \wedge [b/x]g_1(a) & & \\
&& \hspace*{0.57in}\conj \forall b.\forall a.~g_2(b) \wedge [b/x]g_1(a) 
      \Rightarrow g(a), ~g) & \texttt{  } & (\phi^1_1,g_1)=\semof{s_1} \spc
      (\phi^1_2,g_2)=\semof{s_2} \spc \fresh(g)\\
%
\semof{\itel{\phi^0}{s_1}{s_2}} & = & (\phi^1_1 \conj \phi^1_2 \conj
  \forall a.~ g(a) \Leftrightarrow (\phi^0 \Rightarrow g_1(a))& & \\
&&  \hspace*{1.5in}\wedge (\neg\phi^0 \Rightarrow g_2(a)),~g) & \texttt{  } 
      & (\phi^1_1,g_1)=\semof{s_1} \spc
      (\phi^1_2,g_2)=\semof{s_2} \spc \fresh(g)\\
%
\semof{s_1 \cup s_2} & = & (\phi^1_1 \conj \phi^1_2 \conj
  \forall a.~ g(a) \Leftrightarrow g_1(a) \vee g_2(a),~g) 
  & \texttt{ } & (\phi^1_1,g_1)=\semof{s_1} \spc
      (\phi^1_2,g_2)=\semof{s_2} \spc \fresh(g)\\
\end{array}
\end{smathpar}

\caption{The syntax and semantics of the set logic}
\label{fig:logic}
\end{figure}


\subsection{Soundness of Inference}

The correspondence between the inference rules and the RG judgment of
Sec.~\ref{sec:reasoning} is stated thus\footnote{The proof can be found in
the supplementary material.}:
\begin{theorem}
\label{thm:inference-sound}
  \emph{Forall} i,$R$,$I$,$c$,$\F$, if $\stable(\R,I)$ and $c \elabsto \F$,
  then:\\\vspace*{-0.2cm}
  \begin{smathpar}
  \begin{array}{c}
  \R \vdash \hoare{\lambda(\stl,\stg).~\stl=\emptyset \conj
  I(\stg)}{c}{\lambda(\stl,\stg).\stl \subseteq \F(\emptyset,\stg)}
  \end{array}
  \end{smathpar}
\end{theorem}
\noindent where the set expression $\F(\emptyset,\stg)$ has a natural
interpretation as discussed below.
% Where, ${\sf true} = \lambda(\stl,\stg).~true$ and $\stable(\R,I)
% \Leftrightarrow \forall\stl,\stg,\stg'. I(\stg) \conj
% \R(\stl,\stg,\stg') \Rightarrow \I(\stg')$.

\subsection{From $\SL$ to the first-order logic}

Theorem~\ref{thm:inference-sound} lets us replace the local judgment
of the \rulelabel{RG-Txn} rule (Fig.~\ref{fig:rg-rules}) by a state
transformer inference judgment. The soundness of a transaction's guarantee can now
be established w.r.t the effect $\F$ of the body. The
\rulelabel{RG-Txn} rule so updated is shown
below:
\begin{smathpar}
\begin{array}{c}
\RULE
{
  \stable(R,\I)\spc
  \stable(R,I)\\
   \R_e(\stl,\stg,\stg') \Leftrightarrow \exists \stg_1. \I_e(\stl, \stg_1, \stg) \wedge R(\stg, \stg') \wedge \I_e(\stl, \stg_1, \stg') \spc  c \elabsto \F\\
  \R_c(\stl,\stg,\stg') \Leftrightarrow \exists \stg_1. \I_c(\stl, \stg_1, \stg) \wedge R(\stg, \stg') \wedge \I_c(\stl, \stg_1, \stg') \spc \stable(\R_c,\F)\\
  \forall \stg.~ G(\stg, F(\emptyset,\stg) \gg \stg)\spc
  \forall \stg,\stg'.~I(\stg) \wedge G(\stg,\stg') \Rightarrow I(\stg')\\
}
{
  \rg{I,R}{\ctxn{i}{\I}{c}}{G,I}
}
\end{array}
\end{smathpar}
Automating the application of the \rulelabel{RG-Txn} rule for a
transaction requires automating the multiple implication checks in
the premise. While $R$, $\R$, $\R_c$, $\I$ and $I$ are formulas in
first-order logic (FOL) with a relatively simple structure, $\F$
is an expression in the set language $\SL$
(Fig.~\ref{fig:logic-syntax}) with a possibly complex structure.
Fortunately, however, there exists a semantics-preserving translation
from $\SL$ to a restricted subset of first-order logic (FOL) that
lends itself to automatic reasoning. 

The algorithm ($\mssemof{\cdot}$) shown in Fig.~\ref{fig:logic}
encodes an $\SL$ expression in a decidable fragment of FOL. Given a
set expression $s$ of $\SL$, $\mssemof{s}$ is a pair $(\phi,f)$, where
$\phi$ encodes $s$ in FOL, and $f$ is a Boolean function that names
the set defined by $s$.  Naming provides an easy handle to $s$,
simplifying $\mssemof{\cdot}$.  Set variables ($x$, $\stl$, $\stg$,
$\dots$) are encoded as unconstrained Boolean functions ($g_x$,
$g_{\stl}$, $g_{\stg}$, $\ldots$) that are named after the variable.
Encoding of a set literal $\{a\,|\,\varphi\}$ introduces a new Boolean
function $g$ that is \emph{true} for only those $a$ that satisfy $\varphi$.
Thus $g$ faithfully encodes the literal. The encoding of the
$\existsl{}$ expression first eliminates the second-order
quantification over sets.  Existentially bound set variable ($a$) is
skolemized, i.e., instantiated with a new unbound variable $a'$,
before encoding the bound expression $s$ and the first-order
constraint $\phi$. The algorithm for encoding $\phi$, which is already
a first-order formula, is straightforward. It basically involves
replacing set variables with Boolean functions named after the
variables. The first-order encoding for a bind expression describes
the semantics of set monad's bind operator in FOL. Let $s_1$ be a set, and
let $f$ be a function that maps each variable in $s_1$ to a new
set. Then, $s_2 = s_1 \bind f$ if and only if for all $y\in s_2$,
there exists an $x \in s_1$ such that $y = f(x)$, and forall $x\in
s_1$, $f(x)\in s_2$. The encoding adds new constraints to this
effect. Conditional set expressions and set union are encoded
straightforwardly.

Observe that the encoding shown in Fig.~\ref{fig:logic} maps to
a subset of logic that satisfies the following syntactic properties:
\begin{itemize}
  \item All quantification is first-order; second-order objects, such
    as sets and functions, are never quantified.
  \item Quantifiers appear only at the prenex position, i.e., at the
    beginning of a quantified formula.  
  \item All function symbols, modulo those that might appear in
    $\varphi\in\Prop^0$, are uninterpreted and Boolean.
\end{itemize}
This fragment of FOL known as EPR (Effectively Propositional logic)
is known to be decidable~\cite{z3epr}. The language of encoding, however, is a
combination of {\sf EPR} with (a). $\Prop^0$, the theory from which
quantifier-free propositions ($\varphi$) that encode object language
expressions are drawn, and (b).  $\Prop^1$, the theory from which
invariants ($I$) are drawn. We write $\SL[\Prop^0,\Prop^1]$ to
highlight the parameterization of $\SL$ on $\Prop^0$ and $\Prop^1$,
and state the following theorem:
\begin{theorem}
  $\SL[\Prop^0,\Prop^1]$ is decidable if ${\sf EPR}+\Prop^0+\Prop^1$
  is decidable.
\end{theorem}
A useful instantiation of $\SL$ is $\SL[{\sf SLA},{\sf EPR}+{\sf
SLA}]$, where ${\sf SLA}$ is the theory of simple linear arithmetic.
Since {\sf EPR}+{\sf SLA} is known to be decidable~\cite{eprsla}:
\begin{theorem}
  $\SL[{\sf SLA},{\sf EPR}+{\sf SLA}]$ is decidable.
\end{theorem}
The $\SL[{\sf SLA},{\sf EPR}+{\sf SLA}]$ instantiation requires $I$ to
be drawn from {\sf EPR}+{\sf SLA}, which is expressive enough to
describe common database integrity constraints, such as referential
integrity, non-negativeness of all integer values in a column etc.
The isolation specifications presented in Sec.~\ref{sec:isolation} are
already simple first-order formulas that can be encoded in {\sf EPR}.
Furthermore, it is also reasonable to expect the guarantee ($G$) of a
transaction to be expressible in the same logic as its inferred $\F$,
since $\F$ (without the stability check) is essentially a complete
characterization of the transaction, while $G$ is only an abstraction.
Thus, with $\SL[{\sf SLA},{\sf EPR}+{\sf SLA}]$ as the language of
inference, the verification problem for weakly isolated transactions
is decidable. Moreover, off-the-shelf SMT solvers (e.g., Z3) are
equipped with efficient decision procedures for ${\sf EPR}+{\sf SLA}$,
making automated verification a practical exercise.



\section{Implementation}
\label{sec:implementation}

\begin{figure}
\begin{ocaml}
type table_name =  District | Order | Order_line | Stock

type district = {d_id: int; d_next_o_id: int}
type order = {o_id: int; o_d_id: int; o_c_id: int; o_ol_cnt: int}
type order_line = {ol_o_id: int; ol_d_id: int; ol_i_id: int; ol_qty: int}
type stock = {s_i_id: int; s_d_id:int; s_qty: int}
\end{ocaml}
\caption{OCaml type definitions corresponding to the TPC-C schema from
Fig.~\ref{fig:schema}}
\label{fig:ocaml-schema}
\end{figure}

We have implemented our DSL to define transactions as monadic
computations in OCaml (modulo some syntactic sugar), and our automatic
reasoning framework as an extra frontend pass (called \tool) in the
ocamlc 4.03 compiler\footnote{The source code is available at
available at \emph{https://goo.gl/nmUXZK}}. The input to \tool is a
program in our DSL that describes the schema of the database as a
collection of OCaml type definitions, and transactions as OCaml
functions, whose top-level expression is an application of the
\C{atomically\_do} combinator. For instance, TPC-C's schema from
Fig.~\ref{fig:schema} can be described via the OCaml type definitions
shown in Fig.~\ref{fig:ocaml-schema}.  \tool also requires a
specification of the program in the form of a collection of guarantees
($G$), one per transaction, and an invariant $I$ that is a conjunction
of the integrity constraints on the database. An auxiliary DSL that
includes the first-order logic (FOL) combinators has been implemented
for this purpose. \tool's verification pass follows OCaml's type
checking pass, hence the concrete artifact of verification is OCaml's
typed AST. The tool is already equipped with  an axiomatization of
PostgreSQL and MySQL's isolation levels expressed in our FOL DSL.
Other data stores can be similarly axiomatized. The concrete result of
verification is an assignment of an isolation level of the selected
data store to each transaction in the program.

At the top-level, \tool runs a loop that picks an unverified
transaction and progressively strengthens its isolation level until it
passes verification. If the selected data store provides a
serializable isolation level, and if the program is sequentially
correct, then the verification is guaranteed to succeed. Within the
loop, the \tool first computes the various rely relations needed for
verification ($R$, $\R_l$, and $\R_c$). It then traverses the AST of a
transaction, applying the inference rules to construct a state
transformer, check its stability, and weaken it ($\stabilize{\cdot}$)
if it is not stable. The result of traversing the transaction's AST is
therefore a state transformer ($\F$) that is stable w.r.t $\R_l$, which
is also stabilized against $\R_c$ (using $\stabilize{\cdot}$), and
then checked against the transaction's stated guarantee ($G$). If the
check passes, then the guarantee is verified to check if it preserves
the invariant $I$. The successful result from both the checks results
in the transaction being certified correct under the current choice of
its isolation level. Successful verification of all transactions
concludes the top-level execution, returning the inferred isolation
levels as its output.

\tool uses the Z3 SMT solver as its underlying reasoning engine. Each
implication check described above is first encoded in FOL, applying
the translation described in \S\ref{sec:inference} wherever
necessary.

\subsection{Pragmatics}

\textbf{Real-World Isolation Levels} The axiomatization of the
isolation levels presented in \S\ref{sec:isolation} leave out
certain nuances of their implementations on real data stores, which
need to be taken into account for verification to be effective in
practice.  We consider these into account while linking \tool with
store-specific semantics (isolation specifications etc). As an
example, consider how PostgreSQL implements an \C{UPDATE} operation.
\C{UPDATE} first selects the records that meet the search criteria
from the snapshot against which it is executing (the snapshot is
established at the beginning of the transaction if the isolation level
is SI, or at the beginning of the \C{UPDATE} statement if the
isolation level is RC). The selected records are then visited in the
actual database (if they still exist), write locks are obtained, and
the update is performed, provided that the record still meets
\C{UPDATE}'s search criteria. If the record no longer meets the
search criteria (due to a concurrent update) the record is excluded
from the update, and the write lock is immediately released.
Otherwise, the record remains locked until the transaction commits. 

Clearly, this sequence of events is not atomic, unlike the assumption
made by our formal model.  The implementation admits interference
between the updates of individual records that meet the search
criteria.  Nonetheless, through a series of relatively straightforward
deductions, we can show that PostgreSQL's \C{UPDATE} is in fact
equivalent (in behavior) to a sequential composition of two atomic
operations $c_1;c_2$, where $c_1$ is effectively a \C{SELECT}
operation with the same search criteria as \C{UPDATE}, and $c_2$ is
a slight variation of the original \C{UPDATE} that updates a
record only if a record with the same id is present in the set of records
returned by the \C{SELECT}. This transformation is summarized below:
\begin{smathpar}
\begin{array}{lcl}
\updatee{(\lambda x. ~e_1)}{(\lambda x.~e_2)}
&
\longrightarrow
&
\lete{y}{\selecte{(\lambda x.~e_1})}
     {\updatee{(\lambda x.~e_1 \wedge x.\idf\in\dom(y))}
              {(\lambda x.~e_2})}\\
\end{array}
\end{smathpar}
The intuition behind this translation is the observation that all
interferences possible during the execution of the \C{UPDATE} can be
accommodated between the time the records are selected from the
snapshot, and the time they are actually updated.  \tool performs this
translation if the selected store is PostgreSQL, allowing it to reason
about \C{UPDATE} operations in a way that is faithful to its semantics
on PostgreSQL. \tool also admits similar compensatory logic for
certain combinations of isolation levels and operations on MySQL.

\textbf{Set functions} SQL's \C{SELECT} query admits projections of
record fields, and also application of auxiliary functions such as
\C{MAX} and \C{MIN}, e.g., \C{SELECT MAX(ol\_o\_id) FROM
Order\_line WHERE $\ldots$}. We admit such extensions as set functions
in our DSL (e.g., \C{project}, \C{max}, \C{min}), and axiomatize their
behavior. For instance:
\begin{smathpar}
\begin{array}{lcl}
  s_2 \;=\;\C{project}\,s_1\,(\lambda z.~e) & \Leftrightarrow &
  \forall y.~y\in s_2 \Leftrightarrow  \exists(x \in s_1).~y = [x/z]e\\
  x \;=\; \C{max}\,s & \Leftrightarrow & x \in s \conj \forall(y \in
  s).~y\le x\\
\end{array}
\end{smathpar}
There are however certain set functions whose behavior cannot be
completely axiomatized in FOL. These include \C{sum}, \C{count} etc.
For these, we admit imprecise axiomatizations expressed as lemmas
over these functions.

\textbf{Annotation Burden} \tool significantly reduces the annotation
burden in verifying a weakly isolated transactions by eliminating the
need to annotate intermediate assertions and loop invariants.
Guarantees ($G$) and global invariants ($I$), however, still need to be
provided. Alternatively, a weakly isolated transaction $T$ can be
verified against a generic serializability condition,  eliminating
the need for any annotation. In this mode, \tool first infers the
transformer $\F_{SER}$ of $T$ without considering any interference,
which then becomes its guarantee ($G$). Doing likewise for every
transaction results in a rely relation ($R$) that includes $\F_{SER}$
of every transaction. Verification now proceeds by taking interference
into account, and verifying that each transaction still yields the
same $F$ as its $F_{SER}$. The result of this verification is an
assignment of (possibly weak) isolation levels to transactions which
nonetheless guarantees behavior equivalent to a  serializable execution.


\subsection{Evaluation}


To explore the space of correct isolation levels for different transactions, we  also  ran a number  of experiments using  a real transaction OLTP workload   based on TPC-C benchmark.  
 The  workload  involves  a mix of five TPC-C transaction types: 
\C{New\_Order},  \C{Payment},  \C{Delivery}, \C{Stock Level} and  \C{Order Status}, 
operating on top of  a $10$-warehouse database.  
Each warehouse has  $10$ districts, each serving $3000$ customers. 
TPC-C  specification model  requires  maintaining multiple   consistency conditions.  We focus only on   four conditions, named $I_1$, $I_2$, $I_3$, and $I_4$.   
Consistency condition $I_1$  requires that  sales of each  warehouse  equals to the  sum of    its  district's sales.  Conditions $I_2$ and $I_3$  enforce  
 sequentially unique IDs assigned to  orders placed in the same   warehouse and district. Condition $I_4$ requires 
 the  number of  item orders   in each district 
  reflexes  the district's roll-up. 
Given the consistency conditions,  Table  \ref{tab:tpcc-eval} gives the isolation level  for the \C{New\_Order},  \C{Payment},  \C{Delivery} transactions using the  MySql and Postrgres databases   to maintain   each consistency condition. 
Recall that the \C{New\_Order} transaction places a new order for
 items from a particular district in a warehouse.  
The \C{Payment} transaction  processes  a payment for a particular customer and updates the customer, district, and warehouse tables. 
 The  \C{Delivery}  transaction processes  orders for a particular   district at  a warehouse. 
 Our experimental results  shows that  the concurrent executions of   \C{New\_Order}   
    and  \C{Delivery} transactions under a weak isolation level (i.e.,  $RC$) is anomalous, i.e., it may violate the consistency conditions $I_2$ and $I_3$. Depending on the isolation guarantees provided  by  underlying database,   the stronger  isolation level is required. For instance, the isolation level $RR$ in Postgres  is sufficient to maintain the condition while   MySql database  requires  $SER$. 


 	\begin{table*}[t]\small
	\centering
\begin{tabular}{|c|c|c|c|}
		\hline
	\textbf{Transaction}   & \textbf{Invariant} 
	& \textbf{MYSQL-Isolation} & \textbf{Postgres-Isolation} \\ 
	\hline
	\multirow{4}{*}{new\_order }  & $I_1$ 
	& RC &  RC\\ 
	&  $I_2$ &SER & RR \\
	&  $I_3$ & SER  &  RR  \\
	& $I_4$ & RC & RC   \\
	\hline
	\multirow{4}{*}{Payment}  & $I_1 $ 
	& RC &  RC\\ 
	&  $I_2$  &RC & RC \\
	&  $I_3 $ & RC  &  RC  \\
	& $I_4$  & RC & RC   \\
	\hline
	\multirow{4}{*}{Delivery}  & $I_1$  
	& RC &  RC \\ 
	&  $I_2$ &SER & RR \\
	&  $I_3$ & SER  &  RR \\
	& $I_4$  & RC & RC   \\
	\hline
\end{tabular}
	\caption{ Isolation level required by TPC-C transactions. }
	\label{tab:tpcc-eval}
\end{table*}



\section{Related Work}
\label{sec:relatedwork}

\paragraph{Specifying weak isolation.}
Adya~\cite{adyaphd} specifies several weak isolation levels in terms
of \emph{dependency graphs} between transactions, by defining the
kinds of dependency cycles that are forbidden in each case. The
operational nature of Adya's specifications make them suitable for
runtime monitoring and anomaly
detection~\cite{kemmevldb,feketesigmod08,pssi2011}, whereas the
declarative nature of our specifications make them suitable for formal
reasoning about program behaviour. Sivaramakrishnan \emph{et
  al.}~\cite{pldi15} specify isolation levels declaratively as trace
well-formedness conditions, but their specifications implicitly assume
a complete trace with only committed transactions, and is thus not
suitable to reason operationally about a program as it builds the
trace; their framework also requires application invariants be
expressed in terms of the visibility relations used to define
isolation and consistency properties.  Cerone \emph{et
  al.}~\cite{gotsmanconcur15} present a framework to specify isolation
levels with atomic visibility, but their specifications are also for
complete traces; moreover, their formulation implicitly assumes
homogenity, \emph{i.e.}, they require all transactions to execute
using either {\sc si} or {\sc ser}, which is often not the case in
practice. While all the aforementioned specification frameworks use
the vocabulary introduced in~\cite{burckhardt14} to specify replicated
data types and different forms of weak consistency, none of them come
with an associated reasoning framework that can use such
specifications productively towards verifying programs under weak
isolation.

\paragraph{Reasoning under weak isolation} In~\cite{feketessi}, Fekete
\emph{et al.} propose a theory to characterize non-serializable
executions that arise under {\sc si}. Fekete~\cite{fekete2005} also
proposes an algorithm that allocates either {\sc si} or {\sc ser}
isolation levels to transactions while guaranteeing the
serializability of the execution. In~\cite{gotsmanpodc16}, Cerone
\emph{et al.} improve on Adya's {\sc si} specification and use it to
derive a static analysis that determines the safety of substituting
{\sc si} with a weaker variant called \iso{Parallel Snapshot
  Isolation}~\cite{psi}.  These efforts focus on establishing the
equivalence of executions between a pair of isolation levels, without
taking application invariants into account.  Bernstein \emph{et
  al.}~\cite{bern2000} propose informal semantic conditions to ensure
the satisfaction of application invariants under weaker isolation
levels.  However, no formal proof system is given.  Furthermore, these
techniques are tailor-made for a finite set of well-understood
isolation levels (rooted in~\cite{berenson}) under a pre-defined
store consistency model (usually {\sc sc}).

\paragraph{Reasoning under weak consistency} There have been several
recent proposals on reasoning techniques for programs executing under
weak consistency~\cite{bailisvldb, alvarocalm,
  gotsmanpopl16,redblueatc, redblueosdi, ecinec}. All of them assume a
system model that offers a choice between a \emph{coordination-free}
weak consistency level (\emph{e.g.}, {\sc ec}~\cite{redblueosdi,
  redblueatc, ecinec, alvarocalm, bailisvldb}) or causal
consistency~\cite{gotsmanpopl16}). The key technical issue in these
efforts involves proving that atomic and fully isolated operations
preserve application invariants when executed on these consistency
levels.  In contrast, we admit weakly isolated transactions, and our
system model accepts \emph{specifications} of consistency and
isolation levels drawn from an expressive logic.  In recent
work~\cite{gotsmanpopl16}, \iso{Parallel Snapshot Isolation} is
adapted to the aforementioned setting by interpreting it as a
consistency level that serializes writes to objects; a dedicated proof
rule is developed to help prove prove program invariants hold under
this model. By parameterizing our proof system over a gamut of weak
isolation specifications, we avoid the need to define a separate proof
rule for each new isolation level we may encounter.

\paragraph{Reasoning under relaxed memory} The reasoning mechanisms
used to describe and prove properties about weakly-isolated
transactions bear resemblance to those used to formalize relaxed
memory behaviour~\cite{battycpp}.  Ridge~\cite{rgtso} generalizes
rely-guarantee reasoning to the x86-TSO memory model.  Likewise,
Vafeiadis \emph{et al.}~\cite{rsl13} generalize concurrent separation
logic (CSL)~\cite{csl} to the C11 relaxed memory model.  Ferreira
\emph{et al.}~\cite{ferreira10} propose a parameterized operational
semantics for relaxed memory models, but the parameterization is over
a relation between relaxed memory programs and related {\sc sc}
programs. They also prove the soundness of CSL for the weakest memory
model expressible in their semantics, but neither a parameterized (in
the aforementioned sense) CSL, nor a CSL specialized for any relaxed
memory model is proposed.  ~\cite{DLZ+13} presents a \emph{buffered
  memory model} for Java that defines an axiomatic definition for the
JMM in terms of memory reorderings, and an operational instantiation
consistent with the TSO memory model found on x86 processors, and
demonstrates the equivalence of the two definitions.  However, there
is no attempt to parameterize the model over different reordering or
consistency definitions, which is a central focus of this paper.



\section*{Acknowledgements}

We thank KC Sivaramakrishnan for numerous helpful discussions about
weak isolation, and for thorough analysis of the material presented in
this paper. We are grateful to the anonymous reviewers, and our
shepherd, Peter M{\"u}ller, for their careful reading and insightful
comments and suggestions.  This material is based upon work supported
by the National Science Foundation under Grant No. CCF-SHF 1717741 and
the Air Force Research Lab under Grant No.  FA8750-17-1-0006.

% We recommend abbrvnat bibliography style.

%\bibliographystyle{plainnat}
\small
\bibliography{all}

\end{document}
